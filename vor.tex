\setcounter{page}{5}%
\chapter*{Vorwort zur ersten Auflage}

\vspace*{-5mm}
Ausgelöst durch den BOLOGNA-Prozess vollzieht sich zur Zeit an den deutschen 
Universitäten bezüglich Aufbau und Inhalt des Studiums ein radikaler Umbruch. 
Die Umstellung der klassischen Diplomstudiengänge auf die gestuften 
Studiengänge Bachelor (6 Semester) bzw. Master (4 Semester) hat 
vielerorts schon stattgefunden oder steht kurz vor der Realisierung. Wer sich 
jedoch zur Zeit~im klassischen Diplomstudiengang Mathematik befindet, kann 
diesen noch abschließen, mancherorts ({\zB} in Heidelberg) ist auch noch 
ein Einstieg in den Diplomstudiengang zum Wintersemester 2007/2008 möglich.

Was die Studieninhalte betrifft, so ist die Änderung nicht so radikal wie die 
Änderung der Organisationsform mit dem Erwerb von Leistungspunkten 
oder "`Softskills"'. In zahlreichen Modulhandbüchern findet sich der 
bisherige Analysiszyklus "`Analysis 1 bis 3"' unter "`Analysis 1"', 
"`Analysis 2"' und etwa "`Höhere Analysis"' wieder. 
Sieht man sich jedoch die Inhaltsbeschreibungen dieser Vorlesungen an, 
so kann man feststellen, 
dass~der bisherige dreisemestrige Grundkurs über Analysis 
abgespeckt~wur\-de~und~etwas anspruchsvollere Themen und "`Leckerbissen"' der Analysis aus den 
Inhalten verschwunden sind.

An wen richtet sich dieser \slanted{Prüfungstrainer Analysis}?
Dieser Prüfungstrainer richtet sich an Studierende mit Mathematik als Haupt- 
oder Nebenfach, speziell an Studieren\-de der Mathematik (mit den 
Studienzielen Diplom, Bachelor, Lehramt), aber auch an Studierende der 
Naturwissenschaften (speziell der Physik), der Informatik oder auch der 
Wirtschaftswissenschaften, die nach ihrem Studienplan Grundvorlesungen in 
Analysis besuchen und entsprechende Prüfungen (mündlich oder schriftlich) 
ablegen müssen. Der Prüfungstrainer kann und will kein Lehrbuch ersetzen, 
und schon gar nicht den Besuch der entsprechenden Vorlesung. Wir 
gehen von der Voraussetzung aus, dass die Leserinnen und Leser 
die entsprechenden 
Vorlesungen gehört haben {und/\-oder} sich den Stoff anhand eines Lehrbuchs oder 
Scriptums erarbeitet haben, sich jedoch in konzentrierter Form auf 
mögliche Klausur- oder Prüfungsfragen vorbereiten wollen.
Zentrale Begriffe der Analysis werden dafür in einem konzisen Frage- und 
Antwortspiel wiederholt. Der Prüfungstrainer ist keine Aufgabensammlung 
(davon gibt es zahlreiche), sondern er zielt mit seinen Fragen 
und Antworten hauptsächlich auf das Verständnis mathematischer Begriffe und 
Konzepte. Der Nutzen des Prüfungstrainers soll darin liegen, dass die 
Leserinnen und Leser ihr Wissen in Analysis stichpunktartig überprüfen und 
trainieren können. Auch Studierende höherer Semester können schon
 einmal Gelerntes gezielt nachschlagen.

Fundierte Kenntnisse in den Grundlagen der Analysis sind nach unserer 
Meinung für jede weitere Beschäftigung mit Mathematik von grundsätzlicher 
Bedeutung, weshalb bei der Stoffauswahl auch Anwendungsbezüge und 
Zusammenhänge mit anderen mathematischen Gebieten berücksichtigt wurden.
Informationselemente zum Stoff finden sich an den Kapitelanfängen und 
teilweise am Beginn der Unterabschnitte. 
Natürlich musste an einigen Stellen eine Auswahl 
getroffen werden, wir hoffen aber trotzdem, mit den 
$7 \times 11 \times 13-1$ Fragen ein breites Spektrum 
abgedeckt zu haben. Wir weisen die \mbox{Leserinnen} und Leser ausdrücklich 
darauf hin, Prüfungsfragen, die sie im Prüfungstrainer vermisst haben, an den 
Verlag zu senden (siehe letzte Seite im Buch).

Basis für diesen Prüfungstrainer waren Test- und Wiederholungsfragen des 
erstgenannten Autors zu einem mehrfach an der Ruprecht-Karls-Universität 
Heidelberg  gehaltenen dreisemestrigen Analysiszyklus. Erfahrungen aus einem 
gerade abgeschlossenen Zyklus sind in die Fragen und Antworten 
eingeflossen.

\bigskip
\noindent%
\textbf{Einige Tipps zur Prüfungs- und Klausurvorbereitung:}

\begin{itemize}
\item Nehmen Sie frühzeitig vor der Prüfung mit 
ihrem Prüfer persönlichen Kontakt auf, fragen Sie Ihren Prüfer, 
auf welche Sachverhalte er einen besonderen 
Schwerpunkt legt.
\item In einem Prüfungsgespräch wird von Ihnen erwartet, zeigen zu 
können, dass 
Sie den behandelten Stoff verstanden haben. Ein erfahrener Prüfer kann dies 
sehr schnell feststellen. Ein Prüfer erwartet,
\begin{itemize}
\item dass Sie die grundlegenden Definitionen 
beherrschen ({\zB} die Definition 
für den Konvergenzradius einer Potenzreihe),
\item dass Sie die Aussagen der wichtigen Sätze parat haben 
(welche Sätze das sind, hat der Prüfer bestimmt mehrmals in der 
Vorlesung betont),
\item dass Sie die Sätze in konkreten Fällen anwenden können (etwa den 
Banach'schen Fixpunktsatz zum Beweis des lokalen Umkehrsatzes),
\item 
dass Sie bei den fundamentalen Sätzen die Hauptargumente des Beweises kennen
 -- neben den $7\times 11\times 13 -1$ von uns gestellten Fragen 
gibt es unzählige andere, 
ein Prüfer wird in der Regel aber nicht versuchen, eine Frage zu finden, die 
Sie nicht beantworten können.
\end{itemize}
\item Versuchen Sie, sich einen Überblick über die Vorlesung zu 
verschaffen. Dabei 
kann ein "`Roter Faden"' für die Vorlesung, den der Dozent 
hoffentlich häufig erwähnt und dem er hoffentlich auch gefolgt ist, sehr 
nützlich sein.
\item Simulieren Sie mit Kommilitonen die Prüfungssituation, indem Sie sich 
gegenseitig Fragen aus dem Prüfungstrainer stellen und die Antworten mit 
den "`Musterantworten"' vergleichen.
\item Wenn Sie den Prüfungstermin schon über eine längere Zeit 
ausgemacht haben, bringen Sie sich beim Prüfer kurz vor der Prüfung 
nochmals in Erinnerung.
\item Haben Sie Mut zur Lücke, bekennen Sie sich freimütig zu eventuellen 
Wissenslücken und versuchen Sie nicht durch Herumdrucksen Zeit zu gewinnen.
Eine Prüfungszeit von 30 bis 40 Minuten geht ohnehin wie im Flug vorbei.
\end{itemize}
In Klausuren werden neben Rechenfertigkeiten und Rechentechniken auch 
grundlegende Begriffe und Sätze abgefragt. Für die Klausurvorbereitung gelten 
also im Prinzip die gleichen Tipps.

Wir danken dem Verlagsteam von Spektrum Akademischer 
Verlag für die konstruktive Zusammenarbeit, insbesondere Herrn Dr. Andreas 
Rüdinger für die Idee zu diesem Prüfungstrainer und seine kompetente Beratung 
und Unterstützung während der Entstehungsphase.

\medskip\noindent
Heidelberg/Berlin, im Juli 2007 
\hspace*{\stretch{1}}{Rolf Busam\hspace{4pt}}\\
\hspace*{\stretch{1}}Thomas Epp
