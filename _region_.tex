\message{ !name(master.tex)}\documentclass[a4paper,10pt,openany,bibtotoc,indextotoc,liststotoc]
{lehrbuch17x24}
\usepackage{mymacros}

\includeonly{kompl}

\def\incl#1{\setcounter{Frage}{0}\include{#1}}

\begin{document}

\message{ !name(kompl.tex) !offset(67) }
\begin{antwort}
Um nachzuweisen, dass $\CC$ ein Körper ist, muss man die 9 Körperaxiome einzeln nachprüfen. 

$(\CC,\oplus)$ ist eine abelsche Gruppe mit dem neutralen Element $0=(0,0)$ und das 
additive Inverse zu $z=(a,b)$ ist $z':=(-a,-b)$. 

$(\CC\mengeminus \cdot)$ ist eine abelsche Gruppe mit dem neutralen Element $(1,0)$. 

Das multiplikativ Inverse $(x,y)$ zu $(a,b) \not= (0,0)$ ergibt sich 
aus dem eindeutig lösbaren linearen Gleichungssystem 
\[
ax-by = 1, \qquad bx+ay = 0.
\]
Es ist 
\[
x=\frac{a}{a^2+b^2}\quad\text{und}\quad
y=\frac{-b}{a^2+b^2}.
\]
Die Gültigkeit des Distributivegesetzes ist unmittelbar zu sehen, 
und das Assoziativgesetz für die Multiplikation ergibt sich einfach 
durch Nachrechnen aus dem Assoziativgesetz in $\RR$.

Ebenfalls durch elementares Nachrechnen bestätigt man, dass die Menge 
\[
\CC_{\RR} := \left\{ (a,0) | a\in\RR \right\}
\]
mit der auf $\CC_{\RR}$ eingeschränkten Addition und Multiplikation ein Körper ist 
und die Abbildung 
\[
j: \RR\to\RR, a \mapsto (a,0)
\]
ein Körperisomorphismus ist.

Identifiziert man die komplexe Zahl $(a,0)$ mit der reellen Zahl $a$, 
so wird $\RR$ zu einem Unterkörper von $\CC$.

Für das spezielle Element $\i=(0,1) \in \CC$ gilt 
\[
\i^2 = (0,1) \cdot (0,1) = (-1,0) = -1.
\]
Da ein Polynom vom Grad 2 höchstens 2 Nullstellen hat, sind also $\i$ und 
$-\i$ die einzigen Nullstellen der Gleichung $z^2 +1=0$. 

Die Normaldarstellung einer komplexen Zahl $z=(a,b)$ erhält man aus 
\[
z=(a,b) = (a,0) + (0,b) = a + (b,0) (0,1) = a+b\i.
\]

Die Eindeutigkeit ist evident, denn aus 
\[
a+b\i = a'+b'\i, \quad a\not=a', a,a',b,b' \in\RR
\]
folgt 
\[
a-a' = (b'-b) \i.
\]
Wäre $b'\not b$, dann folgte der Widerspruch $\frac{a-a'}{b-b'} = \i \in\RR$.

Also ist $b=b'$ und damit auch $a=a'$.   

\end{antwort}
\message{ !name(master.tex) !offset(-21) }

\end{document}