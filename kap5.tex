\chapter{Funktionenfolgen, Funktionenreihen, 
Potenzreihen}\label{funktionenfolgen}

Bei C.\,F. Gauß (Werke 3, S. 198) findet sich die 
folgende Bemerkung\index{Gauss@\textsc{Gauss}, Carl Friedrich (1777-1855)}
\begin{quote}
"`Die transzendenten Funktionen haben ihre wahre Quelle 
allemal, offenliegend oder versteckt, im Unendlichen. 
Die Operationen des Integrierens, der Summation unendlicher 
Reihen $\ldots$ oder überhaupt die Annäherung an eine 
Grenze durch Operationen, die nach bestimmten 
Gesetzen \slanted{ohne Ende} festgesetzt werden -- dies ist der 
eigentliche Boden, auf dem die transzendenten Funktionen 
erzeugt werden $\ldots$."'
\end{quote}
Durch endlich häufige Anwendung der vier Grundrechenarten 
erhält man aus der identischen Funktion und den konstanten 
Funktionen \slanted{Polynome} und \slanted{rationale Funktionen}, 
die in ihrem jeweiligen Definitionsbereich stetig sind. 
Alle anderen wichtigen Funktionen, die nicht zu dieser Klasse gehören,
erhält man dagegen erst durch (zum Teil mehrfache) 
Grenzprozesse. So ist etwa die Exponentialfunktion 
$\exp\fd \CC\to\CC$ für jedes $z\in\CC$ 
durch den Grenzwert der Folge  
$\left( \left( 1+\frac{z}{n} \right )^n \right)$ bzw. der Folge 
$\left( \sum_{k=0}^n \frac{z^k}{k!} \right)$ definiert. 

Ausgangspunkt aller Grenzprozesse bei Funktionenfolgen $(f_n)$ mit 
$f_n\fd D\to\KK$ ist der Begriff der \slanted{punktweisen Konvergenz}, 
der die Situation beschreibt, dass die Folgen $(f_n(x))$ für jedes $x\in D$ im 
Sinne der für Zahlenfolgen eingeführten Definition konvergieren. 
Durch die Festsetzung $f(x):=\lim f_n(x)$ wird unter dieser Voraussetzung  
eine Funktion $f \fd D \to \KK$ definiert. Naheliegende 
Fragestellungen sind nun:
{\setlength{\labelsep}{4mm}
\begin{itemize}
\item[\desc{i}] Übertragen sich die "`guten"' Eigenschaften wie Stetigkeit, 
Differenzierbarkeit, Integrierbarkeit der $f_n$ auf 
die Grenzfunktion $f$?
\item[\desc{ii}] Unter welchen Bedingungen ist die 
"`Vertauschung von Grenzprozessen"' gerechtfertigt, {\dasheisst},  
wann gelten Beziehungen etwa von der Art
\[
\lim_{n\to\infty} f_n'(x) = f'(x)  
\quad\text{oder}\quad 
\lim_{n\to\infty} \int_a^b  f_n(x) \difx = 
\int_a^b f(x) \difx\quad\text{?}
\]
\end{itemize} }
\noindent
Einfache Beispiele zeigen, dass der Begriff der punktweisen 
Konvergenz zu schwach ist, um die Gültigkeit solcher Übertragungs- 
und Vertauschungsbeziehungen zu gewähr\-lei\-sten. Aus diesem 
Grund nimmt man Zuflucht zum stärkeren Begriff der 
\slanted{gleichmäßigen Konvergenz}, den man elegant mithilfe der 
\slanted{Supremumsnorm} beschreiben kann.

Die gleichen Fragestellungen lassen sich auch im Bezug auf 
Funktionen\slanted{reihen} $\left(\sum_{k=0}^n f_k \right)$ stellen, zu 
deren wichtigsten Vertretern die Potenzreihen zählen. 
Aufgrund der speziellen Bauart von Reihen 
gibt es zahlreiche Kriterien für deren punktweise bzw. 
gleichmäßige Konvergenz, deren bedeutendstes (in Bezug auf gleichmäßige 
Konvergenz) der \slanted{Weierstraß'sche Majorantentest} ist. 

Es sei noch darauf hingewiesen, dass bei der 
Untersuchung von Funktionenreihen die Terminologie in den Lehrbüchern 
leider nicht immer einheitlich gehalten ist. Man muss die manchmal 
kleinen, aber feinen Unterschiede beachten.  


\section{Punktweise und gleichmäßige Konvergenz}

%% --- 1 --- %%
\begin{frage}\label{04_glk}\index{Konvergenz!gleichmäßige}
\index{Konvergenz!punktweise}
Wann heißt eine Folge $(f_n)$ von Funktionen $f_n \fd D \to \KK$ 
($\emptyset \not= D \subset \KK$) \bold{punktweise}, wann 
\bold{gleichmäßig konvergent}?
\end{frage}

\begin{antwort}
\slanted{Punktweise Konvergenz} einer Funktionenfolge $(f_n)$
bedeutet, dass die Folge $(f_n(x))$ der Funktionswerte für 
jedes $x\in D$ im Sinne einer reellen oder komplexen Zahlenfolge 
konvergiert. Bezeichnet man mit $f(x)$ den Grenzwert der Folge $(f_n(x))$, 
so heißt das also: Zu jedem $x\in D$ und jedem 
$\eps >0$ existiert ein $N:=N(x,\eps) \in \NN$, sodass gilt
\[
| f_n (x) - f(x) | < \eps  \qquad\text{für alle $n>N$}.
\]
Man beachte, dass in diesem Fall die Schranke $N$ sowohl von $\eps$ 
\slanted{als auch von $x$} abhängig ist. Hierin liegt der Unterschied 
zur gleichmäßigen Konvergenz. 

Ist die Folge $(f_n)$ \slanted{gleichmäßige konvergent}, so gibt es 
zu jedem $\eps >0$ eine \slanted{von $x$ unabhängige, universelle Schranke}
$N:=N(\eps)$, sodass gilt 
\[
| f_n (x) - f(x ) | < \eps \qquad\text{für alle $n>N$ und alle $x\in D$}.\EndTag
\] 
\end{antwort}

%% --- 2 --- %%
\begin{frage}
Wie lauten die Definitionen für punktweise bzw. 
gleichmäßige Konvergenz in Quantorenschreibweise? 
\end{frage}

\begin{antwort}
 Dir Definitionen lauten:
\begin{align}
\forall \eps > 0\; \forall x\in D\;  
\exists N\in\NN\; \forall n > N:\;| f_n(x)-f(x) | < \eps &\qquad 
\text{(punktweise)} \notag \\
\forall \eps > 0\;  \exists N\in\NN\;
\forall n > N\; \forall x\in D:\; | f_n(x)-f(x) | < \eps &\qquad 
\text{(gleichmäßig)} \EndTag 
\end{align}
\end{antwort} 

%% --- 3 --- %%
\begin{frage}\label{04_gbsp}\index{Grenzfunktion einer Funktionenfolge}
Können Sie ein einfaches Beispiel für eine Folge $(f_n)$ von 
auf $D:= [0,1]$ stetigen Funktionen $f_n\fd D \to \RR$ angeben, für welche 
die Grenzfunktion $f$ nicht stetig ist? Analysieren Sie den Grund für die 
Unstetigkeit.
\end{frage}
 

\begin{antwort}
Ein Beispiel liefert die Folge $(f_n)$ mit 
$f_n(x) =x^n$, \sieheAbbildung\ref{fig:04_glm1}. 
Jede Funktion $f_n$ ist stetig auf $[0,1]$. 
Die punktweise gebildete Grenzfunktion $f\fd [0,1]\to\RR$ 
mit 
\[
f(x)= \left\{ \begin{array}{ll}
0 & \text{für $0\le x < 1$} \\
1 & \text{für $x=1$,} 
\end{array}\right.
\]
ist dagegen unstetig im Punkt $1$. 

\begin{center}
  \includegraphics{mp/04_glm1}
  \captionof{figure}{Die Grenzfunktion dieser Folge stetiger 
    Funktionen ist nicht stetig.}
  \label{fig:04_glm1}
\end{center}

Die Ursache für dieses Phänomen liegt darin, 
dass die Funktionenfolge $(x^n)$ auf $[0,1]$ 
nicht \slanted{gleichmäßig} konvergiert. 

Für jedes $n\in\NN$ ist der maximale Abstand des Graphen von 
$f_n$ zu dem benachbarten Graphen von $f_{n+1}$ größer als 
eine bestimmte positive Zahl. Wegen der 
Stetigkeit der $f_n$ gibt es nämlich zu jedem $n\in\NN$ 
einen Punkt $x\in[0,1]$ mit $x^n=\frac{1}{2}$. 
Dann gilt $x^{n+1}=\frac{1}{4}$ und 
$|x^{n+1}-x^n|=\frac{1}{4}$. Damit erfüllt die Folge $(f_n)$ 
nicht das Cauchy-Kriterium für gleichmäßige Konvergenz, das in Frage 
\ref{04_cau} beschrieben wird. 
\AntEnd
\end{antwort}

%% --- 4 --- %%
\begin{frage}\label{04_glmdiff}
\index{Grenzfunktion einer Funktionenfolge}
Begründen Sie, warum die Funktionenfolge $(f_n)$ mit 
$f_n (x) := \frac{\sin nx}{\sqrt{n}}$
gleichmäßig gegen die Nullfunktion konvergiert, 
die Folge $(f_n')$ der Ableitungen aber divergiert. 
(Hier wird der Ableitungsbegriff als bekannt vorausgesetzt.) 
\end{frage}

\begin{antwort}
Für $\eps > 0$ gilt 
\[
\left| \frac{\sin nx}{\sqrt{n}} \right| \le \frac{1}{\sqrt{n}} 
< \eps \qquad\text{für $n>\frac{1}{\eps^2}$}. 
\]
Die Funktionenfolge $(f_n)$ konvergiert somit auf ganz $\RR$ 
gegen die Nullfunktion, und da die gegebene Abschätzung 
unabhängig vom Punkt $x$ ist, konvergiert sie dort sogar gleichmäßig gegen 
$f(x)\equiv 0$.

Die Folge der Ableitungen 
$f_n' = \sqrt{n}\cos n x$ divergiert allerdings für jedes $x$. Denn aus 
$\lim\limits_{n\to\infty} \sqrt{n}\cos n x = a\in\RR$ würde wegen 
$\lim\limits_{n\to\infty} \sqrt{n} =\infty$ 
zunächst 
$\lim\limits_{n\to\infty} \cos nx =0$ und damit auch 
$\lim\limits_{n\to\infty} \cos 2n x=0$ folgen. 
Mit dem Addtionstheorem für die Cosinus-Funktion 
(vgl. Frage \ref{05_trei} Teil (5))
gilt aber $\cos 2n x= 2\cos^2 nx -1$. Zusammen ergäbe sich  
der Widerspruch $0=-1$. 
\AntEnd   
\end{antwort}

%% --- 5 --- %%
\begin{frage}\label{04_glmint}
\index{Grenzfunktion einer Funktionenfolge}
Können Sie für die in Abbildung~\ref{fig:04_glmint} angedeutete 
Folge von Funktionen zeigen,  dass der Grenzwert der Integrale 
$\int_0^2 f_n$ nicht mit dem Integral der Grenzfunktion 
$\int_0^2 f$ übereinstimmt, also den Zusammenhang
\[
\int_0^2 \left( \limm f_n(x) \right) \difx \not= \limm \int_0^2 f_n(x)\difx.
\]
(Hier wird die Integration vorausgesetzt.)
\end{frage}  


\begin{antwort}
Für die Integrale der $f_n$ gilt 
\[
\int_0^2 f_n(x)\difx = 
\int_0^{1/n} f_n(x) \difx + \int_{1/n}^{2/n} f_n(x) \difx = 1
\quad\text{für alle $n\in \NN$}.
\]
Die Folge der Integrale konvergiert somit gegen $1$. 
Dagegen konvergiert die Folge $(f_n)$ punktweise gegen die 
Funktion $f(x)\equiv 0$, 
und für deren Integral gilt freilich $\int_0^2 f(x)\difx =0$.

\begin{center}
  \includegraphics{mp/04_glmint}
  \captionof{figure}{Das Integral jeder Funktion $f_n$ ist 1, das Integral der 
    Grenzfunktion jedoch 0.}
  \label{fig:04_glmint}
\end{center}

Dass die Vertauschung des Grenzprozesses hier nicht zulässig ist, 
hängt wiederum mit der nicht gleichmäßigen Konvergenz der 
Folge $(f_n)$ zusammen. Diese wiederum ist offensichtlich, 
da die Maximalwerte von $f_n$ und $f_{n+1}$ 
sich für alle $n\in\NN$ um den Wert $n$ unterscheiden.   
\AntEnd 
\end{antwort}

%% --- 6 --- %%
\begin{frage}\label{04_gsup}\index{Supremumsnorm}
Wie kann man die gleichmäßige Konvergenz mithilfe der 
\bold{Supremumsnorm} ${\| f \|}_D := \sup \{ |f(x)|;\; x\in D\}$  
beschreiben?
\end{frage}

\begin{antwort}
Die Äquivalenz
\[
|f_n(x)-f(x)|< \eps \quad\text{für alle $x\in D$}\quad  
\LLa
{\| f_n-f \|}_D < \eps.  
\]
ermöglicht folgende Definition der gleichmäßigen Konvergenz: \slanted{
Eine Folge von Funktionen $f_n\fd D\to \RR$ konvergiert gleichmäßig 
gegen $f$ genau dann, wenn für alle $\eps>0$ ein $N\in \NN$ existiert, 
sodass für alle $n>N$ gilt: $\| f_n-f \|_D < \eps.$} \AntEnd
\end{antwort}



%% --- 7 --- %%
\begin{frage}\index{eps@$\eps$-Schlauch}
Wie kann man die gleichmäßige Konvergenz im Fall reellwertiger Funktionen 
mithilfe des "`$\eps$-Schlauches"' veranschaulichen?
\end{frage}

\begin{antwort}
Bei gleichmäßiger Konvergenz verlaufen die Graphen der $f_n$ ab einem 
bestimmten Index $N$ alle innerhalb des $\eps$-Schlauches 
\[
\{ (x,y)\in D\times \RR; \; |y-f(x)| < \eps \}.
\] 

\picskip{3}\noindent
Abbildung\ref{fig:04_schlauch} zeigt den grau gezeichneten Graphen von $f$ mit  
$\eps$-Schlauch und eine der Annäherungsfunktionen $f_n$ mit $n>N$.\AntEnd

\begin{center}
  \includegraphics{mp/04_schlauch}
  \captionof{figure}{$\eps$-Schlauch einer reellen Funktion $f$.}
  \label{fig:04_schlauch}
\end{center}
\end{antwort}

%% --- 8 --- %%
\begin{frage}\label{04_cau}\index{Cauchy-Kriterium!für gleichmäßige Konvergenz}
Wie lautet das \bold{Cauchy-Kriterium} für gleichmäßige Konvergenz?
\end{frage}

\begin{antwort}
Mit diesem Kriterium lässt sich die gleichmäßige Konvergenz einer 
Funktionenfolge ohne Bezug auf eine eventuelle Grenzfunktion untersuchen. 
Es besagt: 

\medskip
\noindent\satz{Eine Folge $(f_n)$ von Funktionen 
$f_n\fd D\to \RR$ konvergiert gleichmäßig genau 
dann, wenn zu jedem $\eps>0$ ein $N\in n$ existiert, sodass 
für alle $m,n>N$ gilt:
\begin{equation}
\| f_n- f_m\| <\eps.\notag
\end{equation}}
Wie in den anderen Versionen beweist man auch dieses Cauchy-Kriterium 
unter Ausnutzung der 
Eigenschaften der Norm mit einfachen Standardmethoden.  
\AntEnd
\end{antwort}

%% --- 9 --- %%
\begin{frage}\label{04_stet}\index{Grenzfunktion einer Funktionenfolge}
Warum ist für eine \bold{gleichmäßig konvergente} Folge 
von stetigen Funktionen $f_n \fd D\to \KK$ die Grenzfunktion $f$ ebenfalls 
stetig?
\end{frage}

\begin{antwort}
 Sei $a$ ein beliebiger Punkt aus $D$. Es geht darum, zu zeigen, 
dass $f$ in $a$ stetig ist.  
Wegen der gleichmäßigen Konvergenz der $f_n$ 
gibt es ein $N\in \NN$, sodass  
\[
|f_N(x)-f(x)|<\eps/3
\qquad\text{für jedes $x\in D$}
\] 
gilt. Mit der Dreiecksungleichung folgt daraus 
\begin{align*}
|f(x)-f(a)| &\le |f(x)-f_N(x)| + |f_N(x)-f_N(a)| + |f_N(a)-f(a)| \\
&< |f_N(x)-f_N(a)| + \frac{2}{3}\eps. 
\end{align*}
Nun ist $f_N$ nach der Voraussetzung stetig, und somit 
gibt es ein $\delta>0$ derart, dass für alle $x\in U_\delta(a)$ 
die Differenz $|f_N(x)-f_N(a)|$ 
ebenfalls kleiner als $\eps/3$ ausfällt. Insgesamt folgt  
\[
|f(x)-f(a)| < \eps \qquad\text{für alle $x\in U_\delta(a)$},
\]
also die Stetigkeit von $f$ in $a$.
\AntEnd
\end{antwort}

%% --- 10 --- %%
\begin{frage}\index{Konvergenz!bei Funktionenreihen}
Wie wird die punktweise bzw. gleichmäßige Konvergenz bei Funktionenreihen 
erklärt?
\end{frage}

\begin{antwort}
Für eine Funktionenreihe $\sum_{k} f_k$ erklärt man diese Begriffe, 
indem man die in Frage \ref{04_glk} gegebene Definition 
auf die Folge der Partialsummenfunktionen 
$\left( \sum_{k=0}^n f_k \right)$ anwendet. 

Eine Reihe $\sum_k f_k$ von Funktionen 
$f\fd D\to \KK$ konvergiert somit 
\slanted{punktweise} gegen die 
Funktion $f$, wenn zu jedem $\eps>0$ und jedem $x\in D$ 
ein $N\in\NN$ existiert, sodass gilt:
\[
\left| f(x) - \sum_{k=0}^n f_k(x) \right| < \eps 
\qquad\text{für alle $n>N$}. 
\]
Die Funktionenreihe 
konvergiert \slanted{gleichmäßig}, 
wenn die Schranke $N$ unabhängig von $x$ gewählt 
werden kann, wenn also zusätzlich gilt:
\begin{equation}
\left| f(x) - \sum_{k=0}^n f_k(x)\right| < \eps 
\qquad\text{für alle $n>N$ und alle $x\in D$}. \EndTag
\end{equation}
\end{antwort}

%% --- 11 --- %%
\begin{frage}\label{04_caur}
\index{Cauchy-Kriterium!für gleichmäßige Konvergenz}
Wie lautet das Cauchy-Kriterium für die gleichmäßige Konvergenz einer 
Funktionenreihe?
\end{frage}

\begin{antwort}
Das Cauchy-Kriterium lautet: 

\medskip
\noindent\satz{Eine Funktionenreihe 
$\sum_k f_k$ mit $f_k\fd D\to\RR$ konvergiert 
gleichmäßig genau dann, wenn zu jedem $\eps > 0$ ein $N\in\NN$ existiert, 
sodass für alle $n,m>N$ mit $n>m$ gilt 
\[
\left\| \sum_{k=0}^n f_k - \sum_{k=0}^m f_k \right\|_D = 
\left\| \sum_{k=m+1}^n f_k  \right\|_D <\eps.
\]}
\noindent
Dieses Kriterium ist eine unmittelbare Folge des Cauchy-Kriteriums 
für Funktionenfolgen.  
\AntEnd
\end{antwort}

%% --- 12 --- %%
\begin{frage}\index{absolute Konvergenz}
\index{Konvergenz!absolute}\index{Konvergenz!absolut gleichmäßige}
Wann heißt eine Funktionenreihe $\sum f_k$ auf $D$ \bold{absolut konvergent} 
bzw. \bold{absolut gleichmäßig konvergent}?
\end{frage}

\begin{antwort}
Eine Funktionenreihe $\sum f_k$ heißt 
\slanted{absolut konvergent}, wenn die 
Reihe $\sum |f_k|$ für alle $x\in D$ punktweise konvergiert, 
sie heißt \slanted{absolut gleichmäßig konvergent}, 
wenn $\sum |f_k|$ gleichmäßig konvergiert.
\AntEnd
\end{antwort}

%% --- 13 --- %%
\begin{frage}\label{04_abgl}
Wieso folgt aus absolut gleichmäßiger Konvergenz die gleichmäßige 
Konvergenz?
\end{frage}

\begin{antwort}
Ist $\sum f_k$ absolut gleichmäßig konvergent, so gilt  
$\sum_{k=m}^n |f_k(x)|<\eps$ für alle 
$x\in D$ und alle $n,m\in\NN$ mit $n>m>N$ mit einem hinreichend großen 
$N$. Daraus folgt $
\left| \sum_{k=m}^n f_k(x) \right| \le 
\sum_{k=m}^n |f_k(x)| < \eps$ für alle $x\in D$ und alle 
$m,n>N$ mit $n>m$. 
Die Reihe $\sum f_k$ konvergiert somit gleichmäßig aufgrund des 
Cauchy-Kriteriums aus Frage \ref{04_caur}.
\AntEnd 
\end{antwort}

%% --- 14 --- %%
\begin{frage}\index{Konvergenz!normale}\index{normal konvergent}
Wann heißt eine Reihe $\sum_n f_n$ von Funktionen 
$f_n \fd D\to \RR$ auf einem Intervall $D\subset\RR$ 
\bold{normal konvergent}?
\end{frage}

\begin{antwort}
Die Reihe heißt \slanted{normal konvergent}, wenn jeder Summand 
$f_n$ auf $D$ beschränkt und die Reihe der Normen bezüglich 
$D$ konvergiert: $\sum_{n=1}^\infty \n{f}_D  < \infty$.
\AntEnd
\end{antwort}

%% --- 15 --- %%
\begin{frage}
Bei verschiedenen Lehrbuchautoren werden im Zusammenhang mit der Konvergenz 
von Funktionenreihen $\sum f_k$ folgende Konvergenzbegriffe verwendet: 

\medskip
\begin{tabular}{lp{3mm}l}
\rule{4pt}{4pt}\quad punktweise Konvergenz & &
\rule{4pt}{4pt}\quad gleichmäßige Konvergenz \\
\rule{4pt}{4pt}\quad absolut gleichmäßige Konvergenz & &
\rule{4pt}{4pt}\quad normale Konvergenz
\end{tabular}

\medskip\noindent%
Ge Sie je ein Beispiel für eine 
\bold{punktweise konvergente} 
Funktionenreihe angeben, die 
\begin{itemize}[4mm]
\item[\desc{1}] gleichmäßig und absolut konvergiert.\\[-4mm]
\item[\desc{2}] gleichmäßig und nicht absolut konvergiert.\\[-4mm]
\item[\desc{3}] nicht gleichmäßig und absolut konvergiert.\\[-4mm]
\item[\desc{4}] nicht gleichmäßig und nicht absolut konvergiert.\\[-4mm]
\item[\desc{5}] in jedem kompakten Teilintervall $[a,b]\subset D$ 
gleichmäßig, aber in $D$ nicht normal konvergiert.
\end{itemize}
\end{frage}


\begin{antwort}
\desc{1} Die Reihe $\sum  x^k$ konvergiert 
gleichmäßig und absolut auf jedem kompakten Intervall 
$[a,b]\subset \open{-1,1}$.  
Die gleichmäßige Konvergenz folgt dabei aus der Konvergenz der 
Reihe $\sum M^k$ mit $M:=\max\{|a|,|b|\}$. Für ein hinreichend 
großes $N\in\NN$ gilt $\sum_{k=n}^m M^k < \eps$ für alle $n,m>N$, 
und wegen $|x| \le M$ für alle $x\in [a,b]$ folgt $\sum_{k=n}^m |x|^k < \eps$ 
für alle $x\in \open{a,b}$ und alle $n,m>N$ mit $m>n$. 
Mit dem Cauchy-Kriterium aus Frage \ref{04_cau} 
und dem Ergebnis von Frage \ref{04_abgl} folgt daraus die 
gleichmäßige Konvergenz von $\sum x^k$ auf $[a,b]$. 

\medskip
\noindent
\desc{2} Die Reihe $\sum x^k/k$ konvergiert gleichmäßig auf jedem 
kompakten Intervall $[-1,a]$ mit $a<1$. Auf dem Intervall $\open{-1,a}$ 
folgt die (sogar absolute) Konvergenz durch Vergleich mit 
der geometrischen Reihe. Im Fall $x=-1$ handelt es sich um die alternierende 
harmonische Reihe, die konvergiert, aber nicht absolut konvergiert. 

\medskip
\noindent
\desc{3} Die Reihe $\sum x^k$ konvergiert absolut für jedes 
$x\in \open{-1,1}$. 
Sie konvergiert auf dem Intervall $\open{-1,1}$ 
allerdings nicht gleichmäßig, denn für $x>1/2$ gilt 
\[
\left| \sum_{k=N}^\infty x^k \right|  =  
x^N \sum_{k=0}^\infty x^k > \frac{1}{2^N}\cdot \frac{1}{1-x}. 
\]  
Der Term rechts ist unbeschränkt für $x\to 1$, es kann also kein 
$N\in\NN$ geben, für das $\left|\sum_{k=N}^\infty x^k\right|< \eps$ 
für alle $x\in \open{-1,1}$ gilt. Die Reihe hat somit nicht die 
Cauchy-Eigenschaft aus Frage \ref{04_cau} und kann daher nicht 
gleichmäßig konvergent sein.

\medskip
\noindent
\desc{4} 
Aus der Kombination von Teilantwort (2) und (3) folgt, 
dass die Reihe $\sum_k x^k$ im Intervall $\ropen{-1,1}$ zwar konvergiert, 
aber weder gleichmäßig noch absolut. 

\medskip
\noindent
\desc{5} Die Reihe $\sum x^k$ ist gleichmäßig konvergent auf 
jedem kompakten Teilintervall von $\open{-1,1}$, aber wegen 
$\| x^k \|_{\open{-1,1}} = 1$ kann sie im Intervall $\open{-1,1}$ 
nicht normal konvergieren.
\AntEnd
  
 
\end{antwort}
%% --- 16 --- %%
\begin{frage}\label{04_norm}
\index{Konvergenz!normale}
\index{Konvergenz!absolute}
\index{Konvergenz!gleichmäßige}
Warum folgt aus der normalen Konvergenz einer Funktionenreihe die 
absolute und die gleichmäßige Konvergenz der Reihe?
\end{frage}

\begin{antwort}
 Konvergiert die Reihe $\sum f_k$ normal, so besitzen die 
Reihen $\sum | f_k(x) |$ für jedes $x\in D$ in $\sum \| f_k \|$ eine 
konvergente Majorante und sind damit ebenfalls konvergent. Die Reihe 
$\sum f_k$ konvergiert also absolut.

Ferner gilt dann für jedes $\eps>0$ und hinreichend großes $N\in\NN$ 
\[
\left| \sum_{k=n}^\infty f_k(x) \right| \le 
\sum_{k=n}^\infty | f(x) | \le
\sum_{k=n}^\infty \| f_k \|_D < \eps\qquad 
\text{für alle $n>N$ und alle $x\in D$},
\]
und somit konvergiert $\sum f_k$ nach dem Cauchy-Kriterium gleichmäßig.
\AntEnd 
\end{antwort}

%% --- 17 --- %%
\begin{frage}
\index{Weierstrassches Majorantenkriterium@
Weierstraß'sches Majorantenkriterium}
\index{Majorantenkriterium!für eine Funktionenreihe}
\index{Weierstrass@\textsc{Weierstra{\ss}}, Karl (1815-1897)}
Was besagt das \bold{Weierstraß'sche Majorantenkriterium} für die 
normale Konvergenz einer Funktionenreihe?
\end{frage}

\begin{antwort}
Das Kriterium besagt: 

\medskip\noindent
\slanted{Gilt für eine Funktionenreihe $\sum f_k$ für 
jedes $k\in \NN$ und jedes $x\in D$ die Ungleichung $|f_k(x)| \le c_k$ 
und konvergiert die Reihe $\sum c_k$, 
so konvergiert die Funktionenreihe $\sum f_k$ normal und damit gleichmäßig auf $D$.}

\medskip\noindent
Dieses Kriterium erhält man durch eine Anwendung  
des Majorantenkriteriums für reelle oder 
komplexe Zahlenfolgen auf die Reihe $\sum \| f_k \|_D$. 
Diese besitzt unter den gegebenen Bedingungen in $\sum c_k$ eine 
konvergente Majorante und ist damit selbst konvergent. 
Nach der Antwort zu Frage \ref{04_norm} 
folgt daraus die gleichmäßige Konvergenz von $\sum f_k$.\AntEnd
\end{antwort} 
  

\section{Potenzreihen}

\slanted{Potenzreihen} wurden schon in Abschnitt \index{Potenzreihe}
\ref{elmenentare_potenzreihen} als Reihen 
bestimmter Bauart eingeführt (vgl. Frage \ref{02_potr}\,ff.).

Ein Aspekt, der in diesem Kapitel allerdings noch nicht 
deutlich werden konnte, war die Rolle, 
die Potenzreihen als spezielle \slanted{Funktionenfolgen bzw. -reihen} 
spielen. Auf der Menge aller $z\in\CC$, für 
die eine Potenzreihe $\sum_k c_k(z-a)^k$ konvergiert, 
definiert sie eine Funktion, die sich hinsichtlich ihrer spezifischen 
Funktionseigenschaften wie Stetigkeit, 
Differenzierbarkeit usw. untersuchen lässt. Es zeigt sich, dass 
die durch Potenzreihen gegebenen Funktionen alle diese angenehmen 
Eigenschaften besitzen. 
Potenzreihen stellen daher eines der wirkungsvollsten Mittel der Analysis dar, 
um neue Funktionen zu definieren bzw. konstruieren 
oder andere, die zunächst nur durch 
bestimmte Eigenschaften charakterisiert und gegeben sind, 
durch einen durchschaubaren analytischen Ausdruck darzustellen, 
mit dem sich "`fast"' so rechnen lässt wie mit Polynomen. Die 
Klasse derjenigen Funktionen, die sich durch Potenzreihen darstellen 
lässt, nennt man daher auch \slanted{analytische Funktionen}.


%% --- 18 --- %%
\begin{frage}\index{Potenzreihe}
Was versteht man unter einer \bold{Potenzreihe} in $\KK$ 
(mit $\KK=\RR$ oder $\KK=\CC$)?
\end{frage}

\begin{antwort}
Eine \slanted{Potenzreihe} im Punkt $z\in \KK$ zum Entwicklungspunkt 
$a\in \KK$ ist eine Reihe der Gestalt  
\[
\sum_{k=0}^\infty c_k(z-a)^k,\quad{a_k \in \KK}.
\]
Bezeichnet $D$ die Menge aller $z\in\KK$, für die die Potenzreihe 
konvergiert, so ist durch 
\[
P(z) := \sum_{k=0}^\infty c_k(z-a)^k
\]
eine Funktion $P\fd D\to\KK$ definiert. 
\AntEnd
\end{antwort}

%% --- 19 --- %%
\begin{frage}\index{Konvergenzradius}
Was versteht man unter dem \bold{Konvergenzradius} einer Potenzreihe?
\end{frage}

\begin{antwort}
 Nach Frage \ref{02_kok} gibt es zu jeder Potenzreihe 
$\sum a_k(z-a)^k$ ein $R\in \RR_+\cup\{\infty\}$, sodass die Reihe für alle 
$z\in U_R(a)$ konvergiert (sogar absolut) und außerhalb davon 
divergiert. 
Die Zahl $R$ nennt man den \slanted{Konvergenzradius} der Potenzreihe. 

Die durch 
\[
P(z):=\sum_{k=0}^\infty a_k(z-a)^k
\]
definierte Funktion ist somit 
zumindest auf der Menge $U_R(a)$ definiert. Unter Umständen  
macht diese Definition auch noch auf dem Rand von $U_R(a)$ Sinn, 
allerdings nur unter bestimmten Bedingungen. Der in Frage 
\ref{04_abel} besprochene \slanted{Abel'sche Grenzwertsatz}
 liefert dazu für 
reelle Potenzreihen ein erstes Ergebnis.\AntEnd 
  
\end{antwort}

%% --- 20 --- %%
\begin{frage}\index{Cauchy-Hadamard Formel}
\index{Konvergenzradius}
Welche Formeln zur Ermittlung des Konvergenzradius einer 
Potenzreihe sind Ihnen geläufig?
\end{frage}

\begin{antwort}
 Wichtig sind hier 
die Formeln von Cauchy-Hadamard und Euler, die 
jeweils eine Folge des Wurzel- bzw. Quotientenkriteriums für 
reelle oder komplexe Zahlenreihen sind. Sie wurden bereits in Frage 
\ref{02_kokf} formuliert und bewiesen.
\AntEnd
\end{antwort}

%% --- 21 --- %%
\begin{frage}\label{04_part}\index{partielle Summation}
Wie lässt sich eine Funktionenreihe $\sum_k a_k f_k$ mittels 
\bold{partieller Summation} darstellen?
\end{frage}

\begin{antwort}
Die \slanted{partielle Summation} einer Funktionenreihe 
lässt sich als diskrete Variante 
der aus der Integralrechnung bekannten \slanted{partiellen 
Integration} verstehen. 

Seien $a_1,\ldots, a_n$ und $f_1,\ldots,f_n$ beliebige 
Zahlen oder Funktionen (wobei man sich natürlich 
hauptsächlich für den Fall interessiert, dass diese als Glieder einer 
Folge gegeben sind). Mit $A_k := \sum_{j=1}^k a_j$ gilt dann
\[
\boxed{
\sum_{k=1}^n a_k f_k = \sum_{k=1}^n A_k ( f_k-f_{k+1} ) + A_n f_{n+1}, }
\]
wobei $f_{n+1}$ sogar beliebig gewählt werden kann.

Der Beweis ergibt sich rein rechnerisch durch Manipulation 
endlicher Summen. Setzt man $A_0 :=0$, so gilt $a_k=(A_k-A_{k-1})$ 
für alle $k\in\{ 1,\ldots n \}$. Durch geeignetes Zusammenfassen  
der Summanden und Indexverschiebungen erhält man
\begin{align}
\sum_{k=1}^n a_k f_k &= 
\sum_{k=1}^n ( A_k-A_{k-1} ) f_k = 
\sum_{k=1}^n A_k f_k - \sum_{k=1}^n A_{k-1} f_k \notag \\
&=
\sum_{k=1}^n A_k f_k -\sum_{k=1}^{n-1} A_k f_{k+1} = 
\sum_{k=1}^n A_k f_k -\sum_{k=1}^{n} A_k f_{k+1} + A_n f_{n+1} \notag \\ 
&= 
\sum_{k=1}^n A_k ( f_k-f_{k+1} ) + A_n f_{n+1}.\EndTag
\end{align}
\end{antwort}

%% --- 22 --- %%
\begin{frage}
Warum haben die Potenzreihen
\[
\sum \frac{z^k}{k!},\quad
\sum (-1)^k \frac{z^{2k}}{(2k)!},\quad
\sum (-1)^k \frac{z^{k+1}}{(2k+1)!},\quad
\sum \frac{z^{2k}}{(2k)!},\quad
\sum \frac{z^{2k+1}}{(2k+1)!}
\]
alle den Konvergenzradius $R=\infty$? 
\end{frage}

\begin{antwort}
Die Exponentialreihe $\sum \frac{z^k}{k!}$ 
konvergiert absolut für jedes $z\in \CC$, was in Frage 
\ref{02_expd} gezeigt wurde. 
Hieraus ergibt sich auch für jedes $z\in \CC$ die absolute 
Konvergenz der anderen vier Reihen, da die Beträge von deren Summanden 
alle auch in der Reihe $\sum \left| \frac{z^k}{k!}\right|$ vorkommen. 
Jede der vier Reihen besitzt also in der Exponentialreihe eine konvergente 
Majorante.   
\AntEnd
\end{antwort}

%% --- 23 --- %%
\begin{frage}\label{04_glpo}\index{Konvergenzradius}
Hat eine Potenzreihe $\sum c_k (z-a)^k$ einen Konvergenzradius 
$R>0$ und ist $P\fd U_R(a) \to \KK$ definiert durch $
 z\mapsto P(z) = \sum_{k=0}^\infty c_k( z-a )^k$, 
warum ist dann $P$ in $U_R(a)$ stetig?
\end{frage}

\begin{antwort}
 Für jedes $r<R$ konvergiert die Potenzreihe wegen 
\[
\left| \sum_{k=0}^\infty c_k (z-a) \right| 
\le \sum_{k=0}^\infty |c_k| r^k 
\qquad\text{für alle $z$ mit $|z-a|\le r$}
\] 
gleichmäßig in der abgeschlossenen Umgebung 
$\overline{U_r (a)}$. Nach Frage \ref{04_stet} ist $P$ somit stetig 
in jeder abgeschlossenen Umgebung $\overline{U_r (a)}$ mit $r<R$, 
und daraus folgt die Stetigkeit von $P$ in  $U_R(a)$, da es für jedes 
$z\in U_R( a )$ ein $r<R$ mit $z\in  \overline{U_r (a)}$ gibt, 
\sieheAbbildung\ref{fig:04_potglm}.
\AntEnd

\begin{center}
  \includegraphics{mp/04_potglm}
  \captionof{figure}{Jeder Punkt $z\in U_R(a)$ liegt in einer 
    \slanted{abgeschlossenen} Kreisscheibe $\overline{U_r(a)}\subset U_R(a)$.}
  \label{fig:04_potglm}
\end{center}
\end{antwort} 

%% --- 24 --- %%
\begin{frage}
Können Sie eine komplexe Potenzreihe angeben, deren Konvergenzradius 
$R=1$ ist, die aber für keinen Punkt $|z|=1$ konvergiert?
\end{frage}

\begin{antwort}
Die geometrische Reihe $\sum z^{k}$
liefert ein entsprechendes Beispiel. 
Für alle Punkte $|z|<1$ folgt die (absolute) Konvergenz aus derjenigen 
der geometrischen Reihe. Für $|z|=1$ kann die Reihe aber nicht konvergieren, 
da in diesem Fall die Summanden überhaupt keine Nullfolge bilden. 
\AntEnd
\end{antwort}


%% --- 25 --- %%
\begin{frage}\label{04_abel}
\index{Abelscher_Grenzwertsatz@Abel'scher Grenzwertsatz}
\index{Abel@\textsc{Abel}, Nils Henrik (1802-1829)}
Was besagt der \bold{Abel'sche Grenzwertsatz} für reelle Potenzreihen?
\end{frage}

\begin{antwort}
Nach Frage \ref{04_glpo} konvergiert eine reelle Potenzreihe 
$\sum a_kx^k$ innerhalb ihres Konvergenzradius gleichmäßig 
und stellt dort eine stetige Funktion $f$
dar. Es stellt sich die naheliegende Frage, ob in den 
Fällen, in denen die Reihe auch noch in den Randpunkten 
ihres Konvergenzintervalls konvergiert, sich die Definition 
$f(x)=\sum a_kx^k$ auf diese Randpunkte ausdehnen lässt, so dass 
$f$ dort immer noch stetig ist. Der Abel'sche Grenzwertsatz liefert 
dafür die positive Antwort. Er lautet:

\medskip
\noindent\satz{Die reelle Potenzreihe 
$\sum a_kx^k$ konvergiere für die positive Zahl $x=R$. 
Dann konvergiert sie gleichmäßig auf dem Intervall $[0,R]$ und stellt 
dort eine stetige Funktion dar. (Ein entsprechender Satz gilt, wenn 
die Reihe für $x=-R$ konvergiert.)}
 
\medskip\noindent
Es genügt, den Abel'schen Grenzwertsatz für den Fall $R=1$ zu beweisen 
(da sich der allgemeine Fall durch Übergang zur Reihe $\sum a'_k x^k$ 
mit $a'_k=a_k/R^k$ auf diesen zurückführen lässt).  
Unter dieser Voraussetzung ist die Reihe $\sum_{k=0}^\infty a_k$ konvergent. 
Mit $A:= \sum_k a_k$ liefert die Abel'sche partielle Summation 
(s. Frage \ref{04_part}) 
\begin{eqnarray*}
\sum_{k=n+1}^m a_k x^k &=&
\sum_{k=n}^{m-1} A_k( x^k -x^{k+1} ) + A_m x^m - A_n x^n \\
&=&
\sum_{k=n}^{m-1} (A_k-A)(x^k-x^{k+1}) + (A_m-A)x^m - (A_n-A)x^n. 
\end{eqnarray*}
Wegen der Konvergenz von $\sum_k a_k$ gibt es ein $N\in\NN$, sodass  
$|A-A_k|<\eps$ ist für alle $k>N$. Folglich gilt für alle $m>n>N$
\[
\left| \sum_{k=n+1}^m a_k x^k \right|  \le 
\eps \sum_{k=n}^{m-1} |x^k-x^{k+1}| +2 \eps \le 4\eps 
\quad\text{für alle $x\in [0,1]$.}
\] 
Mit dem Cauchy-Kriterium folgt hieraus die gleichmäßige Konvergenz der 
Reihe auf dem Intervall $[0,1]$.
\AntEnd
\end{antwort}

%% --- 26 --- %%
\begin{frage}\label{04_caup}\index{Cauchy-Produkt}
Wie ist das \bold{Cauchy-Produkt (die Faltung)} zweier Potenzreihen 
$P(z)=\sum_k a_k z^k$ und $Q(z)=\sum_n b_n z^n$ definiert?
\end{frage}

\begin{antwort}
Das Cauchy-Produkt ist die durch 
\begin{equation}
\sum_{k=0}^\infty \left( \sum_{n=0}^k a_n b_{k-n} \right) z^k \tag{$\ast$}
\end{equation}
definierte Potenzreihe. 
Man erhält sie also durch formale Cauchy-Multiplikation der beiden 
Potenzreihen für $\sum a_kz^k$ und $\sum b_n z^n$. Diese konvergieren 
beide zusammen in dem kleineren ihrer Konvergenzkreise absolut. 
Innerhalb dieses Konvergenzkreises konvergiert nach Frage 
\ref{02_cprod} damit auch deren Cauchy-Produkt ($\ast$) und stellt 
dort die Funktion $p(z)q(z)$ dar.
\AntEnd
\end{antwort}



%% --- 27 --- %%
\begin{frage}\index{Identitatssatz fur Potenzreihen@
Identitätssatz für Potenzreihen}
Was besagt der \bold{Identitätssatz für Potenzreihen}? 
\end{frage}

\begin{antwort}
Der Satz lautet: 

\medskip\noindent
\slanted{Seien 
\[
f(z) = \sum_{\ell=0}^\infty a_\ell( z-z_0)^\ell, \qquad
g(z) = \sum_{\ell=0}^\infty b_\ell( z-z_0)^\ell
\]
zwei Potenzreihen, die beide in $U_R( z_0 )$ mit $R>0$ konvergieren. 
Stimmen dann die Funktionen $f$ und $g$ nur 
auf irgendeiner Folge $(z_k)\subset U_R(z_0)$ 
mit $\lim z_k = z_0$ überein, 
so sind beide Funktionen vollkommen identisch, es gilt also 
$a_\ell=b_\ell$ für alle $\ell\in\NN_0$.} \AntEnd
\end{antwort}

%% --- 28 --- %%
\begin{frage}
Wie lässt sich der Identitätssatz beweisen?
\end{frage} 

\begin{antwort}
Der Satz lässt sich induktiv beweisen. Zunächst gilt wegen 
der Stetigkeit von $f$ und $g$ und der Übereinstimmung von 
$f$ und $g$ auf der Folge $(z_k)$
\[
f(z_0)=\lim_{k\to\infty} f(z_k) = \lim_{k\to\infty} g(z_k) = g(z_0).
\]
Daraus folgt $a_0=b_0$, was uns den Induktionsanfang liefert. 

Angenommen, $a_\ell=b_\ell$ sei nun bereits für alle $\ell\le n$ gezeigt. 
Die beiden Potenzreihen 
\[
f_n (z) := \sum_{\ell=n+1}^\infty a_\ell(z-z_0)^\ell, \qquad
g_n (z) := \sum_{\ell=n+1}^\infty b_\ell(z-z_0)^\ell 
\]
konvergieren beide auf $U_R(z_0)$ und stimmen auf der Folge 
$(z_k)$ überein. Hieraus folgt mit demselben Argument wie oben folgt
\[
f_n(z_0)=\lim_{k\to\infty} f_n(z_k) = \lim_{k\to\infty} g_n(z_k) = g_n(z_0),
\]
und damit $a_{n+1}=b_{n+1}$, was im Induktionsschritt zu zeigen war. 

Insgesamt gilt also $a_\ell=b_\ell$ für alle $\ell\in\NN_0$, 
und selbstverständlich 
folgt daraus, dass die beiden Funktionen $f$ und $g$ identisch sind.
\AntEnd 
\end{antwort} 



  

 

 
