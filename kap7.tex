\chapter{Grundlagen der Integral- und Differenzialrechnung}

Sowohl die Differenzial- als auch die Integralrechnung 
gehören zum Kernbestand der Analysis, sie bilden den Inhalt des 
sogenannten "`Calculus"'. Beide gehen ursprünglich von geometrischen 
Fragestellungen aus. Bei der Differenzialrechnung etwa das 
\slanted{Tangentenproblem} für Kurven oder die Bestimmung von 
\slanted{Extremwerten}. In physikalischer 
Hinsicht entspricht das Problemen wie der Bestimmung von 
Momentangeschwindigkeiten oder Momentanbeschleunigungen, allgemeiner 
der Bestimmung der \slanted{momentanen Änderungsrate} einer Größe.

Bei der Integralrechnung steht dagegen geometrisch die 
Ermittlung von \slanted{Kurvenlängen}, \slanted{Flächeninhalten} 
oder \slanted{Rauminhalten} am Ausgangspunkt. Damit verwandte 
Probleme sind etwa die Bestimmung von \slanted{Dichten}, 
\slanted{Schwerpunkten} und \slanted{Mittelwerten} oder in 
physikalischen Anwendungen die Berechnung der Arbeit in einem 
nichtkonstanten Kraftfeld. Ferner führt die Aufgabe, aus der
Änderungsrate einer Größe die Größe selbst zu rekonstruieren, 
auf die Methoden der Integralrechnung.  

Der von Leibniz und Newton um 1670 entdeckte Zusammenhang von 
Differenzial- und Integralrechnung, der als \slanted{Hauptsatz der 
Differenzial- und Integralrechnung} bezeichnet wird, macht 
diese Kalküle zu einem schlagkräftigen Werkzeug. 

Als Integralbegriff wählen wir in diesem Kapitel das 
Integral für \slanted{Regelfunktionen}. Das Riemann-Integral spielt nur 
eine untergeordnete Rolle. Das Lesbesgue-Integral führen wir erst
bei der  mehrdimensionalen Integrationstheorie in 
Kapitel 11 ein. 

\section{Das Integral für Treppenfunktionen und Regelfunktionen}

Zunächst wird für eine einfache Klasse von Funktionen 
-- die Treppenfunktionen -- ein Integral definiert. Dieses wird anschließend 
auf die umfassendere Klasse der \slanted{Regelfunktionen} erweitert. 
Zu den Regelfunktionen (auf einem kompakten Intervall) gehören insbesondere 
die \slanted{stetigen} und \slanted{monotonen} Funktionen. 

%% --- 1 --- %%
\begin{frage}\label{06_step}\index{Treppenfunktion}
Was versteht man unter einer Treppenfunktion 
$t \fd [a,b]\to \RR$?
\end{frage}

\begin{antwort}
$t\fd [a,b]\to\RR$ heißt Treppenfunktion genau dann, wenn es 
Punkte $x_0,\ldots,x_n$ mit $a=x_0 < x_1 < \ldots < x_n =b $ gibt, 
sodass $t$ in jedem offenen Intervall $]x_{k-1},x_k[$ konstant ist, 
\sieheAbbildung\ref{fig:05_treppenfunktion}.

\begin{center}
  \includegraphics{mp/05_treppenfunktion}
  \captionof{figure}{Graph einer Treppenfunktion}
  \label{fig:05_treppenfunktion}
\end{center}

Die Werte der Funktion in den Zerlegungspunkten unterliegen dabei 
keiner Einschränkung.
\AntEnd
\end{antwort}


%% --- 2 --- %%
\begin{frage}\label{06_stpi}\index{Integral!einer Treppenfunktion}
Wie ist das \bold{Integral einer Treppenfunktion} $t\fd [a,b]\to\RR$ erklärt? 
\end{frage}

\begin{antwort}
Ist $c_k$ der konstante 
Wert der Funktion $t$ im Intervall $]x_{k-1}, x_k[$ und  
$\Delta x_k := x_k - x_{k-1}$ die Länge dieses Intervalls, so 
ist das Integral von $t$ definiert durch
\begin{equation}\boxed{
\int_a^b t(x)\difx := \sum_{k=1}^n c_k \Delta x_k.}\notag
\end{equation}

Der Wert des Integrals entspricht geometrisch dem orientierten 
Flächeninhalt, den der Graph der Treppenfunktion 
mit der $x$-Achse einschließt. 
Das Integral ist also eine endliche Summe von Rechtecksinhalten, 
wobei diese mit positiver Bilanz eingehen, wenn der Funktionswert 
in dem betreffenden Intervall positiv ist, und mit negativer Bilanz, 
falls der Funktionswert negativ ist, \sieheAbbildung\ref{fig:06_stepf1}. 
\AntEnd

\begin{center}
  \includegraphics[width=7cm]{mp/06_stepf1}
  \captionof{figure}{Das Integral einer Treppenfunktion ist die Summe orientierter 
    Rechtecksinhalte.}
  \label{fig:06_stepf1}
\end{center}

\end{antwort}


%% --- 3 --- %%
\begin{frage}\index{Zerlegung!eines Intervalls}
Warum ist die Definition des Integrals einer Treppenfunktion 
unabhängig von der zugrunde gelegten Partition (Zerlegung) von $M$? 
\end{frage}


\begin{antwort}
Man betrachte zwei Zerlegungen
$Z_1$ und $Z_2$ von $[a,b]$ mit der Eigenschaft, 
dass $t$ auf den offenen Teilintervallen dieser Zerlegungen jeweils 
konstant ist. Ferner sei $Z$ diejenige feinere Zerlegung, die man durch 
Zusammenfassung der Zerlegungspunkte von $Z_1$ und $Z_2$ erhält. 
$I(Z_1)$, $I(Z_2)$ und $I(Z)$ 
bezeichne den Wert des Integrals von $t$ bezüglich 
der jeweiligen Zerlegung. Da die Einfügung eines zusätzlichen   
Teilungspunktes $\overline{x}$ 
offensichtlich (vgl. die Abbildung in der vorigen Frage) nichts am Wert 
der Summe in der Integraldefinition ändert, gilt also $I(Z)=I(Z_1)$ 
genauso wie $I(Z)=I(Z_2)$, und damit $I(Z_1)=I(Z_2)$.      
\AntEnd
\end{antwort}


%% --- 4 --- %%
\begin{frage}\index{Treppenfunktion}
\nomenclature{$\calli{T}(M)$}{Raum der Treppenfunkionen auf $M=[a,b]$}
Welche algebraische Struktur besitzt 
die Menge $\calli{T}(M)$ der Treppenfunktionen auf $M:=[a,b]$?
\end{frage}


\begin{antwort}
 
Die Menge der Treppenfunktionen auf $[a,b]$ 
bildet einen $\RR$-Vektorraum. 

Zu Treppenfunktionen $\varphi$ und $\psi$ wähle man eine Zerlegung derart, 
dass sowohl $\varphi$ als auch $\psi$ auf den offenen Teilintervallen 
konstant sind. Dann ist mit beliebigen 
reellen Zahlen $\alpha$ und $\beta$ 
auch $\alpha\varphi+\beta\psi$ auf diesen Teilintervallen 
konstant und somit eine Treppenfunktion. 
Dies kennzeichnet $\calli{T}(M)$ als einen $\RR$-Vektorraum. 
\AntEnd
\end{antwort}



%% --- 5 --- %%
\begin{frage}\label{06_trve}\index{Integral!einer Treppenfunktion}
Welche Haupteigenschaften hat die Abbildung
\[
I \fd \calli{T}(M)\to \RR; \qquad 
\varphi \mapsto I(\varphi) := \int_a^b \varphi(x)\difx \text{?}
\]
\end{frage}

\begin{antwort}
Seien $\varphi,\psi$ Treppenfunktionen auf $M$ und $a,b\in \RR$. 
Die Abbildung $I$ besitzt die folgenden Eigenschaften: \satz{
\[
\begin{array}{rp{2mm}lp{15mm}r}
\text{\desc{a}} & &
I(a\varphi+b\psi)= a \cdot I( \varphi)+b\cdot I(\psi),  & &
\text{(Linearität)} \\[1mm]
\text{\desc{c}} & &
\varphi \le \psi \Ra I(\varphi) \le I( \psi ). & &
\text{(Montonie)}\\[1mm]
\text{\desc{b}} & &
\left| I(\varphi) \right| \le I( |\varphi| ) \le (b-a)\cdot \| \varphi \|, & &
\text{(Beschränktheit)} 
\end{array}
\]}
Die Abbildung $I\fd T\to \RR$ ist somit ein 
\slanted{lineares, monotones, beschränktes Funktional}
\index{Funktional}

Zum Beweis zerlege man $[a,b]$ derart, dass sowohl $\varphi$ 
als auch $\psi$ auf den offenen Teilintervallen konstant sind. 
Mit der Integraldefinition aus Frage \ref{06_stpi} sind die drei 
Sachverhalte dann nichts anderes als einfache Feststellungen über 
endliche Summen.
\AntEnd
\end{antwort}



%% --- 6 --- %%
\begin{frage}
Wie erweitert man die Integraldefinition, wenn $a=b$ bzw. $b<a$ ist?
\end{frage}

\begin{antwort}
Für $a=b$ setzt man $\int_a^b \varphi(x)\difx =0,$ und für 
$b<a$ definiert man 
\[
\int_b^a \varphi(x)\difx = - \int_a^b \varphi(x)\difx.
\] 
Damit gelten die Eigenschaften $a$ und $b$ aus Frage $\ref{06_trve}$ auch in 
diesen Fällen. Eigenschaft \desc{c} bleibt für $a=b$ erhalten 
und gilt für $b<a$ sinngemäß mit verändertem Vorzeichen. 
\AntEnd

\smallskip
\end{antwort}

\smallskip
%% --- 7 --- %%
\begin{frage}\index{Regelfunktion}
Was versteht man unter einer \bold{Regelfunktion} $f\fd [a,b]\to\RR$?
\end{frage}

\begin{antwort}
Eine Regelfunktion lässt sich als 
\slanted{gleichmäßiger Limes von Treppenfunktionen} definieren. 
Das heißt:

\medskip
\noindent
Eine Funktion 
$f\fd [a,b]\to\RR$ ist eine Regelfunktion genau dann, wenn 
es eine Folge von Treppenfunktionen $\varphi_n \fd [a,b]\to\RR$ 
gibt, die gleichmäßig auf $[a,b]$ gegen $f$ konvergiert, wenn 
also gilt
\begin{equation}
\lim_{n\to\infty}\| f-\varphi_n \| = 0, \notag
\end{equation}
wobei $\n{\;\,}$ (hier und im Folgenden) 
die Supremumsnorm auf $[a,b]$ bezeichnet.
\AntEnd

\smallskip
\end{antwort}

\smallskip
%% --- 8 --- %%
\begin{frage}\index{Regelfunktion}
Wieso ist jede Regelfunktion beschränkt?
\end{frage}

\begin{antwort}
Wegen 
\[
\n{f} \le \n{f-\varphi_n}+\n{\varphi_n},
\] 
und weil jede Treppenfunktion beschränkt ist.
\AntEnd
\end{antwort}



%% --- 9 --- %%
\begin{frage}\index{Regelfunktion}
Wie lässt sich die Eigenschaft, eine Regelfunktion zu sein, 
geometrisch veranschaulichen?
\end{frage}


\begin{antwort}
Anschaulich sind Regelfunktionen dadurch gekennzeichnet, 
dass zu jedem $\eps>0$ eine Treppenfunktion existiert, deren Graph 
vollständig im $\eps$-Schlauch von $f$ verläuft, 
\sieheAbbildung\ref{fig:06_regel}. 

\begin{center}
  \includegraphics{mp/06_regel}
  \captionof{figure}{Bei einer Regelfunktion gibt es zu jedem 
    $\eps>0$ eine Treppenfunktion, die vollständig im $\eps$-Schlauch von 
    $f$ verläuft.}
  \label{fig:06_regel}
\end{center}

Man beachte, dass in dieser Definition die Forderung 
der \slanted{gleichmäßigen} Konvergenz entscheidend ist. 
Die nur punktweise 
Approximierbarkeit durch eine Folge von Treppenfunktion 
reicht im Allgemeinen nicht dafür aus, eine Regelfunktion zu sein. 
Ein Beispiel dafür wird in Frage \ref{06_pkon} gegeben. 
\AntEnd
\end{antwort}

%% --- 10 --- %%
\begin{frage}\index{Integral!einer Regelfunktion}
\nomenclature{$\int_a^b f(x) \difx$}{Regelintegral}
Wie ist das \bold{Integral einer Regelfunktion} definiert?  
\end{frage}
  
\begin{antwort}
Sei $f \fd [a,b] \to\RR$ 
eine Regelfunktion und $(t_n)$ eine Folge von Treppenfunktionen 
auf $[a,b]$, die gleichmäßig gegen $f$ konvergiert. Das Integral von $f$ 
ist dann definiert als der Grenzwert 
\begin{equation}
\boxed{
\int_a^b f(x) \difx = \lim_{n\to\infty} \int_a^b t_n (x) \difx.
}
\tag{$\ast$} 
\end{equation}
Um sicherzustellen, dass diese Definition auch sinnvoll ist, muss 
zweierlei gezeigt werden:
\satz{\setlength{\labelsep}{4mm}
\begin{enumerate}
\item[\desc{i}] Für jede gleichmäßig konvergente Folge von Treppenfunktionen 
$t_n\fd [a,b]\to\RR$ existiert der Grenzwert 
$\lim_{n\to\infty} \int_a^b t_n (x) \difx$.
\item[\desc{ii}] Für zwei gleichmäßig gegen $f$ konvergente Folgen 
$(t_n)$ und $(\psi_n)$ von Treppenfunktionen gilt 
\[
\lim_{n\to\infty} \int_a^b t_n (x) \difx = 
\lim_{n\to\infty} \int_a^b  \psi_n (x) \difx.
\]
Mit anderen Worten, das Integral von $f$ in der Definition ($\ast$) ist 
unabhängig von der approximierenden Folge von Treppenfunktionen.
\end{enumerate}}

\noindent
Die Eigenschaft \desc{i} erhält man als eine Folge 
der Beschränktheit des Integrals für Treppenfunktionen. 
Wegen
\[
\left| \int_a^b t_k(x) \difx - \int_a^b t_m(x) \difx \right| \le 
(b-a) \cdot \| t_k -t_m \| \le 
(b-a) \cdot ( \|t_k-f\| + \|f-t_m \| )
\]
und $\| f-t_n \| \to 0$ ist die Folge der Integrale 
$\int_a^b t_n(x) \difx$ eine Cauchy-Folge und damit konvergent. 

Um \desc{ii} zu zeigen, betrachte man die nach dem 
"`Reißverschlussprinzip"'  
gebildete Folge $t_1,\psi_1,\ldots,t_k,\psi_k,\ldots$. 
Diese Folge konvergiert gleichmäßig gegen $f$, 
und somit ist nach $\desc{i}$ auch 
die zugehörige Folge der Integrale konvergent. 
Deren Teilfolgen 
$\left( \int_a^b t_n(x)\difx \right)$ und 
$\left( \int_a^b \psi_n(x)\difx \right)$ besitzen somit denselben 
Grenzwert. 
\AntEnd
\end{antwort}

%% --- 11 --- %%
\begin{frage}\index{Regelfunktion}
\nomenclature{$\calli{R}(M)$}{Raum der Regelfunktionen auf $M=[a,b]$} 
Warum ist die Menge
$\calli{R}(M) := \{ f\fd M \to \RR\sets \text{$f$ Regelfunktion} \}$
ein $\RR$-Vektorraum?
\end{frage}


\begin{antwort}
Sind $f$ und $g$ Regelfunktionen,  
$(\varphi_n)$ und $(\psi_n)$ Folgen von Treppenfunktionen mit 
$\varphi_n \stackrel{glm}{\to}f$ und 
$\psi_n \stackrel{glm}{\to} g$, so ist mit $\alpha,\beta\in \RR$ auch 
auch $(\alpha\varphi_n + \beta\psi_n)$ eine Folge von Treppenfunktionen, und diese 
konvergiert gleichmäßig 
gegen die Funktion $\alpha f+\beta g$, die somit ebenfalls eine 
Regelfunktion ist.
\AntEnd
\end{antwort}


%% --- 12 --- %%
\begin{frage}\label{06_rgei}\index{Integral!einer Regelfunktion}
Welche Haupteigenschaften besitzt die Abbildung 
\[
I\fd \calli{R}(M) \to \RR; 
\qquad f\mapsto I(f) := \int_a^b f(x) \difx \text{?}
\]
\end{frage}

\begin{antwort}
Die Abbildung ist -- wie im Fall der Treppenfunktionen -- ein 
\slanted{lineares, monotones und beschränktes Funktional}. 
Mit $f,g\in \calli{R}(M)$ und $\alpha,\beta\in\RR$ gilt also
\begin{equation}
\boxed{
\begin{array}{rp{2mm}lp{15mm}r}
\text{\desc{i}} & &  
I(\alpha f+ \beta g)= \alpha I(f)+\beta I(g), & &
\text{(Linearität)} \\[2mm] 
\text{\desc{ii}} & &  
\left| I(f) \right| \le I( |f| ) \le (b-a) \cdot \| f \|, & &
\text{(Beschränktheit)} \\[2mm]
\text{\desc{iii}} & &  
f \le g \Ra I(f) \le I( g ). & &
\text{(Montonie)}
\end{array}}
\notag
\end{equation}
Seien $(\varphi_n)$ und $(\psi_n)$ Folgen von Treppenfunktionen 
mit $\varphi_n \stackrel{glm}{\to} f$ 
und $\psi_n \stackrel{glm}{\to} g$. 

\medskip
\noindent
\desc{i} Es gilt 
$(\alpha\varphi_n + \beta
\psi_n) \stackrel{glm}{\to} (\alpha f +\beta g)$ und damit 
\[
I(\alpha f + \beta g)  =  
\lim_{n\to\infty} \left( \int_a^b \alpha \varphi_n (x) \difx + 
\int_a^b \alpha \psi_n (x) \difx \right)= 
\alpha I(f)+\beta I(g).
\]

\medskip\noindent
\desc{ii} Aus $\| f-\varphi_n \| \to 0$ folgt 
$\| \, |f| - | \varphi_n| \, \| \to 0$ und damit
$\int_a^b |f(x)| \difx = \lim\limits_{n\to\infty} \int_a^b |\varphi_n(x)|\difx$. 
Also ist
\[
\left| \int_a^b f(x) \difx \right| = 
\lim_{n\to\infty} \left|  \int_a^b \varphi_n(x) \difx \right| \le
\lim_{n\to\infty} \| \varphi_n \| \cdot (b-a) = \| f \| \cdot (b-a).
\]

\medskip\noindent
\desc{iii} Man setze ${\varphi_*}_n := \varphi_n -\| f-\varphi_n \|$ und 
$\psi_n^* := \psi_n +\| g-\psi_n \|$. Dann sind ${\varphi_*}_n$ und 
$\psi^*_n$ Treppenfunktionen mit ${\varphi_*}_n \stackrel{glm}{\to} f$,  
$\psi_n^* \stackrel{glm}{\to} g$ 
und ${\varphi_*}_n \le f \le g \le \psi_n^*$ für alle 
$n\in \NN$. Hieraus folgt mit den Monotonieeigenschaften des Integrals 
für Treppenfunktionen  
\begin{equation}
\int_a^b f(x) \difx = \lim_{n\to\infty} \int_a^b{\varphi_*}_n (x) \difx 
\le  \lim_{n\to\infty} \int_a^b \psi_n^* (x)\difx = \int_a^b g (x)\difx.
\EndTag
\end{equation}
\end{antwort}


%% --- 13 --- %%
\begin{frage}\index{Intervalladditivität des Integrals}
Was versteht man unter der \bold{Intervalladditiviät} des Integrals?
\end{frage}

\begin{antwort}
Für eine Regelfunktion $f\fd [a,b] \to \RR$ und $c\in [a,b]$ gilt
\[
\int_a^b f (x)\difx = \int_a^c f(x)\difx + \int_c^b f(x)\difx.
\]
Dies ist für Treppenfunktionen eine einfache Aussage über endliche Summen. 
Daraus folgt der allgemeine Zusammenhang durch Übergang zum Grenzwert einer 
gleichmäßig konvergenten Folge von Treppenfunktionen.  
\AntEnd 
\end{antwort}



%% --- 14 --- %%
\begin{frage}\index{Stabilitätssatz für das Regelintegral}
\index{Vertauschbarkeit von Limesbildung und Integration}
Was besagt der \bold{Stabilitätssatz für das Regelintegral}? 
\end{frage}

\begin{antwort}
Der Satz liefert eine Aussage über die Vertauschbarkeit 
von Limesbildung und Integration für gleichmäßig konvergente 
Folgen von Regelfunktionen. Er besagt: 

\medskip\noindent
\satz{Für eine \slanted{gleichmäßig} konvergente Folge 
von Regelfunktionen $f_n \fd [a,b]\to\RR$ ist auch die Grenzfunktion eine 
Regelfunktion, und es gilt  
\[
\boxed{\int_a^b \lim_{n\to\infty} f_n \difx 
= \lim_{n\to\infty} \int_a^b f_n \difx.}
\]  }

\noindent
Für den Beweis des ersten Teils sei $f$ die Grenzfunktion der $f_n$ und 
$n$ so groß, dass $\| f-f_n \| < \frac{\eps}{2}$ gilt. 
Da $f_n$ eine Regelfunktion ist, gibt es ein 
$\varphi\in\calli{T}([a,b])$ mit $\| f_n - \varphi \| < \frac{\eps}{2}$. 
Daraus folgt
\[
\| f- \varphi \| \le \| f-f_n \| + \| f_n - \varphi \| < \eps. 
\] 
Die Funktion $f$ lässt sich also beliebig genau durch eine 
Treppenfunktion approximieren und gehört damit zu $\calli{R}([a,b])$.

Die Übereinstimmung der Integrale schließlich folgt aus
\begin{align*}
\left| \int_a^b f \difx - \int_a^b f_n \difx \right| 
&=
\left| \int_a^b (f - f_n) \difx \right| \le  
\int_a^b | f-f_n| \difx \\
 &\le (b-a) \cdot \| f-f_n \|  \le (b-a)\cdot \eps. \EndTag
\end{align*}
\end{antwort}

%% --- 15 --- %%
\begin{frage}\index{Banachraum}
Warum ist $\calli{R}(M)$ bezüglich der Supremumsnorm ein 
Banachraum (s.~Frage~\ref{q:banachraum})? 
Wieso ist $\calli{T}(M)$ kein Banachraum?
\end{frage}

\begin{antwort}
Ist $(f_n)$ eine Cauchy-Folge in 
$\calli{R}(M)$ bezüglich der Supremumsnorm, dann 
konvergiert sie gleichmäßig auf $M$, und nach dem Stabilitätssatz ist 
die Grenzfunktion ebenfalls eine Regelfunktion. Jede Cauchy-Folge in 
$\calli{R}(M)$ besitzt also einen Grenzwert in 
$\calli{R}(M)$, folglich ist $\calli{R}(M)$ ein Banachraum. 

Der Raum $\calli{T}(M)$ kann freilich 
kein Banachraum sein, da die Grenzfunktion einer gleichmäßig 
konvergenten Folge von Treppenfunktionen im Allgemeinen
keine Treppenfunktion ist. 
\AntEnd
\end{antwort}


%% --- 16 --- %%
\begin{frage}
Können Sie mit dem 
Stabilitätssatz das Integral $\int_a^b \exp(x) \difx $ 
berechnen?
\end{frage}

\begin{antwort}
Die Folge $(f_n)$ von Regelfunktionen 
mit $f_n =\sum_{k=0}^n \frac{x^k}{k!}$  
konvergiert auf jedem kompakten Intervall gleichmäßig gegen $\exp$. 
Mit dem Stabilitätssatz gilt also
\begin{align}
\int_a^b \exp(x) \difx &=
\int_a^b \lim_{n\to\infty} \sum_{k=0}^n \frac{x^k}{k!}\difx = 
\lim_{n\to\infty} \int_a^b \sum_{k=0}^n \frac{x^k}{k!}\difx =
\lim_{n\to\infty} \sum_{k=0}^n \int_a^b \frac{x^k}{k!}\difx 
\notag
\\
&=
\lim_{n\to\infty} \sum_{k=0}^n 
\frac{1}{k!} 
\left( \frac{b^{k+1}}{k+1} 
-  \frac{a^{k+1}}{k+1} \right) 
\notag
\\
&=
\lim_{n\to\infty} 
\sum_{k=1}^{n+1} 
\left( \frac{b^k}{k!} -  \frac{a^k}{k!}  \right)
=
\lim_{n\to\infty}
\sum_{k=0}^n 
\left( \frac{b^k}{k!} - \frac{a^k}{k!} \right) 
= \exp(b)-\exp(a). 
\EndTag
\end{align}
\end{antwort}


%% --- 17 --- %%
\begin{frage}\label{06_regf}\index{Regelfunktion}
Wieso sind \slanted{stetige Funktionen} Regelfunktionen?
\end{frage}

\begin{antwort}
 Eine stetige Funktion auf der kompakten Menge $M$ ist dort sogar 
gleichmäßig stetig. Es gibt also ein $\delta>0$, sodass 
$|f(x)-f(y)| < \eps$ für alle $x\in M$ und alle $y\in U_\delta(x)\cap M$ 
gilt. Man zerlege $M$ in endlich viele Intervalle $[x_{k-1},x_k]$ 
($1\le k \le n$ für ein geeignetes $n\in\NN$), 
die alle eine kleinere Länge als $\delta$ besitzen. Für die durch 
\[
\varphi( x ) = f(x_k) \text{ für $x\in \ropen{ x_{k-1},x_k }$}
\quad\text{ und }\quad
\varphi( x ) = f(b) \text{ für $x=b$}  
\]
gegebene Treppenfunktion 
$\varphi\fd M\to\RR$ gilt dann 
\[
\| \varphi-f \| < \eps. \EndTag
\]
\end{antwort}

%% --- 18 --- %%
\begin{frage}\label{06_regfm}\index{Regelfunktion}
Welche weitere wichtige Funktionenklasse 
gehört zu den Regelfunktionen?
\end{frage}

\begin{antwort}
 Die \slanted{monotonen Funktionen} sind ebenfalls Regelfunktionen.  
Für die Konstruktion einer 
approximierenden Treppenfunktion gehe man hier vom 
Intervall $[f(a),f(b)]$ auf der $y$-Achse aus und  
unterteile dieses in eine endliche Anzahl 
von Intervallen $[y_{k-1},y_k]$ mit einer kleineren Länge 
als $\eps$. Als entsprechende Zerlegungspunkte   
auf der $x$-Achse wähle man 
$x_k := \sup \{ x\in[a,b]; \, f(x)< y_k \}$. Die Funktion 
\[
\varphi( x ) = f(x_k)  \quad\text{für $x\in \open{x_k,x_{k+1}}$},  
\]
ist dann eine $\eps$-approximierende Treppenfunktion zu $f$.
\AntEnd
\end{antwort} 

%% --- 19 --- %%
\begin{frage}\index{stuckweise stetig@stückweise stetig (monoton)}
Die Ergebnisse aus Frage \ref{06_regf} und \ref{06_regfm} lassen 
sich unmittelbar auf eine größere Klasse von Funktionen verallgemeinern.  
Auf welche? 
\end{frage}

\begin{antwort}
 
Aus den Antworten folgt sofort, dass auch die 
\slanted{stückweise stetigen} bzw. 
\slanted{stückweise monotonen} Funktionen 
Regelfunktionen sind. Dabei heißt eine Funktion auf $[a,b]$ 
stückweise stetig bzw. monoton, wenn eine Zerlegung 
$a=x_0 < x_1 < \ldots < x_n=b$ des Intervalls 
$[a,b]$ existiert, sodass 
$f$ auf den offenen Intervallen $\open{x_k,x_{k+1}}$ 
stetig bzw. monoton ist. 
\AntEnd 
\end{antwort}


%% --- 20 --- %%
\begin{frage}\label{06_pkon}
\index{Vertauschbarkeit!von Limesbildung und Integration}
Können Sie ein Beispiel dafür angeben, dass der 
Stabilitätssatz (Vertauschungssatz) bei nur punktweiser Konvergenz im 
Allgemeinen nicht gilt?
\end{frage}

\begin{antwort}
Ein Beispiel für das Versagen der Vertauschbarkeit von Limesbildung 
und Integration bei 
nicht gleichmäßiger Konvergenz wurde schon in Frage 
\ref{04_glmint} gegeben. 

Auch für Folgen von Treppenfunktionen lässt sich ein sehr ähnlich 
geartetes Beispiel konstruieren. Die in der Abbildung~\ref{fig:06_stabil} 
angedeutete Folge $(\varphi_n)$ von Treppenfunktionen auf $[0,1]$ 
konvergiert gegen die Nullfunktion -- nur eben nicht gleichmäßig. 

\begin{center}
  \includegraphics{mp/06_stabil}
  \captionof{figure}{Eine nicht-gleichmäßig konvergente Folge von 
    Treppenfunktionen. 
    Vertauschung von Limesbildung und Integration ist hier nicht möglich.}
  \label{fig:06_stabil}
\end{center}

Die Folge der Integrale der $\varphi_n$ ist konstant $\frac{1}{4}$, 
konvergiert also nicht gegen das Integral der Grenzfunktion.
\AntEnd
\end{antwort}

%% --- 21 --- %%
\begin{frage}
Nach \ref{06_regfm} sind alle monotonen Funktionen auf kompakten Intervallen 
Regelfunktionen. 
Können Sie das für $f(x)=x^2$ das Integral 
$\int_0^b f(x) \difx$
mittels einer direkten Approximation durch eine Treppenfunktion berechnen? 
\end{frage}

\begin{antwort}
Für $n\in \NN$ zerlege man das Intervall $[0,b]$ in $n$ gleichlange 
Intervalle der Länge $ \frac{b}{n}$. Der $k$-te Zerlegungspunkt 
liegt dann an der Stelle $x_k:=\frac{kb}{n}$. Für $n\to\infty$ konvergieren  
die Längen der Teilintervalle gegen $0$, und da $f$ eine Regelfunktion 
ist folgt $\n{ f-\varphi_n }\to 0$.  

Für das Integral $I_n$ von $\varphi_n$ erhält man 
\[
I_n := \sum_{k=0}^{n-1} \left( \frac{kb}{n} \right)^2 \cdot \frac{b}{n} =
\left(\frac{b}{n}\right)^3 \sum_{k=1}^n k^2.
%\tag{$\ast$}   
\]
Die Summe lässt sich mit der Formel 
$\sum_{k=1}^n k^2 = \frac{n(n+1)(2n+1)}{6}$ auswerten. Dies liefert
\[
\int_0^b x^2 \difx = 
\lim_{n\to\infty}\frac{ n(n+1)(2n+1) }{ 6 } \cdot \frac{b^3}{n^3} = 
\lim_{n\to\infty}
\frac{b^3}{6} \left( 1+\frac{1}{n} \right) \left( 2+\frac{1}{n} \right)
=\frac{b^3}{3}.
\]
Mit der Intervalladditivität des Regelintegrals erhält man hieraus auch 
noch den allgemeineren Zusammenhang
$
\int_a^b x^2 \difx = \frac{b^3-a^3}{3}.
$\AntEnd
\end{antwort}

%% --- 22 --- %%
\begin{frage}\index{Mittelwertsatz!der Integralrechnung}
Was besagt der 
\bold{Erste Mittelwertsatz der Integralrechnung}? Kennen Sie eine 
Anwendung?
\end{frage}

\begin{antwort}
Aus der Beschränktheitseigenschaft (Frage \ref{06_rgei}\desc{b}) 
des Regelintegrals folgt unmittelbar die Existenz einer Zahl 
$\mu\in\RR$ mit $\mu \le \| f \|$ und 
$\int_a^b f \difx = \mu \cdot (a-b)$.

Für \slanted{stetige reellwertige} Funktionen $f$ gibt es aufgrund des 
Zwischenwertsatzes eine Zahl $\xi\in [a,b]$ mit $\mu=f(\xi)$. In diesem 
Fall gilt dann 
\[
\boxed{ \int_a^b f \difx = f(\xi) \cdot (a-b). }
\]
\picskip{0}\noindent

Dieser Sachverhalt 
(also die Existenz eines $\xi\in ]a,b[$ mit dieser Eigenschaft) ist ein 
Spezialfall des Mittelwertsatzes und wird auch oft 
\slanted{Erster Mittelwertsatz} genannt. Er besagt anschaulich, dass 
der Flächeninhalt unter dem Graphen mit dem Flächeninhalt eines 
Rechtecks der Seitenlänge $(b-a)$ und der Höhe $f(\xi)$ übereinstimmt, 
\sieheAbbildung\ref{fig:06_zwischenwertsatz}. \AntEnd

\begin{center}
  \includegraphics{mp/06_zwischenwertsatz}
  \captionof{figure}{Zur Veranschaulichung des Mittelwertsatzes: Die Fläche 
    des grauen Rechtecks ist gleich dem Integral von $f$ über dem Intervall 
    $[a,b]$}
  \label{fig:06_zwischenwertsatz}
\end{center} 
\end{antwort}

%% --- 23 --- %%
\begin{frage}\label{06_allgmittelwertsatz}
Was besagt der \index{Mittelwertsatz!der Integralrechnung}
\bold{Verallgemeinerte Mittelwertsatz} oder 
einfach \slanted{Mittelwertsatz der Integralrechnung}? 
\end{frage}

\begin{antwort}
 Der \slanted{Verallgemeinerte Mittelwertsatz} besagt: 

\medskip
\noindent\satz{Sei $f\fd [a,b]\to \RR$ stetig und 
sei $g\fd [a,b]\to \RR$ eine Regelfunktion mit $g\ge 0$. Dann gibt es ein 
$\xi\in [a,b]$ mit
\begin{equation}
\boxed{
\int_a^b f(x) g(x) \difx = f(\xi) \cdot \int_a^b g(x)\difx.
}
\asttag
\end{equation}
}

\medskip\noindent
Der Satz ergibt sich als 
Folge der Monotonieeigenschaften des Regelintegrals sowie  
des Zwischenwertsatzes für stetige reelle Funktionen. Bezeichnet man mit 
$m$ das Minimum und mit $M$ das Maximum von $f$ auf $[a,b]$, so gilt 
\[
m \int_a^b g(x) \difx \le \int_a^b f(x)g(x)\difx \le 
M \int_a^b g(x) \difx.
\]
Es gibt also eine Zahl $\mu$ zwischen $m$ und $M$ mit 
$\int_a^b f(x)g(x)\difx = \mu$, und da $f$ stetig ist, existiert 
ein $\xi\in [a,b]$ mit $\mu=f(\xi)$. Damit erhält man {\astref}. 
\AntEnd 
\end{antwort} 

%% --- 24 --- %%
\begin{frage}\index{Mittelwertsatz!der Integralrechnung}
Kennen Sie eine Anwendung des Mittelwertsatzes?
\end{frage}

\begin{antwort}
 
Eine wichtige Anwendung des Ersten Mittelwertsatzes kommt beim Beweis 
des Hauptsatzes der Differenzial- und Integralrechnung 
(s. Frage \ref{06_fa} und \ref{06_hpts}) vor.
\AntEnd
\end{antwort}

%% --- 25 --- %%
\begin{frage}\label{06_lips}
Ist $f\fd[a,b]\to\RR$ eine Regelfunktion. 
Warum ist dann die "`Integralfunktion"' 
\[
F_a \fd [a,b] \to \RR; \qquad x\mapsto \int_a^x f(t) \dift
\]
Lipschitz-stetig, also insbesondere stetig?
\end{frage}

\begin{antwort}
Aufgrund der Beschränktheitseigenschaft des Regelintegrals gilt
\[
\left| F_a(x) - F_a(y) \right| = 
\left| \int_y^x f(t) \dift \right| \le 
\nnb{ f }_{[a,b]} \cdot |x-y|.  
\] 
Damit ist die Integralfunktion Lipschitz-stetig im Sinne der Definition 
\ref{03_lips} (und zwar mit $\nnb{ f }_{[a,b]} =L$). \AntEnd
\end{antwort}

%% --- 26 --- %%
\begin{frage}\label{06_fa}
\index{Hauptsatz der Differenzial- und Integralrechnung}
Wenn $f$ in Frage \ref{06_lips} zudem stetig ist, 
warum ist dann die \index{Integralfunktion} $F_a$ sogar differenzierbar, wobei 
für alle $x\in[a,b]$ gilt: 
$F_a'(x)=f(x)$.
\end{frage}

\begin{antwort}
Sei $x_0 \in [a,b[$ beliebig. Nach dem Mittelwertsatz gibt es 
zu jedem $h>0$ mit $x_0+h\le b$ 
ein $\xi_h \in [x_0,x_0+h]$ mit 
$\int_{x_0}^{x_0+h} f(t)\dift  = h \cdot f(\xi_h)$. 
Dann gilt $\lim\limits_{h\to 0} \xi_h=x_0$, und daraus folgt 
zusammen mit der Stetigkeit von $f$
\[  
F_a' (x_0) = 
\lim_{h\to 0} \frac{F_a(x_0+h )-F_a(x_0)}{h} = 
\lim_{h\to 0} \frac{1}{h}  \int_{x_0}^{x_0+h} f(t)\dift = 
\lim_{h\to 0} f(\xi_h) = f(x_0 ).
\]
Auf dieselbe Weise zeigt man den Zusammenhang für $h<0$ und 
$x_0 \in ]a,b]$. Danach ist die Behauptung bewiesen, die uns auf 
direktem Weg zum \slanted{Hauptsatz der Differenzial- und Integralrechnung} 
führt, dem das Kapitel \ref{hauptsatz} gewidmet ist. 
\AntEnd
\end{antwort}


\section{Grundlagen der Differenzialrechnung}

%% --- 27 --- %%
\begin{frage} 
\index{Newton@\textsc{Newton}, Isaac (1643-1727)}
\index{Leibniz@\textsc{Leibniz}, Gottfried Wilhelm (1646-1716)}
Die Differenzialrechnung wurde unabhängig von I. Newton und 
G.\,W. Leibniz im 17. Jahrhundert entwickelt. Was war für Newton bzw. 
Leibniz ein wesentliches Motiv, das schließlich zur Differenzialrechnung 
führte?
\end{frage}

\begin{antwort}
Newtons hauptsächliche Motivation war \slanted{physikalischer} Natur,  
ihm ging es um die Berechnung von 
Momentangeschwindigkeiten, Momentanbeschleunigungen und dergleichen. 
Bei Leibniz stand mehr die Geometrie im Vordergrund, speziell
das \slanted{Tangentenproblem} 
für Kurven. 

Das Gemeinsame beider Ansätze liegt darin, aus 
\slanted{mittleren Änderungsraten}  
(Durchschnittsgeschwindigkeit, Sekantensteigung) durch Grenzübergang 
\slanted{momentane Änderungsraten}  
(Momentangeschwindigkeit, Tangentensteigung) zu ermitteln. 

Newton und Leibniz beschäftigten sich auch schon mit dem umgekehrten 
Problem, nämlich aus der momentanen Änderungsrate einer Größe 
die Größe selbst zu rekonstruieren. 
\AntEnd       
\end{antwort}

%% --- 28 --- %%
\begin{frage}\index{Differenzierbarkeit!einer Funktion einer Variablen}
\index{Differenzialquotient}
\index{Ableitung!einer Funktion einer Variablen}\label{06_diffdef}
\nomenclature{$f'(x)$}{Ableitung von $f$ in $x$}
Wann heißt eine Funktion $f\fd M\to \RR$ 
($M$ ein echtes Intervall) im Punkt $x_0\in M$ 
differenzierbar? Wann heißt sie auf $M$ differenzierbar?
\end{frage}

\begin{antwort}
$f$ heißt differenzierbar in $x_0 \in M$, wenn der Grenzwert 
\begin{equation}
\lim_{x\to x_0} \frac{f(x)-f(x_0)}{x-x_0} 
\tag{D$_1$}
\end{equation}
existiert. Gegebenenfalls bezeichnet man den Grenzwert mit $f'(x_0)$ und 
nennt ihn die \slanted{Ableitung von $f$ in $x_0$}.

Die Funktion heißt differenzierbar auf $M$, wenn der Grenzwert 
für jedes $x\in M$ existiert. Die durch $x\mapsto f'(x)$ gegebene 
Funktion $f'\fd M\to \RR$ heißt dann die \slanted{Ableitung von $f$}. 
\end{antwort}

%% --- 29 --- %%
\begin{frage}\index{Differenzierbarkeit!einer Funktion einer Variablen}
Was bedeutet in der Definition der Differenzierbarkeit die Formulierung 
"`der Grenzwert existiert"'?
\end{frage}

\begin{antwort}
Die Redewendung lässt sich auf drei äquivalente Weisen präzise 
ausdrücken (vgl. dazu die Fragen aus Kapitel 4). Der Grenzwert (D$_1$) existiert 
genau dann und hat den Wert $l$, 
wenn einer der folgenden (äquivalenten) Sachverhalte zutrifft:
{\setlength{\labelsep}{4mm}
\begin{itemize}
\item[\desc{i}] \slanted{$\eps\delta$-Definition}: 
Zu jedem $\eps > 0$ gibt es ein $\delta>0$, sodass 
gilt: \index{Differenzierbarkeit!eps@$\eps\delta$-Kriterium}
\[ \left| \frac{f(x)-f(x_0)}{x-x_0} - l \right| < \eps
\quad\text{für alle $x$ mit $0<|x-x_0|<\delta$}.
 \]
\item[\desc{ii}] \slanted{Folgenkriterium}: Für jede Folge 
$(x_n) \subset M$ mit \index{Differenzierbarkeit!eps@Folgenkriterium}
$\lim x_n = x_0$ und $x_n\not=x_0$ konvergiert die Folge der Quotienten 
$\left( \frac{ f(x_n)-f(x_0) }{ x_n - x_0 } \right)$ gegen $l$, 
{\dasheisst}, es gibt ein $N\in \NN$, sodass gilt:
\[ 
\left| \frac{f(x_n)-f(x_0)}{x_n-x_0} -a  \right| < \eps 
\quad\text{für alle $n>N$}.
\]
\item[\desc{iii}] \slanted{Stetige Fortsetzbarkeit}: Die Funktion 
$\varphi \fd  M\mengeminus \{x_0\} \to \RR$ mit  
$\varphi( x ) := \frac{f(x)-f(x_0)}{x-x_0}$ besitzt eine 
stetige Fortsetzung in $x_0$, und konvergiert bei Annäherung an 
$x_0$ gegen $l$, {\dasheisst}, die Funktion $\tilde{\varphi} \fd M \to \RR$ 
mit \index{Differenzierbarkeit!eps@im Sinne stetiger Fortsetzbarkeit}
\[
\tilde{\varphi}(x) := \left\{ \begin{array}{ll} 
\dis \frac{f(x)-f(x_0)}{x-x_0} & \text{für $x\not=x_0$} \\
l & \text{für $x=x_0$} \end{array}\right.
\]
ist stetig in $x_0$. \AntEnd 
\end{itemize}}
\end{antwort}


%% --- 30 --- %%
\begin{frage}\index{Differenzierbarkeit!geometrische Interpretation}
\index{Sekante}\index{Tangente}
Wie kann man die Differenzierbarkeit geometrisch interpretieren?
\end{frage}

\begin{antwort}
Der Quotient $\dis \frac{f(x)-f(x_0)}{x-x_0}$ beschreibt 
geometrisch die Steigung der 
Sekante an den Graphen von $f$ durch die 
Punkte $\big(x,f(x)\big)$ und $\big(x_0,f(x_0)\big)$. 

\begin{center}
  \includegraphics{mp/06_steigung}
  \captionof{figure}{%
    Die Differenzenquotienten beschreiben die Steigung von 
    Sekanten an den Graphen, die Ableitung als Grenzwert 
    einer Folge von Differenzenquotienten beschreibt die Steigung der 
    Tangente.}
  \label{fig:06_steigung}
\end{center}

Bei Annäherung des Punktes $x$ an den Punkt $x_0$  nähert sich die 
Sekante einer Geraden mit der Steigung $f'(x_0)$ an, 
\sieheAbbildung\ref{fig:06_steigung}. Diese Gerade 
durch $\big( x_0, f(x_0) \big)$ ist die Tangente $T$ an den Graphen 
von $f$ im Punkt $\big( x_0, f(x_0) \big)$. $T$ ist eine affin-lineare 
Funktion mit der Gleichung 
\[
T( x ) = f(x_0) + f'(x_0) \cdot (x-x_0 ).
\]
Es gilt $T(x_0)=f(x_0)$ und $T$ approximiert $f$ in $x_0$ bis auf einen 
kleinen Fehler (vgl. auch Frage \ref{06_linapprox}).
\AntEnd
\end{antwort}


%% --- 31 --- %%
\begin{frage}\index{Stetigkeit!einer differenzierbaren Funktion}
Warum ist eine in $x_0$ differenzierbare Funktion dort auch stetig?
\end{frage}

\begin{antwort}
Aus der Existenz des Grenzwerts in Frage \ref{06_diffdef} 
folgt insbesondere für jede gegen $x_0$ konvergente Folge $(x_n)$: 
$\lim\limits_{n\to\infty} f(x_n)-f(x_0)=0$, also 
$\lim\limits_{n\to\infty } f(x_n) = f(x_0)$. Damit ist $f$ gemäß 
des Folgenkriteriums stetig.
\AntEnd
\end{antwort} 


%% --- 32 --- %%
\begin{frage}\label{06_linapprox}
\index{linear approximierbar}
\index{Differenzial!einer Funktion einer reellen Veränderlichen}
\nomenclature{$\dd f$}{Differenzial von $f$}
Differenzial einer Funktion.  
Wann heißt eine Funktion $f\fd M\to\RR$ in $x_0\in M$ 
\bold{linear approximierbar}? Warum sind Differenzierbarkeit 
und lineare Approximierbarkeit von $f$ in $x_0$ äquivalente Aussagen. 
\end{frage}

\begin{antwort}
Eine Funktion $f$ heißt \index{linear approximierbar}, wenn eine 
lineare Abbildung $L\fd \RR \to \RR$ existiert, für die gilt
\begin{equation}
\lim_{h\to 0} \frac{f(x_0+h)-f(x_0)-L(h)}{h} = 0.
\tag{D2}
\end{equation}
Ist $f$ in $x_0$ differenzierbar, 
dann ist die Abbildung durch $L(h) = f'(x_0)\cdot h$ gegebene 
Abbildung eine lineare Approximation in diesem Sinne.

Gilt andersherum ($\ast$) für eine lineare Abbildung $L$, 
dann ist $L(h)=l\cdot h$ für ein $l\in\RR$,  und es folgt
$0 = \lim\limits_{h\to 0} \frac{1}{h}
\big(f(x_0+h)-f(x_0)-lh\big)$, also 
$\lim\limits_{h\to 0} \frac{1}{h} \big(f(x_0+h)-f(x_0)\big)=l.$
Das heißt, $f$ in $x_0$ differenzierbar mit $f'(x_0)=l$.

Die lineare Abbildung $L\fd \RR\to\RR$ heißt, die $(\ast)$
erfüllt, heißt \slanted{Differenzial von $f$ in $x_0$} und wird mit 
$\diff (x_0)$ bezeichnet, \sieheAbbildung\ref{fig:06_differential} Es gilt 
\[
\diff (x_0) h = f'(x_0)\cdot h.
\]
\begin{center}
  \includegraphics{mp/06_differential}
  \captionof{figure}
  {%
    Das Differenzial $\diff (x_0)$ ist eine lineare Funktion, die $f$ 
    im Punkt $x_0$ approximiert.
  }
  \label{fig:06_differential}
\end{center}

Die Charakterisierung der Differenzierbarkeit durch lineare 
Approximierbarkeit wird erst im Höherdimensionalen wirklich fruchtbar. 
In $\RR$ bleibt die Unterscheidung zwischen Differenzial und Ableitung 
ohne größere praktische Konsequenzen. Man sollte sich aber trotzdem 
jetzt schon einprägen, dass für eine differenzierbare Abbildung 
$f\fd U \to V$ zwischen beliebigen normierten Räumen $\diff (x_0)$ 
stets eine \slanted{lineare Abbildung} $U\to V$ bezeichnet. \AntEnd
\end{antwort}


%% --- 33 --- %%
\begin{frage}\label{06_tang}\index{Tangentenfunktion}
Warum ist die Tangentenfunktion
\[
T\fd M\to\RR, \qquad x\mapsto f(x_0)+f'(x_0)(x-x_0)
\]
unter allen linear-affinen Funktionen 
$x\mapsto f(x_0)+ m (x-x_0)$ die \bold{beste lineare Approximation} 
von $f$ im Punkt $(x_0,f(x_0))$?
\end{frage}

\begin{antwort}
Der Existenz des Grenzwerts (D$_2$) in Frage \ref{06_linapprox} 
impliziert, dass der Unterschied $f(x_0+h)-f(x_0)-L(h)$ 
für $h\to 0$ schneller gegen $0$ geht als $h$ selbst. 
Für jede andere lineare Funktion $L^*$ mit $L^*(h)=bh$ und 
$b\not= f'(x_0)$ gilt dies wegen 
\[
\lim_{h\to 0} \frac{f(x_0+h)-f(x_0)-L^*(h)}{h} = f'(x_0)-b \not=0
\]
nicht. In genau diesem Sinne ist die Tangentenfunktion die 
\slanted{beste} lineare Approximation. 
\AntEnd
\end{antwort}

%% --- 34 --- %%
\begin{frage}\index{stetig differenzierbar}
Können Sie Beispiele angeben für Funktionen $f\fd M\to \RR$ mit 
$0\in M$, die 
\begin{itemize}[4mm]
\item[\desc{a}] stetig, im Nullpunkt aber nicht differenzierbar sind?\\[-3.5mm]
\item[\desc{b}] differenzierbar sind, aber deren Ableitung im Nullpunkt 
nicht stetig ist? \\[-3.5mm]
\item[\desc{c}] stetig differenzierbar, aber nicht 2-mal differenzierbar 
sind?.
\end{itemize}
\end{frage}

\begin{antwort}
\desc{a} Die Betragsfunktion $x\mapsto |x|$ ist stetig, aber im 
Nullpunkt nicht differenzierbar.

\medskip

\noindent
\desc{b} Die Funktion $x\mapsto x^2 \cdot \sin\frac{1}{x}$ 
besitzt die gesuchte Eigenschaft.

\medskip
\noindent
\desc{c} Die durch 
\[
f(x) := \left\{ \begin{array}{rl} 
\frac{x^2}{2} & \text{für $x\ge 0$}, \\ 
-\frac{x^2}{2} & \text{für $x < 0$} 
\end{array} \right.
\]
definierte Funktion $\RR\to\RR$ ist auf $\RR$ differenzierbar und 
hat die Ableitung $f'(x)=|x|$. Wegen \desc{a} 
ist $f$ nicht 2-mal differenzierbar.
\AntEnd
\end{antwort}

%% --- 35 --- %%
\begin{frage}\label{06_algd}\index{Differenziationsregeln}
Was besagen die \bold{algebraischen Differenziationsregeln}? 
Können Sie diese beweisen?
\end{frage}

\begin{antwort}
\satz{Seien $f$ und $g$ in $x$ differenzierbar. Dann sind auch 
$f+g$, $f\cdot g$ und für $g(x)\not=0$ auch $\frac{1}{g(x)}$ in 
$x$ differenzierbar, und es gilt
\[\boxed{
\begin{array}{lp{3mm}lp{3mm}l}
\text{\desc{a}} & & \dis (f+g)'(x)=f'(x)+g'(x) &  &  
\text{(Summenregel)}, \\[2mm]
\text{\desc{b}} & &  \dis (fg)'(x)=f'(x)g(x)+f(x)g'(x) &  &  
\text{(Produktregel)}
\\[2mm]
\text{\desc{c}} &  & \dis \left(\frac{f}{g}\right)' (x) = 
\frac{f'(x)g(x)-f(x) g(x) }{g(x)^2}  & & \text{(Quotientenregel)}
\end{array}}
\]
Speziell folgt aus \desc{b} für eine konstante Funktion $f(x)=c$ 
($c\in \RR$) für alle $x$ wegen $f'(x)=0$ auch die Regel 
\[
(c\cdot g)'(x)
= c \cdot g'(x). \]
}

\medskip\noindent
Die Regeln zeigt man durch die folgenden Umformungen des 
Differenzenquotienten für $f+g$, $fg$ und $f/g$
\[
\begin{array}{lp{3mm}l}
\text{\desc{a}} & & 
\dis \frac{f(x+h)-f(x)}{h}+\frac{g(x+h)-g(x)}{h}, \\[2mm]
\text{\desc{b}} & &
\dis \frac{f(x+h)-f(x)}{h}g(x+h)+
\frac{g(x+h)-g(x)}{h}f(x), \\[2mm]
\text{\desc{c}} & &
\dis \frac{1}{g(x+h)g(x)} 
\left( \frac{f(x+h)-f(x)}{h} g(x) -  \frac{g(x+h)-g(x)}{h} f(x) \right),
\end{array}
\]
Aus diesen Darstellungen folgen die Regeln dann für $h\to 0$.
Bei \desc{c} muss man nur beachten, dass es eine Umgebung $U_h(x)$ 
gibt mit $g(y) \not= 0$ für alle 
$y\in U_h(x)$.  
\AntEnd
\end{antwort}



%% --- 36 --- %%
\begin{frage}\index{Differenzierbarkeit!Kriterium}\label{06_dif3}
Wie lässt sich die Differenzierbarkeit durch die  
Existenz einer stetigen Funktion mit bestimmten Eigenschaften 
charakterisieren?
\end{frage}

\begin{antwort}
\satz{Eine Funktion $f\fd M\to\RR$ 
ist genau dann differenzierbar in $x_0$, wenn es eine 
in $x_0$ stetige Funktion $\varphi\fd M\to\RR$ gibt, 
für die gilt:
\[
f(x)-f(x_0)= \varphi(x)\cdot (x-x_0).
\tag{D3}
\]}
Ist nämlich $f$ in $x_0$ differenzierbar, dann besitzt die Funktion 
\[
\varphi^*( x ) := \frac{f(x)-f(x_0)}{x-x_0}\qquad 
\text{für $x\in M \mengeminus\{ x_0 \}$}
\]
eine in $x_0$ stetige Fortsetzung $\varphi$. 
In diesem Fall gilt $\varphi(x_0)=f'(x_0)$. 

Diese Formulierung liefert neben \desc{D1} und \desc{D2} eine weitere 
Charakterisierung der Differenzierbarkeit, die in Beweisen häufig 
leichter anzuwenden ist. In Frage \ref{06_kettenregelbeweis} und bzw. 
\ref{06_diffumkehr} wird sie zum Beweis der Kettenregel 
bzw. des Satzes von der Differenziation der Umkehrfunktion
herangezogen.
\AntEnd 
\end{antwort}


%% --- 37 --- %%
\begin{frage}\index{Kettenregel!für Funktionen einer Veränderlichen}
Was besagt die Kettenregel? 
\end{frage}

\begin{antwort}
Die Kettenregel besagt: 

\medskip
\noindent\satz{Sind $f\fd  M\to I \subset \RR$ und 
$g\fd I\to \RR$ Funktionen so, dass $f$ in $x_0$ und 
$g$ in $y_0=f(x_0)$ differenzierbar sind. 
Dann ist auch $g\circ f$ in $x_0$ differenzierbar, und es gilt
\[
\boxed{ ( g\circ f )' (x_0) = g'\big(f(x_0)\big) \cdot f'(x_0). }
\EndTag\]}
\end{antwort}

%% --- 38 --- %%
\begin{frage}\label{06_kettenregelbeweis}
Können Sie die Kettenregel beweisen?
\end{frage}

\begin{antwort}
 Für den Beweis benutzt man am besten die Formulierung 
der Differenzierbarkeit aus Frage \ref{06_dif3}. 
Demnach gibt es in $x_0\in M$ bzw. $y_0\in I$ stetige Funktionen 
$\varphi$ und $\gamma$ mit 
\[
f(x)-f(x_0) = \varphi(x)\cdot (x-x_0), \qquad
g(y)-g(y_0) = \gamma(y) \cdot (y-y_0).
\]
Einsetzen der ersten Gleichung in die zweite liefert
\[
g\big( f(x) \big)-g \big(f(x_0)\big) = 
\gamma\big( f(x) \big) \cdot \big[ f(x)-f(x_0) \big] 
=\gamma \big( f(x) \big) \cdot \varphi(x) \cdot (x-x_0),
\]
Die Funktion $\gamma\left( f(x) \right)\cdot \varphi(x)$ 
ist stetig in $x_0$, nach Frage \ref{06_dif3}  
ist $f$ dort also differenzierbar. 
Wegen $\varphi(x_0)=f'(x_0)$ und $\gamma\big( f(x_0) \big)= 
g'\big( f(x_0) \big)$ folgt die Kettenregel. \AntEnd
\end{antwort} 

%% --- 39 --- %%
\begin{frage}
Wie lautet die Ableitung der Funktion $f(x)=(x^2+1)^{2014}$?
\end{frage}

\begin{antwort}
Mit der Kettenregel folgt $f'(x)=2014\cdot (x^2+1)^{2013} \cdot 2x$.
\AntEnd 
\end{antwort}

%% --- 40 --- %%
\begin{frage}\index{Differenzierbarkeit!der Umkehrfunktion}
Was besagt der \bold{Satz über die Differenzierbarkeit der Umkehrfunktion}? 
\end{frage}

\begin{antwort}
Der Satz besagt: 

\medskip
\noindent\slanted{Ist $M$ ein Intervall und 
$f\fd M \to \RR$ eine streng monotone, 
in $y_0 \in M$ differenzierbare Funktion mit $f'(y_0) \not=0$,   
dann ist die Umkehrfunktion $f^{-1}$ in 
$x_0 = f(y_0)$ differenzierbar und es gilt
\[
\boxed{(f^{-1})'(x_0) = \frac{1}{f'(y_0)} = \frac{1}{f'( f^{-1}( x_0 ) ) }.}
\EndTag
\]}
\end{antwort} 

%% --- 41 --- %%
\begin{frage}\label{06_diffumkehr}
\index{Differenzierbarkeit!der Umkehrfunktion}
Wie lässt sich der Satz über die Differenzierbarkeit der 
Umkehrfunktion beweisen?
\end{frage}

\begin{antwort}
 Zum Beweis benutzt man wieder die Charakterisierung 
der Differenzierbarkeit 
aus Frage \ref{06_dif3}. Es gibt eine in $y_0\in M$ stetige Funktion 
$\varphi \fd M \to \RR$ mit 
\[
f(y)-f(y_0)=\varphi( y ) \cdot( y-y_0 ). 
\]
Wegen der strengen Monotonie von $f$ und 
wegen $\varphi(y_0)=f'(y_0) \not=0$ folgt aus dieser Gleichung  
$\varphi( y ) \not=0$ für alle $y\in M$. 

Die Substitution $y = f(x)$, $y = f^{-1}(x)$ liefert 
\[
x - x_0 = \varphi \big( f^{-1} ( x ) \big) \cdot 
\big[ f^{-1}(x)-f^{-1}(x_0) \big]. 
\]
Die Funktion $\frac{1}{\varphi \circ f^{-1}}$ 
ist stetig  in $x_0$ und hat 
dort den Wert $\frac{1}{f'(y_0)}$. 
Nach Frage \ref{06_dif3} folgt also 
$(f^{-1})'(x_0) = \frac{1}{f'(y_0)}$.

(Man beachte: Man benötigt die Stetigkeit der Umkehrfunktion 
an der Stelle $x_0$! Das folgt aber aus den Voraussetzungen, 
da die Umkehrfunktion einer streng monotonen Funktion auf einem Intervall 
als Definitionsbereich automatisch stetig ist.)
\AntEnd
\end{antwort}

%% --- 42 --- %%
\begin{frage}\index{Logarithmus!Ableitung}\index{Arcus-Tangens!Ableitung}
Können Sie den Satz aus der vorigen Frage 
anwenden, um die Ableitung des natürlichen 
Logarithmus und des Arcustangens zu berechnen.
\end{frage}

\begin{antwort}
Wegen $\exp'(x)=\exp(x)$ und $\exp(x) \not =0$ für alle 
$x\in\RR$ folgt mit dem Satz 
\[
\log'(x) = (\exp^{-1})'(x) = \frac{1}{\exp'( \log(x) )} = \frac{1}{x}, 
\quad\text{für alle $x\in\RR_+$}.
\]
Für die Ableitung des Arcus-Tangens berechnet man zunächst mit 
der Regel \ref{06_algd}\,\desc{c}
\[
\tan'(x)= \frac{\sin'(x)\cos(x)-\sin(x)\cos'(x)}{\cos^2 (x)} = 
\frac{\cos^2(x)+\sin^2(x)}{\cos^2(x)} = 
1+\tan^2 (x). 
\] 
Damit erhält man $
\arctan'(x)= \frac{1}{1+ \tan^2 ( \arctan x ) } = \frac{1}{1+x^2}.$\AntEnd
\end{antwort}


%% --- 43 --- %%
\begin{frage}\label{06_ferm}
\index{Fermatsches Kriterium@Fermat\sch es Kriterium}
\index{Extremum!Existenzkriterium}
Was besagt das \bold{Fermat\sch e Kriterium} bezüglich der  
\bold{Existenz eines Extremums} (Maximum oder Minimum) für eine 
differenzierbare Funktion $f\fd M\to \RR$ in einem 
inneren Punkt $x_0 \in M$? 
\end{frage}

\begin{antwort}
Das Kriterium besagt: \satz{Ist $f$ in $x_0$ differenzierbar und 
besitzt dort ein Extremum, so gilt $f'(x_0)=0$.} 

Der Beweis der Aussage ist sehr einfach. Liegt in $x_0$ ein lokales 
Extremum, etwa ein Maximum vor, dann gilt $f(x)-f(x_0)\le 0$ 
innerhalb einer Umgebung von $x_0$. Es folgt
\[
\lim_{x\uparrow x_0}\frac{f(x)-f(x_0)}{x-x_0} \le 0  \quad\text{und}\quad
\lim_{x\downarrow x_0}\frac{f(x)-f(x_0)}{x-x_0} \ge 0, 
\]
also $f'(x_0)=0$. 
Mit dem gleichen Argument beweist man den Zusammenhang für den 
Fall, dass bei $x_0$ ein Minimum vorliegt. \AntEnd
\end{antwort} 

\smallskip
%% --- 44 --- %%
\begin{frage}\index{Fermatsches Kriterium@Fermat'sches Kriterium}
Warum gilt die Aussage des Fermat\sch en Kriteriums nicht 
in den Randpunkten?
\end{frage}

\begin{antwort}
Das in der letzten Antwort gegebene Argument lässt sich nicht 
auf die Randpunkte von $M$ übertragen, 
da hier entweder nur ein links- \slanted{oder} rechtsseitiger Grenzwert 
existiert. Dass der Satz in den Randpunkten so nicht gelten kann, wird 
durch das einfache Beispiel der Funktion $x\mapsto x$ auf dem 
kompakten Intervall $[0,1]$ deutlich. 
\AntEnd
\end{antwort}

%% --- 45 --- %%
\begin{frage}
Ist das Fermat\sch e Kriterium \slanted{hinreichend} 
für die Existenz eines lokalen Extremums?
\end{frage}

\begin{antwort}
 
Das Kriterium formuliert eine \slanted{notwendige} Bedingung 
für das Vorliegen eines (lokalen) Extremums. 
Dass es nicht hinreichend ist, zeigt das Beispiel  
$f(x)=x^3$ und $x_0=0$.
\AntEnd
\end{antwort}

%% --- 46 --- %%
\begin{frage}\index{Extremum!Existenzkriterium}
Kennen Sie  \bold{hinreichende Kriterien} für das Vorliegen eines 
lokalen Extremums?
\end{frage}

\begin{antwort}
Eine differenzierbare Funktion $f \fd M\to \RR$ besitzt in einem  
inneren Punkt $x_0 \in M$ ein \slanted{Maximum}, wenn ein 
$\delta > 0$ existiert mit 
\[
\text{
$f'(x) \ge 0$ für $x\in \open{x_0-\delta,x_0}$\qquad und\qquad
$f'(x) \le 0$ für $x\in \open{x_0,x_0+\delta}.$ }
\]
Beim Vorliegen eines \slanted{Minimums} existiert analog ein 
$\delta$ mit 
\[
\text{
$f'(x) \le 0$ für $x\in \open{x_0-\delta,x_0}$\qquad und\qquad
$f'(x) \ge 0$ für $x\in \open{x_0,x_0+\delta}.$ }
\]
Ein hinreichendes Kriterium für das Vorliegen eines lokalen 
Extremums ist also, dass die 
Ableitung von $f$ beim Durchgang durch $x_0$ das Vorzeichen wechselt. 

\begin{center}
  \includegraphics{mp/06_maximum}
  \captionof{figure}{Jeder linksseitige 
    Differenzenquotient ist größer oder gleich null, 
    jeder rechtsseitige kleiner oder gleich null.}
  \label{fig:06_maximum}
\end{center}

Der Beweis ist hier wiederum nicht schwierig. 
Zum Beispiel bedeutet das Vorliegen eines lokalen Maximums 
von $f$ in $x_0$, dass in einer Umgebung $U$ von $x_0$ 
stets $f(x_0)-f(x) \ge 0$ gilt. Damit ist der 
linksseitige Grenzwert des Differenzenquotienten in $x_0$ 
größer oder gleich $0$, 
der rechtsseitige kleiner 
oder gleich $0$, \sieheAbbildung\ref{fig:06_maximum}. 
Zusammen ergibt dies $f'(x_0)=0$.   
\AntEnd
\end{antwort}


%% --- 47 --- %%
\begin{frage}\label{06_rolle}\index{Satz!von Rolle}
\index{Rolle@\textsc{Rolle}, Michel (1652-1719)} 
\index{Mittelwertsatz!der Differenzialrechnung}
Was besagt der \bold{Satz von Rolle}, was der 
\bold{Mittelwertsatz der Differenzialrechnung}? Warum sind 
beide Aussagen äquivalent?
\end{frage} 

\begin{antwort}
Sei $f\fd [a,b] \to \RR$ stetig und auf $(a,b)$ differenzierbar.  
Dann gilt

\begin{itemize}
\item \slanted{Satz von Rolle}: 
Ist $f(a)=f(b)$, so gibt es ein $\xi\in [a,b]$ mit $f'(\xi)=0$.

\medskip
\item \slanted{Mittelwertsatz}: Es gibt ein $\xi \in (a,b)$ mit 
$\dis \frac{ f(b)-f(a) }{b-a} = f' (\xi ).$
\end{itemize}

Zunächst zum Satz von Rolle. 
Die stetige Funktion $f$ besitzt auf dem kompakten Intervall 
$[a,b]$ ein 
Minimum und ein Maximum. Ist $f$ konstant, dann ist $f'(x)=0$
für alle $x\in [a,b]$, im anderen Fall ist mindestens 
einer der beiden Extremwerte von $a$ und von $b$ verschieden, 
und nach Frage \ref{06_ferm} verschwindet die Ableitung an dieser 
Stelle, \sieheAbbildung\ref{fig:06_rolle}.

\begin{center}
  \begin{minipage}{60mm}
  \includegraphics{mp/06_rolle}
  \captionof{figure}{Satz von Rolle}
  \label{fig:06_rolle}
\end{minipage}
\qquad
\begin{minipage}{60mm}
  \includegraphics{mp/06_mws}
  \captionof{figure}{Mittelwertsatz}
  \label{fig:06_mws}
\end{minipage}
\end{center}

Zum Beweis der Mittelwertsatzes wende man den Satz von Rolle 
auf die Funktion 
\[
\varphi(x) := f(x)-\frac{f(b)-f(a)}{b-a} (x-a).  
\]
an. Wegen $\varphi(a)=\varphi(b)$ gibt es ein $\xi \in (a,b)$ mit 
$\varphi'(\xi)=0$, also  
\[
f'(\xi)=\frac{f(b)-f(a)}{b-a},
\]
\sieheAbbildung\ref{fig:06_mws}. 

Der Mittelwertsatz wurde hier ohne weitere Voraussetzungen aus dem 
Satz von Rolle abgeleitet. Da umgekehrt der Satz von Rolle 
offensichtlich ein Spezialfalls des Mittelwertsatzes ist, 
sind beide Aussagen äquivalent.  
\AntEnd
\end{antwort}



%% --- 48 --- %%
\begin{frage}\label{q:mittelwertsatz-der-differenzialrechnung}\index{Mittelwertsatz!der Differenzialrechnung}
Wie lautet der \bold{verallgemeinerte Mittelwertsatz der 
Differenzialrechnung}?
\end{frage}

\begin{antwort}
Dieser Satz besagt: 

\medskip
\noindent\satz{Sind die Funktionen 
$f$ und $g$ auf dem kompakten Intervall $[a,b]$ stetig und auf 
$\open{a,b}$ differenzierbar, so gibt es mindestens eine Stelle 
$\xi \in \open{a,b}$ mit 
\[
\big( f(b)-f(a) \big) \cdot g'(\xi) = 
\big( g(b)-g(a) \big) \cdot f'(\xi). 
\]}
Zum Beweis wende man den Satz von Rolle auf die 
Funktion
\[
\varphi( x ) := \big( f(b)-f(a) \big)\cdot g(x) - 
\big( g(b)-g(a) \big)\cdot f(x), \quad  x\in [a,b] 
\]
an. Wegen $\varphi(a)=\varphi(b)$ gibt es ein $\xi\in (a,b)$ 
mit $\varphi'(\xi)=0$. Damit ist der Satz bewiesen. 

Gilt zudem $g'(x) \not=0$ für alle $x\in [a,b]$, so 
ist nach dem Mittelwertsatz 
$g(b)-g(a) \not=0$. 
In diesem Fall lässt sich die Gleichung 
des verallgemeinerten Mittelwertsatzes auch in folgender Form schreiben:
\begin{equation}
\frac{f(b)-f(a)}{g(b)-g(a)} = \frac{f'(\xi)}{g'(\xi)} . \EndTag  
\end{equation}
\end{antwort}


%% --- 49 --- %%
\begin{frage}\label{06_schr}\index{Schrankensatz}
Was besagt der \bold{Schrankensatz}?
\end{frage}

\begin{antwort}
Der Schrankensatz lautet: 

\medskip
\noindent\satz{Für eine differenzierbare Funktion 
$f\fd M\to \RR$ mit beschränkter Ableitung gilt für beliebige 
Punkte $x_1,x_2 \in M$:
\[
\boxed{ | f(x_1)-f(x_2) | \le \|f'\| \cdot |x_1-x_2|. }
\]
Mit anderen Worten, die Funktion ist unter den angegebenen
Bedingungen Lipschitz-stetig.}

\medskip\noindent
Ohne Beschränkung der Allgemeinheit können wir $x_2>x_1$ annehmen. 
Nach dem Mittelwertsatz gibt es ein $\xi \in (x_1,x_2)$ mit $
\left| \frac{ f(x_2)-f(x_1) }{ x_2-x_1 } \right| = |f'( \xi )| \le \|f'\|.$
Damit ist der Satz schon bewiesen. 
\AntEnd
\end{antwort}



%% --- 50 --- %%
\begin{frage}\index{Regel von de L'Hospital}
Wie lauten die \bold{Regeln von de L'Hospital}?
\end{frage}
\begin{antwort}
 \satz{Seien $f$ und $g$ auf $M$ definierte reelle Funktionen. 
Gilt dann für ein $a\in M$ 
\[
\text{\desc{a}} \quad \lim_{x\uparrow a} f(x) = \lim_{x\uparrow a} g(x)=0 \quad\text{oder}\quad
\text{\desc{b}}\quad \lim_{x\uparrow a} f(x) = \lim_{x\uparrow a} g(x)=\infty, 
\]
so gilt unter der Voraussetzung, dass $g'(x) \not=0$ in einer 
Umgebung von $a$ nicht verschwindet
\[
\lim_{x\uparrow a} \frac{f(x)}{g(x)} = 
\lim_{x\uparrow a} \frac{f'(x)}{g'(x)}.
\]
Analoge Zusammenhänge gelten für $x\downarrow a$ sowie für $x\to\pm \infty$.} 
\AntEnd
\end{antwort}

%% --- 51 --- %%
\begin{frage}\index{Regel von de L'Hospital}
Wie kann man die Regeln von de L'Hospital beweisen?
\end{frage}

\begin{antwort}
 Der Beweis muss für jeden der Fälle \desc{a} und 
\desc{b} extra geführt werden. Beide Male beruht er wesentlich 
auf einer Anwendung des verallgemeinerten Mittelwertsatzes. 

\medskip
\noindent
\desc{a} Nach dem verallgemeinerten Mittelwertsatz gibt es ein 
$\xi \in (a-\delta,a)$ mit
\[
\frac{f(a)-f(a-\delta)}{g(a)-g(a-\delta)} = \frac{f'(\xi)}{g'(\xi)}.
\]
Für $\delta \to 0$ folgt hieraus die Aussage. 

\medskip
\noindent
\desc{b} Zunächst schreibe man den Quotienten $f(x)/g(x)$ in der Form  
\begin{equation}
\frac{f(x)}{g(x)} = \frac{f(x)-f(y)}{g(x)-g(y)} \cdot 
\frac{1-g(y)/g(x)}{1-f(y)/f(x)}. 
\asttag
\end{equation}
Zu $A:= \lim\limits_{x\uparrow a}f'(x)/g'(x)$ wähle man $\delta>0$ so, 
dass $\left|  \frac{f'(x)}{g'(x)} - A \right| <\eps$ 
für alle $x\in (a-\delta, a )$ gilt. Mit 
dem verallgemeinerten Mittelwertsatz folgt dann
\[
A-\eps <   
\frac{f(x)-f(y)}{g(x)-g(y)} < A+\eps  
\quad\text{für alle $x,y \in (a-\delta, a)$.}
\]
Einsetzen dieses Ergebnisses in ($\ast$) liefert
\[
(A-\eps) \left| \frac{1-g(y)/g(x)}{1-f(y)/f(x)} \right|  < 
\frac{f(x)}{g(x)} < 
(A+\eps) \left| \frac{1-g(y)/g(x)}{1-f(y)/f(x)} \right|
\]
Der Faktor in den Betragsstrichen konvergiert bei 
festgehaltenem $y$ für $x\!\uparrow \!a$ gegen $1$. 
Da $\eps$ beliebig klein gewählt werden kann, folgt die Behauptung.
\AntEnd
\end{antwort}

%% --- 52 --- %%
\begin{frage}
Können Sie $\lim\limits_{x\downarrow 0} x\log x$ bestimmen?
\end{frage}

\begin{antwort}
Die Regel von de L'Hospital liefert hierfür 
\begin{equation}
\dis \lim_{x\downarrow 0} x \log x = 
\lim_{x\downarrow 0} \frac{\log x}{1/x} = \lim_{x\downarrow 0}
\frac{1/x}{-1/x^2} = \lim_{x\downarrow 0} -x = 0.
\EndTag
\end{equation}
\end{antwort}

%% --- 53 --- %%
\begin{frage}\index{Monotoniekriterium}
Wie lautet das  \bold{Monotoniekriterium}?
\end{frage}

\begin{antwort}
Das Kriterium lautet: 

\medskip\noindent\satz{Ist $f\fd [a,b] \to \RR$ differenzierbar, 
dann ist $f$ auf $[a,b]$ 
\begin{itemize}[2mm]
\item[\desc{i}] monoton steigend genau dann, wenn 
$f'(x) \ge 0$ für alle $x\in[a,b]$ und \\[-3.5mm]
\item[\desc{ii}] monoton fallend genau dann, wenn 
$f'(x) \le 0$ für alle $x\in[a,b]$ gilt. 
\end{itemize}}
\noindent
Beide Aussagen folgen bei Betrachtung der 
Differenzenquotienten von $f$ leicht aus dem Mittelwertsatz. 
\AntEnd
\end{antwort}

%% --- 54 --- %%
\begin{frage}\label{06_konst}\index{Charakterisierung!konstanter Funktionen}
Wie lassen sich \bold{konstante Funktionen} über ihre Ableitung 
charakterisieren?
\end{frage}

\begin{antwort}
\satz{Eine Funktion $f\fd M \to \RR$ auf einem Intervall 
$M$ ist konstant genau dann, wenn ihre 
Ableitung verschwindet.}

\medskip\noindent
Die Richtung "`$\La$"' offensichtlich. Die andere Richtung 
ergibt sich wiederum aus einer Anwendung des Mittelwertsatzes. 
Für zwei Punkte $x_1,x_2 \in [a,b]$ existiert ein $\xi$ 
zwischen $x_1$ und $x_2$ mit 
$f(x_1)-f(x_2)=(x_1-x_2)f'(\xi)$. Aus $f'(\xi)=0$ folgt dann 
$f(x_1)=f(x_2)$.
\AntEnd
\end{antwort}

%% --- 55 --- %%
\begin{frage}\index{konvex}\label{06_konvex}
Wann heißt eine Funktion $f\fd [a,b] \to \RR$ \bold{konvex}?
\end{frage}

\begin{antwort}
Eine Funktion $f$ ist konvex, wenn die 
Sekante durch je zwei Punkte 
$P_1:=\big(x_1,f(x_1)\big)$ und $P_2:=\big(x_2,f(x_2)\big)$ 
des Graphen stets oberhalb des Graphen verläuft 
(\sieheAbbildung\ref{fig:06_konvex}), wenn also für alle 
$x\in (x_1,x_2)$ gilt
\[
f(x) \le f(x_1)+\frac{f(x_2)-f(x_1)}{x_2-x_1}  x. 
\]

\begin{center}
  \includegraphics{mp/06_konvex}
  \captionof{figure}{Bei einer konvexen Funktion verläuft die 
    Sekante durch zwei Punkte des Graphen stets oberhalb des 
    Graphen.}
  \label{fig:06_konvex}
\end{center}

Parametrisiert man die Punkte des Intervalls $(x_1,x_2)$ durch  
$x=\lambda x_1 + (1-\lambda) x_2$ für 
$\lambda \in \open{0,1}$, so erhält man 
daraus die äquivalente Bedingung 
\[
f(  \lambda x_1 + (1-\lambda) x_2 ) \le 
\lambda f(x_1) + (1-\lambda) f(x_2), 
\quad\text{für $\lambda \in\open{0,1}$}. 
\]
\end{antwort}

%% --- 56 --- %%
\begin{frage}\index{Charakterisierung!konvexer Funktionen}
Wie lässt sich die Konvexität für eine 
differenzierbare Funktion 
$f\fd [a,b]\to \RR$ mittels der zweiten Ableitung charakterisieren?
\end{frage}

\begin{antwort}
 
Ist $f$ differenzierbar, so ist die Konvexität gleichbedeutend 
damit, dass die Ableitung $f'$ auf $[a,b]$ monoton wächst. 
Ist $f$ sogar zweimal differenzierbar, so ist das nach dem 
Monotoniekriterium äquivalent zu $f''(x)\ge 0$ für alle 
$x\in [a,b]$. 
\AntEnd
\end{antwort}   

%% --- 57 --- %%
\begin{frage}\index{Stammfunktion}
Ist $M\subset \RR$ ein echtes Intervall und $f\fd M \to \RR$ eine 
Funktion, die eine Stammfunktion $F_0$ besitzt. Warum ist  
$F\fd M\to \RR$ genau dann eine Stammfunktion von $f$, wenn 
$F=F_0+C$ mit einer Konstanten $C$ gilt?
\end{frage}

\begin{antwort}
Sind $F$ und $F_0$ Stammfunktionen von $f$, so gilt
\[
(F-F_0)'= f-f=0,\]
also ist nach Frage \ref{06_konst} $F-F_0$ konstant. Die Umkehrung ist 
trivial. 
\AntEnd  
 
\end{antwort}

%% --- 58 --- %%
\begin{frage}\index{Differenzialgleichung!der Exponentialfunktion}
\index{Exponentialfunktion!Charakterisierung über ihre Ableitung}
Können Sie begründen, warum die Exponentialfunktion die 
einzige differenzierbare Funktion $f\fd \RR \to \RR$ mit 
$f'=f$ und $f(0)=1$ ist?
\end{frage}

\begin{antwort}
Man betrachte die Funktion $g(x) :=f(x)\cdot \exp(-x)$. Für diese gilt 
\[
g'(x)= 
\big(f'(x)-f(x)\big)\cdot \exp(-x)= 0. 
\]
Es folgt $f(x)=C\cdot\exp(x)$, und wegen $f(0)=\exp(0)$ ist $C=1$.
\AntEnd
\end{antwort}

%% --- 59 --- %%
\begin{frage}\index{Differenzialgleichung}
\index{Schwingungsdifferenzialgleichung}
Können Sie begründen, warum \slanted{jede} Lösung der Differenzialgleichung 
$f''+f=0$ die Form $f=a\cos +b\sin $ hat?
\end{frage}

\begin{antwort}
Ist $f$ konstant null, dann gilt die Behauptung trivialerweise. 
Wir wollen also $f\not\equiv 0$ annehmen. Dann muss $f$ mindestens eine 
Nullstelle besitzen. Denn wäre $f$ zum Beispiel überall negativ, dann 
wäre $f''$ auf ganz $\RR$ positiv, also konkav, und daraus 
würde mit einem einfachen geometrischen Argument folgen, dass der 
Graph von $f$ die $x$-Achse an einem Punkt schneiden muss (vgl. dazu 
die Abbildung in Frage \ref{06_konvex}). $f$ besitzt also eine 
Nullstelle $x_0$.       

Weiter folgt aus $f+f''=0$  
\[
( f'^2 + f^2 )' = 2f''\cdot f' + 2f\cdot f'= 2f'\cdot (f''+f )= 0
\]
und damit $f^2+f'^2 = C$ mit einer Konstanten $C\in \RR$. 
Die Funktion $f$ besitzt zusammenfassend also die drei Eigenschaften
\[
f( x_0 ) = 0, \qquad f'(x_0) = c, \qquad  
f'' + f=0,
\]  
wobei $c=\sqrt{C}$ oder $c=-\sqrt{C}$ ist. Sei $g$ eine weitere Funktion 
mit diesen Eigenschaften. Für die Funktion $ h := f-g$ gilt dann
\[
h(x_0)=h'(x_0)=0, \qquad h+h''=0.
\]
Daraus folgt wie oben $h'^2 + h^2 = K$ und wegen der ersten Eigenschaft 
dann $K=0$, also $h=0$. Damit sind $f$ und $g$ identisch. Es kann also nur 
eine Funktion geben, die die drei Eigenschaften in ($\ast$) besitzt. 

Da die Funktion $a \cos( x+ t)+b \sin(x+t)$ für geeignete 
Parameter $a,b,t$ die Eigenschaften besitzt, ist sie 
die einzige. 
\AntEnd  
\end{antwort}  

%% --- 60 --- %%
\begin{frage}\index{Differenzierbarkeit!komplexwertiger Funktionen}
\index{Ableitung!einer komplexwertigen Funktion}
Wie kann man den Begriff der Differenzierbarkeit von reellwertigen 
Funktionen auf komplexwertige Funktionen verallgemeinern?
\end{frage}

\begin{antwort}
 Zunächst einmal lässt sich der Differenzierbarkeitsbegriff 
für komplexwertige Funktionen auf reellen 
Intervallen natürlich genauso \slanted{definieren} 
wie für reellwertige. Eine Funktion $f\fd D\to \CC$ auf einer 
Teilmenge $D\subset \CC$ heißt demnach \slanted{differenzierbar} in 
$x_0\in D$, wenn der Grenzwert
\[
\lim_{x\to x_0} \frac{f(x)-f(x_0)}{x-x_0}
\]
existiert. Gegebenenfalls heißt dann der Grenzwert $f'(x_0)$ die 
\slanted{Ableitung von $f$ in $x_0$}. 

Da jede komplexwertige Funktion 
$f\fd D\to \CC$ eine Zerlegung 
\[
f(x) := u(x) + \i  v(x),\qquad u,v \fd D\to \RR 
\] 
in Real- und Imaginärteil besitzt und weil der 
Konvergenzbegriff in $\CC$ komponentenweise erklärt ist, folgt daraus 
unmittelbar:

\medskip
\noindent\satz{Eine komplexwertige Funktion 
$f=u+\i v$ auf einem 
reellen Intervall $D$ ist genau dann differenzierbar in $x_0\in D$, 
wenn die reellen Funktionen $u$ und $v$ in $x_0$ differenzierbar sind. 
Gebegenenfalls ist die Ableitung von $f$ in $x_0$ gegeben durch
\[
\boxed{f'(x_0) = u'(x_0) + \i v'(x_0).} \asttag
\]}
Aufgrund der Übereinstimmung der Definition von 
"`Differenzierbarkeit"'  und wegen {\astref} gelten 
sämtliche Permanenzeigenschaften für die Ableitung reellwertiger 
Funktionen auch für komplexwertige.   
Sind die Funktionen $f,g \fd D\to \CC$ in $x_0\in D$ 
also differenzierbar, dann auch die Funktionen 
\[
\lambda f\quad \text{(mit $\lambda\in \CC$)} 
\qquad f+g, \qquad f\cdot g,\qquad 
\frac{f}{g}\quad \text{(falls $g(x_0) \not=0$)},
\]   
und es gelten dieselben algebraischen Ableitungsregeln wie in Frage 
\ref{06_algd}. 
Ferner lassen sich die Charakterisierung \desc{D1} und \desc{D2} und 
\desc{D3} mittels linearer bzw. stetiger Funktionen unmittelbar ins  
Komplexe übertragen. 

In Abschnitt \ref{total} werden wir untersuchen, welche Konsequenzen
sich ergeben, wenn 
man den Differenzierbarkeitsbegriff auf Funktionen 
$f\fd D\to\CC$ anwendet, die auf einer Menge $D\subset \CC$ 
definiert sind. Dort werden sich weiter reichende Besonderheiten zeigen.  
\AntEnd
\end{antwort} 

%% --- 61 --- %%
\begin{frage}
Was ist die Ableitung der Funktion 
$\cis \fd \RR\to\CC$ mit 
$\cis(x):= \cos x +\i \sin x $?
\end{frage}

\begin{antwort}
 Es ist $\cis'(x) = \cos'x+\i \sin' x=-\sin x+\i\cos x = 
\i \cdot \cis(x)$. 
\end{antwort}
  
\section{Der Hauptsatz der Differenzial- und Integralrechnung}\label{hauptsatz}

Der Hauptsatz der Differenzial- und Integralrechnung stellt den fundamentalen 
Zusammenhang zwischen den beiden Säulen des \slanted{Calculus} \index{Calculus}
 her. 
Er besagt, dass Differenziation und Integration zueinander inverse Operationen 
sind. Er garantiert die 
\slanted{Existenz von Stammfunktionen} für alle stetigen Funktionen auf 
kompakten Intervallen (bei Zugrundelegung eines allgemeineren 
Stammfunktionsbegriffs sogar für alle Regelfunktionen). Mithilfe von 
Stammfunktionen lassen sich Integrale bequem und elegant berechnen. 


%% --- 62 --- %%
\begin{frage}\label{06_hpts}\index{Hauptsatz der Differenzial- und 
Integralrechnung}
Was besagt der Hauptsatz der Differenzial- und Integralrechnung?
\end{frage}

\begin{antwort}
Die Aussage des Hauptsatzes gliedert sich in drei Teile. Er besagt 
\satz{\setlength{\labelsep}{5mm}
\begin{enumerate}
\item[\desc{i}] Jede stetige Funktion $f\fd [a,b]\to\RR$ besitzt eine 
Stammfunktion, {\dasheisst} eine Funktion $F\fd [a,b]\to\RR$ mit 
$F'(x)=f(x)$.\\[-3.5mm] 
\item[\desc{ii}] Je zwei Stammfunktionen von $f$ unterscheiden 
sich nur durch eine additive Konstante.\\[-3.5mm]
\item[\desc{iii}] Für eine \slanted{beliebige} Stammfunktion $\Phi$ von $f$ 
gilt 
\[
\boxed{\int_a^b f(t) \dift = \Phi(b)-\Phi(a).}
\]
\end{enumerate}}
\noindent
Der Beweis von Teil \desc{i} wurde schon in Frage \ref{06_fa} erbracht. 
Wie dort gezeigt wurde, ist die "`Integralfunktion"' 
\[
F_a (x) = \int_a^x f(t) \dift 
\]
eine Stammfunktion zu $f$. 

Sind $F$ und $G$ Stammfunktionen von $f$, so gilt 
$F'-G'=(G-F)'=0$ und damit $F=G+C$ mit einer Konstanten $C\in \RR$. 
Daraus folgt \desc{ii} 

Behauptung \desc{iii} ist für die Stammfunktion $F_a$ offensichtlich. 
Die allgemeine Aussage ergibt sich daraus 
zusammen mit \desc{ii}. 
\AntEnd
\end{antwort}

%% --- 63 --- %%
\begin{frage}\index{Hauptsatz der Differenzial- und 
Integralrechnung}
Wie kann man den Hauptsatz in der Sprache der linearen Algebra formulieren?
\end{frage}

\begin{antwort}
Ist $V:= \{ f\fd [a,b]\to\RR;\,\text{$f$ stetig} \}$ der Vektorraum der 
stetigen Funktionen auf $[a,b]$ und 
$W_0:= \{ g\in {\cal C}^1([a,b]);\,  g(a)=0 \}$, dann ist die Abbildung 
\[
I \fd V\to W_0; \qquad 
f \mapsto I(f) = :g  \quad\text{mit } g(x):= \int_a^x f(t)\dift 
\]
ein Isomorphismus mit der Umkehrung 
\begin{equation}
D\fd W_0 \to V; \qquad g \mapsto Dg=g'.
\end{equation}
Somit gelten die Beziehungen 
\begin{equation}
f = D(I(f))\quad\text{ und }\quad g = I(Dg).\EndTag
\end{equation} 
\end{antwort}

%% --- 64 --- %%
\begin{frage}
Wie kann man mithilfe des Hauptsatzes für eine 
differenzierbare Funktion 
$f\fd M\to \RR$ die Funktion selbst und ihre erste Ableitung in Beziehung 
setzen.
\end{frage}

\begin{antwort}
Aus dem Hauptsatz folgt die Beziehung 
\[
f(x)=\int_{x_0}^x f'(t)\dift + f(x_0).
\]
Der Hauptsatz ermöglicht also, bei gegebener Ableitung $f'$ die 
Ausgangsfunktion $f$ bis auf eine Konstante zu rekonstruieren. 
Speziell gilt für $[a,b]\subset M$
\begin{equation}
f(b)-f(a)=\int_a^b f'(t) \dift.
\EndTag
\end{equation}
\end{antwort}

%% --- 65 --- %%
\begin{frage}\index{Integration!glattet@glättet}
Was besagt die Redeweise "`Integration glättet"'?
\end{frage}

\begin{antwort}
Die durch Integration erhaltene Stammfunktion ist genau einmal öfter 
differenzierbar als der Integrand 
(sofern dieser nicht ohnehin schon aus $\calli{C}^\infty$ ist) 
und in diesem Sinne "`glatter"'. Insbesondere ist 
die Stammfunktion jeder \slanted{stetigen}, also nicht 
notwendig differenzierbaren, Funktion differenzierbar.


Abbildung~\ref{fig:06_stammfunktion} veranschaulicht den Zusammenhang für eine 
Treppenfunktion. Es gilt $F(x_0) = \int_0^{x_0} f(t) \dift$, und die 
Funktion $F(x)$ ist im Gegensatz zu $f(x)$ stetig. Sie macht auch 
deutlich, wie die \slanted{Steigung} des Graphen von $F$ an der Stelle, 
mit dem Integral von $f$ zusammenhängt. 
Insofern illustriert sie in gewisser Weise die Aussage des 
Hauptsatzes der Differenzial- und Integralrechnung. 
Streng genommen ist $F$ allerdings keine Stammfunktion von $f$ 
im Sinne der Definition, da $F$ an einigen Stellen nicht 
differenzierbar ist.    

\begin{center}
  \includegraphics{mp/06_stammfunktion}
  \captionof{figure}{Die Stammfunktion $F$ stellt den 
    (grauen) orientierten Flächeninhalt in Abhängigkeit von $x_0$ dar.} 
  \label{fig:06_stammfunktion}
\end{center}

Legt man jedoch einen etwas allgemeineren Stammfunktionsbegriff 
zugrunde (etwa wie in \citep{Koenig}, demzufolge eine 
Stammfunktion nur bis auf eine abzählbare 
Ausnahmemenge differenzierbar sein muss), so ist $F$ in diesem allgemeineren 
Sinn eine Stammfunktion von $f$. 
\AntEnd
\end{antwort} 

%% --- 66 --- %%
\begin{frage}\label{06_reig}\index{Regelfunktion!Charakterisierung}
Durch welche inneren Eigenschaften lassen sich Regelfunktionen 
$f\fd [a,b]\to \RR$ charakterisieren?
\end{frage}

\begin{antwort}
\satz{Eine Funktion $f\fd [a,b]\to\RR$ ist eine Regelfunktion genau dann, wenn 
sie an jeder Stelle $ \xi \in ]a,b[$ sowohl einen rechts- als auch einen 
linksseitigen Grenzwert hat und in den Randpunkten jeweils einseitige 
Grenzwerte.}

\medskip\noindent
Die Richtung "`$\Ra$"' gilt, da Treppenfunktionen an jeder Stelle einen 
links- und rechtsseitigen Grenzwert besitzen, und weil sich diese 
Eigenschaft bei gleichmäßiger Konvergenz auf die Grenzfunktion überträgt.

Die andere Richtung zeigt man mit einem Intervallschachtelungsargumemt. 
Angenommen, $f$ besitzt in jedem Punkt $x_0\in [a,b]$ einen links- und 
rechtsseitigen Grenzwert, ist aber keine Regelfunktion. Dann gibt es 
ein $\eps>0$, sodass $\| f-\varphi \|>\eps$ für alle Treppenfunktionen 
$\varphi$ gilt. Besitzt $f$ nun auf einem Intervall $I\subset[a,b]$ 
keine approximierende Folge von Treppenfunktionen, so gilt dies auch 
für mindestens eine der Hälften von $I$. 
Ausgehend vom Intervall $[a,b]$ lässt sich durch sukzessive Halbierung 
also eine Intervallschachtelung $(I_n)$ mit der Eigenschaft 
\begin{equation}
\| f-\varphi \|_{I_n}  > \eps \qquad\text{für alle $n\in \NN$ und alle 
$\varphi\in \calli{T}$} \notag
\end{equation}
konstruieren. An dieser Stelle 
mus man mit einem $\eps\delta$-Standardargument nur noch zeigen, 
dass das im Widerspruch zur Voraussetzung steht. \AntEnd   
\end{antwort}

%% --- 67 --- %%
\begin{frage}\index{Stammfunktion!einer Regelfunktion}
Warum besitzt eine Regelfunktion $f\fd [a,b]\to \RR$ genau dann eine 
Stammfunktion (im klassischen Sinne), wenn $f$ stetig ist?
\end{frage}

\begin{antwort}
Sei zunächst $x_0$ ein beliebiger innerer Punkt aus $\lopen{a,b}$. 
Eine Regelfunktion $f$ besitzt nach Frage \ref{06_reig} dann 
in $x_0$ einen linksseitigen Grenzwert $f({x_0}_-)$. 
Die Funktion $f_{{x_0}_-}$, die an der Stelle $x_0$ den Wert 
$f({x_0}_-)$ hat und ansonsten mit $f$ übereinstimmt, ist damit 
für ein $\delta>0$ stetig auf dem Intervall $[x_0-\delta, x_0]$,  
hat dort also eine Stammfunktion, und es gilt
\[
\lim_{x\uparrow x_0} \frac{1}{x_0-x} \int_{x}^{x_0} f_{{x_0}_-} (t) \dift 
= f( {x_0}_- )
\] 
Da sich die Werte von $f$ und $f_{{x_0}_-}$ nur in höchstens einem 
Punkt unterscheiden, sind deren Integrale gleich, es folgt also 
\begin{equation}
\lim_{x\uparrow x_0} \frac{1}{x_0-x} \int_{x}^{x_0} f (t) \dift 
= f( {x_0}_- ).
\asttag
\end{equation}
Analog erhält man für $x_0 \in \ropen{a,b}$
\begin{equation}
\lim_{x\downarrow x_0} \frac{1}{x_0-x} \int_{x}^{x_0} f (t) \dift 
= f( {x_0}_+ )
\tag{$\ast\ast$}. 
\end{equation} 
Besitzt $f$ nun eine Stammfunktion, so sind die beiden Grenzwerte 
($\ast$) und ($\ast\ast$) gleich, weil die Approximationen 
$x\uparrow x_0$ und $x\downarrow x_0$ dann durch beliebige 
gegen $x_0$ konvergente Folgen ersetzt werden können. 
Es folgt  
$f({x_0}_-)=f({x_0}_+)$, also die Stetigkeit von $f$ in $x_0$. 
\AntEnd    
\end{antwort}  

%% --- 68 --- %%
\begin{frage}\index{Stammfunktion!unstetiger Funktionen}
Gibt es auch unstetige Funktionen, die eine Stammfunktion besitzen?
\end{frage}  
 
\begin{antwort}
Ja. Als Beispiel betrachte man die Funktion 
$F\fd \RR \to \RR$ mit $F(0)=0$ 
und $F(x)=x^2\sin \frac{1}{x}$. Deren Ableitung 
$f(x):=F'(x)=2x\sin \frac{1}{x}+\cos(\frac{1}{x})$ 
besitzt im Nullpunkt weder einen links- noch einen rechtsseitigen Grenzwert.
\AntEnd 
\end{antwort}


%% --- 69 --- %%
\begin{frage}\index{Riemann'sche Summe}
\index{Riemann@\textsc{Riemann}, Bernhard (1826-1866)}
Was versteht man unter einer \bold{Riemann'schen Summe}?
\end{frage}

\begin{antwort}
Sei $f\fd [a,b]\to\RR$ eine beliebige Funktion und 
sei $Z$ eine Zerlegung von $[a,b]$ mit den Teilungspunkten 
$x_0, \ldots , x_n$. 
Weiter sei $\xi_k$ für $1\le k \le n$ ein beliebiger Punkt aus dem 
Intervall $[x_{k-1},x_k]$. 
Dann ist ist die \slanted{Riemann'sche Summe} zur Funktion 
$f$ bezüglich der Zerlegung $Z$ und dem Zwischenvektor 
$\xi:= (\xi_1,\ldots,\xi_n)$ definiert als die Summe 
\begin{equation}
S(Z,\xi,f):=\sum_{k=1}^n f(\xi_k) \Delta x_k, \qquad 
\Delta x_k := x_k-x_{k-1}.
\notag
\end{equation}

\begin{center}
  \includegraphics[width=40mm]{mp/06_riemann}
  \captionof{figure}{Riemann'sche Summe mit Zwischenvektor 
    $(\xi_1,\ldots,\xi_n)$.}
  \label{fig:06_riemann}
\end{center}

Der Wert der Riemann'schen Summe entspricht somit dem Flächeninhalt, den 
der Graph der durch $\varphi(x)=f(\xi_k)$ für $x\in[x_{k-1},x_k]$ gegebenen  
Treppenfunktion mit der $x$-Achse einschließt, \sieheAbbildung\ref{fig:06_riemann}. 
\AntEnd
\end{antwort}

%% --- 70 --- %%
\begin{frage}\label{riemannsumme}\index{Riemann'sche Summe}
Können Sie folgende Aussage begründen: Ist $f\fd [a,b]\to \RR$ eine 
Regelfunktion und $(Z_j)$ mit 
$Z_j := \{ x_0^{(j)}, \ldots, x_{n_j}^{(j)} \}$ eine Folge 
von Zerlegungen mit 
\[
\lim_{j\to \infty} |Z_j| = 0, \qquad 
|Z_j| := \max\{ x_k^{(j)} - x_{k-1}^{(j)};\, 1\le k \le n_j \},
\]
dann gilt für beliebige Zwischenvektoren $\xi^{(j)}$ von $Z_j$ 
\[
\lim_{j\to\infty} S( Z_j, \xi_j, f ) = \int_a^b f(x) \difx.
\]
In Worten: Das Integral einer Regelfunktion auf einer 
kompakten Menge $[a,b]$ wird durch eine Riemann'sche Summe 
beliebig genau approximiert, solange die Zerlegung nur 
fein genug ist.
\end{frage}

\begin{antwort}
Wir zeigen, dass zu jedem $\eps>0$ 
ein $\delta>0$ existiert, sodass für jede Zerlegung 
$Z$ von $[a,b]$ mit $|Z| \le \delta$ die Abschätzung 
\[
\left| S( Z, \xi, f ) - \int_a^b f \difx \right| < \eps
\]
gilt, wobei $\xi$ ein beliebiger Zwischenvektor zur 
Zerlegung $Z$ ist. Daraus folgt die Behauptung.   

Diese Aussage lässt sich zunächst für \slanted{Treppenfunktionen} 
$\varphi$ unschwer mittels vollständiger Induktion zeigen. 

Sei $f$ daher eine Regelfunktion und $\varphi$ eine Treppenfunktion mit 
$\| f-\varphi \| < \frac{\eps}{3(b-a)}$. Ferner sei $\delta>0$ so gewählt, 
dass für jede Zerlegung $Z$ mit $|Z| < \delta$ die Ungleichung  
$\left| S( Z, \xi, \varphi ) - \int_a^b \varphi \difx\right| < \eps/3$ gilt. 
Für jede Zerlegung $Z=\{ x_0, \ldots, x_n \}$ mit $|Z| < \delta$ und 
beliebigem Zwischenvektor $\xi$ folgt hieraus

\begin{eqnarray*}  
\left| S(Z,\xi,f)  - \int_a^b f \difx \right| 
&\le& 
\bigg| S(Z,\xi,f) - S( Z,\xi,\varphi)  \bigg| 
 + \left| S(Z,\xi,\varphi) - \int_a^b \varphi \difx\right| \\
& & + \left| \int_a^b \varphi \difx - \int_a^b f\difx \right| \\
&\le& 
\sum_{k=1}^n \| f-\varphi_n \| \Delta x_k + \frac{\eps}{3} + 
\int_a^b \| f-\varphi_n \| \difx \le \eps.
\end{eqnarray*}
Damit ist die Behauptung bewiesen. \AntEnd
\end{antwort}

%% --- 71 --- %%
\begin{frage}\index{Obersumme}\index{Untersumme}
Sei $f\fd [a,b]\to\RR$ eine \bold{beschränkte} Funktion. 
Für jedes 
$\underline{\varphi} \in \underline{\calli{T}}(f) := 
\{ \varphi \in \calli{T}; \, \varphi \le f \}$ 
und jedes 
$\overline{\varphi} \in \overline{\calli{T}}(f) := 
\{ \varphi \in \calli{T}; \, \varphi \ge f \}$ 
heißt 
\[
\begin{array}{lp{3mm}l}
I( \underline{\varphi} ):=\int_a^b \underline{\varphi} \difx 
& & \text{\bold{Untersumme} von $f$ und} \\[2mm]
I( \overline{\varphi} ): =\int_a^b \overline{\varphi} \difx
& & \text{\bold{Obersumme} von $f$},
\end{array}
\]
\sieheAbbildung\ref{fig:06_riemann3}. Bezeichnet 
man mit $U_f$ die Menge aller Untersummen und 
mit $O_f$ die Menge aller Obersummen von $f$, 
warum existiert dann stets 
\[
I_* (f) := \sup U_f \quad\text{und}\quad I^*(f) := \inf O_f,
\]
und warum gilt immer $I_* (f) \le I^* (f) \text{?}$
\end{frage} 


\begin{antwort}
Die Mengen $O_f$ und $U_f$ sind nicht leer, 
da sie die Inhalte der konstanten 
Funktionen $x\mapsto \sup f$ bzw. $x\mapsto \inf f$ enthalten. Ferner 
gilt für alle $\underline{\varphi}\in \underline{\calli{T}}(f)$ 
und alle $\overline{\varphi}\in \overline{\calli{T}}(f)$
\[
I(\underline{\varphi}) \le (b-a)\cdot \sup f 
\qquad\text{und}\qquad 
I(\overline{\varphi}) \ge (b-a)\cdot \inf f. 
\]
Die nicht leeren Mengen $U_f$ und $O_f$ sind also nach 
oben bzw. unten beschränkt, und damit existieren $\sup U_f$ und $\inf O_f$. 

\begin{center}
  \includegraphics{mp/06_riemann3}
  \captionof{figure}{Unter- und Obersumme einer Funktion}
  \label{fig:06_riemann3}
\end{center}

Angenommen, es würde $I_* (f) > I^* (f)$ gelten. Dann gäbe es eine 
Treppenfunktion $\underline{\psi} \in \underline{\calli{T}}(f)$ mit 
\begin{equation}
I(\underline{\psi}) > I(\overline{\varphi}) \qquad\text {für alle 
$\overline{\varphi}\in \overline{\calli{T}}(f)$.} \tag{$\ast$}
\end{equation} 
Die Funktion 
$\overline{\psi} := \underline{\psi} + \| f-\underline{\psi} \| +1$ 
liegt dann in $\overline{\calli{T}}(f)$ und es gilt 
$I ( \overline{\psi} ) > I ( \underline{\psi} ),$
im Widerspruch zu ($\ast$).\AntEnd
\end{antwort}

%% --- 72 --- %%
\begin{frage}\label{regelrie}
\nomenclature{$I^*(f)$}{Oberintegral von $f$}
\nomenclature{$I_*(f)$}{Unterintegral von $f$}
Warum gilt für Regelfunktionen $f\fd[a,b]\to\RR$ 
stets $I_*(f) = I^*(f)$?
\end{frage}

\begin{antwort}
Sei $\varphi$ eine Treppenfunktion mit $\| f-\varphi \| < \eps$. 
Dann ist $\varphi-\eps \in \underline{\calli{T}}(f)$ und 
$\varphi+\eps \in \overline{\calli{T}}(f)$. Hieraus folgt
\[
0 \le I^* (f) - I_* (f) < I(\varphi+\eps) - I(\varphi-\eps) = 2\eps\cdot (b-a).
\]
Da $\eps$ beliebig klein gehalten werden kann, ergibt sich 
hieraus die Behauptung.
\AntEnd
\end{antwort}

%% --- 73 --- %%
\begin{frage}\index{Riemann-Integral}\index{Integral!Riemann-Integral}
\index{Riemann@\textsc{Riemann}, Bernhard (1826-1866)}
Wann heißt eine Funktion $f\fd [a,b] \to \RR$ \bold{Riemann-integrierbar}?
\end{frage}

\begin{antwort}
\satz{Eine beschränkte 
Funktion $f\fd [a,b]\to\RR$ 
heißt Riemann-integrierbar, wenn $I_*(f)=I^*(f)$ gilt.} 

\medskip\noindent
Der Unterschied zum Regelintegral besteht also darin, dass beim 
Riemann-Integral \slanted{direkt} versucht wird, den Flächeninhalt unter 
dem Graphen von $f$ durch Einschließung von Ober- und Untersummen zu 
bestimmen, ohne sich weiter darum zu kümmern, ob $f$ selbst durch Funktionen 
einer bestimmten Klasse approximiert werden kann. 
\AntEnd
\end{antwort}

%% --- 74 --- %%
\begin{frage}\label{06_riem}\index{Riemann-Integral}
\nomenclature{$\Rie(M)$}{Raum der Riemann-integrierbaren Funktionen 
auf $M=[a,b]$}
Begründen Sie, warum 
\[
\Rie(M) := \{ f \fd M\to\RR, \quad\text{$f$ Riemann-integrierbar} \} 
\]
mit $M=[a,b]$ ein $\RR$-Vektorrraum ist und warum die Abbildung 
\[
\Rie(M) \to \RR; \qquad f \mapsto \int_a^b f(x) \difx 
\]
ein lineares Funktional ist, dessen Einschränkung auf die Menge der 
Regelfunktionen mit dem Integral für Regelfunktionen übereinstimmt.
\end{frage}

\begin{antwort}
Seien $f$ und $g$ Riemann-integrierbar. 
Für $\alpha,\beta\in\RR$ sowie Treppenfunktionen 
$\underline{\varphi} \in \underline{\calli{T}}(f)$ und  
$\underline{\psi} \in \underline{\calli{T}}(g)$  
gilt 
$I( \alpha \underline{\varphi} + \beta \underline{\psi} ) = 
\alpha \cdot I(  \underline{\varphi} ) + 
\beta \cdot  I ( \underline{ \psi } )$. 
Daraus folgt 
\begin{equation}
I^* (\alpha f + \beta g ) = 
\alpha\cdot I^*( f ) + 
\beta\cdot I^*( g ).
\tag{$\ast$}
\end{equation}
Analog erhält man   
\begin{equation}
I_*  (\alpha f + \beta g ) = 
\alpha\cdot I_*(  f ) + 
\beta\cdot I_*( g ).
\tag{$\ast\ast$}
\end{equation}
Wegen $I^*( f )=I_*(f)$ und $I^*( g )=I_*(g)$ folgt daraus 
$I^*( \alpha f+ \beta g )=I_*(\alpha f+\beta g)$. Die Funktion 
$\alpha f+\beta g$ ist also Riemann-integrierbar. 
Damit ist gezeigt, dass $\Rie(M)$ ein $\RR$-Vektorraum ist.  

Die Gleichungen ($\ast$) und ($\ast\ast$) zeigen auch, 
dass es sich bei der Abbildung 
$\Rie(M) \to \RR$ um ein lineares Funktional handelt. 

Der letzte Teil der Behauptung ergibt sich im Wesentlichen als 
Spezialfall aus Frage \ref{riemannsumme}. Zu einer Zerlegung 
$Z=\{x_0,\ldots,x_n\}$ betrachte man die beiden in 
\Abb~\ref{fig:06_riemann1} skizzierten speziellen Riemann'schen Summen 
\[
\begin{array}{lp{3mm}l}
\overline{S}(Z)  := \sum_{k=1}^n f(\overline{\xi}_k) \Delta x_k,  & &
\overline{\xi}_k := \max\{ f(x); \; x\in [x_{k-1},x_k] \}, \\
\underline{S}(Z) := \sum_{k=1}^n f(\underline{\xi}_k) \Delta x_k,  
& &
\underline{\xi}_k := \min\{ f(x); \; x\in [x_{k-1},x_k] \}.
\end{array}
\]

\begin{center}
  \includegraphics{mp/06_riemann1}
  \captionof{figure}{Spezielle Riemann'sche Summen. Die 
    Komponenten des Zwischenvektors liegen am Maximum 
    bzw. Minimum des entsprechenden Teilintervalls.}
  \label{fig:06_riemann1}
\end{center}

Für alle Zerlegungen $Z$ ist  
$\overline{S}(Z)$ eine Ober- und $\underline{S}(Z)$ eine 
Untersumme von $f$. Nach Frage \ref{riemannsumme} gilt 
$\lim_{|Z|\to 0} \overline{S}(Z)=\lim_{|Z|\to 0 }\underline{S}(Z) 
= \int_a^b f \difx$, und hieraus folgt 
$\lim_{|Z|\to 0} \overline{S}(Z)=I^*(f)=I_*(f)=\int_a^b f \difx$. 
\AntEnd
\end{antwort}

%% --- 75 --- %%
\begin{frage}\label{06_rire}
\index{Riemann-Integral!Vergleich mit Regelintegral}
Können Sie eine Riemann-integrierbare Funktion 
$f\fd [a,b]\to\RR$ angeben, die keine Regelfunktion ist?
\end{frage}

\begin{antwort}
Ein Standardbeispiel liefert die
Funktion $f\fd x\mapsto \sin \frac{1}{x}$ auf $[0,1]$. 
Diese besitzt in $0$ keinen rechtsseitigen Grenzwert, kann also 
keine Regelfunktion sein.

\begin{center}
  \includegraphics{mp/06_sineinsdurchx}
  \captionof{figure}{Die Funktion $x\mapsto \sin\frac1x$ ist auf $[0,1]$ 
    riemann-, aber nicht regelintegrierbar.}
  \label{fig:06_sineinsdurchx}
\end{center}

Ihre Riemann-Integrierbarkeit folgt daraus, dass sie auf jedem 
kompakten Intervall $[\eps,1]$ stetig und damit nach Frage \ref{regelrie} 
erst recht Riemann-integrierbar ist. 
Das Infimum ${I_\eps}_*$ der Obersummen über das 
restliche Intervall ist kleiner als $\eps\cdot 1=\eps$, 
das Supremum $I^*_\eps$ 
der Untersummen größer als $\eps\cdot (-1)=-\eps$, \sieheAbbildung 
\ref{fig:06_sineinsdurchx}. Also gilt auch hier 
${I_\eps}_* = I_\eps^*$. \AntEnd
\end{antwort}


\section{Integrationstechniken}

Weit verbreitet ist die Meinung \slanted{"`Differenziation ist Technik, Integration ist Kunst."'}.

Auch im Zeitalter von Computeralgebrasystemen sollte jede(r) 
Studierende einige elementare Integrationstechniken beherrschen. 

%% --- 76 --- %%
\begin{frage}\index{Substitutionsregel}
Wie lautet die \bold{Substitutionsregel}?
\end{frage}

\begin{antwort}
Die Substitutionsregel erhält man 
über den Hauptsatz der Differenzial- und Integralrechnung 
als Pendant zur Kettenregel. Sie lautet: 

\medskip
\noindent\satz{Sei $f\fd M\to \RR$ eine stetige Funktion und 
$F$ eine Stammfunktion zu $f$. 
Sei weiter $\varphi \fd [a,b] \to M$ 
stetig differenzierbar. Dann ist $F\circ \varphi$ eine 
Stammfunktion zu $(f\circ \varphi)\cdot \varphi'$,  
und es gilt
\[
\boxed{\int_a^b f \big( \varphi(x) \big) \cdot \varphi'(x)  \difx =
\int_{\varphi(a)}^{\varphi(b)} f(t)    \dift.}
\]}
\noindent
Die erste Behauptung folgt unmittelbar aus der Kettenregel, 
die zweite durch Auswerten der Integrale auf beiden Seiten der 
Gleichung. Diese haben nämlich beide 
den Wert $F\big(\varphi(b)\big)-F\big( \varphi(a) \big)$. 
\AntEnd
\end{antwort}

%% --- 77 --- %%
\begin{frage}\label{06_sbsp}\index{Substitutionsregel}
Können Sie mithilfe der Substitutionsregel das Integral 
$\int_{-1}^1 \sqrt{1-x^2} \difx$ berechnen?
\end{frage}

\begin{antwort}
Die Substitution $x = \sin t$, $\difx =\cos t \dift$ liefert  
\[
\int_{-1}^1 \sqrt{1-x^2}\difx = 
\Int_{\arcsin(-1)}^{\arcsin (1)} 
\sqrt{1-\sin^2}\cos t \dift 
= \int_{-\pi/2}^{\pi/2} \cos^2 t \dift 
\]
Nun ist
\[
\int_{-\pi/2}^{\pi/2}\cos^2 t  \dift = 
\int_{-\pi/2}^{\pi/2} 1 \dift - 
\int_{-\pi/2}^{\pi/2} \sin^2 t \dift = 
\pi - \int_{-\pi/2}^{\pi/2} \cos^2 t \dift.
\]
Einsetzen in die oberen Gleichung führt zu dem Ergebnis 
$\int_{-1}^1 \sqrt{1-x^2}\difx = \pi/2$. \AntEnd
\end{antwort}



%% --- 78 --- %%
\begin{frage}\index{partielle Integration}
Wie lautet die \bold{Regel der partiellen Integration}?
\end{frage}   



\begin{antwort}
Die Regel lautet: \satz{
Sind $f,g \fd M \to \RR$ stetig differenzierbar, so ist 
\[
\boxed{\int fg' \difx = fg - \int f'g \difx \qquad
\int_a^b fg' \difx = fg \Big|_a^b - \int_a^b f'g \difx.}
\]}

\medskip\noindent
Der Zusammenhang folgt sofort aus der Produktregel 
für die Differenziation. 
Demnach ist nämlich $fg$ eine Stammfunktion zu 
$fg'+f'g$. 
\AntEnd
\end{antwort}

%% --- 79 --- %%
\begin{frage}\label{06_inberechnung}
Können Sie mit dieser Regel begründen, warum 
\[
x\mapsto \frac{1}{2} ( x\sqrt{1-x^2} + \arcsin x )
\]
eine Stammfunktion von $\sqrt{1-x^2}$ auf dem Intervall 
$[-1,1]$ ist? 
\end{frage}

\begin{antwort}
Bei der folgenden Rechnung beachte man vor allem den Trick 
einer Multiplikation mit $1$, der bei Problemlösungen 
mittels partieller Integration recht häufig zum Einsatz kommt. 
\begin{eqnarray*}
\int 1\cdot \sqrt{1-x^2}\difx &=& 
x \sqrt{1-x^2} - \int \frac{x(-2x)}{2\sqrt{1-x^2}} \difx \\
&=& x\sqrt{1-x^2} + \int \frac{\difx}{\sqrt{1-x^2}} - 
\int \frac{1-x^2}{\sqrt{1-x^2}}\difx \\
&=& 
x\sqrt{1-x^2} + \arcsin x - \int \sqrt{1-x^2}\difx.
\end{eqnarray*}
Addition von $\int \sqrt{1-x^2}\difx$ auf beiden Seiten 
der Gleichung und anschließende Division durch $2$ 
liefert das gesuchte Ergebnis. \AntEnd
\end{antwort}

%% --- 80 --- %%
\begin{frage}
Können Sie das Ergebnis aus Frage \ref{06_inberechnung} 
durch ein geometrisches Argument anschaulich verifizieren?
\end{frage}

\begin{antwort}
 Das Integral 
$\int_{-1}^x \sqrt{1-x^2} \difx$ gibt den in 
Abbildung~\ref{fig:06_kreis} dunkelgrau eingefärbten Flächeninhalt an.
Dieser ist gleich dem Flächeninhalt des 
Kreissektors mit Winkel $\varphi$ 
\slanted{minus} bzw. \slanted{plus} dem Inhalt des 
hellgrau eingefärbten Dreiecks, je nachdem, ob $x$ negativ 
oder positiv ist. In beiden Fällen führt das auf
\[
\int_{-1}^x \sqrt{1-x^2} \difx = \frac{1}{2} ( \varphi + x\sqrt{1-x^2} ).
\]

\begin{center}
  \includegraphics{mp/06_kreis}
  \captionof{figure}{Elementargeometrische "`Berechnung"' von 
    $\int_{-1}^x \sqrt{1-x^2} \difx$.}
  \label{fig:06_kreis}
\end{center}

Nun ist aber $\varphi=\psi+\frac{\pi}{2}=\arcsin(x)+\frac{\pi}{2}$. 
Setzt man das in die Gleichung ein, erhält man eine 
Stammfunktion von $\sqrt{1-x^2}$, die sich von der formal 
hergeleiteten nur um die Konstante $\pi/4$ unterscheidet.
\AntEnd
\end{antwort}

%% --- 81 --- %%
\begin{frage}\label{06_wallis1}\index{Wallis'sche Produktformel}
Können Sie für $
I_n := \int\limits_0^{\pi/2} \sin^n x \difx $
eine Rekursionsformel angeben?
\end{frage}

\begin{antwort}
Mit partieller Integration erhält man zunächst
\begin{eqnarray*}
I_n &=& \int_0^{\pi/2} \sin x \sin^{n-1} x \difx = 
-\cos x \sin^{n-1} x \Bigg{|}_0^{\pi/2} + 
(n-1) \int_0^{\pi/2} \cos^2 x \sin^{n-2} x \difx \\
&=& (n-1) \int_0^{\pi/2} (1-\sin^2 x) \sin^{n-2} x \difx 
= (n-1)\cdot I_{n-2} - (n-1)  \cdot I_n.
\end{eqnarray*}
Damit lautet die gesuchte Rekursionsformel 
$\dis I_n = \frac{n-1}{n} \cdot I_{n-2}$.  
Wegen $I_0 =\pi/2$ und $I_1=1$ folgt aus dieser 
\[
I_{2n} = \frac{2n-1}{2n}\cdots\frac{3}{4}\cdot\frac{1}{2}\cdot\frac{\pi}{2}, 
\qquad 
I_{2n+1} = \frac{2n}{2n+1}\cdots\frac{4}{5}\cdot\frac{2}{3}.\EndTag
\]
\end{antwort}


%% --- 82 --- %%
\begin{frage}\label{06_wallis2}\index{Wallis'sche Produktformel}
\index{Wallis@\textsc{Wallis}, John (1616-1703)}
Können Sie aus dem Ergebnis von Frage \ref{06_wallis1} 
die \bold{Wallis'sche Produktformel}
\[
\dis \frac{\pi}{2} = \lim_{n\to\infty} 
\frac{2\cdot2}{1\cdot 3}\cdot
\frac{4\cdot4}{3\cdot 5}\cdots \frac{2n\cdot2n}{(2n-1)\cdot (2n+1)}
\] 
ableiten?
\end{frage}

\begin{antwort}
Die Wallis\sch e Produktformel entspricht gerade der Aussage 
\begin{equation}
\lim_{n\to\infty} \frac{I_{2n+1}}{I_{2n}}=1,\asttag 
\end{equation}
die es jetzt noch zu zeigen gilt. Der Sinus ist im Intervall $[0,\pi/2]$ 
positiv und $\le 1$. Daher ist $\sin^{2n} x \ge \sin^{2n+1}x \ge \sin^{2n+2}x$. 
Es folgt also $I_{2n} \ge I_{2n+1} \ge I_{2n+2}$ und damit
\[
1 \ge \frac{I_{2n+1}}{I_{2n}} \ge \frac{I_{2n+2}}{I_{2n}} = 
\frac{2n+1}{2n+2}.
\]
Wegen $\lim_{n\to \infty} (2n+1)/(2n+2)=1$ 
folgt daraus der Grenzwert ($\ast$) und somit 
die Wallis'sche Produktformel.
\AntEnd
\end{antwort}

\smallskip
%% --- 83 --- %%
\begin{frage}\label{06_pbz}
\index{Partialbruchzerlegung}
Was besagt der Satz über die \bold{Partialbruchzerlegung} einer 
rationalen Funktion?
\end{frage}

\begin{antwort}
 Der Satz besagt: 

\medskip
\noindent\satz{Sei $R=\frac{P(x)}{Q(x)}$ 
eine rationale Funktion. $Q$ besitze die komplexen Nullstellen 
$\alpha_1,\ldots,\alpha_k$ mit den jeweiligen Vielfachheiten 
$n_1,\ldots,n_k$. Dann gibt es genau eine Darstellung 
\[
R = H_1 + \cdots H_k + q,
\]
wobei $q$ ein Polynom ist, 
und die $H_k$ rationale Funktionen der Gestalt 
\[
H_k (x) = \frac{c_{n_k}}{ (x-\alpha_k)^{n_k}} + 
\frac{c_{n_{k-1}}}{ (x-\alpha_{k-1})^{n_{k-1}}} + \cdots + 
\frac{c_1}{ (x-\alpha_k) }, \quad c_{n_j} \in \CC, \; c_{n_k} \not=0. 
\EndTag
\] }

\smallskip
\end{antwort}

\smallskip
%% --- 84 --- %%
\begin{frage}
Können Sie ein konkretes Beispiel einer Partialbruchzerlegung angeben?
\end{frage}

\begin{antwort}
 Ein Beispiel einer Partialbruchzerlegung wäre etwa
\[
\frac{3x^2-5x+4}{x^3-3x^2+6x-4} = 
\frac{3x^2-5x+4}{(x-2)^2(x-1)} = 
\frac{2}{(x-2)^2} + \frac{1}{x-2} + \frac{2}{x-1}.\EndTag
\]
\end{antwort}

\smallskip
%% --- 85 --- %%
\begin{frage}
Wie kann man den Satz über die Partialbruchzerlegung beweisen?
\end{frage}

\begin{antwort}
Für den Beweis des Satzes sei die rationale Funktion durch 
Polynomdivision mit Rest auf die Form 
$R(x)=f(x)/g(x)+q(x)$ mit $\deg f \le \deg q$ gebracht. 
Sei $g(x)=(x-\alpha)^n h(x)$ mit einem Polynom $h$, das 
in $\alpha$ keine Nullstelle besitzt. 
Es genügt dann zu zeigen, dass eine eindeutige Darstellung
\begin{equation}
R(x)= \frac{f(x)}{h(x)(x-\alpha)^n} + R_0(x) + q 
\asttag
\end{equation}
existiert, wobei $R_0(x)$ eine rationale Funktion ist, 
die in $\alpha$ einen Pol höchstens $(n-1)$-ter Ordnung hat. 

Da das Polynom 
$\frac{1}{h(\alpha)}\big( f(x)h(\alpha)-f(\alpha)h(x) \big)$ 
die Nullstelle $\alpha$ besitzt, gibt es ein Polynom $p$, 
sodass gilt: 
\[   
\frac{f(x)}{h(x)} - \frac{f(\alpha)}{h(\alpha)} 
= \frac{f(x)h(\alpha)-f(\alpha)h(x)}{h(x)h(\alpha)} 
= 
\frac{(x-\alpha)p(x)}{h(x)}. 
\]  
Einsetzen der aus dieser Gleichung 
ermittelten Darstellung für $f(x)/h(x)$ in ($\ast$) liefert 
\[
R(x)= \frac{a_n}{ (x-\alpha)^n }+
\underbrace{\frac{p(x)}{h(x)(x-\alpha)^{n-1}}}_{R_0(x)} + q(x)
\quad\text{mit $a_n := f(\alpha)/h(\alpha)$.} 
\]
Die rationale Funktion $R_0(x)$ 
hat einen höchstens $(n-1)$-fachen Pol an der Stelle $\alpha$. 
Ferner gilt $\deg p = \deg f -1$. Nach spätestens $\deg f$ 
Schritten führt das Verfahren also auf "`Restfunktionen"' der 
Form $\frac{c}{\tilde{g(x)}}$. Das schließt den Fall aus, dass 
am Ende der Konstruktion eine nicht weiter zerlegbare Funktion 
übrigbleibt. Insgesamt zeigt das die Existenz der 
Partialbruchzerlegung. Die Eindeutigkeit folgt daraus, 
dass in der Konstruktion die Polynome $h(x)$ und $p(x)$ 
eindeutig bestimmt sind. Damit ist aber 
auch $a_n$ eindeutig. 
\AntEnd
\end{antwort}

%% --- 86 --- %%
\begin{frage}\label{06_elfk}\index{Integration!der rationalen Funktionen}
Mithilfe welcher elementaren Funktionen kann man eine rationale 
Funktion mit reellen Koeffizienten integrieren?
\end{frage}

\begin{antwort}
Die Integration einer rationalen Funktion kann in jedem Fall mittels 
\slanted{rationaler Funktionen}, des 
\slanted{Logarithmus} sowie des \slanted{Arcus-Tangens}
geleistet werden. 

Nach dem Satz über die Partialbruchzerlegung genügt es, 
Stammfunktionen von Funktionen der Form
\[
\frac{A}{(x-a)^m}, \qquad\text{$a,A\in \CC,\, m \in \NN$}
\]
zu finden. Je nachdem, ob $a$ und $A$ reell oder komplex sind 
und $m>1$ oder $m=1$ gilt, führt das zu verschiedenen Ergebnissen.

\medskip
\noindent
\desc{a} $m>1:$ In diesem Fall ist eine Stammfunktion gegeben durch 
\[
\frac{-1}{(m-1)}\cdot \frac{1}{(x-a)^{m-1}},
\]
und zwar unabhängig davon, ob $a$ reell oder komplex ist. 
(Die Gleichung $(x^{n})'=nx^{n-1}$ für alle $n\not=1$) lässt sich 
unmittelbar aufs Komplexe übertragen).  

\medskip
\noindent
\desc{b} $m=1$ und $a\in \RR$: Eine Stammfunktion ist hier durch den 
Logarithmus gegeben
\[
 \int \frac{\difx}{(x-a)}= \log(x-a).
\]

\medskip
\noindent
\desc{c} $m=1$ und $a\in \RR\mengeminus\CC$. 
Das ist der mit Abstand problematischste Fall. Aufgrund der  
Eigenheiten des komplexen Logarithmus ist hier eine Stammfunktion 
nicht so leicht wie für $a\in \RR$ anzugeben. Eine Auswertung 
des Integrals $\int \frac{1}{x-z} \difx$ für nicht reelle $z$ 
ist eigentlich Sache der Funktionentheorie. Da die Terme 
$\frac{A}{x-z}$ und $\frac{\overline{A}}{x-\overline{z}}$ 
aber in der Partialbruchzerlegung einer rationalen Form immer 
gemeinsam vorkommen, versucht man in der reellen Analysis 
Stammfunktionen zu Funktionen des Typs  
\[
\frac{A}{x-z} + \frac{\overline{A}}{x-\overline{z}} = 
\frac{Bb-C}{x^2+2bx+c} 
\]
allgemein anzugeben. Das ist mit einigem Aufwand und trickreichen 
Substitutionen auf ausschließlich reellem Wege 
auch möglich (vgl. etwa \citep{Koenig}) und führt schließlich auf die 
Formel 
\begin{equation}
\int \frac{Bx+C}{x^2+2bx+c}\difx = 
\frac{B}{2}\cdot\log\big| x^2+2bx+c \big| + 
\frac{C-Bb}{\sqrt{c-b^2}}\cdot \arctan \frac{x+b}{\sqrt{c-b^2}}.
\asttag
\end{equation}
Diese komplizierte Formel wird im Komplexen wesentlich verständlicher. 
Letztendlich steht sie mit dem komplexen Logarithmus in Beziehung, und der 
setzt sich zusammen aus Arcustangens und reellem Logarithmus. 

Jedenfalls sind damit zu allen Typen der in einer 
Partialbruchzerlegung auftretenden 
rationalen Summanden Stammfunktionen angegeben. In diesen 
kommen außer 
rationalen Funktionen, Logarithmus und Arcus-Tangens keine weiteren 
Funktionen vor. 
\AntEnd
\end{antwort}

%% --- 87 --- %%
\begin{frage}
Können Sie begründen, warum die Funktion 
$F\fd \RR\to\RR$ mit 
\[
F(x) = \frac{1}{4\sqrt{2}} \log 
\frac{x^2-\sqrt{2} x +1}{ x^2+\sqrt{2} x +1 } + 
\frac{1}{2\sqrt{2}} \left\{ 
\arctan( \sqrt{2} x-1 )+\arctan( \sqrt{2} x+1 ) \right\}
\]
auf jedem Intervall $[a,b]$ eine Stammfunktion von 
$f(x)=\frac{x^2}{x^4+1}$ ist?
\end{frage}

\begin{antwort}
Diese Stammfunktion erhält man aus der reellen Partialbruchzerlegung 
\[
\frac{x^2}{x^4+1} = 
\frac{ \frac{1}{2\sqrt{2}}\cdot x }{ x^2-\sqrt{2}x+1} - 
\frac{ \frac{1}{2\sqrt{2}}\cdot x }{ x^2+\sqrt{2}x+1}
\]
unter Benutzung der Formel ($\ast$) aus Frage \ref{06_elfk}.
\AntEnd 
\end{antwort} 




 
                                                                                                                                                                                                                      






 





 



 



  
  
 
   
 

 



  
 
 

 








 




 
