\chapter{Stetigkeit, Grenzwerte von Funktionen}

Der \slanted{Funktions-} bzw. \slanted{Abbildungsbegriff} 
ist für die Mathematik und ihre Anwendungen zentral. 
Wir beschäftigen uns daher zunächst etwas systematischer mit diesem 
Begriff, den wir an verschiedenen Stellen schon benutzt und dabei 
eine gewisse Vertrautheit mit ihm bereits vorausgesetzt haben.

\index{Fermat@\textsc{Fermat}, Pierre de (1607-1655)}
\index{Descartes@\textsc{Descartes}, Ren\'e (1596-1650)}
\index{Leibniz@\textsc{Leibniz}, Gottfried Wilhelm (1646-1716)}
\index{Bolzano@\textsc{Bolzano}, Bernard (1781-1848)}
\index{Euler@\textsc{Euler}, Leonhard (1707-1783)}
\index{Fourier@\textsc{Fourier}, Jean-Baptiste Joseph (1768-1830)}
\index{Bernoulli D@\textsc{Bernoulli}, Daniel (1700-1781)}
\index{Bernoulli J@\textsc{Bernoulli}, Johann (1667-1748)}
\index{Cauchy@\textsc{Cauchy}, Augustin Louis (1789-1857)}
 \index{Weierstrass@\textsc{Weierstrass}, Karl Theodor Wilhelm (1815-1897)}
Nach Anfängen bei Fermat (1636) und 
Descartes (1637) und weiteren 
Präzisierungen des Begriffs \slanted{Funktion} (\slanted{functio}) 
durch Leibniz und Johann Bernoulli gegen Ende 
des 17. Jahrhunderts haben Euler 
(1748 bzw. 1755), Daniel Bernoulli (1755), Fourier (1822), 
Bolzano (1817) und Cauchy (1821) wichtige Beiträge 
zur Entwicklung dieses Begriffs geliefert. 
Der "`moderne"' Funktionsbegriff wird häufig mit dem Namen Dirichlet belegt. 
Aus der Namensgebung lässt sich aber --- wie so oft in der Mathematik --- 
nicht ohne Weiteres auf die Urheberschaft schließen. 
Man vergleiche dazu den informativen Artikel von Youschkevitch \citep{Yousch}.

Mithilfe des Abbildungsbegriffs lässt sich auch der Begriff der 
\slanted{abzählbaren Menge} definieren, 
der die Grundlage dafür liefert, unendliche 
Mengen bezüglich ihrer Mächtigkeit miteinander zu vergleichen.

\slanted{Natura non facit saltus} --- "`die Natur macht keine Sprünge"'. 
Diese Raoul Fournier (1627) zugeschriebene Äußerung spiegelt 
jene Einstellung wider, 
wegen der bei der mathematischen Behandlung naturwissenschaftlicher 
Vorgänge lange Zeit nur \slanted{stetige} Funktionen betrachtet wurden. 
Die noch etwas vagen Stetigkeitsvorstellungen der Analytiker des 18. 
Jahrhunderts wurden im 19. Jahrhundert durch Bolzano, Cauchy und vor allem 
Weierstraß präzisiert. Mit der anschaulichen Vorstellung, dass bei stetigen 
Funktionen bei einer "`kleinen"' Änderung der Argumente sich auch die 
Funktionswerte nur wenig ändern (genauer: dass die Funktionswerte sich 
beliebig wenig ändern, wenn nur die Argumente sich hinreichend wenig ändern), 
ist man dem exakten Stetigkeitsbegriff schon recht nahe, der mithilfe 
des Umgebungs- oder Folgenbegriffs definiert wird. Wir beschränken uns im 
Bezug auf die Stetigkeit hier auf \slanted{metrische} Räume, in denen 
man die "`Nähe"' mithilfe eines Abstandsbegriffs messen kann.

Stetige reellwertige Funktionen auf Intervallen und vor allem auf  
kompakten Mengen haben bemerkenswerte Eigenschaften ({\zB} 
Zwischenwerteigenschaft, Satz vom Maximum und Minimum, 
gleichmäßige Stetigkeit). Wir beschränken uns hier bei der Kompaktheit 
auf den Begriff "`folgenkompakt"' --- metrischen Räumen und ihrer Topologie 
ist ein späterer Abschnitt gewidmet. Den Grenzwertbegriff bei Funktionen 
führt man am besten auf den Stetigkeitsbegriff zurück (und nicht umgekehrt!).  


\section{Grundbegriffe}

%% Question 1
\begin{frage}
  \label{03_abdf}
  \index{Abbildung!allgemeine}
  \index{Funktion!allgemeine}
  Was versteht man unter einer Abbildung (Funktion) $f$ von einer Menge 
  $X$ in eine Menge $Y$? Welche Notationen sind Ihnen hierfür bekannt?
\end{frage}

\begin{antwort}
  Unter einer Abbildung $f$ von $X$ nach $Y$ versteht man 
  eine Vorschrift 
  (die zum Beispiel durch ein Naturgesetz, 
  eine Tabelle oder eine Formel gegeben sein kann), die 
  \slanted{jedem} Element $x\in X$ \slanted{genau ein} Element 
  $f(x) \in Y$ zuordnet. 

  Der Begriff "`Abbildung"' ist sehr allgemein und beinhaltet 
  keinerlei Einschränkung bezüglich der Objekte, die einander 
  zugeordnet werden. Den Begriff "`Funktion"' 
  gebraucht man in einem engeren Sinn, und zwar dann, wenn 
  es sich bei der Abbildung um die Zuordnung eines 
  reellen oder komplexen Zahlenwerts handelt, wenn also $Y=\RR$ 
  oder $Y=\CC$ gilt. Man spricht in diesem Fall von reell- oder 
  komplexwertigen Funktionen. 

  Zur Beschreibung einer Funktion $f$ von 
  $X$ nach $Y$ benutzt man die Schreibweise 
  \[
  f\fd X\to Y, \qquad x\mapsto f(x).
  \] 
  Der erste Ausdruck charakterisiert $f$ als eine Abbildung von $X$ nach $Y$, 
  der zweite spezifiziert die Zuordnungsvorschrift, derzufolge 
  $x\in X$ auf das Element $f(x)\in Y$ abgebildet wird. 
  \AntEnd
\end{antwort}

%% Question 2
\begin{frage}
  \label{03_graf}
  \index{Graph}
  Was versteht man unter dem \bold{Graphen} einer Abbildung $f\fd X\to Y$? 
\end{frage}

\begin{antwort}
  Eine Abbildung $f\fd X\to Y$ stellt jedes Element $x\in X$ mit 
  einem bestimmten Element $f(x)\in Y$ zu einem Paar $\big(x,f(x)\big)$ 
  zusammen. Diese Paare bilden eine Teilmenge 
  $G_f$ \nomenclature{$G_f$}{Graph von $f$}
  des kartesischen Produkts $X\times Y$, 
  das die folgenden beiden Eigenschaften besitzt 
  \satz{\setlength{\labelsep}{4mm}
    \begin{itemize}
    \item[\desc{i}] Jedes $x\in X$ tritt als erste Komponente 
      eines Paares $\big(x,f(x)\big)$ auf.\\[-3.5mm]
    \item[\desc{ii}] Die Zuordnung der zweiten 
      Komponente ist eindeutig, {\dasheisst} aus 
      $(x_1,y_1)\in G_f$ und $(x_2,y_2)\in G_f$ 
      und $x_1=x_2$ folgt $y_1=y_2$.
    \end{itemize}}
  \noindent
  Die Menge $G_f:=\{ (x,f(x));\, x\in X \}\subset X\times Y$ 
  nennt man den \slanted{Graphen von $f$}. \AntEnd
\end{antwort} 

%% Question 3
\begin{frage}
  Welcher Zusammenhang besteht zwischen den Begriffen 
  "`Graph einer Abbildung"' und "`Abbildung"'?
\end{frage}

\begin{antwort}
  Eine Abbildung ist durch ihren Graphen eindeutig festgelegt. 
  Zu jeder Teilmenge $X\times Y$, die 
  die Eigenschaften $(a)$ und $(b)$ besitzt, gibt es genau eine Abbildung, 
  die die Teilmenge als Graphen besitzt. 

  Aufgrund dieser eineindeutigen Beziehung lässt sich der Abbildungsbegriff 
  auch rein mengentheoretisch definieren, ohne den vagen Begriff 
  der "`Vorschrift"' 
  benutzen zu müssen. Eine Abbildung von $X\to Y$ ist demnach einfach ein 
  Graph, das heißt eine Teilmenge von $X\times Y$ mit den 
  Eigenschaften \desc{a} und \desc{b}.
  \AntEnd
\end{antwort}

%% Question 4
\begin{frage}
  Welche Visualisierungsmöglichkeiten für Funktionen $f\fd D \to \RR$ 
  mit $D\subset\RR$ bzw. $D\subset\RR\times\RR$ sind Ihnen bekannt?
\end{frage}

\begin{antwort}
  Der Graph einer Funktion $f\fd D\to \RR$ 
  als Teilmenge von $D\times \RR$ lässt sich 
  in aller Regel als Punktmenge 
  in einem zweidimensionalen Koordinatensystem visualisieren. Handelt es 
  sich bei $D$ um ein Intervall, und genügt die Funktion $f$ bestimmten 
  Stetigkeitsanforderungen, so besitzt der Graph der Funktion die Gestalt 
  einer "`Kurve"' wie in \Abb\ref{fig:03_funktion2d}. 

  \begin{center}
    \begin{minipage}{60mm}
      \includegraphics{mp/03_funktion2d}
      \captionof{figure}{Graph einer stetigen Funktion auf 
        einem reellen Intervall.}
      \label{fig:03_funktion2d}
    \end{minipage}
    \qquad
    \begin{minipage}{60mm}
      \includegraphics{mp/03_funktion2da}
      \captionof{figure}{Graph einer Funktion auf einer diskreten Menge.}
      \label{fig:03_funktion2da}
    \end{minipage}
  \end{center}

  Handelt es sich bei $D$ um eine diskrete Menge, etwa 
  $\NN$, so kann der Graph unter Umständen 
  durch einzelne unzusammenhängende Punkte 
  wie in \Abb\ref{fig:03_funktion2da} visualisiert werden, 
  
  Der Graph einer Abbildung $g\fd D\to \RR$ mit 
  $D\subset \RR \times \RR$ ist eine Teilmenge des 
  $\RR^3$. Dieser lässt sich 
  unter bestimmten Voraussetzungen als Punktmenge in einem 
  dreidimensionalen kartesischen Koordinatensystem visualisieren. 
  Bei einer stetigen Abbildung hat der Graph etwa die Gestalt einer 
  Fläche im Raum (\sieheAbbildung\ref{fig:04_graph}).

  \begin{center}
    \includegraphics{povray/04_graph.pdf}
    \captionof{figure}{Der Graph einer stetigen Abbildung 
      $D \to \RR$ mit $D\subset\RR^2$ bildet 
      eine Fläche im Raum.}
    \label{fig:04_graph}
  \end{center}

  Allgemein kann man zur Visualisierung von Abbildungen den Bild- und 
  Wertebereich auch getrennt einzeichnen wie in \Abb\ref{fig:03_exp}. Bestimmte   
  Abbildungseigenschaften 
  lassen sich damit häufig sehr gut veranschaulichen, {\zB} welche 
  Teilmengen des Definitionsbereichs auf welche Teilmengen des Bildbereichs 
  abgebildet werden.  

  \begin{center}
    \includegraphics{mp/03_exp}
    \captionof{figure}{Darstellung von Bild- und Wertebereich einer Abbildung.}
    \label{fig:03_exp}
  \end{center}

  Die Möglichkeit der Visualisierung einer Funktion ist nicht 
  grundsätzlich gegeben, sondern hängt von bestimmten Eigenschaften der 
  Funktion ab. Man hat zum Beispiel keine Chance, die Funktion 
  $\RR \to \RR$, die für rationale Argumente den Wert $0$ und für irrationale 
  den Wert $1$ annimmt, nach diesem Muster zu visualisieren.
  \AntEnd
\end{antwort}

%% Question 5
\begin{frage}
  \label{03_abid}
  Wann sind zwei Abbildungen gleich?
\end{frage}

\begin{antwort}
  Zwei Abbildungen $f_1\fd X_1\to Y_1$ und $f_2\fd  X_2 \to Y_2$ sind 
  genau dann gleich, wenn $X_1=X_2$ und $Y_1=Y_2$ gilt und für jedes 
  $x\in X_1=X_2$ die Werte $f_1(x)$ und $f_2(x)$ identisch sind. 
  \AntEnd
\end{antwort}

%% Question 6
\begin{frage}\label{03_meng}
  \index{Wertemenge}\index{Zielbereich}\index{Definitionsbereich}
  \index{Bildmenge}
  Erläutern Sie die Begriffe 
  \bold{Definitionsbereich}, \bold{Wertemenge}, \bold{Zielbereich}, 
  \bold{Bildmenge} einer Abbildung $f\fd X\to Y$.
\end{frage}

\begin{antwort}
  Der \slanted{Definitionsbereich} einer Abbildung $f\fd X\to Y$ 
  ist die Menge $X$, der \slanted{Zielbereich}  
  die Menge $Y$. Der \slanted{Wertebereich bzw. die Bildmenge} der 
  Abbildung ist die Menge aller 
  derjenigen Elemente des Zielbereichs, die als Funktionswerte eines 
  Elements des Wertebereichs auftreten, also die Menge 
  $ f(X):=\{ f(x)\sets x\in X\}$. 
  \AntEnd
\end{antwort}

%% Question 7
\begin{frage}\label{03_jek}
  \index{injektiv}\index{surjektiv}\index{bijektiv}
  Wann nennt man eine Abbildung 
  \bold{injektiv},
  \bold{surjektiv},
  \bold{bijektiv}? 
  Geben Sie einfache Beispiele. 
\end{frage}

\begin{antwort}
  Eine Abbildung $f\fd X\to Y$ heißt \slanted{injektiv}, wenn 
  verschiedene Elemente aus $X$ stets auf verschiedene 
  Elemente aus $Y$ abgebildet werden, wenn also aus 
  $x_1\not=x_2$ stets $f(x_1)\not= f(x_2)$ folgt. 
  Beispielsweise ist die Abbildung $\RR\to\RR,\; x\mapsto \exp(x)$ 
  injektiv, die Abbildung $\RR\to[0,1],\; x\mapsto\sin x$ 
  hingegen nicht.  

  Die Abbildung heißt \slanted{surjektiv}, wenn der Wertebereich gleich 
  dem Zielbereich ist, wenn also für jedes $y\in Y$ ein $x\in X$
  mit $f(x)=y$ existiert. So ist beispielsweise die Abbildung 
  $\RR\to \ropen{0,\infty},\; x\mapsto x^2$ surjektiv, die Abbildung 
  $\RR\to\RR,\; x\mapsto x^2$ hingegen nicht.

  Die Abbildung heißt \slanted{bijektiv}, wenn sie injektiv und 
  surjektiv ist. Einfache Beispiele bijektiver 
  Abbildungen sind die linearen Funktionen 
  $\RR\to\RR,\; x\mapsto ax$ mit $a\in\RR$.
  \begin{center}
    \includegraphics{mp/03_inj}
    \captionof{figure}{Veranschaulichung der Begriffe 
      "`surjektiv"', "`injektiv"' und "`bijektiv"'.}
    \label{fig:03_inj}
  \end{center}
  Die \Abb\ref{fig:03_inj} illustriert die Begriffe am Beispiel einer Abbildung 
  zwischen zwei endlichen Mengen.
  \AntEnd
\end{antwort}

%% Question 8
\begin{frage}
  \label{03_urb}
  \index{Bild einer Abbildung}
  \index{Urbild}
  Sind $A\subset X$ und $B\subset Y$ Teilmengen und $f\fd X\to Y$ eine 
  Abbildung. Was versteht man unter dem \bold{Bild} von $A$ unter $f$, 
  was unter dem \bold{Urbild} von $B$ unter $f$?
\end{frage}

\begin{antwort}
  Unter dem Bild $f(A)$ von $A$ unter $f$ bzw. dem Urbild $f^{-1}(B)$ 
  von $B$ unter $f$ versteht man die Mengen 
  \begin{align*}
    f(A) &:= \{ f(x) \in Y\sets x\in A \} \subset Y  \\
    f^{-1}(B) &:= \{ x\in X\sets f(x)\in B \} \subset X, 
  \end{align*}
  vgl. auch \Abb\ref{fig:03_urbild}.
  Man beachte, dass das Urbild $f^{-1}(B)$ einer Teilmenge $B\subset Y$ 
  im Gegensatz zur Umkehr\slanted{abbildung} immer existiert. 
  \AntEnd

  \begin{center}
    \includegraphics{mp/03_urbild}
    \captionof{figure}{Bilder $f(A)$, 
      $B$ einer Abbildung und deren Urbilder $A$ und $f^{-1}(B)$.}
    \label{fig:03_urbild}
  \end{center}
\end{antwort}

%% Question 9
\begin{frage}\label{03_umk}\index{Umkehrabbildung}
  Wie ist die \bold{Umkehrabbildung} einer bijektiven Abbildung 
  $f\fd X\to Y$ definiert?
\end{frage}

\begin{antwort}
  Die \slanted{Umkehrabbildung} $f^{-1}$ 
  \nomenclature{$f^{-1}$}{Umkehrabbildung} 
  einer bijektiven Abbildung $f\fd X\to Y$ ist definiert durch 
  \begin{equation}
    f^{-1}\fd Y \to X, \qquad f(x) \mapsto x. \tag{$\ast$}
  \end{equation}

  Dass diese Definition sinnvoll ist, wird durch die Bijektivität 
  von $f$ gewährleistet. Aus der Surjektivität von $f$ folgt, dass zu 
  jedem $y\in Y$ mindestens ein $x\in X$ mit $y=f(x)$ exis\-tiert, und  
  aus der Injektivität, dass es höchstens ein $x\in X$ mit dieser Eigenschaft 
  gibt. Also ist durch ($\ast$) eine Vorschrift gegeben, die 
  \slanted{jedem} $y\in Y$ \slanted{genau ein} $x\in X$ zuordnet, also eine 
  Abbildung $Y\to X$, siehe auch \Abb\ref{fig:03_umkehrabbildung}.
  \AntEnd

  \begin{center}
    \includegraphics{mp/03_umkehrabbildung}
    \captionof{figure}{Graph einer bijektiven Abbildung $f$ und deren Umkehrabbildung.}
    \label{fig:03_umkehrabbildung}
  \end{center}

\end{antwort}

%% Question 10
\begin{frage}\label{03_abz}\index{abzählbar}
  \index{hochstens@höchstens abzählbar}
  Was versteht man unter einer (höchstens) \bold{abzählbaren} Menge?
\end{frage}

\begin{antwort}
  "`Abzählbarkeit"' ist ein Maß für die Mächtigkeit einer 
  unendlichen Menge. Anschaulich steht hinter diesem Begriff die 
  Vorstellung, die Elemente einer unendlichen Menge derart 
  "`durchzunummerieren"' bzw. in einer  Reihenfolge anzuordnen, 
  dass man sinnvollerweise vom "`ersten"' Element, vom "`zweiten"', 
  vom "`dritten"' usw. sprechen kann. Existiert eine solche Anordnung, dann 
  gelangt man beim Durchgang durch die unendliche Reihe zu jedem Element nach 
  endlich vielen Schritten.

  Präziser: 
  Bei einer derartigen Anordnung handelt es sich darum, 
  jedem Element der Menge \slanted{eineindeutig} 
  eine natürliche Zahl zuzuordnen. 
  Das führt auf die folgende Definition der Abzählbarkeit: 
  \slanted{eine Menge $M$ ist abzählbar genau dann, wenn 
    eine bijektive Abbildung $\NN \to M$ existiert.} 
  Zum Beispiel ist die Menge $A$ aller natürlichen 
  Quadratzahlen abzählbar, und zwar vermöge der Abbildung
  \[
  \NN \to A, \qquad n \mapsto n^2.
  \]  
  Eine abzählbare Menge ist definitionsgemäß immer unendlich. 
  \slanted{Höchstens} abzählbar ist sie dann, 
  wenn sie endlich oder abzählbar ist.
  \AntEnd
\end{antwort}

%% Question 11
\begin{frage}\label{03_abca}
  Warum ist für (höchstens) abzählbare Mengen auch das kartesische Produkt 
  $X\times Y$ (höchstens) abzählbar?
\end{frage}

\begin{antwort}
  Sind $X$ und $Y$ abzählbar, dann lassen sich die Elemente 
  aus $X\times Y$ wie in \Abb\ref{fig:03_abz1} in einer Tabelle eintragen. 
  Die Paare $(x_i,y_j)$ kann man nun durchnummerieren, indem man  
  bei $(x_1,y_1)$ beginnend den Pfeilen folgt. Man erkennt dann, 
  dass man mit diesem "`Diagonalverfahren"' jedes Paar $(x_i,y_j)$  
  nach endlich vielen Schritten erreicht. 
  Das zeigt die Abzählbarkeit von $X\times Y$ im Fall abzählbarer Mengen 
  $X$ und $Y$. 
  \index{Cantor@\textsc{Cantor}, Georg (1845-1918)}
  \index{Cantor'sches Diagonalverfahren}

  Ist eine der beiden Mengen, etwa $Y=\{ y_1 , \ldots, y_n \}$ endlich, 
  dann erhält man eine Abzählung durch 
  \[
  (x_1, y_1),\ldots, (x_1,y_n), (x_2,y1), \ldots, (x_2,y_n), (x_3,y_1), \ldots
  \]
  Der Fall, dass beide Mengen endlich sind, ist trivial, da in diesem 
  Fall auch das kartesische Produkt endlich ist.
  \AntEnd

  \begin{center}
    \includegraphics{mp/03_abz1}
    \captionof{figure}{Abzählstrategie des kartesischen Produkts abzählbarer 
      Mengen.}
    \label{fig:03_abz1}
  \end{center}
  
\end{antwort}

%% Question 12
\begin{frage}\label{03_abv}
  Begründen Sie, warum für höchstens abzählbare Mengen $X_n$, $n\in\NN$ auch 
  $X := \bigcup_{n=1}^\infty X_n$ (höchstens) abzählbar ist.
\end{frage}

\begin{antwort}
  Seien zunächst alle Mengen $X_n$ unendlich. 
  Nach Voraussetzung gibt es für jedes $n\in \NN$ eine Bijektion 
  $\NN \to X_n$ mit $k \mapsto x_{n,k}\in X_n$. Das liefert eine Bijektion   
  \[
  \NN \times \NN \leftrightarrow \bigcup_{n=1}^\infty X_n; 
  \qquad (i,j) \mapsto x_{i,j} \in X_i.
  \]
  Da $\NN\times\NN$ nach Frage \ref{03_abca} abzählbar ist, gilt dies 
  auch für $\bigcup_{n=1}^\infty X_n$. 

  Sind mehrere oder alle der Mengen $X_n$ endlich, dann ist die durch 
  $x_{n,k}\mapsto (n,k)$ definierte Abbildung $X\to \NN\times\NN$ injektiv, 
  aber nicht mehr surjektiv. Aufgrund der Abzählbarkeit von $\NN\times\NN$ 
  erhält man damit zunächst eine injektive Abbildung 
  $X\to \NN$. Deren Bild ist eine unendliche Teilmenge 
  von $M\subset \NN$ (da $X$ unendlich ist), 
  und eine solche lässt sich immer bijektiv auf $\NN$ abbilden, indem 
  man die Elemente von $M$ gemäß der Ordnungsstruktur von 
  $\NN$ durchnummeriert. Die Bijektionen $X\leftrightarrow M$ und 
  $M \leftrightarrow \NN$ stiften dann eine Bijektion zwischen 
  $X$ und $\NN$.     
  \AntEnd
\end{antwort}

%% Question 13
\begin{frage}\label{03_abzz}\index{Q@$\QQ$!Abzählbarkeit}
  Warum sind $\ZZ$, $\ZZ\times \ZZ$ und $\QQ$ abzählbar?
\end{frage}

\begin{antwort}
  Die Menge der ganzen Zahlen ist abzählbar, etwa durch 
  \[
  \begin{array}{lrrrrrrrr}
    \NN\quad & 1 & 2 & 3 & 4 & 5 & 6 & 7 & \ldots \\
    &\downarrow&\downarrow&\downarrow&\downarrow&\downarrow&
    \downarrow&\downarrow & \\
    \ZZ\quad & 0 & 1 & -1 & 2 & -2 & 3 & -3 & \ldots 
  \end{array}
  \]
  Zusammen mit der Antwort zu Frage \ref{03_abca} 
  folgt daraus die Abzählbarkeit des 
  kartesischen Produktes $\ZZ \times \ZZ$. 
  Bezüglich $\QQ$ genügt es zu bemerken, dass eine injektive Abbildung  
  \[
  \QQ \to \ZZ\times \ZZ, \qquad 
  \frac{m}{n} \mapsto (m,n), \qquad m\in\ZZ, n\in\NN
  \]
  existiert. Da $\ZZ\times \ZZ$ abzählbar ist, gibt es eine 
  injektive Abbildung $f\fd\QQ\to\NN$. Das Bild $f(\QQ)$ schließlich 
  ist eine unendliche Teilmenge von $\NN$, also existiert nach der 
  Bemerkung in der vorigen Frage eine Bijektion 
  $f(\QQ) \leftrightarrow \NN$, die zusammen 
  mit $\QQ \leftrightarrow f(\QQ)$ eine Bijektion 
  $\QQ \leftrightarrow \NN$ vermittelt. Das heißt, $\QQ$ ist abzählbar. 
  \AntEnd
\end{antwort}

%% Question 14
\begin{frage}\label{03_ueba}\index{uberabzahlbar@überabzählbar}
  Kennen Sie ein Beispiel einer überabzählbaren Menge?
\end{frage}

\begin{antwort}
  Beispiele überabzählbarer Mengen sind die 
  echten reellen Intervalle $[a,b]$ oder $\open{a,b}$ 
  mit $a<b$ (vgl. Frage \ref{03_uint}). 
  Aus deren Überabzählbarkeit folgt unmittelbar, dass auch die Mengen 
  $\RR$, $\CC$ und $\RR^n$ für $n\in\NN$ überabzählbar sind.

  Für eine unendliche abzählbare Menge $M$ ist die Potenzmenge 
  $\mathfrak{P}(M)$ \nomenclature{$\mathfrak{P}(M)$}{Potenzmenge}
  ebenfalls überabzählbar (vgl. Frage \ref{03_upot}). 
  So ist beispielsweise die Menge aller Teilmengen der natürlichen Zahlen 
  überabzählbar.
  \AntEnd
\end{antwort}


%% Question 15
\begin{frage}\label{03_uint}
  Begründen Sie, warum das Intervall $[0,1]$ überabzählbar ist.
\end{frage}

\begin{antwort}
  Für den Beweis stellen wir die reellen Zahlen 
  aus dem Intervall $[0,1]$ in ihrer $2$-adischen Entwicklung dar (vgl. 
  Frage \ref{02_gal}). Jede reelle Zahl $x\in [0,1]$ besitzt nach 
  dem Satz über die $g$-adische Entwicklung eine eindeutige nicht abbrechende 
  Darstellung $x=0,r_1r_2r_3\ldots$ mit $r_i\in\{0,1\}$.  

  Angenommen, es gäbe eine Abzählung $x_1,x_2,x_3,\ldots$ des Intervalls 
  $[0,1]$. Wir schreiben die Zahlen in ihrer $2$-adischen Entwicklung 
  untereinander
  \begin{equation}
    \begin{array}{rcl}
      x_1 &=& 0,r_{11}r_{12}r_{13}r_{14}\ldots \\
      x_2 &=& 0,r_{21}r_{22}r_{23}r_{24}\ldots \\
      x_3 &=& 0,r_{31}r_{32}r_{33}r_{34}\ldots \\
      \cdots &=& \cdots 
    \end{array}\tag{$\ast$}
  \end{equation}
  und betrachten die durch "`Diagonalisierung"' dieser 
  \index{Diagonalisierung} Tabelle erhaltene reelle Zahl 
  $x:=0,r_{11}r_{22}r_{33}\ldots \in[0,1]$.
  In deren $2$-adischer Entwicklung vertauschen wir jedes Vorkommen 
  der Ziffer $0$ durch die Ziffer $1$ und jedes Vorkommen 
  der Ziffer $1$ durch die Ziffer $0$. 
  Die auf diese Weise konstruierte reelle Zahl 
  $\widetilde{x}\in[0,1]$ kommt in der Abzählung ($\ast$) 
  aber nicht vor, denn sie unterscheidet sich von $x_1$ in der ersten 
  Dualstelle, von $x_2$ in der zweiten, von $x_3$ in der dritten usw. 
  Es folgt, dass die  hypothetische Abzählung 
  $x_1,x_2,x_3\ldots$ nicht alle Zahlen des 
  Intervalls $[0,1]$ erfassen kann. Das Intervall $[0,1]$ kann  
  damit nicht abzählbar sein. 
  \index{Cantor@\textsc{Cantor}, Georg (1845-1918)}
  \index{Cantor'sches Diagonalverfahren}

  Der hier vorgeführte Beweisprinzip ist 
  als "`Zweites Cantor'sches Diagonalverfahren"' bekannt. Eine Variante 
  davon wird in Frage \ref{03_upot} vorgeführt.
  \AntEnd  
\end{antwort}

%% Question 16
\begin{frage}\label{03_ueb}\index{R@$\RR$!Überabzählbarkeit von}
  Warum ist jedes kompakte Intervall $[a,b]\subset \RR$, 
  jedes halboffene Intervall $\ropen{a,b}\subset \RR$ und jedes 
  offene Intervall $\open{a,b}\subset \RR$ 
  $(a<b)$ überabzählbar? Warum ist $\RR$ überabzählbar? 
\end{frage}

\begin{antwort}
  Jedes kompakte Intervall $[a,b]\subset \RR$ lässt sich durch 
  \[
  [a,b] \to [0,1], \qquad x \mapsto \frac{x-a}{b-a}
  \]
  bijektiv auf das Intervall $[0,1]$ abbilden. Würde eine Bijektion 
  $[a,b] \to\NN$ existieren, dann gäbe es damit auch eine Bijektion 
  $[0,1] \to \NN$, im Widerspruch zur Überabzählbarkeit von $[0,1]$.

  Weiter lässt sich jedes halboffene oder offene Intervall $M$ 
  durch eine lineare Skalierung und Translation bijektiv auf 
  ein halboffenes bzw. offenes Intervall $M'$ abbilden, für das 
  $[0,1]\subset M'$ gilt. Gäbe es eine injektive Abbildung $M\to \NN$, 
  dann gäbe es damit auch eine von $[0,1]$ nach $\NN$, im Widerspruch 
  zur Überabzählbarkeit von $[0,1]$. 

  Die Überabzählbarkeit von $\RR$ folgt ebenfalls 
  unmittelbar aus derjenigen des Intervalls $[0,1]$. 
  Da es keine surjektive Abbildung $\NN \to [0,1]$ 
  gibt, kann es wegen $[0,1]\subset \RR$ erst recht keine surjektive 
  Abbildung von $\NN$ nach $\RR$ geben.
  \AntEnd
\end{antwort}

%% Question 17
\begin{frage}\label{03_alg}\index{algebraische Zahl}\index{transzendente Zahl}
  Wann heißt eine reelle oder komplexe Zahl 
  \bold{algebraisch}, wann heißt sie 
  \bold{trans\-zen\-dent}?
\end{frage}

\begin{antwort}
  Die algebraischen Zahlen sind definiert als diejenigen 
  komplexen Zahlen, die Lösung einer polynomialen Gleichung mit 
  rationalen Koeffizienten
  \[
  a_n x^n + a_{n-1}x^{n-1}+\cdots + a_1x+a_0 =0, \quad a_i \in \QQ
  \]
  sind. 
  \slanted{Transzendent} heißt eine komplexe Zahl dann, wenn sie nicht 
  algebraisch ist.

  Die algebraischen Zahlen bilden eine echte Obermenge von $\QQ$,  
  denn jedes $r\in \QQ$ ist Lösung der Gleichung $x-r=0$, 
  aber nicht jede algebraische Zahl ist rational. 
  \AntEnd
\end{antwort} 

%% Question 18
\begin{frage}
  Ist $\sqrt{2}+\sqrt{3}$ eine algebraische Zahl?
\end{frage}

\begin{antwort}
  In der Algebra zeigt man, dass die algebraischen Zahlen einen 
  Körper bilden (s. {\zB} \citep{Karpfinger}). 
  Aus diesem allgemeinen Zusammenhang folgt, 
  dass $\sqrt{2}+\sqrt{3}$ algebraisch ist, da dies offensichtlich 
  auf $\sqrt{2}$ und auf $\sqrt{3}$ zutrifft.

  \medskip
  \noindent
  Man kann auch einfach so schließen: $\sqrt{2}+\sqrt{3}$ ist Nullstelle 
  des Polynoms $X^4-10X^2+1\in\ZZ[X]$, wie man durch 
  zweimaliges Quadrieren feststellt, und  damit algebraisch.
  \AntEnd
\end{antwort}

%% Question 19
\begin{frage}\label{03_trz}\index{Transzendenz}
  Kennen Sie (ohne Beweis) Beispiele für transzendente Zahlen?
\end{frage}

\begin{antwort}
  Die prominentesten transzendenten Zahlen sind die Kreiszahl 
  $\pi$ und die Euler'sche Zahl $e$. Die Transzendenz von $e$ wurde 
  im Jahr 1873 von Hermite \index{Hermite@\textsc{Hermite}, Charles (1822-1901)}
  bewiesen, diejenige von $\pi$ im Jahr 
  1882 von Lindemann. \index{Lindemann@\textsc{Lindemann}, Ferdinand (1852-1939)}
  Lindemanns Beweis beantwortete damit auch 
  die klassische Frage der griechischen Mathematik nach der 
  Möglichkeit einer 
  "`Quadratur des Kreises"' \index{Quadratur des Kreises}
  (also die Aufgabe, zu einem gegebenen Kreis 
  nur mithilfe von Zirkel und Lineal ein 
  flächengleiches Quadrat zu konstruieren), und zwar im negativen Sinne. 
  \AntEnd
\end{antwort}

%% Question 20
\begin{frage}\label{03_trex}
  Wie kann man die Existenz transzendenter Zahlen einfach beweisen 
  (ohne eine zu kennen)?
\end{frage}

\begin{antwort}
  Die Existenz transzendenter Zahlen lässt sich durch ein 
  Abzählbarkeitsargument beweisen, bei dem man die Abzählbarkeit 
  der Menge aller \slanted{algebraischen} Zahlen zeigt. 
  Da $\CC$ überabzählbar und die Vereinigung der algebraischen und 
  transzendenten Zahlen ist, folgt daraus, dass die Menge der transzendenten 
  Zahlen ebenfalls überabzählbar (und damit erst recht nicht leer) sein muss. 

  Zum Beweis der Abzählbarkeit der algebraischen Zahlen
  genügt es natürlich, die Abzählbarkeit aller \slanted{ganzzahligen} Polynome
  zu zeigen.  
  Dazu definiere man die "`Höhe"' $h( p )$ 
  eines ganzzahligen Polynoms durch $
  h( p ) := |a_n| + |a_{n-1}| + \cdots + |a_1| + |a_0|$. 
  Für jede natürliche Zahl $N$ gibt es nur endliche viele Polynome 
  mit einer kleineren Höhe als $N$. Wegen $
  \ZZ [X]   = \bigcup_{n=1}^\infty \{ p\in \ZZ[X]\sets h(p) < N \}$
  ist die Menge der ganzzahligen Polynome somit die Vereinigung 
  abzählbar vieler endlicher Mengen und damit nach Frage 
  \ref{03_abv} ebenfalls abzählbar. 
  \AntEnd
\end{antwort}

%% Question 21
\begin{frage}\label{03_upot}\index{Potenzmenge}
  Warum kann es für eine beliebige nichtleere Menge $X$ keine Bijektion 
  von $X$ auf ihre Potenzmenge $\mathfrak{P}(X)$ geben?
\end{frage}

\begin{antwort}
  Für \slanted{endliche} Mengen $X$ mit $m$ Elementen 
  kann es wegen $| \mathfrak{P}(X) | =2^m > m$ keine solche Bijektion geben.  

  Sei also $X$ unendlich. Angenommen, 
  es existiert eine Bijektion $X\to \mathfrak{P}(X)$, die 
  jedes Element $x\in X$ auf eine Teilmenge 
  $M_x \in \mathfrak{P}(X)$ abbildet. 
  Für jede Teilmenge $M_x$ gibt es 
  nun genau zwei Möglichkeiten: entweder $x \in M_x$ oder 
  $x\not\in M_x$. Man betrachte die Menge $M\subset X$, 
  die durch 
  \[
  x \in  M \LLa x\not\in M_x
  \]
  definiert ist. Die Menge $M$ ist dann mit keiner der Teilmengen 
  $M_x$ identisch, denn jedes $x\in X$ liegt \slanted{entweder} in $M$ 
  \slanted{oder} in $M_x$, aber nicht in beiden. 
  Damit existiert auch kein $x\in X$, das in der angenommenen Bijektion auf 
  $M$ abgebildet wird. Die Abbildung 
  $X\to \mathfrak{P}(X)$ ist also im Widerspruch zur Annahme nicht surjektiv.
  \AntEnd
\end{antwort} 

\section{Stetigkeit}

%% Question 22
\begin{frage}\label{03_epde}\index{Stetigkeit}
  Sind $(X,d_X)$ und $(Y,d_Y)$ metrische Räume und ist 
  $f\fd X\to Y$ eine Abbildung. Wann heißt $f$ im Punkt 
  (an der Stelle) $a\in X$ 
  \bold{stetig}? Wann heißt $f$ (schlechthin) stetig?
\end{frage}

\begin{antwort}
  Die Idee, die dem Stetigkeitsbegriff zugrunde liegt, ist die,
  dass eine "`kleine"' Änderung des Arguments an einer Stelle  
  auch nur eine entsprechend "`kleine"' Änderung des Funktionswerts 
  in einer Umgebung dieser Stelle zur Folge hat. 
  Und mehr noch: Die Änderung des Funktionswerts 
  kann \slanted{beliebig klein} gehalten werden, solange die Veränderung 
  des Argumentes innerhalb bestimmter Grenzen bleibt.    
  
  Dies führt auf die folgende Definition der Stetigkeit in einem Punkt:

  \medskip
  \noindent\slanted{
    Eine Funktion $f\fd X\to Y$ heißt stetig im Punkt $a\in X$, 
    wenn zu jedem $\eps>0$ ein 
    $\delta(\eps):=\delta >0$ existiert, so dass für alle $x\in X$ gilt: 
    \[
    d_X( x,a ) < \delta \Ra d_Y\big( f(x),f(a) \big) < \eps. 
    \]}
  \medskip
  Äquivalent dazu ist die Formulierung:

  \medskip
  \noindent\slanted{Eine Funktion $f$ ist 
    stetig im Punkt $a$, wenn zu jeder $\eps$-Umgebung von $f(a)$ 
    eine $\delta$-Umgebung von $a$ existiert, sodass gilt:  
    \[
    f( U_\delta(a) ) \subset U_\eps\big( f(a) \big).
    \]}
  \noindent
  Die $\eps\delta$-Eigenschaft kennzeichnet die Stetigkeit 
  einer Funktion in \slanted{einem} Punkt ihres Definitionsbereichs. 
  Der Begriff wird insofern also als eine lokale Eigenschaft 
  einer Funktion eingeführt. Stetigkeit \slanted{schlechthin} ist dagegen eine 
  globale Eigenschaft einer Funktion. Eine Funktion heißt (schlechthin) stetig, 
  wenn sie in jedem Punkt ihres Definitionsbereichs stetig ist. 
  \AntEnd
\end{antwort}

%% Question 23
\begin{frage}\label{03_svis}\index{epsdelta@$\eps\delta$-Stetigkeit}
  Wie kann man sich die $\eps\delta$-Definition der Stetigkeit 
  für Funktionen $f\fd X\to Y$ veranschaulichen, wenn 
  $X\subset \RR$ und $Y\subset \RR$ bzw. $X\subset \RR^2$ und $Y\subset \RR^2$ gilt?
\end{frage}

\begin{antwort}[]
  \Ant Die \Abb\ref{fig:03_stetig1} veranschaulicht die Stetigkeit 
  im Punkt $a$ für eine Abbildung, 
  bei der Definitions- und Zielbereich reelle Intervalle sind. 
  Der Graph der Funktion verläuft im Intervall $\open{a-\delta,a+\delta}$ 
  innerhalb des $\eps$-Streifens um $f(a)$. Bei der 
  Funktion in \Abb\ref{fig:03_stetig2} sind Definitions- und Zielbereich 
  Teilmengen des $\RR^2$. Die Abbildung illustriert hier auf eine andere 
  Weise die Stetigkeitseigenschaft $f\big(U_\delta(a)\big)
  \subset U_\eps\big( f(a) \big)$ 
  von $f$ im Punkt $a$. \AntEnd

  \begin{center}
    \begin{minipage}{60mm}
      \includegraphics{mp/03_stetig1}
      \captionof{figure}{Stetigkeit einer Funktion $f\fd\RR\to\RR$.}
      \label{fig:03_stetig1}
    \end{minipage}
    \qquad
    \begin{minipage}{60mm}
      \includegraphics{mp/03_stetig2}
      \captionof{figure}{Stetigkeit einer Funktion $f\fd\RR^2\to\RR^2$.}
      \label{fig:03_stetig2}
    \end{minipage}
  \end{center}
\end{antwort}

%% Question 24
\begin{frage}\label{03_lips}\index{Lipschitz-stetig}
  \index{Lipschitz@\textsc{Lipschitz}, Rudolf (1832-1903)}
  Sind $(X,d_X)$ und $(Y,d_Y)$ metrische Räume. Wann heißt eine Abbildung 
  $f\fd X\to Y$ \bold{Lipschitz-stetig}? 
\end{frage}

\begin{antwort}
  Eine Funktion $f\fd X\to Y$ heißt 
  \slanted{Lipschitz-stetig}, wenn eine Konstante 
  $L\in \RR_+$ existiert, sodass für alle $a,b\in X$ die Ungleichung 
  \begin{equation}
    d_Y \big( f(a), f(b) \big) \le L \cdot d_X (a,b) \tag{$\ast$}
  \end{equation}
  gilt. Anschaulich bedeutet dies, dass die Verzerrung des Abstands zweier 
  Punkte $a, b \in X$ unter der Abbildung beschränkt bleibt, was bei einer 
  Abbildung $X\to Y$ mit $X\subset \RR$ und $Y\subset \RR$ 
  darauf hinausläuft, dass die Steigung des Graphen beschränkt bleibt.   
\end{antwort}

%% Question 25
\begin{frage}\index{Lipschitz-stetig}
  Ist die Funktion $\RR\to\RR_+$ mit $x\mapsto x^2$ Lipschitz-stetig?
\end{frage}

\begin{antwort}
  Die Funktion ist nicht Lipschitz-stetig. 
  Für jede Konstante $L\in\RR$ ist nämlich 
  $
  \left| a^2-b^2 \right| = \big| (a+b)(a-b) \big| > L \cdot |a-b|$, 
  sofern nur $|a+b|>L$ gilt. \AntEnd
\end{antwort}

%% Question 26
\begin{frage}\index{Lipschitz-stetig}
  Warum ist eine Lipschitz-stetige Abbildung stetig? 
\end{frage} 

\begin{antwort}
  Die normale Stetigkeit einer Lipschitz-stetigen Funktion ist eine  
  unmittelbare Konsequenz aus der Ungleichung ($\ast$). 
  Aus $|a-b|<\eps/L+1$ (wir wählen $L+1$ statt $L$, denn $L$ könnte Null sein) folgt für Lipschitz-stetiges $f$ nämlich  
  \[
  |f(a)-f(b)| < \eps.
  \]
  Mit $\delta:=\eps/L+1$ entspricht das genau der 
  $\eps\delta$-Definition der Stetigkeit.
  \AntEnd
\end{antwort}

%% Question 27
\begin{frage}\label{03_lipn}\index{Norm!Lipschitz-Stetigkeit}
  Sei $(X, \n{ \,\; })$ ein normierter $\KK$-Vektorraum 
  ($\KK=\RR$ oder $\KK=\CC$). Warum ist die durch 
  $x\mapsto \| x \|$ definierte \slanted{Normabbildung} 
  $N \fd X\to \RR$ Lipschitz-stetig?
\end{frage}

\begin{antwort}
  Es muss gezeigt werden, dass eine Konstante $L\in\RR_+$ existiert mit 
  \[
  \big|\, \n{a} - \n{b} \, \big| 
  \le L  \cdot \n{ a-b }  \quad\text{für alle $a,b\in X$}.
  \]
  Für $L=1$ ist diese Ungleichung aber gerade die Dreiecksungleichung 
  für Abschätzungen nach unten (vgl. Frage \ref{01_dru}), 
  die in jedem normierten Raum gilt. 
  \AntEnd
\end{antwort}

%% Question 28
\begin{frage}\label{03_lipd}\index{Metrik!Lipschitz-Stetigkeit}
  Sei $(X,d)$ ein metrischer Raum, $x_0\in X$ ein fester Punkt. 
  Warum ist die Abbildung $d \fd X\times X \to \RR$ mit  
  $x\mapsto d(x,x_0)$ Lipschitz-stetig?
\end{frage}

\begin{antwort}
  Wegen der Dreiecksungleichung gilt für alle $x,y\in X$:
  \[
  d(x,x_0) \le d(x,y) + d(y,x_0), \qquad 
  d(y,x_0) \le d(x,y) + d(x,x_0).
  \]
  Die beiden Ungleichungen implizieren zusammen 
  \[
  \big| d(x,x_0) -d(y,x_0) \big| \le 
  d( x,y ) \quad\text{für alle $x,y\in X$}.
  \]
  Mit $L=1$ entspricht das der Definition der Lipschitz-Stetigkeit.
  \AntEnd
\end{antwort}

%% Question 29
\begin{frage}\label{03_kont}\index{kontrahierende Selbstabbildung}
  Was versteht man unter einer \bold{kontrahierenden Selbstabbildung} 
  eines metrischen Raumes? 
  Warum ist eine solche Abbildung immer Lipschitz-stetig?
\end{frage}

\begin{antwort}
  Sei $(X,d)$ ein metrischer Raum. Eine 
  \slanted{kontrahierende Selbstabbildung} oder 
  \slanted{Kontraktion} 
  ist eine Abbildung $f\fd X\to X$, für die ein $\lambda < 1$ 
  existiert, sodass gilt:
  \[
  d\big( f(x), f(y) \big) \le \lambda \cdot d(x,y)\quad \text{für alle $x,y\in X$}.
  \]
  Die Lipschitz-Stetigkeit einer kontrahierenden Selbstabbildung 
  ergibt sich unmittelbar aus der Definition, 
  indem man $L = \lambda$ wählt. 
  \AntEnd
\end{antwort}

%% Question 30
\begin{frage}\index{folgenstetig}
  Wann heißt eine Abbildung $f\fd X\to Y$ \bold{folgenstetig} in $a\in X$?
\end{frage}

\begin{antwort}
  Eine Funktion $f\fd X\to Y$ heißt \slanted{folgenstetig im Punkt $a\in X$} 
  genau dann, wenn für \slanted{jede Folge} $(x_n)\subset X$ gilt: 
  \[
  \lim\limits_{n\to\infty} x_n=a \Ra \lim\limits_{n\to\infty} f(x_n) = f(a),
  \]
  vgl. \Abb\ref{fig:03_folgenstetig}.
  \begin{center}
    \includegraphics{mp/03_folgenstetig}
    \captionof{figure}{Die Funktion $f$ ist folgenstetig: Konvergiert die Folge 
      $(a_n)$ gegen $a$, so konvergiert die Folge der Funktionswerte 
      $\big(f(a_n)\big)$ gegen $f(a)$.}
    \label{fig:03_folgenstetig}
  \end{center}
  Frage~\ref{03_aquv} behandelt die Beziehung  dieses Stetigkeitsbegriffs mit 
  der $\eps\delta$-Stetigkeit. 
  \AntEnd
\end{antwort}

%% Question 31
\begin{frage} 
  \label{03_aquv}
  \index{Aquivalenzsatz@Äquivalenzsatz für Stetigkeit}
  (\bold{Äquivalenzsatz für Stetigkeit}) 
  Sind $(X,d_X)$ und $(Y,d_Y)$ metrische Räume und $f\fd X\to Y$ eine 
  Abbildung. Warum sind die Folgenstetigkeit von $f$ und die 
  $\eps\delta$-Stetigkeit von $f$ äquivalent?
\end{frage}

\begin{antwort}
  Sei zunächst $f\fd X\to Y$ stetig 
  im Sinne der $\eps\delta$-Definition, 
  ein $\eps>0$ sei beliebig gegeben,  und $(x_n) \subset X$ sei eine Folge, 
  die gegen $a\in X$ konvergiert. 
  Für jedes $\delta>0$ liegen dann ab einem bestimmten Index $N$ 
  alle Folgenglieder in $U_\delta( a )$, und daraus folgt -- da $f$ die 
  $\eps\delta$-Eigenschaft besitzt -- $f(x_n) \in U_\eps\big( f(a) \big)$. Das 
  aber bedeutet $\lim f(x_n) = f(a)$.

  Sei $f$ nun umgekehrt folgenstetig in $a\in X$. 
  Angenommen, $f$ besäße nicht die 
  $\eps\delta$-Eigenschaft. Dann gäbe es ein $\eps>0$, sodass für alle 
  $n\in \NN$ gilt:
  \[
  f\big( U_{1/n}(a) \big)\not\subset U_\eps \big( f(a) \big).
  \]  
  Das heißt, es gibt zu jedem $n\in \NN$ mindestens ein "`Ausnahmeelement"'
  $x_n\in U_{1/n}$ mit $f(x_n)\not\in U_\eps \big( f(a) \big)$. 
  Die Folge $(x_n)\subset X$ konvergiert dann gegen $a$, 
  die Folge $f(x_n)\subset Y$ der Funktionswerte aber nicht gegen $f(a)$,
  im Widerspruch zur Voraussetzung.
  \AntEnd   
\end{antwort}

%% Question 32
\begin{frage}
  \label{03_zuss}
  \index{Stetigkeit!zusammengesetzter Funktionen}
  Was bedeutet die Aussage "`Die Zusammensetzung stetiger Funktionen 
  ist stetig"'?
\end{frage}

\begin{antwort}
  Seien $f \fd X\to Y$ und $g\fd D \to Z$ zwei stetige Abbildungen mit 
  $f(X)\subset D$. Die Aussage bedeutet, dass unter diesen Voraussetzungen 
  auch die zusammengesetzte Abbildung $g\circ f \fd X \to Z$ stetig ist.

  Das sieht man {\zB} mithilfe des Folgenkriteriums. 
  \AntEnd
\end{antwort}

%% Question 33
\begin{frage}\label{03_perm}\index{Stetigkeit!Permanenzeigenschaften}
  Ist $(X,d_X)$ ein metrischer Raum und sind 
  $f,g\fd X\to \RR$ im Punkt $a\in X$ 
  stetige Funktionen. Warum gelten die \bold{Permanenzeigenschaften} 
  \begin{itemize}[2mm]
  \item[\desc{a}] $f\pm g$ ist stetig in $a$,\\[-3.5mm]   
  \item[\desc{b}] $fg$ ist stetig in $a$, \\[-3.5mm]
  \item[\desc{c}] $f/g$ ist stetig in $a$, 
    falls $g(a)\not =0$ ist?
  \end{itemize}
\end{frage}

\begin{antwort}
  Unter Zuhilfenahme des Folgenkriteriums ergeben sich die Aussagen als 
  direkte Konsequenzen der Permanenzeigenschaften für reelle Zahlenfolgen 
  (vgl. Frage \ref{02_freg}). 
  Sei $(x_n)$ eine gegen $a$ konvergente Folge aus $X$. 
  Aufgrund der Stetigkeit von $f$ und $g$ folgt
  $\lim f(x_n)=f(a)$ und $\lim g(x_n)=g(a)$, und mit den    
  Permanenzeigenschaften für reelle Zahlenfolgen folgert man daraus 
  \[
  \begin{array}{llp{3mm}ll}
    \text{\desc{a}} & \dis\lim\limits_{n\to\infty}\big( f(x_n)+g(x_n) \big)=
    f(a)+g(a), & &
    \text{\desc{b}} & \dis\lim\limits_{n\to\infty}\big( f(x_n)g(x_n) \big)=
    f(a)g(a) \\[2mm]
    \text{\desc{c}} & \dis\lim\limits_{n\to\infty}
    \frac{f(x_n)}{g(x_n)} = \frac{f(a)}{g(a)}, 
    \quad\text{falls $g(a)\not=0$.} & & & 
  \end{array}
  \]
  Gemäß dem Folgenkriterium ergibt sich daraus 
  die Stetigkeit der Funktionen $f\pm g$ und $fg$ im Punkt $a$
  sowie die Stetigkeit von 
  $f/g$ in $a$, falls $g(a)\not=0$ ist.
  \AntEnd
\end{antwort}

%% Question 34
\begin{frage}\label{03_pols}\index{Stetigkeit!von Polynomen}
  Warum ist jedes reelle oder komplexe Polynom in ganz $\RR$ bzw. 
  ganz $\CC$ stetig? 
  Warum ist jede \bold{rationale Funktion} $P/Q$ ($P,Q$ Polynome, 
  $Q$ nicht das Nullpolynom) in ihrem jeweiligen Definitionsbereich stetig?
\end{frage}

\begin{antwort}
  Es genügt zu bemerken, dass die identische Abbildung 
  $\KK \to \KK, \; z\mapsto z$ mit $\KK=\RR$ oder $\KK=\CC$ und die 
  konstanten Abbildungen $\KK\to \KK,\; z\mapsto a$ 
  mit $a\in \KK$ offensichtlich stetig (die erste mit jedem $\delta$, die 
  zweite mit $\delta=\eps$) sind. 
  Die Stetigkeit der Polynome und rationalen Funktionen 
  ergibt sich hieraus durch wiederholte Anwendungen 
  der Permanenzeigenschaften stetiger Funktionen aus Frage \ref{03_perm}.  
  \AntEnd
\end{antwort} 

%% Question 35
\begin{frage}
  \label{03_bolz}
  \index{Nullstellensatz von Bolzano}
  Was besagt der \bold{Nullstellensatz von Bolzano} für eine stetige 
  reellwertige Funktion $f$ auf einem echten Intervall $I\subset \RR$?
\end{frage}

\begin{antwort}
  Der Nullstellensatz von Bolzano lautet:

  \medskip\noindent 
  \slanted{Ist die Funktion $f\fd X\to \RR$ mit $X\subset \RR$ 
    auf dem Intervall $[a,b]\subset X$ 
    stetig und haben $f(a)$ und $f(b)$ unterschiedliche Vorzeichen 
    ($f(a)<0$ und $f(b)>0$ oder $f(a)>0$ und $f(b)<0$), so besitzt die Funktion
    mindestens eine Nullstelle in $(a,b)$.} 

  \medskip\noindent
  Der Satz beschreibt die anschauliche Tatsache, dass 
  ein zusammenhängender Funktionsgraph, der Punkte über- und unterhalb der 
  $x$-Achse durchläuft, diese in mindestens einem Punkt schneiden muss, \sieheAbbildung\ref{fig:03_bolzano} 

  \begin{center}
    \includegraphics{mp/03_bolzano}
    \captionof{figure}{Zum Nullstellensatz von Bolzano.}
    \label{fig:03_bolzano}
  \end{center}
  Wie immer bei solchen Aussagen, die derart eng mit unserem intuitiven 
  Verständnis von "`Kontinuität"' zusammenhängen, muss man zum 
  Beweis auf die axiomatische Beschreibung 
  (in diesem Fall die Supremumseigenschaft)
  der reellen Zahlen zurückgreifen. 
  Dazu nehme man {\oBdA} 
  $f(a)<0$ und $f(b)>0$ an und betrachte die Menge 
  \[
  A:=\big\{ x\in[a,b]\sets \quad f(x) \le 0 \big\}.
  \]
  Diese ist nicht leer und nach oben beschränkt, 
  besitzt somit ein Supremum $\xi$. 
  Wir wollen zeigen, dass $\xi$ eine Nullstelle von $f$ ist. 
  Dazu wähle man eine Folge $(x_n)\subset A$, 
  die gegen $\xi$ konvergiert. Wegen der Stetigkeit von $f$ folgt 
  $\lim f(x_n)=f(\xi)$. 
  Für alle $n\in\NN$ ist $f(x_n)\le 0$, und daher ist auch $f(\xi)\le 0$ 
  (vgl. Frage \ref{02_gmon}). 
  Es bleibt also nur noch zu zeigen, dass $f(\xi)$ nicht $<0$ sein kann. 

  Wäre $f(\xi)<0$, etwa $f(\xi)=-\eps$ mit $\eps>0$, dann gäbe es wegen 
  der Stetigkeit von $f$ in $\xi$ ein $\delta>0$, sodass 
  $|f(\xi+\delta)-f(\xi)|<\eps/2$ ist. $f(\xi+\delta)$ wäre dann 
  ebenfalls negativ, woraus $(\xi+\delta) \in A$ folgen würde, 
  im Widerspruch zu $\xi=\sup A$.
  \AntEnd
\end{antwort}

%% Question 36
\begin{frage}\label{03_zwi}\index{Zwischenwertsatz}
  Was besagt der \bold{Zwischenwertsatz} für stetige Funktionen? 
  Welcher Zusammenhang besteht mit dem Nullstellensatz?
\end{frage}

\begin{antwort}
  Der Satz lautet: 

  \medskip\noindent
  \slanted{Ist die Funktion $f\fd X \to \RR$ mit $X\subset \RR$ 
    stetig auf dem Intervall $[a,b]\subset X$, so nimmt $f$ 
    jeden Wert zwischen $f(a)$ und $f(b)$ an.} 

  \begin{center}
    \includegraphics{mp/03_zwi}
    \captionof{figure}{Eine stetige reellwertige Funktion auf dem 
      Intervall $[a,b]$ nimmt jeden Wert $\eta$ zwischen $f(a)$ und $f(b)$ an.}
    \label{fig:03_zwi}
  \end{center}
  Im Fall $f(a)=f(b)$ ist nichts zu beweisen. Sei daher $f(a)<f(b)$ und  
  $\eta$ ein beliebiger Punkt aus $\open{ f(a), f(b) }$ (\sieheAbbildung\ref{fig:03_zwi}). Dann ist $f(a)-\eta < 0$ und 
  $f(b)-\eta > 0$. Nach dem Nullstellensatz hat die Funktion $f(x)-\eta$ eine 
  Nullstelle $\xi$ in $[a,b]$. Für diese gilt $f(\xi)=\eta$, der Wert $\eta$ 
  wird also tatsächlich mindestens einmal angenommen.
  \AntEnd
\end{antwort}

%% Question 37
\begin{frage}\label{03_zwia}
  Warum ist der Zwischenwertsatz äquivalent zur Aussage: "`Das Bild eines 
  Intervalls unter einer stetigen reellwertigen Funktion ist wieder ein 
  Intervall?"'
\end{frage}

\begin{antwort}
  In den Fällen, in denen der Bild- bzw. Definitionsbereich aus nur einem 
  einzigen Punkt besteht, ist nichts zu zeigen. Im Folgenden sei daher stets 
  von \slanted{echten} Intervallen die Rede.   

  Sei $f\fd X\to \RR$ mit $X\subset \RR$ eine stetige Funktion und 
  $[a,b]\subset X$ ein beliebiges kompaktes Intervall. Ist dann 
  $M:=f([a,b])$ ebenfalls ein Intervall, so enthält $M$ 
  mit $f(a)$ und $f(b)$ auch jeden Punkt dazwischen. Da dies für alle 
  Punkte $a,b\in X$ mit $a<b$ gilt, hat $f$ die Zwischenwerteigenschaft. 

  Die Funktion $f$ besitze nun umgekehrt die Zwischenwerteigenschaft und 
  $I\subset X$ sei ein Intervall. Angenommen, die Bildmenge $M=f(I)$ ist  
  kein Intervall, dann gibt es Punkte $x,y\in I$ 
  mit $f(x)< f(y)$ und ein 
  $\eta \in\RR$ zwischen $f(x)$ und $f(y)$, das nicht in der 
  Bildmenge $M$ enthalten ist. 
  Das steht im Widerspruch zu der Annahme, dass die 
  Funktion $f$ die Zwischenwerteigenschaft besitzt.
  \AntEnd  
\end{antwort} 

%% Question 38
\begin{frage}\label{03_fix}\index{Fixpunktsatz}
  Warum hat jede stetige Abbildung $f\fd [a,b]\to \RR$ mit 
  $f([a,b])\subset [a,b]$ min\-des\-tens einen 
  Fixpunkt ({\dasheisst} einen 
  Punkt $\xi \in [a,b]$ mit $f(\xi)=\xi$)? 
\end{frage}

\begin{antwort}
  Diesen einfachen Fixpunktsatz erhält man, wenn man 
  den Nullstellensatz auf die Funktion 
  $\varphi(x):=f(x)-x$ anwendet.
  Es ist $\varphi(b)\le 0$ und $\varphi(a)\ge 0$, und damit gibt es laut 
  Nullstellensatz ein $\xi$ aus $[a,b]$ mit $\varphi(\xi)=0$. Daraus folgt 
  $f(\xi)=\xi$, und die Zahl $\xi$ 
  ist dann bereits der gesuchte Fixpunkt von $f$. 
  \AntEnd  
\end{antwort}


%% Question 39
\begin{frage}\index{folgenkompakt}
  Wann heißt ein metrischer Raum \bold{folgenkompakt}?
\end{frage}

\begin{antwort}
  Ein metrischer Raum $(X,d)$ heißt \slanted{folgenkompakt}, wenn jede 
  in $X$ verlaufende Folge $(x_n)$ eine Teilfolge besitzt, die gegen ein 
  Element $x\in X$ konvergiert.
  \AntEnd
\end{antwort}

%% Question 40
\begin{frage}\label{03_komp}
  Ist $(X,d_X)$ ein folgenkompakter metrischer Raum 
  und ist $f\fd X\to Y$ eine surjektive stetige Abbildung, 
  dann ist auch $Y$ folgenkompakt. Können Sie diese Aussage begründen?
\end{frage}

\begin{antwort}
  Sei $(y_n) \subset Y$ eine beliebige Folge aus $Y$. Es muss gezeigt werden, 
  dass es unter den gegebenen Voraussetzungen eine Teilfolge von $(y_n)$ gibt, 
  die gegen einen Grenzwert aus $Y$ konvergiert.
  
  Da die Abbildung $f\fd X\to Y$ surjektiv ist, 
  existiert zu jedem Glied $y_n \in Y$ ein Urbild $x_n\in X$ (im Allgemeinen  
  existieren mehrere, in welchem Fall man ein bestimmtes auswählen kann). 
  Wegen der Folgenkompaktheit von $X$ gibt es eine Teilfolge 
  $(x_{n_k})$ von $(x_n)$, die gegen ein $x\in X$ konvergiert. 
  Die Folge $\big(f(x_{n_k})\big)$ ist dann nach Konstruktion eine Teilfolge von 
  $(y_n)$, die aufgrund der Stetigkeit von $f$ gegen den Wert $f(x)\in Y$ 
  konvergiert. Damit ist die Behauptung bewiesen. \AntEnd
\end{antwort}

%% Question 41
\begin{frage}\label{mima}\index{Satz!vom Maximum und Minimum}
  (\bold{Existenz von Minimum und Maximum})
  Ist $X$ ein folgenkompakter metrischer Raum und $f\fd X\to\RR$ stetig. 
  Warum gibt es dann Punkte $x_{\ast}\in X$ und $x^{\ast} \in X$ mit 
  $f(x_{\ast}) \le f(x) \le f(x^{\ast})$ für alle $x\in X$?
\end{frage}

\begin{antwort}
  Aus der Antwort zu Frage~\ref{03_komp} folgt, dass das Bild 
  $f(X) \subset \RR$ folgenkompakt ist. 
  Eine folgenkompakte Teilmenge von $\RR$ muss aber 
  beschränkt sein (denn andernfalls ließe sich leicht eine Folge bestimmen, 
  die keine konvergente Teilfolge besitzt). 
  $f(X)$ besitzt somit ein Supremum $s$ und ein Infimum $t$. Wiederum aufgrund 
  der Folgenkompaktheit von $f(X)$ muss $s\in f(X)$ und $t\in f(X)$ gelten, da 
  sonst die Folgen $\left(s-\frac1n\right)$ bzw. $\left(t+\frac1n\right)$ keine gegen ein Element aus $f(X)$ 
  konvergente Teilfolge besitzen würden.

  Das Bild von $X$ unter $f$ besitzt somit ein Maximum 
  $s$ und ein Minimum $t$. Die Urbilder $x^{\ast}\in X$ von $s$ und 
  $x_{\ast}\in X$ von $t$ besitzen dann die gesuchte Eigenschaft.
  \AntEnd
\end{antwort}

%% Question 42
\begin{frage}\label{03_umke}\index{Umkehrfunktion}
  Ist $D\subset \RR$ ein \bold{echtes Intervall} und $f\fd D\to\RR$ 
  streng monoton wachsend. Warum existiert dann die Umkehrfunktion 
  $g\fd f(D) \to D\subset \RR$ und warum ist diese streng monoton 
  wachsend und automatisch stetig?
  (Beachten Sie, dass nicht vorausgesetzt wurde, dass $f$ selbst stetig ist.)
\end{frage}

\begin{antwort}
  Wegen der strengen Monotonie der Funktion $f$ ist diese injektiv und 
  bildet $D$ somit bijektiv auf $f(D)$ ab. 
  Daraus folgt mit Frage \ref{03_umk} die Existenz 
  der Umkehrabbildung $g\fd f(D) \to D$. 
  Aus dem streng monotonen Wachstum von $f$ 
  ($a<b \Ra f(a) <f(b)$) und $g\cdot f=\mathrm{id}$ folgt zusammen     
  \[
  f(a) < f(b) \Ra  g\big(f(a)\big) < g\big(f(b)\big).
  \]
  Also ist $g$ streng monoton wachsend.

  Schließlich muss noch die Stetigkeit von $g$ gezeigt werden. 
  Sei dazu $\xi$ zunächst ein innerer Punkt aus $D$ und sei $\eta:=f(\xi)$. 
  Für alle hinreichend kleinen $\eps >0$ liegen 
  dann die Punkte $\xi-\eps$ und $\xi+\eps$ auch noch in $D$, 
  und wegen des streng 
  monotonen Wachstums von $f$ gilt $f(\xi-\eps)<\eta<f(\xi+\eps)$. 
  Also gibt es ein $\delta>0$ mit 
  \[
  f(\xi-\eps) < \eta -\delta < \eta +\delta < f(\xi+\eps).
  \]
  Für alle $y\in \open{ \eta-\delta, \eta+\delta }$ liegt dann wegen 
  des streng monotonen Wachstums von $g$ der Wert $g(y)$ innerhalb des 
  Intervalls $\open{\xi-\eps, \xi+\eps}$. Das heißt, 
  die Funktion $g$ ist stetig an der Stelle $\eta=f(\xi)$.

  Ist $\xi$ ein linker Randpunkt von $D$, dann betrachte man 
  zu hinreichend kleinem $\eps>0$ das in $D$ liegende Intervall 
  $\ropen{\xi, \xi+\eps}$. Wie oben folgt daraus die Existenz 
  eines $\delta>0$ mit 
  \[
  f(\xi)=\eta < \eta +\delta \le f(\xi+\eps).
  \]
  An dieser Stelle ist jetzt nur noch zu bemerken, dass wegen des streng 
  monotonen Wachstums von $f$ der Punkt $\eta=f(\xi)$ gleich dem Minimum 
  von $f(D)$ ist. Aus $y\in f(D)$ und $|y-\eta|<\delta$ 
  folgt also bereits $y\in \ropen{\eta,\eta+\delta}$, und somit
  gilt $g(y) \in \ropen{\xi,\xi+\eps}$, also 
  $|g(y)-g(\eta)|\le \eps$. Das entspricht der 
  $\eps\delta$-Definition der Stetigkeit von $g$ im Punkt $\eta$. 
  Für einen rechten Randpunkt $\xi$ von $D$ ist der Nachweis der Stetigkeit 
  von $g$ im Punkt $f(\xi)$ nach genau demselben Muster zu führen.
  \AntEnd
\end{antwort}

%% Question 43
\begin{frage}\label{03_unsb}
  Kennen Sie ein einfaches Beispiel einer Funktion $f\fd [0,1]\to \RR$, 
  die jeden Wert zwischen $f(0)$ und $f(1)$ annimmt, die aber nur an einer  
  Stelle $a\in [0,1]$ stetig ist?
\end{frage}

\begin{antwort}
  Ein Beispiel wäre die Funktion $f\fd [0,1] \to \RR$ mit
  \[
  f \mapsto \left\{ 
    \begin{array}{ll}
      x   & \text{für $x\in [0,1] \cap \QQ$},\\
      1-x & \text{für $x\in [0,1] \cap\RR \mengeminus \QQ$.}
    \end{array}
  \right.
  \]
  Die Funktion nimmt jeden Wert zwischen $f(0)=0$ und $f(1)=1$ an, 
  \sieheAbbildung\ref{fig:03_unstetig}. 
  Da $\RR$ dicht in $\QQ$ und $\QQ$ dicht in $\RR$ ist, 
  gibt es zu jeder $\delta$-Umgebung eines Punktes $x\in [0,1]$ 
  einen Punkt $y$ mit $|f(x)-f(y)| \ge | x-(1-x) | = |2x-1|$. 
  Für $x\in [0,1]$ und $x\not=\frac{1}{2}$ ist dieser Wert größer 
  als ein positives $\eps$. Die Funktion kann also nicht stetig sein.
  \AntEnd

  \begin{center}
    \includegraphics{mp/03_unstetig}
    \captionof{figure}{Die Funktion $f$ nimmt jeden Wert in $[0,1]$, 
      ist aber nur in einem Punkt stetig.}
    \label{fig:03_unstetig}
  \end{center}

\end{antwort}

%% Question 44
\begin{frage}\label{03_injm}
  Warum ist eine injektive stetige Funktion auf einem echten Intervall 
  stets streng monoton? 
  Warum wird die Aussage falsch, wenn $f$ nicht stetig oder der 
  Definitionsbereich von $f$ kein Intervall ist?
\end{frage}

\begin{antwort}
  Seien $a,b$ mit $b>a$ zwei beliebige Punkte des Definitionsintervalls. 
  Wegen der Injektivität von $f$ ist $f(a)\not=f(b)$, 
  etwa $f(a)<f(b)$. Für einen weiteren Punkt $c\in D(f)$ mit $a<c<b$ 
  folgt dann $f(a)<f(c)<f(b)$, denn aus $f(c)<f(a)$ würde mit dem 
  Zwischenwertsatz folgen, 
  dass der Wert $f(a)$ an einer Stelle $\xi\in \open{c,b}$ nochmals 
  angenommen werden würde, sodass also 
  $f(a)=f(\xi)$ und $a\not=\xi$ gelten würde, im 
  Widerspruch zur Injektivität der Funktion. Ebenso würde aus der Annahme 
  $f(c)>f(b)$ folgen, dass der Wert $f(b)$ schon im Intervall $\open{a,c}$ 
  einmal angenommen wird, ebenfalls im Widerspruch zur Injektivität. 
  Hieraus folgt das streng monotone Wachstum von $f$. 

  Dass derselbe Zusammenhang für eine nichtstetige Funktion nicht 
  zu gelten braucht, zeigt das Beispiel der Funktion 
  \[
  f\fd [0,1] \to [0,1], \qquad
  f\mapsto \left\{ \begin{array}{ll}
      x & \text{für $x < \frac{1}{2}$}\\
      -x+1 &\text{für $x \ge \frac{1}{2}$.}
    \end{array}\right.
  \] 
  Diese im Punkt $\frac{1}{2}$ unstetige Funktion ist injektiv auf $[0,1]$, 
  aber nicht streng monoton. 


  Um die Notwendigkeit der Voraussetzung, dass $D$ ein echtes 
  Intervall ist, einzusehen, mache man sich klar, dass eine 
  Funktion von der Art, 
  deren Graph in \Abb\ref{fig:03_stetig_int} gezeigt wird, die 
  Definition der Stetigkeit erfüllt.  

  \begin{center}
    \includegraphics{mp/03_stetig_int}
    \captionof{figure}{Die Funktion $f$ ist stetig!}
    \label{fig:03_stetig_int}
  \end{center}

  Mit diesem Bild vor Augen lässt sich leicht eine stetige injektive Funktion 
  angeben, die nicht streng monoton ist, etwa
  \[
  f\fd [0,1]\cup[2,3] \to \RR, \qquad
  f\mapsto \left\{ \begin{array}{ll}
      x  & \text{für $x\in[0,1]$},\\
      -x+5 & \text{für $x\in[2,3]$.}
    \end{array}\right.
  \EndTag
  \] 
\end{antwort}

%% Question 45
\begin{frage}\label{03_stum}\index{Stetigkeit!der Umkehrfunktion}
  Sind $(X,d_X)$ und $(Y,d_Y)$ metrische Räume, $X$ folgenkompakt und 
  $f\fd X\to Y$ stetig und bijektiv. Warum ist dann die Umkehrabbildung 
  $g\fd Y\to X$ stetig?

  $X$ und $Y$ sind in diesem Fall also \bold{topologisch äquivalent} 
  bzw. \bold{homöomorph}.\index{homöomorph}
\end{frage}

\begin{antwort}
  Sei $y\in Y$ und $(y_n)\subset Y$ eine gegen $y$ konvergierende Folge. 
  Es muss gezeigt werden, dass die Folge $(x_n)$ mit $x_n:=g(y_n)$ gegen 
  $x:=g(y)$ konvergiert. 

  Angenommen, das wäre nicht der Fall. Dann liegen für ein hinreichend 
  kleines $\eps>0$ unendlich viele Glieder 
  von $(x_n)$ außerhalb von $U_\eps( x )$, {\dasheisst}, $(x_n)$ besitzt 
  eine Teilfolge, die außerhalb von $U_\eps(x)$ verläuft. Diese Teilfolge 
  wiederum besitzt wegen der Folgenkompaktheit von $X$ eine konvergente 
  Teilfolge $(x_{n_k})$, es gibt also ein Element $\widetilde{x}\in X$ mit
  \[
  \lim_{k\to\infty} x_{n_k} = \widetilde{x}.
  \]
  Andererseits gilt
  \[
  \lim_{k\to\infty} f(x_{n_k}) = \lim_{k\to\infty} y_{n_k} = y = f(x). 
  \]
  Wegen $\widetilde{x}\not\in U_\eps(x)$, also 
  $\widetilde{x}\not=x$ widerspricht das der Stetigkeit von $f$. Es gilt also 
  \[
  \lim\limits_{n\to\infty} y_n = y
  \Ra \lim_{n\to\infty} g(y_n)= g(y).
  \]
  Damit ist  gezeigt, dass die Umkehrfunktion $g$ stetig ist.   
  \AntEnd   
\end{antwort}

%% Question 46
\begin{frage}\label{03_kbld}\index{Intervall!kompaktes}
  Sei $[a,b]\in\RR$ ein kompaktes Intervall und 
  $f\fd [a,b]\to\RR$ stetig. Warum ist auch das Bild 
  $f([a,b])$ ein kompaktes Intervall?  
\end{frage}

\begin{antwort}
  Sei $(y_n)$ eine Folge in $f([a,b])$. Für den Nachweis 
  der Kompaktheit von $f([a,b])$ muss gezeigt werden, dass sie eine 
  Teilfolge besitzt, die gegen einen Wert aus $f([a,b])$ 
  konvergiert. Dazu wähle man zu jedem $n\in\NN$ 
  ein $x_n\in [a,b]$ mit $f(x_n)=y_n$. Wegen der Kompaktheit 
  von $[a,b]$ gibt es eine Teilfolge $(x_{n_k})$ von $(x_n)$, 
  die gegen ein Element $x \in [a,b]$ konvergiert. 
  Wegen der Stetigkeit von $f$ folgt daraus 
  \[
  \lim_{k\to\infty} f(x_{n_k})= \lim_{k\to\infty} y_{n_k}=f(x) \in f([a,b]).
  \]
  Die Folge $(y_{n_k})$ besitzt also einen 
  Grenzwert in $f([a,b])$, somit ist $f([a,b])$ kompakt. 
  \AntEnd 
\end{antwort}

%% Question 47
\begin{frage} \label{03_ityp}
  Kann sich bei einer stetigen Abbildung $f\fd M\to\RR$ der Intervalltyp 
  ändern?
\end{frage}

\begin{antwort}
  In der Antwort zur vorigen Frage wurde bereits gezeigt, dass 
  das Bild kompakter Intervalle unter stetigen Abbildungen immer kompakt 
  ist. 

  Für offene Mengen $\open{a,b}$ gibt es keinen analogen Zusammenhang. 
  Diese können auf offene, kompakte oder halboffene Intervalle 
  abgebildet werden. Den ersten Fall zeigt das Beispiel  
  der identischen Funktion $\id\fd \open{a,b} \to \RR,\; x\mapsto x$, den 
  zweiten illustriert zum Beispiel die Sinusfunktion, die das 
  offene Intervall $\open{-10,10}$ auf das kompakte Intervall $[-1,1]$  
  und das offene Intervall $\open{0,\frac{3\pi}{4}}$ auf das halboffene 
  Intervall $\lopen{0,1}$ abbildet.

  Es bleibt noch zu klären, was mit \slanted{halboffenen} Intervallen unter 
  stetigen Abbildungen passiert. Diese können halboffen oder kompakt sein, 
  wie die Beispiele $f\fd \lopen{a,b}\to\RR,\; x\mapsto x$ und 
  $f\fd \ropen{0,\frac{3\pi}{4} } \to\RR, \; x\mapsto \sin x $ zeigen. 
  Die Bilder von halboffenen Intervallen unter stetigen Abbildungen 
  können allerdings nicht offen sein. 
  \AntEnd 
\end{antwort}

%% Question 48
\begin{frage}\label{03_sqrt}\index{Wurzelfunktion}
  Können Sie mit einem $\eps\delta$-Beweis nachweisen, dass die 
  Quadratwurzelfunktion $\sqrt{\;\,} \fd \RR_+ \to \RR, \; 
  x\mapsto \sqrt{x}$ stetig ist?
\end{frage}

\begin{antwort}
  Für $x\in \RR_+$ gilt für beliebiges $a>0$ die Abschätzung
  \begin{align*}
    \big| \sqrt{x}-\sqrt{x+a} \big|^2 &= ( \sqrt{x}-\sqrt{x+a} )^2 \\
    &=x-2\sqrt{(x)(x+a)}+x+a < x-2\sqrt{x^2} + x+ a = a.
  \end{align*}
  Hieraus folgt für $x,y \in \RR_+$, $y>x$ und $\eps>0$
  \[
  |x-y| \le \eps^2  \Ra
  | \sqrt{x}-\sqrt{x+\eps^2} | < \eps.
  \]
  Da die Rollen von $x$ und $y$ in den Betragsstrichen vertauscht 
  werden dürfen, gilt dasselbe Ergebnis auch für $y<x$. Mit 
  $\delta:=\eps^2$ entspricht das der 
  $\eps\delta$-Charakterisierung der Stetigkeit.

  Die Stetigkeit der Wurzelfunktion lässt sich mit Frage \ref{03_umke}
  auch zeigen, indem 
  man sie als Umkehrfunktion von $f\,:\, \RR_+ \to \RR_+, \; 
  x\mapsto x^2$ betrachtet. Die Funktion $f$ ist stetig, streng monoton 
  wachsend und auf einem echten Intervall definiert. Nach Frage 236 ist die 
  Umkehrfunktion ebenfalls stetig.
  \AntEnd
\end{antwort}

%% Question 49
\begin{frage}\label{03_ponu}
  Warum hat jedes Polynom ungeraden Grades 
  \[
  p\fd \RR \to \RR,\qquad x\mapsto a_0+a_1x+\cdots + a_{2n+1} x^{2n+1}
  \]
  mindestens eine reelle Nullstelle?
\end{frage}

\begin{antwort}
  Wir wissen bereits aus Frage \ref{03_pols}, dass jedes reelle 
  Polynom auf ganz $\RR$ stetig ist. Die Frage kann damit durch eine 
  Anwendung des Zwischenwertsatzes beantwortet werden, wenn man zeigen 
  kann, dass ein ungerades Polynom auf $\RR$ mindestens je einen negativen 
  und einen positiven Wert annimmt. Das erkennt man, wenn man $p$ in der 
  Form 
  \[ p(x)=x^{2n+1} \left( a_{2n+1}+
    \frac{a_{2n}}{x}+\cdots+\frac{a_1}{x^{2n}} + \frac{a_0}{x^{2n+1}} 
  \right)
  \]
  schreibt. Der rechte Faktor konvergiert für $x \to \pm \infty $ gegen 
  $a_{2n+1}$. Es gibt also eine Schranke $C$, sodass der rechte 
  Faktor für $x>C$ oder $x<-C$ insgesamt dasselbe Vorzeichen wie $a_{2n+1}$ 
  besitzt. Ferner ist $x^{2n+1}=x\cdot (x^n)^2$ eine ungerade Funktion. 
  Zusammen folgt, dass der Wert von $p$ an der Stelle $x$ 
  dasselbe Vorzeichen hat wie $a_{2n+1}$, falls $x>C$ ist, 
  und verschiedenes Vorzeichen, falls $x<-C$ ist. Das Polynom wechselt 
  also in seinem Definitionsbereich mindestens einmal das Vorzeichen, was 
  zu zeigen war. 
  \AntEnd
\end{antwort}


%% Question 50
\begin{frage}\label{03_anw}
  Sei $K$ eine (folgen-)kompakte Teilmenge von $\KK$ ($\KK=\RR$ oder 
  $\KK=\CC$). Warum gibt es dann zu jedem Punkt $p\not\in K$ einen Punkt 
  $k\in K$, sodass für \bold{jeden} Punkt $z\in K$ gilt 
  \[ |k-p| \le |z-p|. \]
\end{frage}

\begin{antwort}
  Die Funktion 
  $f\fd K \to \RR, \; z \mapsto |z-p|$ 
  ist eine stetige Funktion auf einer kompakten Menge und besitzt somit 
  ein Minimum $k\in K$. Dieses Minimum ist das
  Element mit den gesuchten Eigenschaften (\sieheAbbildung\ref{fig:03_mind}).

  \begin{center}
    \includegraphics{mp/03_mind}
    \captionof{figure}{Der Punkt $k$ liegt von allen Punkten aus $K$ am nächsten 
      bei $p$.}
    \label{fig:03_mind}
  \end{center}
\end{antwort}

%% Question 51
\begin{frage}\label{03_stbj}
  Jedes kompakte (echte) Intervall $K\subset \RR$ lässt sich 
  \bold{bijektiv} auf jedes offene (echte) Intervall von $\RR$ abbilden. 
  Gibt es auch eine bijektive Abbildung, die zusätzlich stetig ist?
\end{frage}

\begin{antwort}
  Es kann keine stetige bijektive Abbildung 
  zwischen einem kompakten und einem offenen Intervall 
  geben, da nach Frage \ref{03_kbld} 
  stetige Abbildungen kompakte Mengen stets auf 
  kompakte Mengen abbilden.  \AntEnd
\end{antwort} 

\section{Grenzwerte bei Funktionen}

Wir legen auch hier zunächst die allgemeine Situation zugrunde, dass 
$(X,d_X)$ und $(Y,d_Y)$ metrische Räume sind, $D\subset X$ eine nichtleere 
Teilmenge und $f\fd D\to Y$ eine Abbildung. 
Ferner sei $a\in X$ ein Punkt, der zu $M$ gehören kann, aber nicht gehören 
muss, der aber ein \slanted{Häufungspunkt} von $M$ sein soll, {\dasheisst},   
für \slanted{jede} punktierte $\eps$-Umgebung 
$\dot{U}_\eps (a) := U_\eps\mengeminus\{a\}$ gilt 
$\dot{U}_\eps(a) \cap M \not= \emptyset$. 
Wichtige Spezialfälle erhält man mit $X=\RR$ und $Y=\RR$ oder $Y=\CC$ 
(jeweils mit der natürlichen Metrik)

%% Question 52
\begin{frage}\label{03_fkon}\index{stetige Fortsetzung}
  \index{Grenzwert!für Funktionen}
  Was besagt in der vorausgesetzten Ausgangssituation 
  die Ausdrucksweise: "`$f$ hat bei Annäherung an 
  $a$ den Grenzwert $l$"' (oder: "`$f$ konvergiert bei 
  Annäherung an $a$ gegen $l$"')? 
\end{frage}

\begin{antwort}
  Die Formulierung bedeutet, dass die Funktion 
  $F \fd D \cup \{ a \} \to Y$, die definiert ist durch 
  \begin{equation}
    F \fd 
    x \mapsto \left\{ \begin{array}{ll} 
        f(x) & \text{für $x\in D$, $x\not=a$} \\
        l  & \text{für $x=a$}
        j\end{array}\right.
    \tag{$\ast$}
  \end{equation}
  die Eigenschaft besitzt, im Punkt $a$ stetig zu sein (\sieheAbbildung\ref{fig:03_fortsetzung}). 
  Die Funktion $F$ nennt man dann eine \slanted{stetige Fortsetzung von $F$}.

  Um auszudrücken, dass $f$ bei Annäherung an $a$ gegen 
  $l$ konvergiert, schreibt man
  \begin{equation}
    \lim_{x\to a} f(x) = l,\qquad\text{oder auch}\qquad 
    f(x) \to l\quad\text{für}\quad x\to a. \EndTag
  \end{equation}

  \begin{center}
    \includegraphics{mp/03_fortsetzung}
    \captionof{figure}{Die Funktion $f$ konvergiert bei 
      Annäherung an $a$ gegen $l$.}
    \label{fig:03_fortsetzung}
  \end{center}
\end{antwort} 

%% Question 53
\begin{frage}
  Warum ist der Grenzwert $l$ im Fall der Existenz eindeutig bestimmt?
\end{frage}

\begin{antwort}
  Ist $F^*$ eine weitere 
  stetige Fortsetzung von $f$ in $a$, dann stimmen $F^*$ und $F$ auf 
  $D$ überein. Für jede in $D$ verlaufende Folge $(x_n)$ mit 
  $x_n\to a$ gilt dann $F^*(x_n)=F(x_n)$ und wegen der 
  Stetigkeit von $F^*$ somit 
  \[
  F^*(a)=\lim_{n\to\infty} F^*(x_n)=F(x_n)=l
  . \EndTag\] 
\end{antwort}

%% Question 54
\begin{frage}\label{03_epsd}\index{eps@$\eps\delta$-Charakterisierung des Grenzwerts}
  Wie lautet die $\eps\delta$-Charakterisierung des Grenzwerts?
\end{frage}

\begin{antwort}
  Die Charakterisierung lautet genauso wie bei der Stetigkeit: 

  \medskip\noindent%
  \slanted{Für eine Funktion $f\fd D\to Y$ gilt 
    $f(x)\to l$ für $x\to a$ genau dann, wenn zu jedem $\eps>0$ 
    ein $\delta>0$ existiert, sodass 
    \[
    f(x)\in U_\eps(l) \quad\text{ für alle $x\in U_\delta(a)\mengeminus\{a\}\cap D$} 
    \]
    gilt.}

  \medskip\noindent%
  Dieses Kriterium folgt unmittelbar, wenn man die Stetigkeit 
  der fortgesetzten Funktion $F$ mit $F(a)=l$ im Punkt $a$ mithilfe 
  der $\eps\delta$-Definition beschreibt. Andersherum erhält man 
  aus dem Kriterium auch die Stetigkeit der in den 
  Punkt $a$ hinein fortgesetzten Funktion $F$ mit $F(a)=l$. 
  Die $\eps\delta$-Charakterisierung ist also äquivalent zu der in Frage 
  \ref{03_fkon} gegebenen Definition.
  \AntEnd
\end{antwort}


%% Question 55
\begin{frage}\label{03_aind}
  Was bedeutet im Fall $a\in D$ die Aussage 
  $\lim\limits_{x\to a} f(x) = f(a)$?
\end{frage}

\begin{antwort}
  Für $a\in D$ ist die Aussage gleichbedeutend mit der Stetigkeit 
  von $f$ in $a$. Die "`fortgesetzte"' Funktion 
  $F$ der Definition ($\ast$) aus der vorigen Frage ist unter dieser 
  Voraussetzung nämlich identisch mit 
  der Ausgangsfunktion $f$.    
  \AntEnd
\end{antwort}

%% Question 56
\begin{frage}\label{03_perf}
  Welche \bold{Permanenzeigenschaften} des Grenzwertbegriffs 
  kennen Sie im Fall $Y=\RR$ oder $Y=\CC$?
\end{frage}

\begin{antwort}
  Gelten für zwei Funktionen $f\fd D\to Y$ und $g\fd D\to Y$ 
  in einem Häufungspunkt $a$ von $D$ die Beziehungen 
  $\lim\limits_{x\to a} f(x)= l$ und 
  $\lim\limits_{x\to a} g(x)=s$ mit $l,s\in Y$, so gilt auch
  \[
  \lim_{x\to a} f(x)+g(x)=l+s, \qquad 
  \lim_{x\to a} f(x)g(x)=l\cdot s \qquad 
  \lim_{x\to a}\frac{ f(x) }{ g(x)}=\frac{l}{s},\quad\text{für $s\not=0$}.
  \]   
  Diese Permanenzeigenschaften sind eine unmittelbare Folge derjenigen 
  für stetige Funktionen (s. Frage \ref{03_perm}). Sind nämlich $F$ 
  und $G$ die stetigen Fortsetzungen von $f$ und $g$ im Punkt $a$, so folgt 
  hieraus, dass $F+G$, $F\cdot G$ und $F/G$ ebenfalls stetig in $a$ sind. 
  Die Werte dieser Funktionen in $a$ sind aber gerade $s+t$ bzw. $s\cdot t$ 
  bzw. $s/t$.
  \AntEnd 
\end{antwort}


%% Question 57
\begin{frage}\label{03_flgk}\index{Folgenkriterium}
  Wie lautet das \bold{Folgenkriterium} für die Existenz des Grenzwerts?
\end{frage}

\begin{antwort}
  Das Folgenkriterium besagt: 

  \medskip
  \noindent\satz{Für eine Funktion $f\fd D\to Y$ 
    und einen Häufungswert $a$ von $D$ existiert der Grenzwert 
    $\lim\limits_{x\to a} f(x)$ genau dann, wenn 
    für jede Folge $(x_n)\subset D \mengeminus \{ a \}$ mit 
    $\lim\limits_{n\to\infty} x_n = a$ die Folge $f(x_n)$ in $Y$ konvergiert.}  

  \medskip\noindent
  Ist die Funktion $F$ die stetige Fortsetzung 
  von $f$ in den Punkt $a$, so folgt der Zusammenhang sofort aus  
  \[
  \text{
    Folgenkriterium $\LLa$ 
    $F$ stetig in $a$ 
    $\LLa$
    $\lim_{x\to a} f(x)$ existiert.} \EndTag
  \]
\end{antwort}

%% Question 58
\begin{frage}
  Wie folgt im Fall der Existenz 
  die Eindeutigkeit des Grenzwerts aus dem Folgenkriterium?
\end{frage}

\begin{antwort}
  Seien $(a_n)$ und $(b_n)$ Folgen 
  in $D\mengeminus\{ a\}$ mit $\lim a_n = \lim b_n$. Angenommen, es gilt 
  $\lim f(a_n) \not= \lim f(b_n)$, dann konvergiert
  die nach dem "`Reißverschlussprinzip"' 
  gebildete Folge $a_0,b_0,a_1,b_1,\ldots$ aus $D\mengeminus\{ a \}$ ebenfalls 
  gegen $a$, die Folge der Funktionswerte 
  ist aber wegen $\lim f(a_n) \not= \lim f(b_n)$ divergent. Die Funktion 
  $f$ erfüllt in diesem Fall also im Widerspruch zur Voraussetzung 
  nicht das Folgenkriterium. Hieraus ergibt sich 
  -- im Fall der Existenz -- die Eindeutigkeit des Grenzwerts.
  \AntEnd
\end{antwort} 

%% Question 59
\begin{frage}\index{Cauchy-Kriterium!für stetige Fortsetzbarkeit}
  Wie lautet das Cauchy-Kriterium für die Existenz des Grenzwerts?
\end{frage}

\begin{antwort}
  Sei $a$ ein Häufungspunkt des Definitionsbereichs einer Funktion 
  $f\fd D\to Y$ mit $Y=\CC$ oder $Y=\RR$. 
  Das Cauchy-Kriterium lautet: 

  \medskip
  \noindent\satz{Der Grenzwert 
    $\lim\limits_{x\to a} f(x)$ existiert genau dann, 
    wenn es zu jedem $\eps>0$ ein 
    $\delta>0$ gibt, sodass gilt
    \begin{equation}
      x,y \in U_\delta (a) \mengeminus\{a\}\cap D \Ra 
      d_Y ( f(x),f(y) )< \eps. \EndTag
    \end{equation}}
\end{antwort}

%% Question 60
\begin{frage}
  Warum gilt für $x,a\in\RR$ und $k\in \NN$: $
  \dis\lim_{x\to a} \frac{x^k-a^k}{x-a} = ka^{k-1}$?
\end{frage}  

\begin{antwort}
  Für $x\not=a$ gilt
  \[
  \frac{x^k-a^k}{x-a} = x^{k-1}+ax^{k-2} + \ldots + a^{k-2}x+a^{k-1} :=p(x).
  \]
  Das Polynom auf der rechten Seite der Gleichung ist eine auf ganz $\RR$ 
  stetige Funktion, und es gilt $\lim\limits_{x\to a} p(x)=p(a)=ka^{k-1}$.
  \AntEnd
\end{antwort}

%% Question 61
\begin{frage}\index{links-, rechtsseitiger Grenzwert}
  \index{links-, rechtsseitig stetig}
  Ist $D \subset \RR$ und $f\fd  D\to \CC$ eine Funktion, wie sind dann in 
  einem Häufungspunkt $x_0 \in D_- :=  D\cap \open{-\infty, x_0}$ bzw. 
  $x_0 \in D_+ := D\cap \open{x_0,\infty}$ der 
  \bold{linksseitige bzw. rechtsseitige Grenzwert} in $x_0$ erklärt?
\end{frage}

\begin{antwort}
  Man sagt, die Funktion $f$ habe in $x_0$ den linksseitigen 
  bzw. rechtsseitigen Grenzwert $l$, wenn die Einschränkung von 
  $f$ auf $D_-$ bzw. $D_+$ bei Annäherung an $a$ gegen $l$ konvergiert. Man 
  schreibt in diesem Fall
  \[
  l = \lim_{x\uparrow x_0} f(x)  = f(x_0 - )\qquad\text{bzw.}\qquad 
  l = \lim_{x\downarrow x_0} f(x)  = f(x_0 + ).
  \] 
  Gehört $x_0$ zu $D$ und ist $f(x_0-)=f(x_0)$ bzw. 
  $f(x_0+)=f(x_0)$, so sagt man, $f$ sei in $x_0$ 
  \slanted{links- bzw. rechtsseitig stetig.}
  \AntEnd
\end{antwort}

%% Question 62
\begin{frage}\index{einseitiger Grenzwert}
  Warum hat eine beschränkte monotone Funktion $f\fd\open{a,b}\to\RR$ 
  an jeder Stelle $x_0\in [a,b]$ einseitige Grenzwerte?
\end{frage}

\begin{antwort}
  Wir zeigen den Zusammenhang für monoton wachsende Funktionen 
  (für monoton fallende Funktionen argumentiert man analog).  
  Wie beinahe immer bei derartigen Fragen, 
  die auf die spezifischen Kontinuitätseigenschaften 
  der reellen Zahlen abzielen, 
  führt eine Supremumskonstruktion zum Ziel. In diesem Fall läuft es 
  darauf hinaus, zu zeigen, dass für ein beliebiges $x_0\in \open{a,b}$ 
  die Funktion 
  \[
  F \fd \lopen{a,x_0} \to \RR, \qquad
  x \mapsto \left\{ \begin{array}{ll}
      f(x) & \text{für $x\in \open{a,x_0}$} \\
      s:=\sup \{ f(x); \, x\in \open{a,x_0} \}  & \text{für $x=x_0$}.
    \end{array} \right.
  \]
  in $x_0$ stetig ist und $f$ an dieser Stelle somit einen linksseitigen 
  Grenzwert besitzt. 

  Die Stetigkeit von $F$ in $x_0$ ist schnell gezeigt.  
  Aufgrund der Supremumseigenschaft von $s$ gibt es zu jedem $\eps>0$ 
  ein $\xi \in \open{a,x_0}$ mit $s-\eps <f(\xi) \le s$. Wegen der 
  Monotonie von $f$ gilt dann auch für alle $x\in \open{\xi,x_0}$ 
  die Ungleichung $s-\eps < f(x)=F(x) \le s$, 
  also $F(x)\in U_\eps( s )$, und das heißt nichts anderes, 
  als dass $F$ in $x_0$ (linksseitig) stetig ist. 

  Die Existenz des rechtsseitigen Grenzwerts 
  zeigt man nun mit einer entsprechenden Infimumskonstruktion. 
  Damit ist der Satz bewiesen.   
  \AntEnd
\end{antwort}

%% Question 63
\begin{frage}\index{Gauss-Klammer@Gau{\ss}-Klammer}
  Welche Grenzwerte besitzt die Gauß-Klammer 
  \[
  [ \;\, ] \fd \RR \to \RR, \qquad 
  x \mapsto \{ g\in\ZZ; \; g\le x \} \text{?}
  \]
\end{frage}

\begin{antwort}
  Die Funktion ist für jedes $k\in \ZZ$ unstetig im Punkt $k$. Ferner ist sie   
  konstant auf den Intervallen $\ropen{k,k+1}$. 
  Für eine gegen $k\in\ZZ$ konvergente Folge aus $\ropen{k,\infty}$ konvergiert 
  daher trivialerweise auch die Folge der Funktionswerte gegen 
  $[ k ]$, die 
  Gauß-Klammer hat also in jeder Unstetigkeitsstelle einen rechtsseitigen 
  Grenzwert. Wegen $\big| [k-\eps]  - [ k ] \big|=1$ für alle 
  $\eps\in \open{0,1}$ existieren aber keine linksseitigen Grenzwerte in 
  den Unstetigkeitsstellen.
  \AntEnd
\end{antwort}

%% Question 64
\begin{frage}
  Warum hat eine \slanted{monotone} Funktion $f\fd [a,b]\to \RR$ 
  höchstens abzählbar viele Unstetigkeitsstellen?
\end{frage}

\begin{antwort}
  Für jede Unstetigkeitsstelle $x\in [a,b]$ existieren nach der vorhergehenden 
  Antwort die Grenzwerte $f(x-)$ und $f(x+)$. 
  Durch Auswahl je einer rationalen Zahl aus den Intervallen 
  $\big( f(x-),f(x+) \big)$ erhält man eine Abbildung der Menge der 
  Unstetigkeitsstellen auf $\QQ$. Da $f$ streng monoton ist, sind 
  die Intervalle paarweise disjunkt, und die Abbildung ist somit injektiv. 
  Das zeigt die Behauptung. \AntEnd
\end{antwort}

%% Question 65
\begin{frage}
  Ist $D\subset \RR$ eine nicht nach oben beschränkte Teilmenge,  
  $f\fd D\to \RR$ eine Funktion und 
  $l\in\RR$. 
  Was bedeutet die Schreibweise $\lim_{x\to\infty} f(x)=l$?
\end{frage}

\begin{antwort}
  Anschaulich bedeutet die Ausdrucksweise, dass die Funktionswerte $f(x)$ 
  beliebig nahe bei $l$ liegen, wenn $x$ nur genügend groß 
  ist. Präziser: 
  Es gilt $\lim_{x\to\infty} f(x)=l$ genau dann, wenn 
  zu jedem $\eps>0$ eine reelle Zahl $K > 0$ existiert mit der Eigenschaft
  \begin{equation}
    | f(x)-l | < \eps \quad\text{ für alle $x>K$.} \EndTag
  \end{equation}
  
\end{antwort}

%% Question 66
\begin{frage}\label{03_gbsp}
  Warum gilt 
  $\lim\limits_{x\to\infty} ( \sqrt{x+1} - \sqrt{x})=0 $, und warum 
  $\lim\limits_{x\to\infty}  \frac{3x-1}{4x+5}=\frac{3}{4}$?
\end{frage}

\begin{antwort}
  Es gilt
  \[
  0<\sqrt{x+1}-\sqrt{x} = 
  \frac{(\sqrt{x+1}+\sqrt{x})(\sqrt{x+1}-\sqrt{x})}{\sqrt{x+1}+\sqrt{x}}
  < \frac{1}{2\sqrt{x}} < \eps \quad\text{ für alle $x>\frac{1}{4\eps^2}$}, 
  \]
  das beweist die erste Formel. Die zweite folgt aus 
  \begin{equation}
    \left| \frac{3x-1}{4x+5}-\frac{3}{4} \right| = 
    \left| \frac{ 19 }{4(4x+5)} \right| < \left| \frac{2}{x} \right| < \eps 
    \quad\text{ für alle $x>\frac{2}{\eps}$.}
    \EndTag
  \end{equation}
\end{antwort} 

%% Question 67
\begin{frage}\index{Reduktionslemma}
  Was besagt das sogenannte \bold{Reduktionslemma}?
\end{frage}

\begin{antwort}
  Mithilfe des Reduktionslemmas lässt sich das Konvergenzverhalten 
  einer Funktion $f(x)$ "`im Unendlichen"' durch dasjenige der Funktion 
  $f \left( \frac{1}{x} \right)$ 
  (die für jedes $x$ mit $\frac{1}{x} \in D(f)$ definiert ist)  
  in der Nähe des Nullpunktes beschreiben. 
  Genau besagt das Reduktionslemma: 

  \medskip\noindent
  \slanted{Der Grenzwert  
    $\lim\limits_{x\to\infty} f(x)$ existiert genau dann, 
    wenn $f\left(\frac{1}{x} \right)$ 
    in $0$ einen rechtsseitigen Grenzwert besitzt, und dass in diesem Fall 
    gilt
    \[\boxed{
      \lim_{x\to\infty} f(x) = \lim_{x\downarrow 0}   
      f\left(\frac{1}{x} \right).}
    \]}
  \noindent
  Das Lemma folgt aus
  \[
  | f(x) - l | \le \eps \quad\text{für $x>C$} 
  \quad \LLa \quad
  \left| f\left(\frac{1}{x}\right) - l \right| 
  \le \eps \quad\text{für $x< \delta := \frac{1}{C}$} 
  . \]
  Zum Beispiel erhält man mit einer 
  Anwendung des Reduktionslemmas den zweiten Grenzwert 
  aus Frage \ref{03_gbsp} auch folgendermaßen: 
  \begin{equation}
    \lim_{x\to\infty} \frac{3x-1}{4x+5} =
    \lim_{x\downarrow 0} \frac{3/x-1}{4/x+5} = 
    \lim_{x\downarrow 0} \frac{3-1x}{4+5x} = \frac{3}{4}.
    \EndTag
  \end{equation}
\end{antwort}

%% Question 68
\begin{frage}
  Was bedeuten Schreibweisen wie 
  \begin{eqnarray*}
    \lim_{x\to x_0}f(x) = \infty &\quad\text{bzw.}\quad&
    \lim_{x\to x_0}f(x) = -\infty \quad\text{bzw.}\quad \\
    \lim_{x\to\infty}f(x) = \infty &\quad\text{bzw.}\quad&
    \lim_{x\to\infty}f(x) = -\infty\text{?}
  \end{eqnarray*}
\end{frage}

\begin{antwort}
  Die Schreibweisen $\lim\limits_{x\to x_0} = \infty$ bzw. 
  $\lim\limits_{x\to x_0} = -\infty$ bedeuten, dass $f(x)$ bei 
  Annäherung an $x_0$ jede beliebige positive bzw. negative 
  Schranke über- bzw. unterschreitet, genauer: Zu jedem $K\in \RR_+$ gibt es 
  eine Umgebung $\dot{U}_\delta( x_0 )$, sodass 
  $f(x)>K$ bzw. $f(x)<-K$ für alle $x\in \dot{U}_\delta (x_0)$ ist. 

  Die Ausdrücke $\lim\limits_{x\to\infty} = \infty$ bzw. 
  $\lim\limits_{x\to\infty} = -\infty$ bedeuten entsprechend
  dass zu jeder Zahl $K\in\RR$ eine Zahl $C\in\RR$ existiert, sodass 
  $f(x)>K$ für alle $x>C$ gilt. 
  \AntEnd
\end{antwort}

%% Question 69
\begin{frage}
  Ist $p\fd \RR\to \RR$ ein Polynom,
  \[
  p\mapsto p(x) := a_n x^n + a_{n-1}x^{n-1} + \cdots + a_0, 
  \qquad a_n\not=0, n\in\NN.
  \]
  Warum gilt dann $\lim\limits_{x\to\infty}\frac{p(x)}{a_nx^n}=1$?
\end{frage}

\begin{antwort}
  Für $x\not=0$ gilt $
  \frac{p(x)}{a_nx^n} = 1+ \frac{a_{n-1}}{a_nx} + \cdots + 
  \frac{a_0}{a_nx^n}.$
  Mit $A:=\max\{ |a_1|, \ldots, |a_n| \}$ und $x\ge 1$ ist folglich
  \[
  \left| 
    \frac{p(x)}{a_nx^n} -1 
  \right| \le
  \left| \frac{a_{n-1}}{a_nx} \right| + \cdots + 
  \left| \frac{a_0}{a_nx^n} \right| \le n\left| \frac{A}{a_n x }\right|.  
  \]
  Beide Seiten der Ungleichung sind kleiner als 
  $\eps$ für alle $x>nA/(|a_n|\eps)$, womit die Behauptung gezeigt ist.
  \AntEnd
\end{antwort}








































































































