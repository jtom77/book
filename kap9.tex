
\chapter{Metrische R\"aume und ihre Topologie}

Ein allgemeines begriffliches Konzept für die Behandlung der in der 
Analysis und ihren Anwendungen auftretenden Funktionen und Abbildungen 
stellen die \slanted{metrischen Räume} und ihre Spezialfälle, 
die \slanted{normierten Räume}, dar. 
Allgemeiner ist der Begriff des \slanted{topologischen Raumes}, 
den wir in diesem Kapitel aber nur streifen werden. Wir nehmen 
hier einen schon früher gesponnenen Faden etwas systematischer wieder auf. 

\section{Grundbegriffe}

%% --- 1 --- %%
\begin{frage}\index{Metrik}\index{metrischer Raum}
  Was versteht man unter einer \bold{Metrik} auf einer 
  nichtleeren Menge $X$?
\end{frage}

\begin{antwort}
  Eine Metrik auf $X$ ist eine Funktion 
  $d\fd X\times X \to \RR$ mit den Eigenschaften 
  {\setlength{\labelsep}{7mm}
    \satz{\begin{itemize}
      \item[\desc{M1}] $d(x,x)=0$ \quad und \quad $d(x,y)>0$ für $x\not=y$,\\[-3mm]
      \item[\desc{M2}] $d(x,y)=d(y,x)$,\\[-3mm]
      \item[\desc{M3}] $d(x,y) \le d(x,z)+d(z,y)$. \qquad(Dreiecksungleichung) 
      \end{itemize}} }
  Die Zahl $d(x,y)$ heißt \slanted{Abstand} von $x$ und $y$. \AntEnd
  
\end{antwort}

%% --- 2 --- %%
\begin{frage}\index{metrischer Raum}
  Was ist ein \bold{metrischer Raum}?
\end{frage}

\begin{antwort}
  Ein \slanted{metrischer Raum} $(X,d)$ ist eine nichtleere Menge 
  $X$ zusammen mit einer Abbildung 
  $d\fd X\times X \to \RR$, die die Eigenschaften \desc{M1}, \desc{M2} 
  und \desc{M3} besitzt.
  \AntEnd
\end{antwort}

%% --- 3 --- %%
\begin{frage}\index{Norm}
  Was versteht man unter einer \bold{Norm} 
  auf einem $\KK$-Vektorraum $V$ ($\KK=\CC$ ode $\KK=\RR$)?
\end{frage}

\begin{antwort}
  Eine \slanted{Norm} ist eine Funktion $\|\;\, \| \fd V\to \RR$ mit der 
  Eigenschaft, dass für alle $x,y\in V$ und $\alpha \in \KK$ gilt:
  {\setlength{\labelsep}{7mm}
    \satz{\begin{itemize}
      \item[\desc{N1}] $\| 0 \| =0$ \quad und \quad $\| x \|> 0$ für $x\not=0$.\\[-3mm]
      \item[\desc{N2}] $\| \alpha x\|=|\alpha| \cdot \| x \|$,\\[-3mm]
      \item[\desc{N3}] $\| x+ y \| \le \| x \| + \|y \|$. 
        \qquad(Dreiecksungleichung)  \AntEnd 
      \end{itemize}}}
\end{antwort}

%% --- 4 --- %%
\begin{frage}\index{Metrik!von Norm induzierte}
  Wie erhält man aus einer Norm auf $V$ eine Metrik auf $V$?
\end{frage}

\begin{antwort}
  Für $x,y\in V$ setze man $d(x,y) := \| x-y \|$. Dann erfüllt  
  die Abbildung $d \fd V\times V \to \RR$ 
  alle Eigenschaften einer Metrik, 
  wie man ohne Weiteres nachprüft. \AntEnd
\end{antwort}

%% --- 5 --- %%
\begin{frage}\label{08_diskr}\index{Metrik!diskrete}
  Kennen Sie ein Beispiel eines metrischen Raumes, dessen Metrik 
  \slanted{nicht} von einer Norm induziert wird?
\end{frage}

\begin{antwort}
  Durch
  \[
  \dis d(x,y) := \left\{ \begin{array}{ll} 0, & \text{falls $x=y$,} \\
      1, & \text{falls $x\not=y$.} \end{array} \right.
  \]
  ist auf jeder nichtleeren Menge $X$ eine Metrik (die sogenannte 
  \slanted{diskrete} oder \slanted{triviale Metrik}) gegeben. Diese kann
  nicht von einer Norm induziert sein, denn für 
  $0\not=\alpha\not=1$ und $x\not=y$ erhielte man aus 
  dieser Annahme den Widerspruch    
  \begin{equation}
    1 = \| x-y \| = d(x,y) = d( \alpha x, \alpha y) 
    = \alpha \cdot \| x-y \| = \alpha \not=1.
    \EndTag
  \end{equation} 
\end{antwort}

%% --- 6 --- %%
\begin{frage}\index{metrischer Raum}\index{normierter Raum}
  Können Sie Beispiele für metrische bzw. normierte 
  Räume nennen?
\end{frage} 

\begin{antwort}
  Beispiele für normierte Räume sind: 
  {\setlength{\labelsep}{5mm}
    \begin{itemize}
    \item[\desc{i}] $\KK$-Vektorräume ($\KK=\RR$ oder $\KK=\CC$), wenn man sie 
      zum Beispiel mit einer der $p$-Normen aus Frage \ref{08_pnorm}  
      versieht,
    \item[\desc{ii}] der Raum der stetigen (differenzierbaren, integrierbaren) 
      Funktionen auf $[a,b]$ bezüglich der Supremumsnorm, 
    \item[\desc{iii}] jeder Unterraum eines normierten Raums,
    \item[\desc{iv}] Der Raum der \slanted{quadratintegrierbaren} Funktionen 
      auf einer beliebigen Menge $U$, d.\,i. die Menge aller 
      auf $U$ lokal integrierbaren Funktionen 
      mit $\int_U |f(x)|^2 \difx < \infty$. Die Norm von $f$ ist dann 
      gerade der Wert dieses Integrals, 
    \item[\desc{v}] Banach- und Hilberträume. 
    \end{itemize}}
  \noindent
  Alle diese Räume sind zusammen mit der durch die Norm induzierten Metrik 
  automatisch auch metrische Räume. 
  Beispiele metrischer Räume, deren Metrik nicht von einer Norm induziert ist, 
  sind nach Frage \ref{08_diskr} beliebige mit der diskreten Metrik 
  versehenen nichtleere Mengen $X$.
  \AntEnd
\end{antwort}

%% --- 7 --- %%
\begin{frage}\label{08_pnorm}\index{p-Norm@$p$-Norm}
  Wie sind die sogenannten $p$-Normen auf $\KK^n$ definiert?
\end{frage}

\begin{antwort}
  Für $p \ge 1$ ist die $p$-Norm eines Vektors $x\in \KK^n$ definiert 
  durch   
  \[
  {\| x \|}_p := \big( {\left| x_1 \right|}^p + \cdots + 
  {\left|x_n \right|}^p \big)^{1/p}.  
  \]

  Für $p=2$ ist das gerade die \slanted{Euklidische Metrik}\index{Euklidische Metrik} 
  \[
  {\| x \|}_2 = \sqrt{ x_1^2+\cdots + x_n^2 }.
  \] 
  Weiter definiert man als \slanted{Maximumsnorm}\index{Maximumsnorm}
  \begin{equation}
    {\| x \|}_\infty := \max\{ x_1, \ldots, x_n \}.
    \notag
  \end{equation}

  \begin{center}
    \includegraphics{mp/08_p-norm}
    \captionof{figure}{Einheitskreis im $\RR^2$ bezüglich unterschiedlicher $p$-Normen.}
    \label{fig:08_p-norm}
  \end{center}

  Die Abbildung \ref{fig:08_p-norm} zeigt den Einheitskreis im $\RR^2$ 
  bezüglich unterschiedlicher $p$-Normen. \AntEnd
\end{antwort}

%% --- 8 --- %%
\begin{frage}\label{08_pnormaq}
  Können Sie für $x\in \KK^n$ und $1\le p \le q < \infty$ die 
  Abschätzungen 
  \[
  {\| x \|}_\infty \le {\| x \|}_p \le {\| x \|}_q \le {\| x \|}_1 
  \le n {\| x \|}_\infty .  
  \]
  beweisen?
\end{frage}

\begin{antwort}
  Die Antwort zu dieser Frage wurde schon in Frage 
  \ref{01_pnorm} gegeben.\AntEnd
\end{antwort}

%% --- 9 --- %%
\begin{frage}\index{Translationsinvarianz}
  Was bedeutet die \bold{Translationsinvarianz} einer 
  von einer Norm abgeleiteten Metrik $d$ auf $V$?
\end{frage}


\begin{antwort}
  Der Begriff besagt, dass für je zwei Vektoren $x, y \in V$ 
  und einen beliebigen Vektor $v\in V$ gilt: 
  \[
  d(x+v, y+v )= d(x,y),
  \]
  \sieheAbbildung\ref{fig:08_translation}

  \begin{center}
    \includegraphics{mp/08_translation}
    \captionof{figure}{Der Abstand von $x$ und $y$ ist translationsinvariant.}
    \label{fig:08_translation}
  \end{center}

  Diese Beziehung folgt für jede aus einer Norm abgeleiteten Metrik unmittelbar 
  aus $d(x,y)=\| x-y \|$. 
  \AntEnd
\end{antwort}

%% --- 10 --- %%
\begin{frage}\label{08_par}\index{Parallelogrammidentität}
  Warum erfüllt die von einem Skalarprodukt induzierte Norm 
  $ \| v \| := \sqrt{ \langle v,v \rangle }$ die 
  \bold{Parallelogrammidentität}
  \[
  \| v+w \|^2 + \|v-w \|^2 = 2 \left( \| v \|^2 + \| w \|^2 \right)
  \text{?}
  \]
\end{frage}

\begin{antwort}
  Beweis wörtlich wie in der Antwort zu Frage 
  \ref{01_parallelogramm}, also 
  mit purem Nachrechnen.  
  \AntEnd
\end{antwort}

%% --- 11 --- %%
\begin{frage}
  Gilt zu der Behauptung aus Frage \ref{08_par} auch die Umkehrung?
\end{frage}

\begin{antwort}
  Die Umkehrung gilt ebenfalls. 
  Um das entsprechende Skalarprodukt 
  $\langle \;, \; \rangle$ zu finden, beachte man, 
  dass dieses die Gleichung 
  \[
  \| v+ w \|^2 = \| v \|^2 + \|w \|^2 + 2\langle v,w \rangle 
  \]
  erfüllen muss. Daraus folgt mit der Parallelogrammidentität
  \begin{eqnarray*}
    \langle v, w \rangle = \frac{1}{4} 
    \left( \| v+w \|^2 - \| v-w \|^2 \right).
  \end{eqnarray*}
  Damit hätte man einen Kandidaten gefunden, für den jetzt 
  noch im Einzelnen nachgewiesen werden muss, dass er die Eigenschaften 
  eines Skalarprodukts (Bilinearität, Symmetrie, positive Definitheit) 
  besitzt. Die Symmetrie und positive Definitheit ergeben sich unmittelbar 
  aus den Eigenschaften der Norm. 
  Die Bilinearität weist man in mehreren Schritten nach. 
  Durch eine direkte Rechnung zeigt man 
  $\langle v+v', w \rangle = \langle v, w \rangle + \langle v', w \rangle$ und 
  folgert daraus, dass  $\langle \lambda v, w \rangle 
  = \lambda\cdot \langle v, w \rangle$ zunächst für alle 
  $\lambda \in \NN$ gilt. Aus Linearitätsgründen folgt daraus, dass die  
  Gleichung auch für alle $\lambda\in\QQ$ gilt. 
  In einem letzten Schritt folgert man daraus mit einem 
  Stetigkeitsargument, dass die Gleichung auch noch für 
  beliebige $\lambda\in \RR$ 
  gültig bleibt.\AntEnd 
\end{antwort}

%% --- 12 --- %%
\begin{frage}\index{Maximumsnorm}
  Warum kann die Maximumsnorm ${\| x \|}_\infty$ auf $\KK^n$ nicht von 
  einem Skalarprodukt induziert sein?
\end{frage}

\begin{antwort}
  Für die Maximumsnorm gilt die Parallelogramm-Identität nicht. 
  Für die Vektoren $v:=(1,0,0,\ldots,0)$ und $w:=(0,1,0,\ldots,0)$ 
  aus $\KK^n$ etwa erhält man 
  \begin{equation}
    2 = {\| v+w \|}_\infty^2 + {\| v-w \|}_\infty^2 \not= 
    2 \left( {\| v \|}^2_\infty + {\| w\|}^2_\infty \right)= 4.
    \EndTag
  \end{equation}
\end{antwort}

%% --- 13 --- %%
\begin{frage}\index{eps@$\eps$-Umgebung}\index{offene Kugel}
  Wie ist der Begriff der $\eps$-Umgebung (offene Kugel) 
  eines Punktes in einem metrischen Raum $X$ erklärt?
\end{frage}


\begin{antwort}
  Unter der offenen $\eps$-Umgebung $U_\eps(a)$ eines 
  Punktes $a\in X$ versteht man die Menge aller Punkte, die 
  von $a$ bezüglich der Metrik auf $X$ einen kleineren Abstand als 
  $\eps$ haben, also die Menge
  \begin{equation}
    U_\eps( a ) := \{ x\in X \sets d(x,a) < \eps \}. \EndTag
  \end{equation}

  \begin{center}
    \includegraphics{mp/08_ueps}
    \captionof{figure}{$\eps$-Umgebung von $a$.}
    \label{fig:08_ueps}
  \end{center}
\end{antwort}

%% --- 14 --- %%
\begin{frage}\label{08_ofdef}\index{Umgebung}\index{offene Menge}
  Wie sind die Begriffe \bold{"`offene Menge"'} 
  und \bold{"`Umgebung"'} in einem metrischen Raum $X$ erklärt?
\end{frage}

\begin{antwort}
  Eine Menge $U \subset X$ heißt \slanted{Umgebung} von $a\in X$, wenn 
  sie eine offene $\eps$-Kugel um $a$ enthält.

  Eine Menge $U$ heißt 
  \slanted{offen}, wenn sie für jeden ihrer Punkte eine Umgebung ist,  
  {\dasheisst} wenn zu jedem $a\in U$ eine $\eps$-Kugel $U_\eps(a)$ 
  existiert, die vollständig in $U$ enthalten ist, 
  \sieheAbbildung\ref{fig:08_umgebung}.

  \begin{center}
    \includegraphics{mp/08_umgebung}
    \captionof{figure}{Jede offene Umgebung $U$ von $a$ enthält eine 
      $\eps$-Kugel um $a$.} 
    \label{fig:08_umgebung}
  \end{center}

  In metrischen Räumen wird "`Offenheit"' somit über den Begriff der 
  $\eps$-Umgebung definiert, der seinerseits über den Abstandsbegriff 
  eingeführt wird. 
  \AntEnd
\end{antwort}

%% --- 15 --- %%
\begin{frage}\index{offene Kugel}\index{eps@$\eps$-Umgebung}
  Warum ist die offene Kugel 
  ($\eps$-Umgebung) $U_\eps(a)$ eines Punktes $a\in X$ stets offen?
\end{frage}

\begin{antwort}
  Für einen Punkt $x \in U_\eps(a)$ ist eine vollständige 
  $\eps^*$-Umgebung von $x$ in $K_\eps(a)$ enthalten, \sieheAbbildung\ref{fig:08_ueps2}. 
  Das ist eine Konsequenz aus der Dreiecksungleichung. 
  Setzt man $\eps^* := \eps-d(x,a) >0$, so gilt 
  für jedes $y\in U_{\eps^*}(x)$ 
  \[
  d(a,y) \le d(a,x) + d(x,y) < (\eps - \eps^*) + \eps^* = \eps,
  \]
  also $y\in U_\eps(a)$. Das zeigt $U_{\eps^*}(a) \subset U_\eps(a)$, 
  nach Definition ist $U_\eps(a)$ somit offen. \AntEnd

  \begin{center}
    \includegraphics{mp/08_ueps2}
    \captionof{figure}{Jeder Punkt $x$ einer $\eps$-Umgebung 
      besitzt selbst eine $\eps^*$-Umgebung, die in der ersten enthalten ist.}
    \label{fig:08_ueps2}
  \end{center}
\end{antwort}

%% --- 16 --- %%
\begin{frage}\index{abgeschlossen}
  Wann heißt eine Teilmenge eines metrischen Raumes 
  \bold{abgeschlossen}?
\end{frage}

\begin{antwort}
  Eine Teilmenge $A$ eines metrischen Raumes $X$ 
  heißt abgeschlossen, wenn 
  das Komplement $A^C = X \mengeminus A$ offen ist.
  \AntEnd
\end{antwort}

%% --- 17 --- %%
\begin{frage}\label{08_oua} 
  Sind $\emptyset$ und $X$ offen oder abgeschlossen in $X$?
\end{frage}

\begin{antwort}
  Die leere Menge und $X$ selbst sind 
  beide gleichzeitig offen \slanted{und} abgeschlossen. 
  Offen sind sie, weil sie die Definition 
  der "`Offenheit"' aus Frage \ref{08_ofdef} erfüllen. 
  Daraus folgt aber wegen $X^C = \emptyset$ und 
  $\emptyset^C = X$ auch deren Abgeschlossenheit. 
  \AntEnd
\end{antwort} 

%% --- 18 --- %%
\begin{frage}\label{08_grund}
  Welche \bold{Grundeigenschaften} hat das 
  System $\calli{O}(d)$ der 
  offenen Mengen in einem metrischen Raum?
  \nomenclature{$\calli{O}$}{System der offenen in einem topologischen Raum}
\end{frage}

\begin{antwort}
  Es gilt: 
  {\setlength{\labelsep}{6mm}
    \begin{itemize} 
    \item[\desc{O1}] Der Durchschnitt endlich vieler offener Mengen ist offen.
      \\[-4mm]
    \item[\desc{O2}] Die Vereinigung beliebig vieler offener Mengen ist offen.
    \end{itemize}}

  \begin{center}
    \includegraphics{mp/08_durchschnitt}
    \captionof{figure}{Jeder Punkt im Durchschnitt zweier offener Mengen 
      besitzt eine offene $\eps$-Umgebung, die im Durchschnitt enthalten ist.}
    \label{fig:08_durchschnitt}
  \end{center}

  Die erste Eigenschaft ergibt sich daraus, dass der Durchschnitt je zweier 
  offener Kugeln eines Punktes wieder eine offene Kugel enthält 
  (nämlich zumindest die offene Kugel mit dem kleineren Radius, vgl. 
  Abbildung \ref{fig:08_durchschnitt}), 
  bezüglich der zweiten ist nur zu sagen, dass wenn eine Teilmenge eine 
  offene Kugel enthält, diese Kugel erst recht in jeder Obermenge 
  enthalten ist.  
  \AntEnd
\end{antwort}

%% --- 19 --- %%
\begin{frage}
  Warum ist der Durchschnitt beliebig vieler 
  offener Mengen eines metrischen Raumes im Allgemeinen nicht mehr offen?
\end{frage}

\begin{antwort}
  Als Beispiel betrachte man zu einem Punkt $a\in \RR^n$ die Menge 
  aller offenen Kugeln $U_{1/k}(a)$ mit $k\in \NN$. Dann ist $
  \bigcap_{k\in\NN} U_{1/k} (a) = \{ a \}$
  nicht mehr offen in $\RR^n$. 
  \AntEnd 
\end{antwort}

%% --- 20 --- %%
\begin{frage}\index{induzierte Metrik}
  Ist $(X,d)$ ein metrischer Raum, was versteht man unter der 
  von $d$ \bold{induzierten Metrik} auf einer Teilmenge $M\subset X$?
\end{frage}

\begin{antwort}
  Die induzierte Metrik ist die Einschränkung der Abbildung 
  $d$ auf $M$, also die Abbildung $d|_M \fd M\times M \to \RR$. 
  Auf diese Weise wird $(M, d|_M)$ selbst zu einem metrischen Raum.
  \AntEnd
\end{antwort}

%% --- 21 --- %%
\begin{frage}\label{08_topdef}\index{Topologie}\index{topologischer Raum}
  Was versteht man unter einem \bold{topologischen Raum}?
\end{frage}

\begin{selbeseite}
  \begin{antwort}
    Ein \slanted{topologischer Raum} ist ein Paar $(X, \calli{O})$, 
    bestehend aus einer Menge $X$ und einer Menge 
    $\calli{O}$ von Teilmengen (\slanted{offene} Mengen genannt) 
    von $X$, derart, dass gilt:
    {\setlength{\labelsep}{7mm}
      \begin{itemize}
      \item[\desc{T1}] Beliebige Vereinigungen von offenen Mengen sind offen.
      \item[\desc{T2}] Der Durchschnitt von je zwei offenen Mengen ist offen.
      \item[\desc{T3}] Die leere Menge und $X$ sind offen.\AntEnd
      \end{itemize}}\vss
  \end{antwort}
\end{selbeseite}

%% --- 22 --- %%
\begin{frage}\index{Umgebung!in allgemeinen topologischen Räumen}
  Wie ist der \bold{Umgebungsbegriff} 
  in einem allgemeinen topologischen Raum $X$ definiert?
\end{frage}

\begin{antwort}
  Eine Menge $U\subset X$ heißt \slanted{Umgebung} eines Punktes $a\in X$, 
  wenn eine offene Menge $V$ mit $a\in V$ und $V\subset U$ existiert.
  \AntEnd
\end{antwort}

%% --- 23 --- %%
\begin{frage}\index{offene Menge}
  Wie lassen sich die offenen Mengen in einem topologischen Raum mit dem 
  Umgebungsbegriff charakterisieren?
\end{frage}

\begin{antwort}
  Eine Teilmenge $M\subset X$ ist genau dann offen in $X$, 
  wenn $M$ eine Umgebung für jeden ihrer Punkte ist. 

  Enthält nämlich $M$ zu jedem Punkt $a\in M$ eine offene Umgebung $U(a)$ 
  von $a$, dann ist $M$ gerade die Vereinigung aller $U(a)$ und daher 
  nach \ref{08_topdef}\,\desc{T2} offen. Ist $M$ andersherum offen, so ist 
  $M$ selbst bereits eine \slanted{offene} Umgebung für jeden ihrer 
  Punkte und damit erst recht eine Umgebung.
  \AntEnd
\end{antwort}

%% --- 24 --- %%
\begin{frage}\index{metrischer Raum}\index{topologischer Raum}
  Warum ist jeder metrische Raum ein topologischer Raum?
\end{frage}

\begin{antwort}
  Die durch die Metrik induzierte Menge an offenen Teilmengen 
  eines metrischen Raumes erfüllt nach Frage \ref{08_grund} 
  die drei Eigenschaften \desc{T1}, \desc{T2} und \desc{T3}
  eines topologischen Raumes.
  \AntEnd
\end{antwort}

%% --- 25 --- %%
\begin{frage}\index{Aquivalenz@Äquivalenz!von Metriken}
  Wann heißen zwei Metriken auf einem Raum $X$ \bold{äquivalent}?
\end{frage}

\begin{antwort}
  
  Zwei Metriken $d$ und $d^*$ auf $X$ heißen \slanted{äquivalent}, 
  wenn sie dieselbe Topologie auf $X$ induzieren, wenn also die 
  Systeme offener Mengen $\calli{O}(d)$ und 
  $\calli{O}(d^*)$ gleich sind.\AntEnd
\end{antwort} 

%% --- 26 --- %%
\begin{frage}\label{08_klumpen}\index{diskrete Topologie}
  Was versteht man unter der \bold{diskreten Topologie} 
  auf einer Menge $X$?
\end{frage}

\begin{antwort}
  
  Die \slanted{diskrete Topologie} erhält man, 
  wenn man \slanted{alle} Teilmengen 
  von $X$ als offen kennzeichnet. 
  In dem Fall ist $\calli{O} = \mathfrak{P}(X)$.
  \AntEnd
\end{antwort} 

%% --- 27 --- %%
\begin{frage}\index{chaotische Topologie}\index{Klumpentopologie}
  Was versteht man unter der \bold{chaotischen Topologie} bzw. 
  \bold{Klumpentopologie} auf $X$? 
\end{frage}   

\begin{antwort}
  Die \slanted{Klumpentopologie} ist diejenige Topologie auf $X$, 
  in der nur die Mengen $X$ und $\emptyset$ offen sind, also 
  $\calli{O} = \{ X, \emptyset \}$.
  \AntEnd
  
\end{antwort}

%% --- 28 --- %%
\begin{frage}\label{08_diskto}\index{diskrete Metrik}\index{diskrete Topologie}
  Induziert die \slanted{diskrete Metrik} 
  die \slanted{diskrete Topologie} $d$ auf $X$?
\end{frage}

\begin{antwort}
  Hinsichtlich der diskreten Metrik enthält jede $\eps$-Kugel $U_\eps(a)$ 
  für $\eps<1$ nur den Punkt $a$ selbst, es gilt also $U_\eps(a)=\{ a \}$. 
  Für eine beliebige Teilmenge $M$ von $X$ folgt aus $a\in M$ damit 
  sogleich $U_\eps(a)\subset M$, {\dasheisst} $M$ ist 
  offen. Somit ist \slanted{jede} Teilmenge des metrischen Raums $(X,d)$ 
  offen. Die diskrete Metrik induziert also 
  in der Tat die diskrete Topologie auf $X$.
  \AntEnd
\end{antwort}

%% --- 29 --- %%
\begin{frage}\index{metrisierbar}
  Wann heißt ein topologischer Raum \bold{metrisierbar}?
\end{frage}

\begin{antwort}
  Ein topologischer Raum $(X,\calli{O})$ ist dann metrisierbar, wenn 
  eine Metrik $d$ auf $X$ existiert, die als offene Mengen 
  gerade diejenigen der Topologie $\calli{O}$ bestimmt,  
  mit der also $\calli{O}(d)=\calli{O}$ gilt.
  \AntEnd
\end{antwort}

%% --- 30 --- %%
\begin{frage}
  Ist $(X, \calli{Q})$ mit $\calli{Q} := \{ \emptyset, \mathrm{Pot}(X) \}$ 
  ($\calli{Q}$ ist die \bold{chaotische Topologie}) metrisierbar?
\end{frage}

\begin{antwort}
  
  Ist ein topologischer Raum $X$ metrisierbar, 
  so existieren zu je zwei verschiedenen 
  Punkten $a,b\in X$ disjunkte offene $\eps$-Umgebungen 
  (Hausdorff\sch e Trennungseigenschaft). Das folgt 
  mit $\eps := d(a,b)/2$ aus der Dreiecksungleichung. 
  \index{Hausdorffsche@Hausdorff'sche Trennungseigenschaft}
  \index{Hausdorff}

  Man kann also festhalten, dass ein metrisierbarer topologischer Raum mit
  mehr als einem Punkt stets mindestens zwei 
  nichtleere offene Mengen besitzt, und hieraus folgt schon, dass 
  $(X,\calli{Q})$ (falls $X$ mehr als einen Punkt hat) 
  nicht metrisierbar sein kann, da die Topologie $\calli{Q}$ nur eine 
  einzige nichtleere Menge enthält.   \AntEnd
\end{antwort}

%% --- 31 --- %%
\begin{frage}\index{Haufunspunkt@Häufungspunkt}
  \index{Beruhrpunkt@Berührpunkt}
  \index{isolierter Punkt}
  \index{ausserer@äußerer Punkt}
  \index{innerer Punkt}\index{Randpunkt}
  Wie sind in einem beliebigen topologischen Raum $X$ für eine Teilmenge 
  $M\subset X$ und einen Punkt $a\in X$ die folgenden Begriffe 
  erklärt:
  \vspace*{-2mm}
  \begin{center}
    \begin{tabular}{llll}
      \desc{i}  &   $a$ ist ein  \bold{Häufungspunkt} von $M$, & 
      \desc{ii} &  $a$ ist ein \bold{Berührpunkt} von $M$, \\
      \desc{iii}  & $a$ ist ein \bold{isolierter Punkt} von $M$, &
      \desc{iv}  & $a$ ist ein \bold{äußerer Punkt} von $M$, \\
      \desc{v}  &  $a$ ist ein \bold{innerer Punkt} von $M$, & 
      \desc{vi}  & $a$ ist ein \bold{Randpunkt} von $M$?
    \end{tabular}
  \end{center}
\end{frage}

\begin{antwort}
  Die Bezeichnungen treffen genau dann zu, wenn 
  {\setlength{\labelsep}{5mm}
    \begin{itemize}
    \item[\desc{i}] jede Umgebung von $a$ einen 
      von $a$ verschiedenen weiteren Punkt aus $M$ enthält, \\[-4mm]
    \item[\desc{ii}] jede Umgebung von $a$ mit $M$ einen nichtleeren Durchschnitt 
      hat,\\[-4mm]   
    \item[\desc{iii}] eine offene Menge $V\subset X$ existiert, für 
      die $M\cap V = \{ a \}$ gilt,\\[-4mm]
    \item[\desc{iv}] $X\mengeminus M$ eine Umgebung von $a$ ist (äquivalent: 
      wenn eine Umgebung von $a$ existiert, in der keine Punkte aus $M$ liegen),
      \\[-4mm]
    \item[\desc{v}] $M$ eine Umgebung von $a$ ist,\\[-4mm]
    \item[\desc{vi}] weder $M$ noch $X\mengeminus M$ eine Umgebung von $a$ ist 
      (äquivalent: wenn in jeder Umgebung von $a$ sowohl Punkte aus 
      $M$ als auch Punkte aus $X\mengeminus M$ liegen).\AntEnd
    \end{itemize} }
\end{antwort}

%% --- 32 --- %%
\begin{frage}\label{08_inner}
  Warum liegen \bold{innere Punkte} und \bold{isolierte Punkte} von $M$ 
  stets in $M$?
\end{frage}

\begin{antwort}
  Nach Definition gibt es für einen inneren Punkt $a$ eine Umgebung $U$ 
  mit $U\subset M$, und freilich gilt $a\in U$ und somit $a\in M$. 
  Ist $a$ ein isolierter Punkt von $M$, so gibt es eine offene Menge 
  $V\subset X$ mit $\{ a \} = V\cap M$. Daraus folgt $a\in M$.  
  \AntEnd
\end{antwort}

%% --- 33 --- %%
\begin{frage}
  Warum ist ein Häufungspunkt von $M$ stets auch ein Berührpunkt 
  von $M$? Gilt hiervon auch die Umkehrung?
\end{frage}

\begin{antwort}
  Ist $U$ eine Umgebung von $a$ und $a$ ein Häufungspunkt von $M$, 
  so enthält $U$ einen Punkt aus $M$. Damit gilt 
  $U\cap M \not= \emptyset $, und $a$ ist ein Berührpunkt 
  von $M$. 

  Ein Berührunspunkt muss aber kein Häufungspunkt sein, 
  da die Definition von "`Berührpunkt"' den Fall 
  $U\cap M = \{ a \}$ (also den Fall, dass 
  $a$ ein isolierter Punkt ist) miteinbezieht. 
  In diesem Fall ist $a$ aber kein Häufungspunkt. 
  \AntEnd
\end{antwort}

%% --- 34 --- %%
\begin{frage}
  Wieso besteht jede offene Teilmenge von $M$ nur aus inneren 
  Punkten von $M$?
\end{frage}

\begin{antwort}
  \nomenclature{$\overline{M}$}{topologischer Abschluss von $M$}
  Andernfalls läge ein Randpunkt $\overline{m}\in M$ 
  in $U$. Aber dann enthält 
  jede Umgebung von $\overline{m}$ definitionsgemäß 
  einen Punkt, der nicht zu $U\subset M$ gehört, {\dasheisst}, 
  $U$ kann nicht offen in $M$ sein. 
  \AntEnd
\end{antwort}



%% --- 35 --- %%
\begin{frage}\nomenclature{$M^\circ$}{Menge der inneren Punkte von $M$}
  Warum ist die Menge $M^\circ$ der inneren 
  Punkte von $M$ stets offen und warum gilt 
  \[
  \text{$M$ offen in $X$ \quad $\LLa$ \quad $M=M^\circ$ ?}
  \]
\end{frage}

\begin{antwort}
  Ist $M$ offen in $X$, dann gilt nach der vorigen Frage  
  gilt $M \subset M^\circ$. Da andererseits nach Frage 
  \ref{08_inner} stets $M^\circ \subset M$ ist, folgt  
  daraus die Implikation "`$\Ra$"'.

  Da jede offene Teilmenge von $M$ nur innere Punkte enthält, gilt      
  \[
  \bigcup_{\substack{U\subset M \\ \text{$U$ offen in $X$}}} U 
  \subset M^\circ.
  \] 
  Andererseits liegt jeder innere Punkt aus $M$ in einer offenen Teilmenge 
  von $M$, woraus sich 
  \[
  M^\circ \subset \bigcup_{\substack{U\subset M \\ \text{$U$ offen in $X$}}} U
  \] 
  ergibt. Daraus folgt, dass $M^\circ$ gerade die 
  Vereinigung aller in $X$ offenen Teilmengen von $M$ ist. 
  Nach Frage \ref{08_topdef} \desc{T2} ist dann $M^\circ$  
  offen in $X$. 
  Das zeigt die Behauptung und die Richtung "`$\La$"'.  
  \AntEnd
\end{antwort}

%% --- 36 --- %%
\begin{frage}\label{08_ber}
  Ist $\overline{M}$ die Menge aller Berührpunkte von $M$, dann 
  ist $\overline{M}$ stets abgeschlossen und es gilt:
  \[
  \text{$M$ abgeschlossen in $X$ \quad $\LLa$ \quad $M=\overline{M}$}.
  \]
\end{frage}

\begin{antwort}
  Für den Beweis der ersten Behauptung und der Richtung "`$\La$"' 
  müssen wir zeigen, dass $X\mengeminus \overline{M}$ offen ist. 
  Angenommen, das ist nicht der Fall. 
  Dann existiert ein $x\in X\mengeminus \overline{M}$ derart, dass 
  \slanted{jede} offene Umgebung $U$ von $x$ einen Punkt 
  $\overline{m} \in \overline{M}$ enthält. 
  Da $U$ offen ist, gibt es dann aber auch eine Umgebung 
  $V$ von $\overline{m}$ mit $V\subset U$. 
  Nun ist $\overline{m}$ ein Berührpunkt von $M$, folglich 
  liegt ein $a \in M$ in $V$. Jede Umgebung von $x$ 
  enthält also einen Punkt aus $M$. 
  Daraus folgt $x\in \overline{M}$, im Widerspruch zur Annahme.

  Um "`$\Ra$"' zu zeigen, sei $M$ jetzt 
  abgeschlossen, {\dasheisst}, $X\mengeminus M$ ist offen in 
  $X$. Sei $a$ ein beliebiger Punkt aus $\overline{M}$. Da $a$ Berührpunkt  
  von $M$ ist, liegt in jeder Umgebung von $a$ ein Punkt aus $M$. 
  Das bedeutet aber, dass $a$ nicht in $X\mengeminus M$ liegen kann, weil 
  $X\mengeminus M$ andernfalls nicht offen wäre. Folglich gilt $a\in M$ 
  und insgesamt $M=\overline{M}$. \AntEnd  
\end{antwort}

%% --- 37 --- %%
\begin{frage}\label{08_haufab}
  Ist $M'$ die Menge aller Häufungspunkte von $M$, warum gilt dann 
  $\overline{M} = M \cup M'$, 
  und warum ist $M$ genau dann abgeschlossen in $X$, wenn $M\supset M'$ gilt?
\end{frage}

\begin{antwort}
  
  $\overline{M}$ besteht aus der Menge aller Häufungspunkte $M'$ 
  zusammen mit den isolierten Punkten von $M$. Da Letztere in 
  $M$ liegen, folgt $\overline{M} = M \cup M'$. 

  Gilt $M\supset M'$, so folgt daraus $M=\overline{M}$ und schließlich 
  mit der Äquivalenz aus Frage \ref{08_ber} die Abgeschlossenheit von $M$. 
  \AntEnd
\end{antwort}

%% --- 38 --- %%
\begin{frage} 
  Durch Anwendung der De Morgan'schen Regeln für die Komplementbildung 
  erhält man aus den Grundeigenschaften der offenen Mengen auch 
  Grundeigenschaften abgeschlossener Mengen. Welche sind das?
\end{frage} 

\begin{antwort}
  Durch Anwendung der Regel $(A \cup B)^C = A^C \cap B^C$ 
  erhält man aus den Grundeigenschaften \desc{T1} bis \desc{T3} für offene 
  Mengen als korrespendierende Eigenschaften abgeschlossener Mengen:
  {\setlength{\labelsep}{9mm}
    \begin{enumerate}
    \item[\desc{TA1}] Beliebige Durchschnitte abgeschlossener Mengen sind abgeschlossen.
    \item[\desc{TA2}] Die Vereinigung von je zwei abgeschlossenen 
      Mengen ist abgeschlossen.
    \item[\desc{TA3}] Die leere Menge und $X$ sind abgeschlossen.
    \end{enumerate}}
  Diese drei "`Axiome"' liefern eine alternative Definition des 
  Begriffs "`topologischer Raum"'.
  \AntEnd 
\end{antwort}

%% --- 39 --- %%
\begin{frage}
  Warum ist eine \slanted{beliebige Vereinigung} abgeschlossener Mengen 
  im Allgemeinen nicht abgeschlossen?
\end{frage}

\begin{antwort}
  Ein Gegenbeispiel liefert etwa die Vereinigung 
  aller abgeschlossenen Mengen 
  $\RR \supset A_n := [ -1+1/n, 1-1/n ]$ mit $n\in \NN$. Die Menge  
  $\bigcup_{n\in\NN} A_n = \open{-1,1}$ ist nicht abgeschlossen in $\RR$.
  \AntEnd
\end{antwort}

%% --- 40 --- %%
\begin{frage}\index{folgenabgeschlossen}
  Was versteht man unter eine \bold{folgenabgeschlossenen Teilmenge} 
  eines metrischen Raumes?
\end{frage}

\begin{antwort}
  Eine Teilmenge $M\subset X$ heißt \slanted{folgenabgeschlossen}, wenn 
  der Grenzwert $a\in X$ jeder konvergenten Folge $(a_n) \subset M$ 
  in $M$ liegt.
  
  So ist beispielsweise die Menge $\QQ$ als Teilmenge von $\RR$ 
  \slanted{nicht} folgenabgeschlossen, $\ZZ\subset \RR$ 
  aber schon, da die einzig konvergenten Folgen in $\ZZ$ diejenigen sind, die 
  ab einem bestimmten Glied konstant sind.
  \AntEnd
\end{antwort}

%% --- 41 --- %%
\begin{frage}%\label{08_folgabg}
  Warum ist eine Teilmenge $M$ eines metrischen Raumes 
  $X$ genau dann abgeschlossen, wenn sie folgenabgeschlossen ist?
\end{frage}

\begin{antwort}
  Sei $(a_n)$ eine Folge in $M$ mit dem Grenzwert $a$ und sei $M$ 
  abgeschlossen. Wenn $a$ nicht in $M$ liegt, dann gibt es -- da 
  $X \setminus M$ offen ist -- eine offene 
  $\eps$-Umgebung von $a$, die vollständig in 
  $X\setminus M$ liegt. Dann gilt 
  $d(a_n,a)>\eps$ für alle $n\in \NN$ und $(a_n)$ 
  kann somit nicht konvergieren.

  Sei jetzt $M$ folgenabgeschlossen und sei $h\in X$ ein beliebiger  
  Häufungspunkt von $M$. Nach Frage \ref{08_haufab} ist für die 
  Abgeschlossenheit von $M$ nur $h\in M$ zu zeigen. Da $h$ 
  ein Häufungspunkt ist, enthält die Umgebung 
  $K_{\frac{1}{n}}(h)$ für jedes $n\in \NN$ einen Punkt $b_n\in M$. 
  Die Folge $(b_n)$ konvergiert gegen $h$, und da $M$ nach Voraussetzung 
  folgenabgeschlossen ist, folgt $h\in M$. Das zeigt die Behauptung.
  \AntEnd   
\end{antwort}

%% --- 42 --- %%
\begin{frage}\nomenclature{$\partial M$}{Menge der Randpunkte der Menge $M$}
  Ist $\partial M $ die Menge aller Randpunkte von $M$, dann ist 
  $\partial M$ stets abgeschlossen, Können Sie das begründen?
\end{frage}

\begin{antwort}
  Für jeden Punkt $a\in X\setminus \partial M$ ist nach Definition entweder 
  $M$ oder $X\setminus M$ eine Umgebung. $X\setminus \partial M$ ist also 
  offen, $\partial M$ folglich abgeschlossen.\AntEnd
\end{antwort}

%% --- 43 --- %%
\begin{frage}
  Kennen Sie ein Beispiel eines metrischen Raumes $M$, für den 
  $M\subset \partial M$ gilt, der also in seinem Rand enthalten ist?
\end{frage}

\begin{antwort}
  Die Menge $\QQ \subset \RR$ mit der durch den Absolutbetrag induzierten 
  Metrik $d(x,y)=|x-y|$ ist {\zB} 
  ein metrischer Raum mit dieser Eigenschaft. Da sowohl die 
  rationalen als auch die irrationalen Zahlen dicht in $\RR$ liegen, besitzt 
  jede reelle Zahl die Eigenschaft, ein Randpunkt von $\QQ$ zu sein. 
  Es ist also in der Tat $\QQ \subset \partial \QQ = \RR$   
  \AntEnd 
\end{antwort} 





\section{Konvergenz, Cauchy-Folgen, Vollständigkeit}

Die Begriffe "`Cauchy-Folgen"',"`Konvergenz"' und "`Vollständigkeit"' 
werden in allgemeinen metrischen Räumen in völliger Analogie zu 
$\RR$ oder $\CC$ erklärt.  

%% --- 44 --- %%
\begin{frage}\index{Konvergenz!in einem metrischen Raum}
  Wann heißt eine Folge $(x_k)$ von Elementen eines metrischen Raumes 
  \bold{konvergent}?
\end{frage}

\begin{antwort}
  Eine Folge $(x_k)$ in einem metrischen Raum $(X,d)$ 
  heißt \slanted{konvergent}, wenn ein 
  $a\in X$ mit der folgenden Eigenschaft existiert: 
  Zu jedem $\eps>0$ gibt es ein $N\in \NN$, sodass $d(a,x_k)< \eps$ 
  für alle $k>N$ gilt. In diesem Fall heißt $a$ dann 
  der \slanted{Grenzwert} Folge. 

  Äquivalent dazu ist die Formulierung, dass für jedes $\eps>0$ 
  \slanted{fast alle} Folgenglieder in $U_\eps(a)$ liegen.  
  \AntEnd
\end{antwort}

%% --- 45 --- %%
\begin{frage}\index{Grenzwert!Eindeutigkeit}
  Warum ist im Falle der Konvergenz einer Folge der Grenzwert eindeutig 
  bestimmt?
\end{frage}

\begin{antwort}
  Ein metrischer Raum besitzt 
  die \slanted{Hausdorff\sch e Trennungseigenschaft}. Damit folgt 
  die Eindeutigkeit des Grenzwerts wie in Frage \ref{01_hausdorff}.
  \index{Hausdorffsche@Hausdorff'sche Trennungseigenschaft}  
  \AntEnd
\end{antwort}

%% --- 46 --- %%
\begin{frage}\heavy \index{Konvergenz!in einem topologischen Raum}
  Wie kann man die Konvergenz einer Folge in einem beliebigen 
  \slanted{topologischen Raum} definieren?
\end{frage}

\begin{antwort}
  Eine Folge $(x_k)$ in einem topologischen Raum konvergiert, wenn ein 
  $x\in X$ existiert, sodass zu \slanted{jeder} Umgebung $U$ von $x$ ein 
  $N\in \NN$ derart existiert, dass für alle $k>N$ gilt: $x_k \in U$. 
  \AntEnd
\end{antwort}

%% --- 47 --- %%
\begin{frage}\index{Klumpentopologie}
  Ist $X$ eine nichtleere Menge, die mit der "`Klumpentopologie"' 
  aus Frage \ref{08_klumpen} versehen ist, warum konvergiert dann \slanted{jede} 
  Folge aus $X$ gegen \slanted{jedes} Element aus $X$?
\end{frage}

\begin{antwort}
  Bezüglich der Klumpentopologie besitzt jedes Element aus $X$ nur 
  eine einzige Umgebung, nämlich $X$. Ist $a\in X$ ein beliebiges Element 
  und $(a_k)$ eine beliebige Folge aus $X$, dann liegen also alle 
  Folgenglieder $a_k$ in jeder Umgebung von $X$. 
  Die Folge konvergiert also im Sinne der Definition gegen $x$.
  \AntEnd 
\end{antwort} 

%% --- 48 --- %%
\begin{frage} \index{diskrete Metrik}
  Ist $d$ die \slanted{diskrete Metrik} auf einer Menge $X$ 
  (vgl. Frage \ref{08_diskr}). 
  Wie kann man die konvergenten Folgen aus $X$ charakterisieren?
\end{frage}

\begin{antwort}
  Wegen $d(x,y)=1$ für alle $x\not= y$ sind die konvergenten Folgen genau 
  die ab einem bestimmten Index konstanten. 

  Alternativ kann man mit dem Ergebnis aus Frage \ref{08_diskto} auch 
  rein topologisch argumentieren: 
  Die diskrete Metrik induziert die triviale Topologie auf $X$, 
  und bezüglich dieser sind alle Teilmengen offen. 
  Insbesondere hat dann jedes Element $a$ die Menge 
  $\{ a \}$ als ihre Umgebung, und diese kann nur dann 
  fast alle Glieder einer Folge $(a_k)$ enthalten, wenn ab einem 
  bestimmten Index $a_k = a$ gilt.\AntEnd
\end{antwort}

%% --- 49 --- %%
\begin{frage}\index{p-Norm@$p$-Norm}
  Versieht man den $\RR^n$ mit einer der $p$-Normen 
  (vgl. Frage \ref{08_pnorm}), 
  dann ist eine Folge $(a_k)_{k\in \NN}$ 
  mit $a_k := (\alpha_{1,k}, \ldots \alpha_{n,k} )$ genau dann konvergent 
  gegen $a:=(\alpha_1,\ldots,\alpha_n)$, wenn jede Komponentenfolge 
  $(\alpha_{\nu,k})_{k\in \NN}$, $\nu=1,\ldots,n$, 
  gegen $\alpha_\nu$ konvergiert. Können Sie das begründen?
\end{frage}

\begin{antwort}
  Aus 
  \begin{equation}
    \n{ a_k - a }_p = \Big( |\alpha_{1,k}-\alpha_1|^p + 
    \cdots +|\alpha_{n,k}-\alpha_n|^p 
    \Big)^{1/p} 
    < \eps \qquad\text{für alle $k>N$}
    \asttag
  \end{equation}
  folgt $| \alpha_{\nu,k} - \alpha_\nu |^p \le \eps^p $ und damit 
  $| \alpha_{\nu,k} - \alpha_\nu | < \eps $ für alle 
  $k>N$ und $\nu=1,\ldots, n$. Die 
  Komponentenfolgen sind also konvergent.

  Gilt umgekehrt $| \alpha_{\nu,k} - \alpha_\nu | < \eps $ 
  für alle $\nu=1,\ldots, n$ 
  und alle $k>N$, dann folgt $\| a_k-a \|_p \le \sqrt[p]{n}\eps$ für 
  alle $k>N$. Das zeigt die Behauptung auch in der anderen Richtung. 
  \AntEnd
\end{antwort}


%% --- 50 --- %%
\begin{frage}\label{08_folgabg}\index{folgenabgeschlossen}
  Ist $X$ ein metrischer Raum und $A\subset X$ eine Teilmenge. Können Sie 
  begründen, warum $A$ genau dann abgeschlossen in $X$ ist, wenn $A$ 
  \bold{folgenabgeschlossen} ist in dem Sinne, dass für jede Folge 
  $(a_k) \subset A$, die gegen einen Punkt $x\in X$ konvergiert, bereits 
  $x\in A$ gilt? Kurz gesagt, können Sie zeigen, dass eine Teilmenge 
  $A$ eines metrischen Raumes genau dann abgeschlossen ist, 
  wenn sie folgenabgeschlossen ist?
\end{frage}

\begin{antwort}
  Man betrachte einen beliebigen Punkt 
  $x\in X \setminus A$. Ist $A$ folgenabgeschlossen, dann kann nicht 
  jede Umgebung von $x$ Punkte aus $A$ enthalten. Daraus folgt die Offenheit 
  von $X\setminus A$. Aus dieser folgt umgekehrt, dass jeder Punkt aus 
  $X\mengeminus A$ eine Umgebung besitzt, die keine Elemente aus $A$ enthält 
  und damit nicht Grenzwert einer Folge aus $A$ sein kann.
  \AntEnd
\end{antwort}


%% --- 51 --- %%
\begin{frage}\index{Cauchy-Folge!in einem allgemeinen metrischen Raum}
  Wann heißt eine Folge $(x_k)$ eines metrischen Raums $(X,d)$ 
  eine Cauchy-Folge?
\end{frage}

\begin{antwort}
  Eine Folge $(x_k)$ ist eine \slanted{Cauchy-Folge} in $(X,d)$ genau dann, 
  wenn sie die Eigenschaft besitzt, dass für jedes $\eps>0$ ein $N\in \NN$ 
  existiert, sodass gilt:
  \[
  d(x_m,x_n)<\eps \quad\text{für alle $n,m>N$}.
  \EndTag
  \] 
\end{antwort}

%% --- 52 --- %%
\begin{frage}
  Können Sie begründen, warum jede \slanted{konvergente} Folge in einem 
  metrischen Raum eine Cauchy-Folge ist?
\end{frage}

\begin{antwort}
  Sei $x$ der Grenzwert der Folge $(x_k)$. 
  Der Zusammenhang beruht auf der Dreiecksungleichung:
  \begin{equation}
    d(x_m, x_n ) \le d(x_m,x) + d( x_n, x ). 
    \notag
  \end{equation}
  Die beiden hinteren Summen sind beide kleiner als $\eps/2$, wenn 
  $n$ und $m$ beide größer als ein bestimmtes $N\in \NN$ sind.
  \AntEnd
\end{antwort}

%% --- 53 --- %%
\begin{frage}
  Können Sie Beispiele für metrische Räume angeben, für welche die 
  Umkehrung \slanted{nicht} gilt? 
\end{frage} 

\begin{antwort}
  Das erste Beispiel, das einem dafür in der Mathematik begegnet, ist die 
  Menge $\QQ$ der rationalen Zahlen. 
  Bekanntlich (oder wie man in der Antwort zu Frage \ref{02_eirr} 
  nachlesen kann) 
  besitzt etwa die konvergente 
  Folge $(e_n)\subset \QQ$ mit $e_n:= (1+1/n)^n$ keinen Grenzwert in $\QQ$. 

  Ein weiteres Beispiel ist der Raum der Treppenfunktionen 
  auf einem Intervall $[a,b]$ bezüglich der 
  Supremumsnorm. Der Grenzwert einer konvergenten 
  Folge von Treppenfunktionen auf $[a,b]$ ist eine Regelfunktion, also 
  im Allgemeinen keine Treppenfunktion mehr. 
  \AntEnd
\end{antwort}

%% --- 54 --- %%
\begin{frage}\label{09_voll}\index{vollständiger metrischer Raum}
  Was ist ein \bold{vollständiger metrischer Raum}?
\end{frage}

\begin{antwort}
  Ein metrischer Raum $X$ 
  ist vollständig genau dann, wenn er die Eigenschaft besitzt, dass 
  \slanted{jede} Cauchy-Folge aus $X$ einen Grenzwert in $X$ besitzt. 

  Beispielsweise ist $\QQ$ \slanted{nicht} vollständig, $\RR$ aber schon. 
  \AntEnd
\end{antwort}



%% --- 55 --- %%
\begin{frage}\label{q:banachraum}\index{Banachraum}\index{Hilbertraum}
  Was ist ein \bold{Banachraum}, was ein \bold{Hilbertraum}? 
  Können Sie jeweils zwei Beispiele nennen?
\end{frage}

\begin{antwort}
  Banach- und Hilberträume haben gemeinsam, beides 
  $\KK$-Vektorräume zu sein, die mit einer \slanted{Norm} 
  versehen sind und die 
  \slanted{vollständig} im Sinne von Frage \ref{09_voll} sind. 

  \slanted{Hilberträume} besitzen darüber hinaus die Eigenschaft, dass 
  ihre Norm durch ein Skalarprodukt $\langle \;,\; \rangle$ gegeben ist,  
  sie sind also \slanted{euklidische bzw. unitäre Vektorräume}. 

  Der Begriff des Banachraums ist damit allgemeiner. Jeder Hilbertraum ist 
  auch Banachraum, aber nicht umgekehrt. 

  Das naheliegendste Beispiel für einen 
  Hilbertraum ist der $\RR^n$ mit dem Standard\-skalar\-produkt. 
  Speziellere Kandidaten sind etwa der Folgenraum $\ell^2$,  
  der in Frage \ref{09_ell} behandelt wird, 
  oder der in der Theorie der Fourierreihen auftretende  
  Raum der $2\pi$-periodischen $L^2$-Funktionen auf $[0,2\pi]$ mit 
  dem Skalarprodukt 
  \[
  \langle f, g \rangle = \frac{1}{2\pi} \int_{0}^{2\pi} f(x) g(x) \difx.  
  \]
  Dies alles sind auch Beispiele für Banachräume. 
  Banachräume, die keine Hilberträume sind, sind {\zB} der 
  $\RR^n$ versehen mit der Maximumsnorm 
  $\n{\,\;}_\infty$ oder der Raum $\calli{C}([a,b])$ 
  der stetigen Funktionen auf $[a,b]$ bezüglich der Supremumsnorm. 
  Dass diese Räume keine Hilberträume sind, kann man mit dem Ergebnis 
  von Frage \ref{08_par} daran sehen, dass in ihnen die Parallelogrammidentität 
  nicht gilt.
  \AntEnd
\end{antwort}

%% --- 56 --- %%
\begin{frage}\label{09_ell}\index{l@$\ell^2$-Folge}
  \index{Hilbert'scher Folgenraum $\ell^2$}
  \nomenclature{$\ell_2$}{Hilbert'scher Folgenraum}
  \index{Hilbert@\textsc{Hilbert}, David (1862-1943)}
  Warum ist der \bold{Hilbert'sche Folgenraum} $\ell^2$, das ist der 
  Raum aller Folgen $a:=(\alpha_\nu)$ komplexer Zahlen mit 
  \[
  \n{ a }_2 := \left( \sum_{\nu=1}^\infty |\alpha_\nu|^2 \right)^{1/2} < \infty,
  \]
  ein Vektorraum?
\end{frage}

\begin{antwort}
  Seien  
  $a:= (\alpha_\nu)$ und $b := ( \beta_\nu )$ zwei Folgen aus $\ell^2$. 
  Es muss gezeigt werden, dass $\sum_{\nu=1}^\infty 
  | \alpha_\nu + \beta_\nu |^2 < \infty$ gilt (die Abgeschlossenheit 
  des Raumes bezüglich Skalarmultiplikation ist offensichtlich). 
  Mit Binomischer Formel und Dreiecksungleichung erhält man für alle 
  $n\in\NN$ die Abschätzung  
  \[
  \sum_{\nu=1}^n \left| \alpha_\nu + \beta_\nu \right|^2  
  \le 
  \n{ a }_2^2 + 2 \sum_{\nu=1}^n \left| \alpha_\nu \beta_\nu \right| 
  +
  \n{ b }_2^2. 
  \]
  Für die Abschätzung des mittleren Terms 
  beachte man, dass nach der Cauchy-Schwarz'schen Ungleichung 
  für alle $n\in \NN$ gilt: 
  $ \sum_{\nu=1}^n \left| \alpha_\nu \beta_\nu \right|  \le 
  \| a \|_2 \cdot \| b \|_2$. Damit ist 
  $\sum_{\nu=1}^\infty \left| \alpha_\nu + \beta_\nu \right|^2 < \infty$, 
  und folglich ist mit $a$ und $b$ auch die Folge $(a+b)$ 
  ein Element aus dem Folgenraum $\ell^2$, 
  der damit tatsächlich ein $\CC$-Vektorraum ist. 
  Ein Skalarprodukt ist gegeben durch
  \[
  \langle a, b \rangle := \sum_{\nu=1}^\infty \alpha_\nu \overline{\beta}_\nu 
  \EndTag
  \]
\end{antwort}

%% --- 57 --- %%
\begin{frage}\index{Hilbert'scher Folgenraum $\ell^2$}
  Können Sie zeigen, dass $\ell^2$ vollständig ist?
\end{frage}

\begin{antwort}
  Sei $(a_k)_{k\in \NN}$ mit 
  $a_k := (\alpha_{k,\nu})_{\nu\in \NN}$ eine Cauchy-Folge in $\ell^2$ 
  und $N$ so gewählt, dass für $k,l>N$ die Abschätzung 
  \begin{equation}
    \nnb{ a_k - a_l }_2^2  = \sum_{\nu=1}^\infty 
    \left| \alpha_{k,\nu} - \alpha_{l,\nu} \right|^2 < \eps^2
    \asttag
  \end{equation}
  zutrifft. Dann gilt 
  $\left| \alpha_{k,\nu} - \alpha_{l,\nu} \right| < \eps$ 
  für alle $\nu\in \NN$. Die Komponentenfolgen 
  $(\alpha_{k,\nu})_{k\in \NN}$ sind also Cauchy-Folgen, 
  folglich konvergent gegen einen Grenzwert $\alpha_{.,\nu} \in \CC$. 
  Mit $a := ( \alpha_{.,1}, \alpha_{.,2},\ldots )$  folgt damit 
  für  $l\to \infty$ aus {\astref} zunächst für endliche Summen 
  \begin{equation}
    \sum_{\nu=1}^K 
    \left| \alpha_{k,\nu} - \alpha_{.,\nu} \right|^2 \le \eps^2 
    \qquad\text{für alle $k > N$ und alle $K\in \NN$},
    \notag
  \end{equation}
  und daraus schließlich für $K\to\infty$
  \begin{equation}
    \nnb{ a_k - a }_2^2  = \sum_{\nu=1}^\infty 
    \left| \alpha_{k,\nu} - \alpha_{.,\nu} \right|^2 \le \eps^2 
    \qquad\text{für alle $k > N$}.
    \aasttag
  \end{equation}
  Demnach gehört die Folge $a_N-a$ zu $\ell^2$ und damit wegen 
  der Vektorraumeigenschaft von $\ell^2$ auch $a$ selbst. 
  $a$ ist aber nach {\astastref} der Grenzwert der Folge 
  $(a_k)_{k\in \NN}$. Damit ist die Vollständigkeit von $\ell^2$ gezeigt.
  \AntEnd 
\end{antwort}

%% --- 58 --- %%
\begin{frage}\index{Aquivalenz@Äquivalenz!von Normen}
  Wann nennt man zwei Normen auf einem $\KK$-Vektorraum $V$ \bold{äquivalent}?
\end{frage}

\begin{antwort}
  Zwei Normen $\| \;\, \|$ und $\| \;\, \|^*$ 
  heißen \slanted{äquivalent}, wenn es positive reelle Zahlen 
  $c$ und $C$ gibt, sodass für alle $v\in V$ gilt
  \begin{equation}
    c \cdot \|  v \| \le \| v \|^* \le C \cdot \| v \|.
    \notag
  \end{equation}
  Das Ergebnis aus Frage \ref{08_pnormaq} besagt zum Beispiel, dass alle 
  $p$-Normen im $\RR^n$ äquivalent sind. In Frage \ref{09_normaq} wird das
  wichtige Ergebnis bewiesen, dass dies für 
  \slanted{sämtliche} Normen auf einem 
  \slanted{endlichdimensionalen} Vektorraum zutrifft. 
  \AntEnd 
\end{antwort}

%% --- 59 --- %%
\begin{frage}
  Warum ergeben äquivalente Normen auf einem 
  $\KK$-Vektorraum $V$ denselben Konvergenzbegriff?
\end{frage}

\begin{antwort}
  Sei $(a_k)\subset V$ eine Folge und $a$ deren Grenzwert. 
  Gilt bezüglich einer Norm 
  $\lim \n{ a- a_k } = 0$, so 

  gilt wegen $\n{ a-a_k }^* < C \n{ a-a_k }$ 
  auch $\lim \n{ a-a_k }^*=0$ bezüglich einer 
  äquivalenten Norm $\n{ \;\, }^*$. 
  \AntEnd 
\end{antwort}

%% --- 60 --- %%
\begin{frage}
  Warum ist $\KK^n$ für $\KK=\CC$ oder $\KK=\RR$ ein Banachraum?
\end{frage}

\begin{antwort}
  Die Räume $\RR^n$ bzw. $\CC^n$ sind vollständig, weil 
  $\RR$ bzw. $\CC$ vollständig sind und der Grenzwert einer Folge von 
  Vektoren in $\KK^n$ komponentenweise gebildet wird.
  \AntEnd
\end{antwort}

%% --- 61 --- %%
\begin{frage}
  \label{q:61}
  Sei $\calli{C}([a,b])$ der Vektorraum der stetigen Funktionen 
  auf $[a,b]$ mit der Supremumsnorm ${\| \; \, \|}_\infty$ und 
  mit der $1$-Norm ${\| f \|}_1 := \int_a^b \left| f(x) \right| \difx$. 
  Sind diese beiden Normen äquivalent?
\end{frage}

\begin{antwort}
  Die Normen sind nicht äquivalent. Angenommen, 
  es würde $C\cdot {\|f\|}_1 \ge  {\|f\|}_\infty$ 
  für ein $C\in \RR_+$ 
  gelten. Man wähle ein $K>C$ und betrachte
  die stetige Funktion $g \fd [a,b]\to\RR$ mit 
  \[
  g(x) :=  \left\{\begin{array}{ll}
      K^2 \cdot (x-a),   & \text{für $x\in [a,a+\frac{1}{K}]$}  \\
      2K-K^2\cdot (x-a), & \text{für $x\in (a+\frac{1}{K},a+\frac{2}{K}]$} \\
      0, & \text{sonst,}
    \end{array} \right.,
  \]
  \sieheAbbildung\ref{fig:09_spitz}.  Es gilt dann 
  $C\cdot {\| g \|}_1 = C \cdot 1 < K  = {\| g \|}_\infty$, im 
  Widerspruch zur Annahme.
  \AntEnd

  \begin{center}
    \includegraphics{mp/09_spitz}
    \captionof{figure}{Konstruktion der Funktion $g$ aus Frage \ref{q:61}.} 
    \label{fig:09_spitz}
  \end{center}
\end{antwort}

%% --- 62 --- %%
\begin{frage}\index{Schachtelungsprinzip, allgemeines}
  Was besagt das \bold{allgemeine Schachtelungsprinzip} für vollständige 
  metrische Räume?
\end{frage}

\begin{antwort}
  \index{Durchmesser}
  Für die Formulierung des Schachtelungsprinzips ist der Begriff des 
  \slanted{Durchmessers} 
  \[
  \mathrm{diam}(A) = \sup\{ d(a,b) \sets a,b \in A \}
  \]
  einer \slanted{beschränkten} Teilmenge $A$ eines 
  metrischen Raumes $X$ hilfreich. Dabei heißt eine eine Teilmenge 
  $A$ eines metrischen Raumes \slanted{beschränkt}, wenn eine Kugel 
  $U_R \subset X$ mit $A\subset U_R$ existiert.


  Das allgemeine Schachtelungsprinzip besagt: 

  \medskip\noindent
  \slanted{
    Ist $(A_k)$ eine Folge 
    von abgeschlossenen, beschränkten Intervallen in einem \slanted{vollständigen} 
    metrischen Raum $X$, für die $A_{k+1} \subset A_k$ für alle $k\in \NN$ 
    sowie $\mathrm{diam}\,A_k \to 0$ für $k\to\infty$ gilt, dann 
    gibt es genau einen Punkt $a\in X$, der in allen $A_k$ enthalten 
    ist: $a\in \bigcap_{k\in \NN} A_k$.}

  \medskip\noindent
  Der Beweis ist nicht schwierig. 
  Man wähle eine Folge $(a_k)$ von Punkten $a_k \in A_k$. 
  Da die Durchmesser der $A_k$ gegen $0$ konvergieren, gilt 
  $d(a_n,a_m)< \eps$, falls $n$ und $m$ hinreichend groß sind. Somit ist 
  $(a_k)$ eine Cauchy-Folge und besitzt wegen der Vollständigkeit von 
  $X$ einen Grenzwert $a\in X$. 

  Für ein beliebiges $N\in \NN$ gilt $a_n \in A_N$ für alle 
  $n \ge N$. Damit ist $a$ ein Häufungspunkt von $A_N$, und wegen 
  der Abgeschlossenheit von $A_N$ folgt $a \in A_N$. Da das für eine 
  beliebige Wahl von $N$ gilt, ist der Grenzwert $a$ 
  also in allen Mengen $A_k$ enthalten. 
  Das beweist die Existenzbehauptung, die Eindeutigkeit folgt aus 
  $\lim\limits_{k\to\infty} \mathrm{diam}\, A_k=0$.   \AntEnd
\end{antwort}

%% --- 63 --- %%
\begin{frage}\label{09_produktmetrik}\index{Produktmetrik}
  Sind $(X,d_X)$ und $(Y,d_Y)$ metrische Räume. Wie ist die 
  \bold{Produkt-Metrik} auf $X\times Y$ definiert?
\end{frage}

\begin{antwort}
  Seien $(x_1,y_1)$ und $(x_2,y_2)$ Elemente aus $X\times Y$.  
  Die Produktmetrik $d$ ist gegeben durch 
  \[
  d \big( (x_1, y_1 ), (x_2,y_2) \big) := 
  \max\{ d_X(x_1, x_2), d_Y( y_1, y_2 ) \}.
  \]
  Man prüft ohne Probleme nach, dass die so definierte Funktion 
  alle drei Eigenschaften einer Metrik besitzt. 


  Die universelle Eigenschaft der Produktmetrik besteht darin, dass 
  eine Abbildung $F \fd Z\to X\times Y$ von einem beliebigen 
  metrischen Raum 
  $Z$ bezüglich der Produktmetrik genau dann stetig ist, wenn die 
  Zusammensetzungen mit den Projektionen, also die beiden 
  Abbildungen $\mathrm{pr}_1\circ F$ und 
  $\mathrm{pr}_2\circ F$ stetig sind. 

  \begin{center}
    \includegraphics{mp/09_produktmetrik}
    \captionof{figure}{Offene Kugel $U_\eps \big( (x.y) \big)$ in der 
      Produktmetrik.}
    \label{fig:09_produktmetrik}
  \end{center}

  Die offene Kugel $U_\eps \big( (x.y) \big)$ um ein Element 
  $(x,y)\in X\times Y$ ist bezüglich der Produktmetrik damit 
  gerade das direkte Produkt der offenen Kugeln 
  $U_\eps(x) \subset X$ und $U_\eps( y ) \subset Y$, \sieheAbbildung\ref{fig:09_produktmetrik}.
  \AntEnd
\end{antwort}

%% --- 64 --- %%
\begin{frage}\label{09_banach}\index{Banachscher Fixpunksatz@Banach\sch er Fixpunktsatz}
  \index{Banach@\textsc{Banach}, Stefan (1892-1945)}
  \index{Fixpunkt}
  \index{Kontraktion}
  \index{kontrahierende Selbstabbildung}
  Was besagt der \bold{Banach\sch e Fixpunktsatz} für  
  vollständige metrische Räume?
\end{frage}

\begin{antwort}
  Der Satz besagt:

  \medskip\noindent
  \slanted{Ist $M$ eine nichtleere abgeschlossene Teilmenge eines 
    vollständigen metrischen Raumes und $\varphi \fd M\to M$ eine 
    kontrahierende Selbstabbildung, dann besitzt $\varphi$ genau 
    einen Fixpunkt, also genau einen Punkt $\xi\in M$ mit 
    $\varphi(\xi)=\xi$. Für jeden Startwert $x_0\in M$ konvergiert die 
    durch  
    \[
    x_{n+1} = \varphi( x_n) 
    \]
    rekursiv definierte Folge $(x_n)$ gegen $\xi$. 
  }

  \medskip
  \noindent
  Der Beweis ist sehr einfach. Sei $x_0$ irgendein Punkt aus $M$.  
  Mit der Kontraktionskonstanten $\lambda<1$ erhält man wegen 
  $d\big( \varphi(x), \varphi(y) \big) \le \lambda d( x,y )$ für alle $x,y\in M$ 
  induktiv
  \[
  d (x_n,x_{n+1}) \le \lambda^n d ( x_0, x_1 ).
  \]
  Es folgt mit $C:=d(x_0,x_1)$
  \[
  d\big( x_n, x_{n+k} ) \le 
  C\cdot (\lambda^n+\lambda^{n+1} +\cdots + \lambda^{n+k-1})
  \le C\cdot\frac{\lambda^n}{1-\lambda}.
  \]
  Der Term auf der rechten konvergiert für $n\to\infty$ gegen Null, die Folge $(x_n)$ ist also eine Cauchy-Folge. 
  Da $M$ als Teilraum eines vollständigen metrischen Raumes 
  selbst vollständig ist (vgl. Frage \ref{09_vollstteilraum}), 
  existiert dazu ein Grenzwert 
  $\xi\in M$. Dieser ist 
  ein Fixpunkt von $\varphi$. Das folgt aus der Stetigkeit von $\varphi$:
  \[
  \varphi( \xi) = \lim_{n\to\infty} \varphi( x_n )= \lim_{n\to\infty} 
  x_{n+1} = \xi.
  \]
  Ferner ist $\xi$ der einzige Fixpunkt. Denn für einen 
  weiteren Fixpunkt $\eta$ gälte
  \[
  d( \xi, \eta ) = d \big( \varphi(\xi), \varphi(\eta) \big) \le 
  \lambda d( \xi,\eta ),\nodpagebreak
  \]
  wegen $\lambda<  1$ also $d(\xi,\eta)=0$. \AntEnd 

  \begin{center}
    \includegraphics{mp/09_banach}
    \captionof{figure}{Die Folge 
      $\varphi(M),\varphi\left(\varphi(M)\right),\ldots$ 
      konvergiert gegen einen Fixpunkt.}
    \label{fig:09_banach}
  \end{center}
\end{antwort} 

%% --- 65 --- %%
\begin{frage}
  Kennen Sie eine Anwendung des Banach\sch en Fixpunktsatzes?
\end{frage}

\begin{antwort}
  Der Banach\sch e Fixpunktsatz wird beispielsweise 
  an einer zentralen Stelle  
  beim Beweis des \slanted{lokalen Umkehrsatzes} verwendet 
  (s. Kapitel \ref{umkehrsatz}, insbesondere Frage \ref{10_umkbanach}). 
  Die gesuchte Umkehrfunktion wird dort als Lösung einer 
  Fixpunktgleichung konstruiert. \index{lokaler Umkehrsatz}

  \index{Differenzialgleichung!lokaler Existenzsatz}
  \index{Picard-Lindelöf, lokaler Existenzsatz}
  \index{Picard@\textsc{Picard}, Charles-Emile (1856-1941)}
  \index{Lindelof@\textsc{Lindelöf}, Ernst Leonard (1870-1946)}
  Ein weitere interessante Anwendung findet sich beim 
  Beweis der Existenz von Lösungen gewöhnlicher Differenzialgleichungen, 
  speziell dem Beweis des Satzes von \slanted{Picard-Lindelöf}. 
  Dabei geht es um die folgende Fragestellung:  
  Sei $F \fd U\to \RR^n$ eine Funktion auf einer Menge 
  $U\subset [a,b] \times V$ mit $V\subset\RR^n$. 
  Gesucht ist eine Lösung der Differenzialgleichung 
  \[
  x'(t) = F \big( t, x(t) \big)  
  \]
  Damit eine Lösung zu einem gegebenen 
  Anfangswert existiert, muss $F$ nur die Bedingung erfüllen, 
  \slanted{lokal Lipschitz-stetig}\index{lokal Lipschitz-stetig}
  bezüglich $x$ zu sein, das heißt, dass es zu jedem Punkt 
  $x_0\in V$ eine Umgebung 
  $V_0\subset V$ gibt, für die eine Konstante $L>0$ existiert, 
  sodass für alle $x_1,x_2\in V_0$ und alle 
  $t\in[a,b]$ gilt: $
  \nb{ F(t,x_1)-F(t,x_2) } \le L \cdot \nb{ x_1-x_2}$.
  Mit diesen Bezeichnungen lautet der Satz von Picard-Lindelöf: 

  \medskip\noindent
  \satz{Sei $F\fd U\to  \RR^n$ mit 
    $U \subset [a,b]\times \RR^n$ eine Funktion, die 
    lokal Lipschitz-stetig bezüglich $x$ ist. Dann gibt es 
    zu jedem Punkt $(x_0,t_0)\in U$ ein Intervall 
    $I_\delta:=\open{t_0-\delta,t_0+\delta}$, auf dem das
    Anfangswertproblem 
    \[
    x'(t) = F\big(t,x(t)\big), \qquad x(t_0)=x_0\asttag
    \]
    genau eine Lösung besitzt.}  

  \medskip\noindent
  Um eine eventuelle Lösung mithilfe des Banach\sch en Fixpunktsatzes 
  zu finden, müssen wir die Suche in einem 
  ersten Schritt auf einen \slanted{vollständigen}  
  metrischen Raum $\calli{Q}$ stetiger Funktionen einschränken. 
  Sei dazu $C$ irgendeine Zahl, sodass 
  $F$ auf $M:=[a,b]\times \overline{K}_C (x_0)$ Lipschitz-stetig 
  bezüglich $x$ mit der Konstanten $L$ ist. 
  Sei $\delta > 0$ dazu so gewählt, dass gilt: 
  \[
  I_\delta := \open{ t_0-\delta, t_0+\delta } 
  \subset [a,b],\qquad
  \delta \n{ F }_M \le C, \qquad 
  \delta L < 1. 
  \]
  Den Raum $\calli{Q}$, in dem eine Lösung gesucht wird, definieren wir 
  durch 
  \[
  \calli{Q} := \big\{ 
  \psi \fd I_\delta\to\RR^n\sets \text{$\psi$ stetig}, \; 
  \nnb{ \psi(t)-x_0 } \le b \, \text{ für alle $t\in I_\delta$}
  \big\} 
  \tag{$\dagger$}
  \]
  Um $\calli{Q}$ zu einem vollständigen metrischen Raum zu 
  machen, führen wir darauf die von der Supremumsnorm im $\RR^n$ 
  induzierte Metrik ein, definieren also 
  \[
  d( \psi_1, \psi_2 ) := \sup \,\big\{ 
  \nnb{\psi_1(t)-\psi_2(t) } \sets t\in I_\delta \big\}.
  \]
  Die Cauchy-Folgen in $\calli{Q}$ bezüglich $d$ sind dann 
  genau die gleichmäßig konvergenten Folgen. Die Grenzfunktion 
  ist in diesem Fall ebenfalls 
  stetig und erfüllt aus Monotoniegründen 
  die Ungleichung $\dagger$, 
  also ist $(\calli{Q}, d )$ ein vollständiger metrischer Raum. 

  Um eine Lösung der Differenzialgleichung {\astref} zu finden, 
  genügt es nun, eine Funktion $\varphi$ zu konstruieren, die die 
  Integralgleichung  
  \[
  \varphi( t ) = x_0 + \int_{t_0}^t F\big(u, \varphi(u)\big) \difu
  \aasttag
  \]
  löst. Wir formulieren diese Gleichung zu einer 
  Fixpunktgleichung bezüglich einer Abbildung 
  $T \fd \calli{Q}\to\calli{Q}$ um. 
  Für $\psi \in\calli{Q}$ sei 
  $T\psi\fd I_\delta \to \RR^n$ definiert durch 
  \[
  T\psi(t) := x_0 + \int_{t_0}^t F\big( u, \psi(u) \big) \difu. 
  \]
  $T\psi$ ist dann stetig und damit $T(T\psi), T(T(T\psi)),\ldots$ 
  wohldefiniert.
  
  Nun ist jede Lösung der Fixpunktgleichung $T\varphi = \varphi$ eine 
  Lösung der Differenzialgleichung {\astref}. Aufgrund des 
  Banach\sch en Fixpunktsatzes 
  genügt es also, zu zeigen, dass die Abbildung $T$ eine 
  kontrahierende Selbstabbildung 
  $\calli{Q}\to\calli{Q}$ induziert. Das ist nun, nach all den 
  Vorbereitungen, einfach geradlinig zu verifizieren:  
  Einerseits ist
  \[
  \nnb{ T\psi (t) - x_0 } = 
  \nnb{ \int_t^{t_0} F \big( u, \psi(u) \big) \difu } \le 
  \left| \int_t^{t_0} \nnb{F \big(u,\psi(u) \big)}\difu \right|
  \le  \delta \nnb{F}_M \le C.
  \]
  Für $\psi\in\calli{Q}$ liegt also $T\psi$ ebenfalls in $\calli{Q}$. 
  Damit ist $T$ eine Selbstabbildung $\calli{Q}\to\calli{Q}$. 
  Ferner ist $T$ eine Kontraktion, da gilt:
  \begin{align*}
    d\big( T\psi_1, T\psi_2 \big) &= 
    \sup_{I_\delta} \nnb{ 
      \int_{t_0}^t \Big( F\big( u, \psi_1(u) \big) -  
      F\big( u, \psi_2(u) \big) \Big) \difu } \\
    &\le 
    \sup_{I_\delta} \left| 
      \int_{t_0}^t \nnb{ F\big( u, \psi_1(u) \big) -  
        F\big( u, \psi_2(u) \big) } \difu 
    \right| \\
    &\le 
    \sup_{I_\delta} \left| 
      \int_{t_0}^t L \nnb{ \psi_1(u)  - \psi_2(u) } \difu 
    \right| \le \delta L \cdot d(\psi_1,\psi_2). 
  \end{align*} 
  Nach Voraussetzung ist $\delta L <1$, $T$ also eine kontrahierende 
  Selbstabbildung von $\calli{Q}$. Die nach dem Banach\sch en 
  Fixpunktsatz existierende Lösung der Fixpunktgleichung 
  $T\psi = \psi$ ist die gesuchte Lösung der Differenzialgleichung 
  {\astref}. \AntEnd
\end{antwort} 

%% --- 66 --- %%
\begin{frage}\index{Anfangswertproblem}
  Können Sie mithilfe des Iterationsverfahrens 
  von Picard-Lindelöf die Differenzialgleichung 
  (das Anfangswertproblem) $x'=tx$ f\"ur $x(0)=1$ 
  lösen?
\end{frage}

\begin{antwort}
  Als Startwert für die Picard-Lindelöf'sche Iteration 
  wählen wir die Funktion $x_0(t)\equiv 1$. Daraus erhält man 
  nach den ersten zwei Schritten
  \begin{align*}
    x_1(t) := Tx_0(t) &= 1 + \int_0^t u x_0(u) \difu =1+\frac{t^2}{2},\\
    x_2(t) := Tx_1(t) &= 1 + \int_0^t u x_1(u) \difu 
    =x_1(t)+ \int_0^t \frac{u^3}{2} \difu = 
    1+\frac{t^2}{2}+\frac{t^4}{8},  
  \end{align*}
  und durch $n$-malige Wiederholung dieser Rechnung 
  \[
  x_n(t) := Tx_{n-1}(t) = x_{n-1}(t)+\int_0^t 
  \frac{u^{2n-1}}{2\cdot 4\cdots (2n-2)} \difu = 
  1+\sum_{k=1}^n \frac{t^{2k}}{2^k k!}.    
  \]
  Die Lösung des Anfangswertproblems lautet damit
  \[
  x(t) = \lim_{n\to\infty} x_n(t) = \sum_{k=0}^\infty  
  \frac{t^{2k}}{2^k k!}.\EndTag
  \]
\end{antwort}  



\section{Stetigkeit, gleichmäßige Konvergenz, stetige 
  Fortsetzbarkeit, Grenzwerte}

%% --- 67 --- %%
\begin{frage}
  \index{Stetigkeit!in allgemeinen metrischen Räumen}
  \index{Stetigkeit!gleichmäßige}
  Sind $(X,d_X)$ und $(Y,d_Y)$ metrische Räume, $D\subset X$ eine 
  nichtleere Teilmenge und $f\fd D\to Y$ eine Abbildung. Wann heißt $f$
  {\setlength{\labelsep}{4mm}
    \begin{itemize}
    \item[\desc{a}] \bold{stetig in einem Punkt} $\mathbf{a\in D}$,\\[-3mm]
    \item[\desc{b}] \bold{stetig auf $\mathbf{D}$},\\[-3mm]
    \item[\desc{c}] \bold{gleichmäßig stetig} auf $\mathbf{D}$?
    \end{itemize}}
\end{frage}

\begin{antwort}
  Diese Begriffe wurden für reelle Funktionen bereits erklärt 
  und bringen auch im Zusammenhang allgemeiner metrischer Räume 
  keine konzeptionellen Neuerungen mit sich. Die Definitionen 
  bleiben dieselben. Zur Wiederholung: 
  {\setlength{\labelsep}{4mm}
    \begin{itemize}
    \item[\desc{a}] $f$ ist \slanted{stetig in $a\in D$}, wenn für 
      jedes $\eps>0$ ein $\delta>0$ existiert, sodass aus 
      $d_X(a,x) < \delta$ stets $d_Y\big( f(a), f(x) \big)<\eps$ 
      folgt. Das ist die adäquate Formulierung der zugrunde liegenden 
      Idee, dass eine "`kleine"' Veränderung des Arguments $a$ auch nur  
      eine "`kleine"' Änderung des Funktionswerts bewirkt. \\[-3mm] 
    \item[\desc{b}] $f$ ist \slanted{stetig auf $D$}, wenn in $f$ in jedem 
      Punkt $a\in D$ stetig ist.\\[-3mm]
    \item[\desc{c}] $f$ ist \slanted{gleichmäßig stetig auf $D$}, wenn für 
      \slanted{jedes} $\eps>0$ ein $\delta >0$ existiert, so dass 
      für alle $x,y\in D$ gilt:
      \begin{equation}
        d_X(x,y) < \delta \Ra d_Y \big( f(x),f(y) \big) < \eps  
        \notag
      \end{equation}
      Die Zahl $\delta$ ist im Fall gleichmäßiger Stetigkeit also nicht abhängig 
      vom jeweiligen Stetigkeitspunkt, sondern gilt global auf ganz $X$. \AntEnd
    \end{itemize}}
\end{antwort}

%% --- 68 --- %%
\begin{frage}\index{Folgenstetigkeit}
  Können Sie den Begriff "`Folgenstetigkeit"' erläutern? 
\end{frage}

\begin{antwort}
  Die Funktion $f\fd D\to Y$ heißt \bold{folgenstetig} 
  in $a\in D$, wenn für jede gegen $a$ konvergente  
  Folge $(a_n)$ mit $a_n \in D$ gilt: 
  \[
  \lim_{n\to\infty} f(a_n)=f(a).   \EndTag
  \]
\end{antwort}

%% --- 69 --- %%
\begin{frage}
  Warum sind Folgenstetigkeit und $\eps\delta$-Stetigkeit 
  für einen metrischen Raum äquivalente Eigenschaften? 
\end{frage}

\begin{antwort} 
  Die Äquivalenz beider Formulierung erhält man, wenn man den 
  Konvergenzbegriff in metrische Terminologie übersetzt. 
  Der Ausdruck $\lim a_n = a $ ist dann so zu verstehen, dass für jedes 
  $\delta>0$ ab einem bestimmten Index $N$ alle Folgenglieder in 
  $U_\delta(a)$ liegen. Wählt man für eine stetige Funktion $f$ zu 
  einem vorgegebenen $\eps$ speziell $\delta$ so, 
  wie es in der $\eps\delta$-Definition der 
  Stetigkeit gefordert ist, so gilt damit:
  \[
  f(a_n) \subset U_\eps\big( f(a) \big) \quad \text{für alle $n>N$}. 
  \]
  Da $\eps$ beliebig klein sein kann, folgt daraus 
  $\lim f(a_n) = f(a)$. 

  Gelte nun umgekehrt $\lim f(a_n)=f(a)$ für jede 
  gegen $a$ konvergente Folge $(a_n)\subset D$. 
  Angenommen, $f$ erfüllt in $a$ 
  nicht die $\eps\delta$-Definition der Stetigkeit. 
  Dann gibt es ein $\eps_0>0$ mit der Eigenschaft, dass in jeder 
  Umgebung von $a$ ein Punkt $a^*$ liegt, sodass $f(a^*)$ nicht 
  in $U_{\eps_0}\big( f(a) \big)$ enthalten ist. Insbesondere 
  existiert zu jedem $n\in\NN$ ein Punkt $a_n^*$ mit 
  \[
  a^*_n \in U_{1/n}(a), \qquad 
  f(a^*_n) \not \in U_{\eps_0}( a ).
  \]
  Die Folge $(a^*_n)$ konvergiert gegen $a$, aber keines 
  ihrer Glieder ist in $U_{\eps_0}\big( f(a) \big)$ enthalten. 
  Somit kann auch nicht 
  $\lim f(a^*_n) = f(a)$ gelten, im Widerspruch zur Voraussetzung. 
  \AntEnd
\end{antwort}

%% --- 70 --- %%
\begin{frage}\label{09_zusammensetzung}
  \index{Stetigkeit!zusammengesetzter Funktionen}
  Können Sie das Schlagwort "`Die Zusammensetzung stetiger Funktionen 
  ist stetig"' erläutern und begründen?
\end{frage}

\begin{antwort}
  Die Aussage bedeutet, dass für zwei stetige Funktionen 
  $g\fd D \to X$ und $f\fd X\to Y$ die zusammengesetzte Funktion 
  $f\circ g \fd D \to Y$ ebenfalls stetig auf $D$ ist. 

  Das folgt {\zB} unmittelbar aus dem Folgenkriterium. Ist $a_n$ eine 
  Folge in $D$ mit $\lim a_n=a$, dann gilt nach Voraussetzung 
  $\lim g(a_n)=g(a)$, und daraus folgt wegen der Stetigkeit wiederum 
  $\lim f\big(g(a_n)\big)=f\big(g(a)\big)$
  \AntEnd
\end{antwort}



%% --- 71 --- %%
\begin{frage}\label{09_komponenten}
  \index{Stetigkeit!von Abbildungen in den 
    $\RR^n$}
  Warum ist für einen beliebigen metrischen Raum $X$ eine Abbildung 
  $f\fd X\to \RR^n$ 
  mit $f(x) := \big( f_1(x),\ldots, f_n(x) \big)$ 
  genau dann stetig, wenn die Komponentenfunktionen 
  $f_\nu \fd X \to \RR$ für alle $\nu \in \{1,\ldots, n \}$ 
  stetig sind?
\end{frage}

\begin{antwort}
  Das hängt damit zusammen, dass der Grenzwert einer 
  Folge im $\RR^n$ komponentenweise gebildet wird. Genauer: 
  Konvergieren für eine Folge $(x_k)\subset X$ 
  die Folgen $\big(f_\nu (x_k) \big)$ gegen den Wert $f_\nu (x)$ 
  ($\nu=1,\ldots,n$), so gilt 
  \begin{equation}
    \lim_{k\to\infty} f(x_k)= 
    \big( f_1(x), \ldots, f_n(x) \big) = f(x). 
    \asttag
  \end{equation}
  Sind die Komponentenfunktionen stetig, dann gilt 
  für jede gegen $x$ konvergente Folge $(x_k)$: 
  $\lim_{k\to\infty} f_\nu(x_k) = f_\nu(x)$ und somit wegen {\astref}  
  $\lim_{k\to\infty} f (x_k) = f (x)$.  
  Nach dem Folgenkriterium ist $f$ also stetig.
  \AntEnd
\end{antwort}

%% --- 72 --- %%
\begin{frage}\label{09_addstet}
  \index{Stetigkeit!der Addition, Multiplikation, Division}
  Können Sie begründen, warum die folgenden Abbildungen 
  stetig sind:
  \[
  \begin{array}{ll}
    \mathrm{add}\fd \RR\times \RR, & (x,y) \mapsto x+y,  \\
    \mathrm{mult}\fd \RR\times \RR, & (x,y) \mapsto x\cdot y, \\ 
    \mathrm{quot}\fd \RR\times \RR^*, &  (x,y) \mapsto x/y.
  \end{array}
  \]
\end{frage}

\begin{antwort}
  Ist $\big( (x_n,y_n) \big)$ eine Folge in $\RR\times\RR$, die 
  gegen $(x,y)$ konvergiert, dann gilt $x_n \to x $ und $y_n \to y$, und 
  daraus folgt mit den Permanenzeigenschaften für reelle Zahlenfolgen 
  (vgl. Frage \ref{02_freg}):
  \[
  \lim_{n\to\infty} (x_n + y_n)=x+y, \qquad
  \lim_{n\to\infty} x_n \cdot y_n = x\cdot y, \qquad
  \lim_{n\to\infty} x_n / y_n = x/y.
  \]
  Daraus folgt die Stetigkeit der drei Abbildungen 
  mithilfe des Folgenkriteriums. 

  (Im Fall des Quotienten schränkt man die Folge $\left( (x_n,y_n) \right)$ 
  auf eine Umgebung von $(x,y)$ ein, in der die $y_n$ nicht Null sind, 
  die im Fall $y\not=0$ existiert). \AntEnd
\end{antwort}

%% --- 73 --- %%
\begin{frage}\index{Stetigkeit!von Monomen}
  Warum sind Monome, {\dasheisst} Abbildungen 
  \[
  \RR^n \to \RR; \qquad (x_1, \ldots, x_n ) 
  \mapsto x_1^{k_1}x_2^{k_2}  \cdots x_n^{k_n} 
  \qquad\text{mit $k_1,\ldots,k_n \in \NN_0$}
  \]
  stetige Funktionen (und damit auch Polynome)?  
\end{frage}

\begin{antwort}
  Induktiv zeigt man zunächst, dass die Abbildung 
  $x\mapsto x^m$ für jedes $m\in \NN$ stetig ist. 
  Für $m=0$ ist das offensichtlich. 
  Ist es für $m-1$ bereits gezeigt, so ist die Abbildung 
  \begin{equation}
    \RR \to \RR\times \RR \stackrel{\mathrm{mult}}{\to} \RR, 
    \quad\text{mit}\quad
    x \mapsto (x,x^{m-1}) \mapsto x\cdot x^{m-1} = x^m 
    \asttag
  \end{equation}
  eine Verkettung stetiger Abbildungen und damit nach Frage 
  \ref{09_zusammensetzung} stetig. 

  Die Stetigkeit der Abbildung 
  \begin{equation}
    \varphi\fd \RR^n\to \RR; \qquad (x_1,\ldots,x_n) \mapsto x_1\cdots x_n
    \aasttag
  \end{equation}
  zeigt man ebenfalls induktiv. Dabei 
  bezieht man sich im Induktionsschritt auf die Abbildung  
  \[
  \varphi_{n-1} \fd \RR^n \to \RR^{n-1}; \qquad \big(x_1,\ldots,x_n\big) \mapsto 
  \big( x_1,\ldots, x_{n-2}, \mathrm{mult}\,(x_{n-1},x_n) \big),
  \]
  die, wie wir aufgrund von Frage 
  \ref{09_komponenten} und \ref{09_addstet} bereits wissen, stetig 
  ist. Daraus folgt dann die Stetigkeit von $\varphi=\varphi_1 
  \circ \cdots \circ \varphi_{n-1}$.  

  Schließlich folgt die Stetigkeit der Monome durch Zusammenfassung 
  und Einsetzung der Abbildungen in {\astref} und {\astastref} sowie 
  einer wiederholten 
  Anwendung der in Frage 
  \ref{09_zusammensetzung} und \ref{09_komponenten} bewiesenen Zusammenhänge.
  \AntEnd
\end{antwort}

%% --- 74 --- %%
\begin{frage}\index{Homöomorphismus}\index{homöomorph}
  Wann heißen zwei metrische Räume \bold{homöomorph} 
  (oder \bold{topologisch äquivalent})?
\end{frage}

\begin{antwort}
  Zwei topologische Räume $X$ und $Y$ heißen homöomorph, wenn ein 
  \slanted{Homöomorphismus} zwischen ihnen existiert, {\dasist} eine 
  \slanted{bijektive} Abbildung $f \fd X \to Y$ derart, dass sowohl 
  $f$ als auch $f^{-1}$ stetig sind. 
  \AntEnd
\end{antwort}

%% --- 75 --- %%
\begin{frage}\label{09_homokugel}
  Können Sie begründen, warum der $\RR^n$ und die offene euklidische 
  Kugel 
  \[
  U_1 (0 ) := \{ x\in \RR^n \sets {\| x \|}_2 < 1 \}
  \]
  homöomorph sind?
\end{frage}

\begin{antwort}
  Ein Homöomorphismus ist gegeben durch die Abbildung $f$ (und 
  ihre Umkehrabbildung $f^{-1}$) mit   
  \[
  f \fd \RR^n \to U_1(0); \quad 
  x \mapsto \frac{x}{1+{\| x\|}_2 }, \qquad
  f^{-1} \fd U_1(0) \to \RR^n; \quad 
  x \mapsto \frac{x}{1-{\| x\|}_2 }.  
  \]
  Beide Abbildungen sind stetig, da das für die Identität, 
  die Normfunktion, die Division in $\RR$ 
  und die Multiplikation mit einem Skalar in $\RR^n$ gilt.
  \AntEnd
\end{antwort}

%% --- 76 --- %%
\begin{frage}\index{Konvergenz!gleichmäßige} 
  Ist $X$ eine beliebige nichtleere Menge und $(Y,d_Y)$ ein metrischer Raum, 
  wann heißt eine Folge $(f_n)$ von Abbildungen $f_n \fd X\to Y$ 
  \slanted{gleichmäßig konvergent} gegen eine Abbildung 
  $f \fd X\to Y$?
\end{frage}

\begin{antwort}
  Die Folge von Abbildungen ist genau dann \slanted{gleichmäßig konvergent} 
  gegen die Grenzfunktion $f$,   
  wenn es zu jedem $\eps >0$ einen Index $N\in \NN$ gibt, 
  sodass gilt
  \[
  d_Y \big( f_n(x), f (x) \big) < \eps 
  \quad\text{\slanted{für alle} $x\in X$ und alle $n>N$}.
  \]
  (Im Unterschied dazu ist bei nur punktweiser Konvergenz die 
  Schranke $N$ von $x$ abhängig.) \AntEnd
\end{antwort}

%% --- 77 --- %%
\begin{frage}\index{Stetigkeit!von Grenzfunktionen}
  Sind $X$ und $Y$ metrische Räume und ist $(f_k)$ eine Folge 
  von \slanted{stetigen} Abbildungen $f_n \fd X\to Y$, 
  die \slanted{gleichmäßig} gegen die Abbildung $f \fd X\to Y$ 
  konvergiert. Warum ist dann die Grenzfunktion ebenfalls wieder stetig?
\end{frage}

\begin{antwort}
  Sei $a$ ein beliebiges Element aus $X$. 
  Mit der Dreiecksungleichung bekommt man 
  die für alle $x \in X$ und $n \in \NN$ 
  gültige Abschätzung 
  \begin{equation}
    d \big( f(a), f(x) \big)\le
    d \big( f(a), f_n(a) \big) + 
    d \big( f_n(a), f_n(x) \big) + 
    d \big( f_n(x), f(x) \big).
    \asttag  
  \end{equation}
  Wegen der gleichmäßigen Konvergenz der Folge $(f_k)$ lässt sich 
  $n$ nun speziell so wählen, dass der erste und letzte Summand in der 
  rechten Summe jeweils kleiner als $\eps/3$ sind. Man fixiere 
  ein $N\in\NN$ mit dieser Eigenschaft. Aufgrund der Stetigkeit 
  von $f_N$ in $a$ gibt es dann ein $\delta>0$ derart, 
  dass der mittlere Summand für $n=N$ ebenfalls kleiner als $\eps/3$ ist, 
  falls $x\in U_\delta( a )$ gilt. 
  Mit $n=N$ folgt damit aus {\astref} 
  \[
  d \big( f(a), f(x) \big) < \eps \quad \text{für alle $x\in K_\delta(a)$},
  \]
  also die Stetigkeit von $f$ in $a$. 
  \AntEnd
\end{antwort}

%% --- 78 --- %%
\begin{frage}\heavy\index{Fortsetzungslemma von Tietze}
  \index{Tietze@\textsc{Tietze}, Heinrich Franz Friedrich (1880-1964)}
  Was besagt das \bold{Fortsetzungslemma von Tietze}?
\end{frage}

\begin{antwort}
  Das Fortzsetzungslemma von Tietze ist ein wichtiges Hilfsmittel 
  bei der Konstruktion stetiger Funktionen auf metrischen Räumen, deren 
  Verhalten lokal (also auf bestimmten Teilmengen des metrischen 
  Raumes) vorgegeben ist. Es besagt:


  \medskip\noindent
  \slanted{Jede stetige Funktion 
    $f \fd A\to \RR$ auf einer \slanted{abgeschlossenen} 
    Teilmenge $A$ eines metrischen Raumes $X$ kann zu einer 
    stetigen Funktion $F\fd X\to \RR$ fortgesetzt 
    werden.}

  \medskip\noindent
  Wir geben hier nur eine Beweisskizze (für einen vollständigen 
  Beweis s. \citep{Koenig}). 
  Man beweist das Fortsetzungslemma durch die Konstruktion einer 
  gleichmäßig konvergenten Folge $(f_k)$ von auf $X$ 
  stetigen Funktionen, die auf $A$ punktweise gegen $f$ konvergiert. 

  Entscheidend dabei ist, dass es zu jeder stetigen 
  Funktion $g\fd A\to \RR$ eine stetige "`Näherungsfunktion"' 
  $v \fd X\to \RR$ gibt, für die gilt: 
  \begin{align}
    |v(x)|  &\le \frac{1}{3} \cdot \sup_{x\in A} |g(x)| 
    \qquad\text{ für alle $x\in X$ und } \asttag  \\
    |v(x)-g(x)| &\le \frac{2}{3} \cdot \sup_{x\in A} 
    |g(x)| \qquad \text{ für alle $x\in A$,} \aasttag 
  \end{align}
  \sieheAbbildung\ref{fig:09_tietze}.


  \begin{center}
    \includegraphics{mp/09_tietze}
    \captionof{figure}{Die Näherungsfunktion $v$ zu $g$ im Beweis 
      des Fortsetzungslemmas von Tietze.}
    \label{fig:09_tietze}
  \end{center}

  Die Existenz einer Funktion $v$ mit diesen Eigenschaften  
  folgt für metrische Räume direkt aus dem 
  \slanted{Urysohn'schen Lemma}, sie lässt sich aber 
  auch ohne Weiteres explizit angeben.  

  Die gesuchte Funktionenfolge $(f_k)$ kann man mit 
  diesem Hilfsmittel nun induktiv konstruieren. 
  Man beginne mit $f_0:=0$. Ist $f_k$ 
  bereits gegeben, so finde man eine Näherungsfunktion $v_k$ zu 
  der "`Fehlerfunktion"' $g_k := f-f_k$ 
  mit den Eigenschaften {\astref} und {\astastref} und setze 
  $f_{k+1}:= f_k + v_k$. Dann gilt für jedes $k\in\NN$ mit 
  $a:= \sup_{x\in A} |f(x)|$ 
  \begin{gather}
    | f(x)-f_{k+1}(x) | = | f(x)-f_{k}(x)- v_k (x) | \le 
    a \left( \frac{2}{3} \right)^{k+1} \quad\text{für alle $x\in A$ sowie } 
    \notag\\
    | v_k(x) | \le  a \cdot \frac{1}{3} \left( \frac{2}{3} \right)^k
    \quad\text{für alle $x\in X$.}\notag
  \end{gather}
  Man prüft dann leicht nach, dass die Folge $(f_k)$   
  auf $X$ gleichmäßig konvergiert, und dass 
  deren Grenzfunktion $F$ auf $A$ mit $f$ übereinstimmt.  
  Die Funktion $F$ ist dann die gesuchte stetige Fortsetzung von $f$.  
  \AntEnd  
\end{antwort}

%% --- 79 --- %%
\begin{frage}\label{09_op}\index{lineare Abbildung!Stetigkeit}
  Können Sie zeigen, dass für eine \slanted{lineare Abbildung} 
  $L \fd V \to W$ von normierten $\KK$-Vektorräumen $V$ und $W$ die 
  folgenden Aussagen äquivalent sind:
  {\setlength{\labelsep}{4mm}
    \begin{itemize}
    \item[\desc{a}] Es gibt ein $C\ge 0$, sodass für alle 
      $x\in V$ gilt: 
      \[
      {\| L(x) \|}_W \le C\cdot {\| x \| }_V,
      \]
      wobei ${\|\;\,\|}_W$ und ${\| \;\, \|}_V$ die Normen 
      auf $V$ bzw. $W$ bezeichnen. 
    \item[\desc{b}] $L$ ist gleichmäßig stetig (also insbesondere 
      stetig) auf $V$, \\[-3mm]
    \item[\desc{c}] $L$ ist stetig in $0$, \\[-3mm]
    \item[\desc{d}] Es ist $
      { \| L \| }_{V,W} := \sup\{ {\| L(x) \|}_W \sets {\| x \|}_V \le 1 \} 
      < \infty.$
    \end{itemize}}
\end{frage}

\begin{antwort}
  \desc{a} $\Ra$ \desc{b}: Ist $\delta<\eps/C$, so gilt wegen der 
  Linearität von $L$ für alle $x,y$ mit ${\| x-y \|}_V < \delta$: 
  \[
  {\| L(x)-L(y) \|}_W = {\| L( x-y ) \|}_W 
  \le C {\| x-y \|}_V < \eps.
  \]
  Das zeigt die gleichmäßige Stetigkeit von $L$ (und insbesondere 
  die Stetigkeit in $0$). 

  \medskip
  \noindent
  \desc{c} $\Ra$ \desc{d}: Sei $L$ stetig in $0$. Dann gibt es 
  ein $\delta>0$ so, dass ${\| L(\xi) \|}_W < 1 $ für alle 
  $\xi\in V$ mit 
  ${ \| \xi \| }_V < \delta$ gilt. Für ${ \| x \| }_V \le 1$ folgt 
  daraus 
  \[
  {\| L(x) \|}_W  =   
  \frac{ {\| x \|}_V }{\delta} \cdot 
  \underbrace{
    {\left\| L \left( \delta \cdot \frac{x}{ {\| x \|}_V } \right) \right\|}_W 
  }_{ \le 1 } \le \frac{1}{\delta} {\| x \|}_V \le \infty.
  \]
  
  \medskip
  \noindent
  \desc{d} $\Ra$ \desc{a}: Aus der Voraussetzung \desc{d} folgt 
  insbesondere ${\|} L( \xi ) {\|}_W \le C$ für ein $C\in \RR$ und 
  alle $\xi \in V$ mit ${\|} \xi {\|}_V = 1$. Damit gilt 
  für alle $x\in V$
  \[
  {\| L(x) \|}_W = 
  { \left\| L\left( 
        {\| x \|}_V \cdot \frac{x}{ {\| x \|}_V } \right) \right \|}_W = 
  {\| x \|}_V \cdot {\left\| L 
      \left( \frac{x}{ {\|} x {\|}_V } \right) \right\|}_W  \le 
  C {\| x \|}_V.
  \]
\end{antwort}

%% --- 80 --- %%
\begin{frage}\label{09_oper}\index{Operatornorm}
  Wie ist die \bold{Operator-Norm} im Raum der \slanted{stetigen} 
  linearen Abbildungen $L \fd V \to W$ definiert und welche Haupteigenschaften hat sie?
\end{frage}

\begin{antwort}
  Die Operatornorm ist für eine stetige lineare Abbildung 
  $L \fd V \to W$ definiert durch
  \begin{equation}
    \boxed{
      {\| L \|}_{\mathrm{Lin(V,W)}} := 
      \sup \bigg\{ {\| L (x) \|}_W \sets x \in V, \,\, {\| x \|}_V \le 1 \bigg\}.}
    \asttag
  \end{equation}
  Dabei bezeichnet $\mathrm{Lin}(V,W)$ 
  den Raum der stetigen linearen Abbildungen $V\to W$. 
  Wenn keine Verwechslungsgefahr besteht, schreibt man auch 
  einfach ${\| L \|}$. 

  Aus der Linearität von $L$ folgt, dass die Bedingung $\n{ x }_V \le 1$ in 
  {\astref} auch durch $\n{ x }=1$ ersetzt werden kann. Wegen 
  $L\left( \frac{x}{ \n{x}_V } \right) = \frac{Lx}{ \n{x}_V }$ gilt ferner 
  \begin{equation}
    {\|} L {\|} = \sup\left\{  
      \frac{ {\| L(x) \|}_W }{ {\| x \|}_V } \sets x\in V,\,\, x\not=0
    \right\}.
    \aasttag
  \end{equation}
  Diese Darstellung erlaubt es, 
  die reelle Zahl $\n{L}$ als den größten 
  \slanted{Streckungsfaktor} der linearen Abbildung $L$ zu interpretieren. 
  \AntEnd
\end{antwort} 

%% --- 81 --- %%
\begin{frage}
  Wieso ist durch {\astref} auch wirklich eine 
  Norm auf $\mathrm{Lin}(V,W)$ gegeben?
\end{frage}

\begin{antwort} 
  Nach Frage \ref{09_op} ist $\n{L} < \infty$ für alle 
  $L\in\mathrm{Lin}(V,W)$.  
  Die Eigenschaften \desc{N1} und \desc{N2} einer 
  Norm folgen aus deren Gültigkeit 
  für die Norm $\n{\;\,}_W$ und der Linearität von $L$, 
  die Dreiecksungleichung ergibt sich mit der in der nächsten 
  Frage unter \desc{i} erwähnten Eigenschaft. 
  Nach dieser gilt für alle $x\in V$: 
  \[
  \n{ (L_1+L_2) x }_W \le \n{ L_1 x }_W+ \n{ L_2x }_W  
  \stackrel{\text{\desc{i}}}{\le} 
  ( \n{L_1} + \n{L_2} ) \cdot \n{x}_V,
  \]
  und daraus folgt $\n{ L_1 + L_2 } \le \n{L_1}+ \n{L_2}$.\AntEnd
\end{antwort}  

%% --- 82 --- %%
\begin{frage}
  Welche Eigenschaften hat die Operatornorm?
\end{frage}

\begin{antwort}
  Es gelten die folgenden Eigenschaften:
  {\setlength{\labelsep}{5mm}
    \begin{enumerate}
    \item[\desc{i}] Aus \ref{09_oper} {\astastref} folgt unmittelbar 
      \[
      {\| L(x) \|}_W \le {\| L \|} \cdot {\| x\|}_V \quad\text{für alle $x\in V$}.
      \]
    \item[\desc{ii}] Sind $L\fd U \to V$ und $L' \fd V \to W$ stetige 
      Abbildungen zwischen normierten Räumen, so gilt 
      \[
      {\| L' \circ L \| }_{\mathrm{Lin}( U,W )}\le 
      {\| L'\| }_{\mathrm{Lin}( V,W )}\cdot 
      {\| L \| }_{\mathrm{Lin}( U,V )}.
      \]
      Nach \desc{i} ist nämlich  
      ${\| (L' \circ L)(x) \|} \le {\|L'\|} \cdot {\| L(x) \|} \le 
      {\|L'\|} \cdot {\|L\|} \cdot {\| x \|}$ 
      für alle $x\in U$, und daraus folgt
      \[
      \sup_{\n{x}_U=1} \big\{ (L'\circ L)(x) \big\} \le \n{L'}\cdot \n{L}.
      \EndTag
      \]
    \end{enumerate}}
\end{antwort}

%% --- 83 --- %%
\begin{frage}\index{Stetigkeit!von linearen Abbildungen $\KK^n\to\KK^m$}
  \index{Lipschitz-Stetigkeit}
  Warum ist jede \slanted{lineare} Abbildung 
  $L\fd V \to W$ mit $V := \KK^n$ und $W := \KK^m$ 
  stetig, sogar Lipschitz-stetig und damit gleichmäßig stetig?
\end{frage}

\begin{antwort}
  Seien $e_1, \ldots, e_n$ die Einheitsvektoren 
  des $\KK^n$ und sei 
  \[
  M := \max \big\{ \n{L( e_i )}_W \sets 1\le i \le n \big\}.
  \]
  Dann gilt für jeden Vektor $x := (x_1,\ldots,x_n)$ aus $V$:  
  \[
  \nnb{L(x)}_W = 
  \nnb{ L \left( \sum_{k=1}^n e_k \cdot x_k \right) }_W 
  \le \sum_{k=1}^n \nnb{ L(e_k) \cdot x_k }_W \le 
  M \cdot \sum_{k=1}^n \left|x_k\right| = 
  M \cdot \nnb{ x }_1.
  \] 
  Da alle Normen auf einem endlich-dimensionalen Vektorraum 
  äquivalent sind, folgt daraus für jede Norm 
  $\n{\,\;}_W$ auf $W$ die Existenz eines $C\in\RR$ mit 
  $\nnb{ L(x) } \le C \nnb{ x }_V$. Aus Linearitätsgründen 
  gilt dann für $x,y \in V$
  \[
  \nnb{ L(x)-L(y) } = \nnb{ L(x-y) } \le C \nnb{ x-y }_V.
  \]
  Das zeigt die Lipschitz-Stetigkeit von $L$. 
  \AntEnd    
\end{antwort}

%% --- 84 --- %%
\begin{frage}\index{Zeilensummennorm}\index{Spaltensummennorm}
  Was versteht man unter \bold{Zeilensummennorm}
  einer Matrix $A=(a_{ij})\in \KK^{m \cdot n}$?
\end{frage}

\begin{antwort}
  Die \slanted{Zeilensummennorm} ist die durch die Maximumsnorm 
  induzierte Operatornorm. Das heißt, die Zeilensummennorm 
  ist für eine Matrix $A$ definiert durch
  \[
  \n{ A }_\infty := \sup_{\n{ x }_\infty=1} \n{ Ax }_\infty.
  \]
  Das führt auf die Darstellung
  \[
  \n{ A }_\infty = \sup_{\n{ x }_\infty=1} 
  \left\{ \max_i \sum_{j=1}^n \left| a_{ij} x_j \right| \right\} = 
  \max_i \sum_{j=1}^n \left| a_{ij} \right|.
  \]
  Der letzte Schritt ist hier gerechtfertigt, da das Supremum gerade für 
  die Vektoren der Form $x=(\pm 1, \ldots, \pm 1)$ angenommen wird. 
  \AntEnd
\end{antwort} 

%% --- 85 --- %%
\begin{frage}
  Wie ist die \bold{Spaltensummennorm}  
  einer Matrix $A=(a_{ij})\in \KK^{m \cdot n}$ definiert?
\end{frage}

\begin{antwort}
  Die \slanted{Spaltensummenorm} ist die 
  durch die $1$-Norm induzierte Operatornorm auf $K^{m\times n}$, also 
  definiert durch
  \[
  \n{ A }_1 := \sup_{\n{ x }_1=1} \n{ Ax }_1.
  \]
  Auch die Spaltensummennorm lässt sich durch die Einträge der 
  Matrix $A$ ausdrücken. Es gilt
  \[
  \n{ A }_1 = \sup_{\n{ x }_1=1} = \max_{j} \sum_{i=1}^m | a_{ij} |,
  \]
  was sich mit einem ähnlichen Argument wie für die Zeilensummennorm zeigen 
  lässt.
  \AntEnd
\end{antwort} 

%% --- 86 --- %%
\begin{frage}\index{Hilbert-Schmidt-Norm}
  Wie ist die \bold{Hilbert-Schmidt-Norm} für eine Matrix 
  $A:= (a_{ij}) \in K^{m\times n}$ definiert?
\end{frage}


\begin{antwort}
  Die Hilbert-Schmidt-Norm ist definiert durch
  \[
  \n{ A }_{HS} := \sqrt{ \sum_{i,j}  a_{ij}^2  }.
  \]
  Von den drei Eigenschaften einer Norm ist allein die Gültigkeit 
  der Dreiecksungleichung hier nicht offensichtlich.
  Diese folgt aus 
  \[ 
  \n{ A+B }_{HS} = \sqrt{ \sum_{i,j} (a_{ij}+b_{ij} )^2 } \le 
  \sqrt{ \sum_{i,j} a^2_{ij} + \sum_{i,j} b^2_{ij} + 
    2 \sum_{i,j}  \left| a_{ij}b_{ij} \right| } \le 
  \n{A}_{HS} + \n{B}_{HS}.
  \]
  Dabei wurde im letzten Schritt von der für alle 
  $x,y \in \RR_+$ gültigen Ungleichung 
  $\sqrt{x+y} \le \sqrt{x}+\sqrt{y}$ Gebrauch gemacht.
  \AntEnd 
\end{antwort}


%% --- 87 --- %%
\begin{frage}\index{Verträglichkeit von Normen}
  Wann heißt eine Matrix-Norm \bold{verträglich} mit irgendwelchen 
  Normen auf $V=\KK^n$ bzw. $W=\KK^m$.
\end{frage}


\begin{antwort}
  Eine Matrix-Norm heißt \slanted{verträglich} mit Normen $\n{\,\;}_V$ und 
  $\n{\;\,}_W$, wenn für alle $x\in V$ gilt 
  \begin{equation}
    \n{ Ax }_W \le \n{ A } \cdot \n{ x }_V
    \notag
  \end{equation}
  Nach Frage \ref{09_oper}, \desc{1} sind {\zB} alle Operatornormen 
  verträglich mit den Normen auf $\KK^n$ und $\KK^m$. 
  \AntEnd 
\end{antwort}


%% --- 88 --- %%
\begin{frage}\index{Hilbert-Schmidt-Norm}
  Warum ist die Hilbert-Schmidt-Norm 
  (Quadratsummennorm) auf $\KK^{m\times n}$ keine 
  Operatornorm?
\end{frage}


\begin{antwort}
  Für eine Operatornorm $\n{\;\,}$ gilt stets $\n{\mathrm{id}}=1$, 
  die Hilbert-Schmidt-Norm liefert aber für die Einheitsmatrix 
  $\nnb{ E_n }_{HS} = \sqrt{n}$.
  \AntEnd 
\end{antwort}


%% --- 89 --- %%
\begin{frage}\label{09_globstet}
  \index{Stetigkeit!globale Charakterisierung}
  Wie lässt sich die (globale) Stetigkeit einer Abbildung 
  $X\to Y$ zwischen metrischen Räumen mithilfe offener Mengen 
  charakterisieren?
\end{frage}

\begin{antwort}
  Das Kriterium lautet:


  \medskip\noindent%
  \slanted{Eine Abbildung $f\fd X\to Y$ ist stetig genau dann, wenn 
    das Urbild $f^{-1}(V)$ jeder offenen Menge $V\subset Y$ 
    offen in $X$ ist.}


  \medskip\noindent
  Ist nämlich $f$ stetig im Sinne dieser Definition, dann ist für 
  jedes $x\in X$ die Menge $f^{-1} \big( U_\eps(f(x))\big)$ offen in 
  $X$ und enthält damit eine volle $\delta$-Umgebung von $X$. Das heißt, 
  $f$ ist stetig im Sinne der $\eps\delta$-Definition.

  Sei umgekehrt $f$ $\eps\delta$-stetig und sei $V\subset Y$ offen. 
  Für hinreichend kleine $\eps$ ist $U_\eps\big( f(x) \big) \subset V$.  
  Dann gilt $f\big( U_\delta(x) \big ) \subset  
  U_\eps\big( f(x) \big)$ für ein hinreichend kleines, also 
  $U_\delta (x) \subset f^{-1}(V)$. Das heißt, $f^{-1}(V)$ ist offen.  
  \AntEnd 
\end{antwort} 





%% --- 90 --- %%
\begin{frage}
  Wie lautet eine äquivalente Definition der Stetigkeit mithilfe 
  des Begriffs der abgeschlossenen Menge?
\end{frage}

\begin{antwort}
  Mit der Charakterisierung abgeschlossener Mengen 
  als Komplemente offener Mengen erhält man aus Antwort \ref{09_globstet} 

  \medskip
  \noindent 
  \slanted{
    Eine Abbildung $f\fd X\to Y$ ist stetig genau 
    dann, wenn das Bild jeder in $X$ abgeschlossenen Menge 
    abgeschlossen in $Y$ ist.} 
  \AntEnd 
\end{antwort}



%% --- 91 --- %%
\begin{frage}
  Warum sind für eine stetige Abbildung $f\fd X\to \RR$ eines metrischen 
  Raumes für jedes $c\in\RR$ die Mengen 
  $U := \{ x\in X \sets f(x) < c \}$ offen und 
  $A := \{ x\in X \sets f(x) \le c \}$ abgeschlossen in $X$?
\end{frage}

\begin{antwort}
  Die Menge $U$ ist das Urbild der offenen Menge 
  $\{ y \in Y\sets y < c \} \subset Y$ unter $f$. 
  Wegen der Stetigkeit  von $f$ ist $U$ damit offen in $X$.

  Die Menge $A^C$ ist das Urbild der offenen Menge 
  $\{ y \in Y\sets y > c \}$ und damit offen in $X$, folglich ist 
  $A$ abgeschlossen.
  \AntEnd
\end{antwort}

%% --- 92 --- %%
\begin{frage}\index{Einheitskugel}
  Warum sind im $\RR^n$ die Einheitskugeln bezüglich der 
  Normen ${\| \; \, \|}_1$, ${\| \; \, \|}_2$ und 
  ${\| \; \, \|}_\infty$ homöomorph?
\end{frage}

\begin{antwort}
  Die in Frage \ref{09_homokugel} 
  speziell für die euklidische Norm 
  betrachtete Abbildung 
  \[
  f_{\n{\;}} \fd x \mapsto \frac{x}{1+\n{x}} 
  \]
  liefert für beliebige Normen $\n{\;\,}$ einen Homöomorphismus 
  zwischen $\RR^n$ und der jeweiligen Einheitskugel $U_{1,\n{\;}}$. 
  Da jede Norm im $\RR^n$ bezüglich einer anderen eine stetige Abbildung 
  darstellt, liefert die Verkettung dieser Funktionen bzw. ihrer 
  Umkehrungen umkehrbare stetige Abbildungen zwischen den jeweiligen 
  Einheitskugeln.
  \AntEnd  
\end{antwort}

%% --- 93 --- %%
\begin{frage}\index{stetige Fortsetzung}\index{Grenzwert!für Funktionen}
  Sind $(X,d_X)$ und $(Y,d_Y)$ beliebige metrische Räume, 
  $D\subset X$ eine nichtleere Teilmenge, $a\in X$ ein 
  Häufungspunkt von $D$ und $l\in Y$. 
  Was besagt dann für eine Funktion 
  $f\fd D \to Y$ die Schreibweise 
  \[
  \lim_{x\to a} f(x) = l \text{?}
  \]
  Welche Sprechweise verwendet man hierfür?
\end{frage}

\begin{antwort}
  Die Aussage $\lim_{x\to a} = l$ bedeutet, dass die 
  Funktion 
  \[
  \tilde{f} \fd D \cup \{ a \} \to Y, \qquad
  \tilde{f} (x) := \left\{ 
    \begin{array}{ll} 
      f(x),& \text{für $x\in D$, $x\not=a$} \\
      l,   & \text{für $x=a$} 
    \end{array} \right.
  \]
  im Punkt $a$ stetig ist (vgl. auch Frage \ref{03_fkon}).

  Man verwendet in diesem Zusammenhang die Sprechweise 
  "`$f$ hat bei Annäherung an 
  $a$ den Grenzwert $l$."' 
  \AntEnd
\end{antwort}


%% --- 94 --- %%
\begin{frage}
  Warum ist $l$ in diesem Fall eindeutig bestimmt?
\end{frage}

\begin{antwort}
  Wäre $l^*\not=l$ ein anderer Grenzwert, dann besäßen $l$ und $l^*$ disjunkte 
  $\eps$-Umgebungen, und für ein hinreichend kleines $\delta$ lägen die 
  Funktionswerte $\tilde{f}(x)$ für alle $x\in D$ mit $d(x,a) < \delta$ 
  in beiden disjunkten Umgebungen -- Widerspruch.
  \AntEnd 
\end{antwort}



%% --- 95 --- %%
\begin{frage}
  Warum gilt im Fall $a\in D$ 
  \[
  \lim_{x\to a} f(x)=f(a) \LLa 
  \text{$f$ stetig in $a$?}
  \]
\end{frage}

\begin{antwort} 
  Ist $f$ stetig in $a$, dann muss nach der vorigen Frage $f(a)=l$ 
  gelten. Gilt andersherum $f(a)=l$, dann sind $f$ und $\tilde{f}$ 
  identische Funktionen.
  \AntEnd
\end{antwort}

%% --- 96 --- %%
\begin{frage}\index{eps@$\eps\delta$-Charakterisierung des Grenzwerts}
  Wie lautet die $\eps\delta$-Charakterisierung 
  bzw. das Folgenkriterium für die Aussage $\lim_{x\to a} f(x)=l$? 
\end{frage} 

\begin{antwort}
  Die Funktion $f$ hat bei Annäherung an $a$ den Grenzwert $l$,  
  {\setlength{\labelsep}{4mm}
    \begin{itemize}
    \item[\desc{i}] wenn für jedes $\eps>0$ ein 
      $\delta>0$ existiert, 
      sodass für alle $x\in D$ mit $d_X(a,x)<\delta$ gilt: 
      $d_Y(a,x)< \eps$ (\slanted{$\eps\delta$-Charakterisierung}) \\[-3.5mm]
    \item[\desc{ii}] wenn für jede gegen $a$ konvergente Folge $(x_k)\subset D$ 
      die Folge $\big( f(x_k) \big)$ der Funktionswerte gegen $l$ konvergiert 
      (\slanted{Folgenkriterium}).\AntEnd
    \end{itemize}}
\end{antwort}




\section{Kompaktheit, stetige Funktionen auf kompakten Räumen}

Im Folgenden werden verschiedene Kompaktheitsbegriffe 
vorgestellt und das Verhalten stetiger Funktionen auf kompakten 
Mengen untersucht. Man erhält auf diese Weise auch neue 
Beweise und Verallgemeinerungen 
für bereits aus der reellen Analysis bekannte Tatsachen.

%% --- 97 --- %%
\begin{frage}\index{Uberdeckung@Überdeckung}\index{Teilüberdeckung}
  Können Sie die Begriffe \bold{Überdeckung} erläutern?
\end{frage}

\begin{antwort}
  Ist $X$ ein topologischer Raum und  
  $\frak{U} := \{ U_\lambda \}_{\lambda \in \Lambda}$ 
  ($\Lambda$ bezeichnet hier irgendeine Indexmenge)
  ein System von Mengen $U_\lambda \subset X$, dann heißt 
  $\frak{U}$ eine \slanted{Überdeckung von $X$}, wenn 
  $X = \bigcup_{\lambda\in \Lambda} U_\lambda$ gilt. 

  Analog kann man von einer Überdeckung 
  einer Teilmenge $M\subset X$ sprechen, wenn 
  $M \subset \bigcup_{\lambda\in \Lambda} U_\lambda$ 
  für ein System $\{ U_\lambda \}_{\lambda\in \Lambda}$ 
  von Teilmengen aus $X$ gilt.

  Im Folgenden spielen vor allem die \slanted{offenen Überdeckungen} 
  eine Rolle. Eine Überdeckung heißt \slanted{offen}, wenn 
  die Überdeckungsmengen $U_\lambda$ offen in $X$ sind. 
  \AntEnd
\end{antwort} 

%% --- 98 --- %%
\begin{frage}\index{Teiluberdeckung@Teilüberdeckung}
  Was versteht man unter einer \bold{Teilüberdeckung}?
\end{frage}

\begin{antwort}
  Ist $\frak{U}=\{ U_\lambda \}_{\lambda\in \Lambda}$ 
  eine Überdeckung von $X$ (bzw. $M\subset X$), dann heißt jede 
  Teilmenge $\{ U_{k} \}_{k \in K \subset \Lambda}$ 
  von $\frak{U}$ eine \slanted{Teilüberdeckung} von $X$ (bzw. $M$), 
  wenn $\{ U_k \}_{k \in K}$ ebenfalls eine Überdeckung 
  von $X$ (bzw. $M$) ist. 

  Zum Beispiel ist die Menge  
  $\frak{U}_1 := \{\open{ k,k+3 }\subset \RR \sets k\in \ZZ \}$ 
  eine Überdeckung von $\RR$.  
  Da bereits die Intervalle des Typs $\open{2k,2k+3}$ die reellen Zahlen 
  überdecken, ist $\frak{U}_2 := \{\open{2k,2k+3} \subset \RR \sets k\in \ZZ \}$ 
  eine Teilüberdeckung von $\frak{U}_1$. 
  \AntEnd 
\end{antwort}

%% --- 99 --- %%
\begin{frage}\index{kompakt}\index{uberdeckungskompakt@überdeckungskompakt}
  \index{Heine-Borel'sche Überdeckungseigenschaft}
  \index{Borel@\textsc{Borel}, Emile (1871-1956)}
  Wann heißt ein metrischer Raum $X$ \bold{(überdeckungs-)kompakt}?
\end{frage}


\begin{antwort}
  
  Ein metrischer Raum $X$ heißt kompakt, wenn er die 
  \slanted{Heine-Borel'sche Überdeckungseigenschaft} besitzt. 

  $X$ besitzt diese Eigenschaft, wenn jede 
  (wohlgemerkt: \slanted{jede}) \slanted{offene} Überdeckung 
  $\{ U_\lambda \}$ eine endliche Teilüberdeckung besitzt, wenn also 
  endlich viele $U_{\lambda_1}, 
  \ldots, U_{\lambda_n}$ aus $\{ U_\lambda \}_{\lambda\in \Lambda }$ 
  existieren, sodass 
  $X = U_{\lambda_1} \cup \cdots \cup U_{\lambda_n}$ gilt. \AntEnd 
\end{antwort}

%% --- 100 --- %%
\begin{frage}
  Wann heißt eine Teilmenge $K\subset X$ kompakt?
\end{frage}

\begin{antwort}
  Eine Teilmenge $K\subset X$ heißt kompakt, 
  wenn sich aus \slanted{jeder} Überdeckung von $K$ 
  durch offene Mengen aus $X$ endlich viele auswählen lassen, die 
  zusammen bereits eine Überdeckung von $K$ bilden, für die   
  (mit den Bezeichnungen von oben) dann gilt: 
  $M \subset U_{\lambda_1} \cup \cdots U_{\lambda_n}$. 
  \AntEnd
\end{antwort}

%% --- 101 --- %%
\begin{frage}\label{09_kompaktbsp}
  Ist $(a_k)$ eine konvergente Folge in einem metrischen Raum $X$, 
  und $a\in X$ deren Grenzwert, warum ist die 
  Menge $A_0:=\{ a_k\sets k\in \NN \} \cup \{ a \}$ dann kompakt? 
\end{frage}

\begin{antwort}
  Sei $\mathfrak{U }:=\{ U_\lambda \}_{\lambda\in\Lambda}$ eine beliebige 
  offene Überdeckung von $A_0$. Es muss gezeigt werden, dass 
  $A_0$ bereits von endlich vielen der Mengen $U_\lambda$ überdeckt 
  wird. Nun, mindestens eine der Mengen aus $\mathfrak{U}$ -- sagen wir 
  $U_{\lambda_0}$ -- enthält den Grenzwert $a$, und nach der Definition 
  der Konvergenz liegen in $U_{\lambda_0}$ dann fast alle Folgenglieder. 
  Die endlich vielen anderen Folgenglieder $a_1,\ldots,a_r$ liegen 
  jeweils in (nicht notwendig verschiedenen) 
  Überdeckungsmengen $U_{\lambda_1},\ldots,U_{\lambda_r}$ aus $\mathfrak{U}$. 
  Somit wird $A_0$ bereits von den endlich vielen 
  Mengen  $U_{\lambda_0},U_{\lambda_1},\ldots,U_{\lambda_r}$ aus 
  $\mathfrak{U}$ überdeckt. $A_0$ ist damit nach Definition eine 
  kompakte Teilmenge von $X$. 
  \AntEnd 
\end{antwort} 

%% --- 102 --- %%
\begin{frage}
  Ist in der Situation aus der vorigen Frage auch die Menge  
  $A:=\{ a_k \sets k\in \NN \}$ stets kompakt?
\end{frage}

\begin{antwort}
  Nein. Man betrachte etwa die Folge $(a_k)\subset \RR$ 
  mit $a_k = 1/k$. Die Mengen aus 
  $\mathfrak{U}:=\{ U_{1/(1+k)^3} (a_k) \sets k\in \NN \}$ 
  bilden zusammen eine Überdeckung von $A$,  
  es existiert dazu aber keine endliche Teilüberdeckung, denn 
  jedes Element aus $\mathfrak{U}$ enthält nur ein Folgenglied. 
  \AntEnd  
  
\end{antwort}

%% --- 103 --- %%
\begin{frage}
  Warum ist der $\RR^n$ (versehen etwa mit der euklidischen Metrik) kein 
  kompakter metrischer Raum?
\end{frage}

\begin{antwort}
  Die Menge $\{ U_k (0 )\sets k \in \NN \}$ von offenen Kugeln 
  um den Nullpunkt ist eine Überdeckung von $\RR^n$. Diese besitzt aber 
  keine endliche Teilüberdeckung.
  \AntEnd
\end{antwort}

%% --- 104 --- %%
\begin{frage}\index{folgenkompakt}
  Wann heißt ein metrischer Raum $X$ bzw. ein Teilraum 
  $K\subset X$ (mit der induzierten 
  Metrik) \bold{folgenkompakt}?
\end{frage}

\begin{antwort}
  $X$ heißt folgenkompakt, wenn jede Folge in $X$ eine konvergente 
  Teilfolge besitzt. 
  Entsprechend heißt ein Teilraum $K\subset X$ folgenkompakt, 
  wenn jede Folge in $K$ eine in $K$ konvergente Teilfolge besitzt. 

  Zum Beispiel ist $\RR$ ersichtlich \slanted{nicht} folgenkompakt, 
  da etwa die Folge der natürlichen Zahlen keine konvergente Teilfolge 
  besitzt.  
  Beispiele für folgenkompakte Räume sind alle kompakten 
  \slanted{metrischen} Räume, was in der 
  nächsten Frage gezeigt wird.
  \AntEnd 
\end{antwort}

%% --- 105 --- %%
\begin{frage}\label{09_kompfolgkomp}
  Können Sie zeigen, dass 
  ein kompakter metrischer Raum $X$ stets folgenkompakt ist?
\end{frage}

\begin{antwort}
  Sei $(a_k)$ eine Folge in $X$ und sei $A\subset X$ die Menge 
  ihrer Folgenglieder. Nimmt die Folge nur endlich 
  viele verschiedene Werte an, so kann man offensichtlich eine (dann sogar 
  konstante) konvergente Teilfolge auswählen. Wir können also davon 
  ausgehen, dass $A$ unendlich ist. Dann muss $A$ einen Häufungspunkt 
  besitzen. Denn angenommen, das wäre nicht der Fall. Dann gäbe es zu jedem 
  $x\in X$ eine Umgebung $U(x)$, in der nur endlich viele Folgenglieder 
  liegen. Die Mengen $U(x)$ bilden zusammen eine offene Überdeckung von 
  $X$, und da $X$ kompakt ist, existiert dazu eine endliche Teilüberdeckung 
  $\{ U(x_1), \ldots ,U(x_r) \}$. Insbesondere gilt dann 
  $A\subset U(x_1)\cup \ldots  \cup U(x_r)$, und da $A$ unendlich ist, 
  muss mindestens eine der Umgebungen $U(x_i)$ unendlich viele 
  Folgenglieder enthalten, im Widerspruch zur Annahme. Also besitzt 
  $A$ einen Häufungspunkt $h$. 
  
  Ausgehend von $h$ konstruieren wir jetzt eine konvergente Teilfolge 
  nach wohlbekanntem Muster. Die 
  Umgebungen $K_{1/\nu} (h)$ enthalten für alle $\nu\in \NN$ unendlich viele 
  der Folgenglieder $a_k$. Daher lässt sich eine monoton wachsende Folge von 
  Indizes $(k_\nu)_{\nu\in \in \NN}$ so bestimmen, dass 
  $a_{k_\nu} \in K_{1/\nu}(h)$ für alle $\nu\in \NN$ gilt. 
  Die Folge $(a_{k_\nu})$ ist dann eine gegen $h$ konvergente  
  Teilfolge von $(a_k)$. Das zeigt die Folgenkompaktheit von $X$.  \AntEnd
\end{antwort}

%% --- 106 --- %%
\begin{frage}\label{09_beschrabgeschlossen}
  Wieso ist jede \slanted{folgenkompakte} Teilmenge $K$ eines metrischen 
  Raumes stets beschränkt und abgeschlossen? (Eine Teilmenge 
  $K$ heißt \slanted{beschränkt}, wenn eine offene Kugel $U \subset X$ mit 
  $K \subset U$ existiert.) 
\end{frage}

\begin{antwort}
  Ist $K$ folgenkompakt, so muss $K$ beschränkt sein. 
  Denn andernfalls ließe sich zu jedem beliebigen Punkt $a\in K$ eine 
  Folge $(x_k)$ mit $d(a,x_k)>k$ konstruieren. Wegen 
  $d(x_m,x_n) \ge | d(a,x_m) - d(a,x_n )| = |m-n|$ für alle 
  $m,n\in \NN$ könnte diese aber keine konvergente Teilfolge besitzen.  

  Wäre $K$ nicht abgeschlossen, dann hätte $K$ einen 
  Häufungspunkt $h$, der nicht zu $K$ gehört, 
  und es gäbe eine Folge in $K$, die gegen $h$ konvergiert. 
  Diese kann dann aber keine in $K$ konvergente Teilfolge besitzen.
  \AntEnd
\end{antwort}

%% --- 107 --- %%
\begin{frage}
  Ist ein \slanted{folgenkompakter} metrischer Raum stets kompakt?
\end{frage}

\begin{antwort}
  Dieser Zusammenhang gilt auch, für den Beweis s. etwa \citep{JaenichTop}. 
  Für metrische Räume sind die beiden Begriffe "`kompakt"' und 
  "`folgenkompakt"' somit äquivalent. Das gilt allerdings nicht 
  für allgemeine topologische Räume, die durchaus eine der beiden 
  Eigenschaften haben können, ohne die andere zu besitzen. 
  \AntEnd
\end{antwort}

%% --- 108 --- %%
\begin{frage}\label{09_abesch}
  Ist eine beschränkte und abgeschlossene Teilmenge $K$ eines 
  metrischen Raumes stets kompakt?
\end{frage}

\begin{antwort}
  Nein. Man betrachte etwa die Teilmenge   
  $E := \{ \exp(\i kx)\sets k \in \NN\}$ des metrischen Raumes 
  $\calli{C}[0,\pi]$ bezüglich der Supremumsnorm. 
  $E$ ist beschränkt und abgeschlossen, aber wegen 
  $\nb{\exp(\i kx)-\exp(\i n x) }_\infty = 2$ für alle $k\not=n$ ist 
  sie nicht folgenkompakt und daher auch nicht kompakt.
  \AntEnd
\end{antwort}

%% --- 109 --- %%
\begin{frage}\label{09_abkompakt}
  Können Sie begründen, warum im Standardvektorraum $\KK^n$ für eine 
  Teilmenge $K$ folgende Eigenschaften äquivalent sind:
  \begin{itemize}[2mm]
  \item[\desc{a}] $K$ ist beschränkt und abgeschlossen,
  \item[\desc{b}] $K$ ist kompakt (im Sinne der Überdeckungseigenschaft),
  \item[\desc{c}] $K$ ist folgenkompakt.
  \end{itemize}
\end{frage}

\begin{antwort}
  Nach den in Frage \ref{09_kompfolgkomp}
  und \ref{09_beschrabgeschlossen} gegebenen Beweisen ist hier nur 
  noch die Implikation \desc{a} $\Ra$ \desc{b} zu zeigen. 
  Dies beweist man zunächst für $\KK=\RR$  
  mit einem Schachtelungsargument.  

  Sei also $K\subset \RR^n$ beschränkt und abgeschlossen. 
  Dann gibt es einen  abgeschlossenen Würfel $W_0 \subset \RR^n$ mit 
  $K\subset W_0$. Angenommen, es gäbe eine Überdeckung 
  $\{U_\lambda \}_{\lambda\in \Lambda}$ von 
  $K$, die keine endliche Teilüberdeckung von $K$ enthält. 
  Ausgehend von $W_0$ ließe sich dann durch sukzessive 
  Halbierung der Seitenlängen eine Folge 
  $W_0 \supset W_1 \supset W_2 \supset \ldots$ von 
  abgeschlossenen Würfeln konstruieren (\sieheAbbildung\ref{fig:test}), 
  bei der $W_k$ für jedes $k\in \NN$ so ausgewählt wird, dass gilt: 
  \begin{equation}
    \text{$W_k \cap K$ wird von keiner 
      endlichen Teilmenge von $\{ U_\lambda \}_{\lambda \in\Lambda }$
      überdeckt}
    \asttag
  \end{equation}

  \begin{center}
    \includegraphics{mp/test}
    \captionof{figure}{Die Konstruktion der Folge abgeschlossener Würfel 
      im Beweis von Frage \ref{09_abkompakt}.}
    \label{fig:test}
  \end{center}

  Eine derartige Auswahl ist aufgrund der Annahme, dass $K$ nicht kompakt ist, 
  in jedem Schritt möglich. Wegen $\mathrm{diam}\,( W_k \cap K )\to 0$ 
  und der Abgeschlossenheit von $K$ gibt es nach dem allgemeinen 
  Schachtelungsprinzip genau ein $x\in K$, das in allen Mengen $W_k\cap K$ 
  enthalten ist. Andererseits liegt $x$ in mindestens einer offenen Menge 
  $U_x\in \{ U_\lambda \}_{\lambda \in\Lambda }$, und für hinreichend  
  hinreichend große $k$ sind die Mengen $W_k \cap K$ alle in $U_x$ enthalten, 
  im Widerspruch zu {\astref}.

  Um aus diesem Ergebnis für reelle Vektorräume den 
  allgemeinen Fall abzuleiten, betrachte man einen Isomorphismus 
  $\varphi \fd \KK^n \to \RR^n$. Dieser ist in jedem 
  Fall auch ein Homöomorphimus und gibt daher Anlass für die Schlusskette
  \[
  \begin{array}{ccc}
    \text{$K \subset \KK^n$ beschränkt und abgeschlossen} 
    &  & \text{$K\subset \KK^n$ kompakt} \\
    \Downarrow & & \Uparrow \\
    \text{$\varphi(K)\subset\RR^n$ beschränkt und abgeschlossen} & \Ra & 
    \text{$\varphi(K)\subset\RR^n$ kompakt} 
  \end{array}
  \]
  Damit ist der Zusammenhang auch für allgemeine endlichdimensionale 
  Vektorräume gezeigt.
  \AntEnd
\end{antwort}

%% --- 110 --- %%
\begin{frage}\index{orthogonale Gruppe $O(n,\RR)$}
  Fasst man die \slanted{orthogonale Gruppe}  
  \[
  O(n,\RR) := \{ A \in \RR^{n\times n} \sets A^T A = E_n \} 
  \]
  als Teilmenge von $\RR^{n^2} \simeq \RR^{n\times n}$ auf, dann 
  ist $O(n,\RR)$ kompakt. Können Sie das begründen?
\end{frage}

\begin{antwort}
  Jedes Element aus $O(n,\RR)$ liegt 
  bezüglich der Maximumsnorm in der abgeschlossenen 
  Einheitskugel von $\RR^{n^2}$ ({\dasheisst}, 
  eine Matrix aus $O(n,\RR)$ besitzt keine Einträge, die 
  betragsmäßig größer als $1$ sind). Die orthogonale 
  Gruppe ist also beschränkt. 

  Ferner ist $\RR^{n^2}\mengeminus O(n,\RR)$ offen. Das liegt 
  daran, dass für jede Matrix $X = (x_{ij})$ die Komponenten 
  von $X^T X$ \slanted{stetig} von den $x_{ij}$ abhängen. 
  Aus $X^T X \not=E_N$ folgt damit, dass auch in einer Umgebung 
  von $X$ die Ungleichung gilt.  

  Somit ist $O(n,\RR)$ abgeschlossen und beschränkt, nach Frage 
  \ref{09_abkompakt} also kompakt.\AntEnd
\end{antwort}

%% --- 111 --- %%
\begin{frage}
  Warum ist eine \slanted{abgeschlossene} Teilmenge $A$ eines 
  \slanted{kompakten} metrischen Raumes $X$ wieder kompakt?
\end{frage}

\begin{antwort}
  Sei $\frak{U} := \{ U_\lambda \}_{\lambda \in \Lambda}$ eine offene 
  Überdeckung von $A$. 
  Da $A$ abgeschlossen ist, ist $X\mengeminus A$ offen, 
  und $X \mengeminus A$ und $\frak{U}$ bilden zusammen eine Überdeckung  
  von $X$. Wegen der Kompaktheit von $X$ existiert dazu eine endliche 
  Teilüberdeckung 
  $\{ (X \mengeminus A),  U_{\lambda_1}, \ldots , U_{\lambda_r} \}$. 
  Somit bilden die Menge 
  $\{ U_{\lambda_1}, \ldots, U_{\lambda_r} \} \subset \frak{U}$ 
  eine endliche Teilüberdeckung von $A$.
  \AntEnd
\end{antwort}

%% --- 112 --- %%
\begin{frage}\label{09_vollstteilraum}
  \index{vollständiger metrischer Raum}
  Warum ist eine nichtleere abgeschlossene Teilmenge $A$ eines 
  vollständigen metrischen Raumes $X$ wieder vollständig?
\end{frage}

\begin{antwort}
  Sei $(a_k)$ eine Cauchy-Folge in $A$. Wegen der Vollständigkeit von $X$ 
  besitzt diese einen Grenzwert $a\in X$. In jeder Umgebung 
  von $a$ liegen dann Elemente aus $A$. 
  Würde $a$ nicht zu $A$, sondern zu $X\mengeminus A$ 
  gehören, dann folgte daraus, dass $X\mengeminus A$ 
  nicht offen und damit $A$ nicht abgeschlossen sein kann -- Widerspruch.
  \AntEnd
\end{antwort}

%% --- 113 --- %%
\begin{frage}\label{09_schlagwort}
  Können Sie das Schlagwort "`stetige Bilder kompakter Räume sind kompakt"' 
  erläutern und begründen?
\end{frage}

\begin{antwort}
  
  Die Sprechweise besagt, dass für eine \slanted{stetige} Abbildung 
  $f\fd X\to Y$ zwischen metrischen Räumen und eine kompakte 
  Teilmenge $K\subset X$ das Bild $f(K)$ eine kompakte Teilmenge 
  von $Y$ ist. 

  Für den Beweis sei $\{ U_\lambda \}_{\lambda\in\Lambda}$ 
  eine offene Überdeckung 
  von $f(K)$. Wegen der Stetigkeit von $f$ sind die Urbilder 
  der $U_\lambda$ offen in $X$, und 
  $\{ f^{-1} (U_\lambda) \}_{\lambda\in\Lambda}$ ist eine 
  offene Überdeckung von $K$. Wegen der Kompaktheit von $K$ existiert 
  dazu eine endliche Teilüberdeckung 
  $\{ f^{-1} (U_{\lambda_1}), \ldots, f^{-1} (U_{\lambda_r}) \}$, 
  und die Mengen $U_{\lambda_1}, \ldots, U_{\lambda_r} $ bilden 
  dann eine endliche Teilüberdeckung von $f(K)$. \AntEnd
\end{antwort}

%% --- 114 --- %%
\begin{frage}\index{Satz!von Weierstraß}\index{Satz!vom Maximum und Minimum}
  Können Sie diesen \bold{Satz von Weierstraß} begründen: Ist $X$ ein 
  kompakter metrischer Raum und $f\fd X\to\RR$ eine stetige Funktion, 
  dann nimmt $f$ auf $X$ ein absolutes Maximum und absolutes Minimum 
  an, {\dasheisst}, es gibt ein $x_{\min} \in X$ und ein $x_{\max} \in X$ 
  mit $f( x_{\min} ) \le  f(x)  \le f( x_{\max} )$ für alle 
  $x\in X$?
\end{frage}

\begin{antwort} 
  Der Beweis für diesen wichtigen Satz ergibt sich mit der erarbeiteten 
  Terminologie wie von selbst: 
  $f(X)\subset \RR$ ist als stetiges Bild einer kompakten Menge kompakt, 
  folglich beschränkt und besitzt ein Infimum $t$ und ein Supremum $s$. 
  Diese müssen wegen der Abgeschlossenheit von $f(X)$ zu $f(X)$ gehören. 
  Es gibt also ein $x_{\min}$ und ein $x_{\max}$ aus $X$ mit 
  $f(x_{\min})=t$ und $f(x_{\max})=s$.\AntEnd   
\end{antwort}

%% --- 115 --- %%
\begin{frage}\index{Stetigkeit!gleichmäßige}\label{09_glmstet}
  Können Sie zeigen, dass eine stetige Abbildung 
  $f\fd X\to Y$ zwischen metrischen Räumen sogar 
  \slanted{gleichmäßig stetig} ist, falls $X$ kompakt ist?
\end{frage}

\begin{antwort}
  
  Es muss gezeigt werden, 
  dass es zu jedem $\eps>0$ ein $\delta>0$ gibt, 
  sodass \slanted{für alle} 
  $x_1,x_2 \in X$ gilt:
  \[
  d_X (x_1, x_2 ) < \delta \Ra d_Y \big( f(x_1), f(x_2) \big) < \eps.
  \]
  Angenommen, es gibt ein $\eps_0$, für das kein 
  $\delta$ mit dieser globalen Eigenschaft existiert. Dann gibt es zu jedem 
  $n\in \NN$ zwei Punkte $x_n, x_n' \in X$ mit 
  $d(x_n,x_n') < 1/n$ aber $d\big( f(x_n), f(x_n') \big) > \eps_0$. 
  Wegen der Kompaktheit von $X$ besitzt $(x_n)$ eine  
  Teilfolge $(x_{n_k})$, die gegen einen Punkt $\xi \in X$ konvergiert. 
  Dann konvergiert aber auch $( x'_{n_k} )$ gegen $\xi$. 
  Wegen der Stetigkeit von $f$ folgt 
  $\lim\limits_{ k\to \infty } f(x_{n_k}) =f(\xi) =
  \lim\limits_{ k\to \infty } f(x'_{n_k}) $, 
  und das steht im Widerspruch zu 
  $d_Y\big( f(x_n), f(x_n') \big) > \eps_0$ für alle $n\in \NN$. 
  \AntEnd
\end{antwort}

%% --- 116 --- %%
\begin{frage}\label{09_normaq}\index{Aquivalenz@Äquivalenz!von Normen}
  Warum sind im $\KK^n$ ($\KK=\RR$ oder $\KK=\CC$) alle Normen äquivalent?
\end{frage}

\begin{antwort}
  Zum Beweis genügt es, die Äquivalenz einer beliebigen 
  Norm $\n{\;\,}$ mit der $1$-Norm $\n{\;\,}_1$ zu zeigen. 
  Das Argument beruht im Wesentlichen auf einer Anwendung des Satzes 
  vom Maximum und Minimum und benutzt die beiden Tatsachen
  {\setlength{\labelsep}{2mm}
    \begin{itemize}[2mm]
    \item[\desc{i}] 
      \slanted{Die Norm $\n{\;\,}$ ist stetig bezüglich $\n{\;\,}_1$.}  \\[-3mm]
    \item[\desc{ii}] 
      \slanted{Die Sphäre $S^{n-1}_1 := \{ x\in \KK^n\sets \nnb{x}_1 = 1 \}$ 
        ist kompakt.} 
    \end{itemize}}
  \noindent
  Aus \desc{i} und \desc{ii} folgt, dass 
  die Funktion $\n{\;\,}$ auf $S^{n-1}_1$ ein Maximum und 
  ein Minimum annimmt, es gibt also reelle Zahlen 
  $c$ und $C$ mit 
  \begin{equation}
    c \le  \n{ x } \le C \quad\text{für alle $x\in S^{n-1}_1$}, 
    \asttag
  \end{equation}
  wobei $c$ und $C$ wegen $0\not\in S^{n-1}_1$ 
  beide $\not=0$ sind. Sei nun 
  $x\in \RR^n \mengeminus\{ 0 \}$ beliebig. Dann ist 
  $\frac{x}{\n{x}_1} \in S_1$, und 
  aus {\astref} folgt $c \le \nnb{ \frac{x}{\n{x}_1} } \le C$ und damit   
  \[
  c\cdot \n{x}_1 \le \n{x} \le C \cdot \n{x}_1.
  \]
  Diese Ungleichungskette drückt die Äquivalenz 
  von $\n{\,\;}$ und $\n{\,\;}_1$ auf $\KK^n$ aus. 
  \AntEnd
\end{antwort}

%% --- 117 --- %%
\begin{frage}
  Warum können die Einheitskreislinie 
  \[
  S^1 := \{ (x,y) \in \RR^2\sets x^2+y^2 =1 \} 
  \]
  und das Intervall $[ 0, 2\pi[ $ nicht homöomorph sein?
\end{frage}

\begin{antwort}
  Bekanntlich lässt sich zwar das Intervall $\ropen{0,2\pi}$ stetig 
  aus $S^1$ abbilden (etwa durch $t\mapsto \left( \cos t, 
    \sin t \right)$), die 
  Umkehrung kann aber nicht stetig sein, und zwar deswegen, weil 
  $S^1$ kompakt ist und $\open{0,2\pi}$ nicht (\sieheAbbildung\ref{fig:09_homo3}), und weil die Bilder kompakter Mengen 
  unter stetigen Abbildungen nach Frage \ref{09_schlagwort} 
  immer kompakt sind. 
  \AntEnd

  \begin{center}
    \includegraphics{mp/09_homo3}
    \captionof{figure}{Das Intervall $\ropen{0,2\pi}$ und die 
      Einheitskreislinie $S^1$ sind nicht homöomorph.}
    \label{fig:09_homo3}
  \end{center}
\end{antwort}

%% --- 118 --- %%
\begin{frage}
  Ist $X$ ein metrischer Raum und sind $A,B \subset X$ kompakte Teilmengen. 
  Warum sind dann $A\cap B$ und $A\cup B$ ebenfalls kompakt? 
\end{frage}

\begin{antwort}
  \desc{i} 
  Sei $\{ U_\lambda \}_{\lambda\in \Lambda}$ eine Überdeckung 
  von $A\cap B$. Die Menge $A\cap B$ ist als Durchschnitt 
  abgeschlossener Mengen abgeschlossen, und somit ist 
  $\{ X\mengeminus ( A\cap B ) \} \cup \{ U_\lambda \}$ eine 
  offene Überdeckung von $A$ und von $B$. Davon 
  gibt es jeweils endliche Teilüberdeckungen 
  für $A$ ebenso wie  für $B$. 
  Die Zusammenfassung dieser beiden Teilüberdeckungen ist 
  dann eine endliche Teilüberdeckung für $A\cap B$. Das zeigt, dass 
  der Durchschnitt kompakt ist.

  \desc{ii} Ist $\{ U_\lambda \}_{\lambda\in \Lambda}$ eine 
  Überdeckung von $A\cup B$, so auch von $A$ und von $B$. Es existieren 
  also jeweils endliche Teilüberdeckungen für $A$ und für $B$, und durch 
  Zusammenfassung von diesen erhält man eine endliche Teilüberdeckung 
  für $A\cup B$.
  \AntEnd
\end{antwort}

%% --- 119 --- %%
\begin{frage}\label{q:119}\heavy
  Sind $X$ und $Y$ kompakte metrische Räume, ist dann auch das kartesische 
  Produkt $X\times Y$ bezüglich der Produktmetrik kompakt?
\end{frage}

\begin{antwort}
  Das kartesische Produkt ist in diesem Fall auch kompakt. 

  Für den Beweis sei $\{ U_\lambda \}_{\lambda\in\Lambda}$ 
  eine offene Überdeckung von $X\times Y$. 
  Zu jedem $(x,y)\in U_\lambda$ gibt es eine offene Umgebungen 
  $V_{(x,y)}$ in $X$ und $W_{(x,y)}$ in $Y$,  
  sodass das offene "`Kästchen"' 
  $V_{(x,y)}\times W_{(x,y)}$ 
  in $U_\lambda$ enthalten ist 
  (vgl. die erste Grafik in Abbildung \ref{fig:09_beweis}). 

  Die Mengen $V_{(x,y)}$ bilden bei festgehaltenem $y$ eine 
  offene Überdeckung von $X$, und wegen der Kompaktheit von 
  $X$ gibt es endlich viele Punkte $x_1,\ldots, x_{r(y)} \in X$,   
  sodass gilt:  
  \begin{equation}
    X \subset V_{(x_1,y)} \cup \cdots \cup V_{(x_r,y)}.
    \tag{vgl. die zweite Abbildung}
  \end{equation}
  Im nächsten Schritt bilden wir für jedes $y\in Y$ die Menge 
  \begin{equation}
    W(y) := W_{(x_1,y)} \cap \cdots \cap W_{(x_{r(y)},y)}. 
    \tag{vgl. die dritte Grafik in Abbildung \ref{fig:09_beweis}} 
  \end{equation}
  Die $W(y)$ sind dann eine Überdeckung von $Y$, und wegen der 
  Kompaktheit von $Y$ lassen sich endlich viele $y_1,\ldots,y_s$ 
  auswählen, für die bereits 
  \[
  Y \subset W(y_1)\cup \cdots \cup W(y_s)
  \]
  gilt. Die Mengen $V_{(x_{r(yj)},y_j)}\times W(y_j)$, $1\le j\le s$ 
  sind dann eine endliche Überdeckung von $X\times Y$, 
  und jedes dieser Kästchen ist 
  in einem $U_\lambda$ enthalten. Man wähle zu jedem Kästchen eines, 
  und man erhält eine endliche Teilüberdeckung der ursprünglichen 
  Überdeckung von $X\times Y$.

  \medskip

  \begin{center}
    \includegraphics[width=\textwidth]{mp/09_beweis}
    \captionof{figure}{Konstruktionsschritte im Beweis zu Frage \ref{q:119}}
    \label{fig:09_beweis}
  \end{center}

  Die Umkehrung dieses Zusammenhangs gilt im Übrigen auch und ist eine 
  Folge davon, dass die kanonischen Projektionen $X\times Y \to X$ bzw. 
  $X\times Y \to Y$ stetige Abbildungen sind.  
  \AntEnd
\end{antwort}



\section{Wege, Zusammenhangsbegriffe}

%% --- 120 --- %%
\begin{frage}\index{wegweise zusammenhängend}\index{bogenweise zusammenhängend}
  Wann heißt ein metrischer Raum $X$ bzw. eine Teilmenge $D\subset X$ 
  \bold{wegweise} oder \bold{bogenweise zusammenhängend}?
\end{frage}

\begin{antwort}
  $X$ heißt bogenweise zusammenhängend, wenn es zu je zwei Punkten 
  $a,b\in X$ einen \slanted{Weg}, {\dasheisst} eine stetige Abbildung 
  $\gamma\fd [0,1]\to X$ mit $\gamma(0)=a$ und $\gamma(1)=b$ gibt.
  \AntEnd
\end{antwort}

%% --- 121 --- %%
\begin{frage}
  Was sind die bogenweise zusammenhängenden Teilmengen von $\RR$ 
  (mit der natürlichen Metrik)?
\end{frage}

\begin{antwort}
  Die bogenweise zusammenhängenden Teilmengen von 
  $\RR$ sind gerade alle Typen von Intervallen.

  Dass für je zwei Punkte eines reellen Intervalls ein  
  Weg existiert, ist offensichtlich. Die Umkehrung ergibt sich als  
  Folgerung aus dem Zwischenwertsatz.
  \AntEnd
\end{antwort}




%% --- 122 --- %%
\begin{frage}
  Warum ist das stetige Bild eines bogenweise zusammenhängenden Raums wieder 
  bogenweise zusammenhängend (allgemeiner Zwischenwertsatz)?
\end{frage}


\begin{antwort}
  Ist $f \fd X\to Y$ stetig und $\gamma$ ein Weg von $a\in X$ 
  nach $b\in X$, so ist $f \circ \gamma$ ein Weg 
  von $f(a)$ nach $f(b)$.
  \AntEnd   
\end{antwort}




%% --- 123 --- %%
\begin{frage}\index{sternförmig}
  Wann heißt eine Teilmenge $S$ eines normierten $\KK$-Vektorraums 
  \bold{sternförmig}?
\end{frage}



\begin{antwort}
  $S$ heißt  \slanted{sternförmig}, wenn ein Punkt 
  $s^*\in S$ existiert, sodass für jeden Punkt $s\in S$ auch die 
  "`Verbindungsstrecke"'
  \[
  \{ s+t\cdot( s^*-s ) \sets t\in [0,1] \}
  \]
  vollständig in $S$ enthalten ist, \sieheAbbildung\ref{fig:09_stern}.

  \begin{center}
    \includegraphics{mp/09_stern}
    \captionof{figure}{Eine sternförmige Menge.}
    \label{fig:09_stern}
  \end{center}

  \begin{center}
    \includegraphics{mp/09_sterngebiet}
    \captionof{figure}{Die "`geschlitzte"' komplexe Zahlenebene 
      $\CC_-$ ist ein Sterngebiet.}
    \label{fig:09_sterngebiet}
  \end{center}

  Beispielsweise ist die längs der negativen reellen Achse 
  "`geschlitzte"' komplexe Zahlenebene 
  $\CC_-:= \CC \mengeminus \{x\in\RR\sets x\le 0\}$ eine sternförmige Menge, 
  \sieheAbbildung\ref{fig:09_sterngebiet}.  
  Jeder Punkt auf der positiven reellen Achse ist ein Sternpunkt von $\CC_-$. 
  \AntEnd
\end{antwort}


%% --- 124 --- %%
\begin{frage}\index{polygonzusammenhängend}
  Wann heißt eine Teilmenge eines normierten $\KK$-Vektorraums $X$ 
  \bold{polygonzusammenhängend}?
\end{frage}


\begin{antwort}
  Eine Menge $M\subset X$ heißt polygonzusammenhängend, wenn zu je zwei 
  Punkten $a, b\in M$ Punkte $a=a_0, a_1, \ldots, a_n=b$ aus $M$ 
  existieren derart, dass die Strecken 
  \[
  S_{a_{k-1}, a_k} := \{ 
  x\in X \sets x = a_{k-1}+t(a_k - a_{k-1}), \, t\in [0,1] \}
  \]
  für $1\le k \le n$ ganz in $M$ enthalten sind. Diese einzelnen 
  Strecken lassen sich dann zu einem \slanted{Streckenzug} zusammenfassen, 
  der $a$ und $b$ verbindet.  
  \AntEnd
\end{antwort}

%% --- 125 --- %%
\begin{frage}
  Warum ist eine sternförmige Menge $S$ stets bogenweise zusammenhängend?
\end{frage}

\begin{antwort}
  Für zwei Elemente $s_1$ und $s_2$ eines sternförmig zusammenhängenden 
  Raumes gibt es Verbindungsstrecken von $s_1$ nach $s^*$ und 
  von $s_1$ nach $s^*$. 
  Diese lassen sich zu Weg zwischen $s_1$ und $s_2$ zusammenfassen.
  \AntEnd
\end{antwort}

%% --- 126 --- %%
\begin{frage}\index{S@$S^{n-1}$!bogenweiser Zusammenhang}
  Warum sind für $n \ge 2$ der Raum $\RR^n \mengeminus \{ 0 \}$ und 
  die Sphäre $S^{n-1}$ bogenweise zusammenhängend?
\end{frage}

\begin{antwort}
  Seien $a$ und $b$ beliebige Punkte auf $S^{n-1}$. Verläuft 
  deren Verbindungsstrecke $\gamma(t)$ nicht durch den Nullpunkt, 
  dann ist 
  \[
  \gamma\fd t\to S^{n-1};\quad t \mapsto \frac{\gamma(t)}{ \nnb{\gamma(t)} } 
  \]
  ein stetiger Weg auf $S^{n-1}$ von $a$ nach $b$, \sieheAbbildung\ref{fig:09_s1}.

  \begin{center}
    \includegraphics{mp/09_s1}
    \captionof{figure}{Ein stetiger Weg auf $S^{n-1}$ von $a$ nach $b$.}
    \label{fig:09_s1}
  \end{center}

  Für den Fall, dass die Verbindungsgerade von $a$ und $b$ durch den Nullpunkt 
  läuft, gibt es einen weiteren Punkt $c$ derart, dass weder die Strecke 
  von $a$ nach $c$ noch die von $b$ nach $c$ den Nullpunkt 
  schneidet. Es gibt dann nach obigem 
  stetige Wege von $a$ nach $c$ und von $c$ nach $b$. Diese lassen
  sich zu einem stetigen Weg von $a$ nach $b$ verbinden lassen.
  \AntEnd
\end{antwort}

%% --- 127 --- %%
\begin{frage}
  Können Sie begründen, warum für einen normierten Raum $X$ und eine 
  nichtleere Teilmenge $D\subset X$ folgende Aussagen äquivalent sind:
  {\setlength{\labelsep}{5mm}
    \begin{enumerate}
    \item[\desc{a}] $D$ ist polygonzusammenhängend,\\[-3.5mm]
    \item[\desc{b}] $D$ ist bogenweise zusammenhängend,\\[-3.5mm]
    \item[\desc{c}] Ist $\emptyset \not=M \subset D$ und $M$ in $D$ offen 
      und abgeschlossen bezüglich der induzierten Metrik ist, dann ist $M=D$.
    \end{enumerate}}
\end{frage}


\begin{antwort}
  \desc{a} $\Ra$ \desc{b} ist offensichtlich.

  \medskip\noindent
  \desc{b} $\Ra$ \desc{c} Sei $D$ bogenweise zusammenhängend 
  und $M\subset D$ offen und abgeschlossen in $D$. 
  Angenommen, es ist $M\not=D$. Dann sind $D\mengeminus M$ und $M$ beide 
  offen, nichtleer und disjunkt. Für einen stetigen Weg 
  $\gamma\fd [0,1] \to D$ mit $\gamma(0)=a \in M$ und 
  $\gamma(1)=b \in  D\mengeminus M$ sind dann die 
  Urbilder $\gamma^{-1}(M)$ und $\gamma^{-1}(D\mengeminus M)$ beide 
  offene, nichtleere und disjunkte Teilmengen von $[0,1]$, deren 
  Vereinigung gerade das Intervall $[0,1]$ ist. 
  Zwei Mengen mit dieser Eigenschaft kann es aber nicht geben 
  (vgl. Frage \ref{09_intzus}). 
  Also muss wirklich $M=D$ gelten. 


  \medskip\noindent
  \desc{c} $\Ra$ \desc{a} Sei $a$ irgendein Punkt aus $D$. 
  Man betrachte die Teilmenge 
  \[
  U := 
  \{ x \in D\sets \text{Es gibt einen Streckenzug in $D$ von $a$ nach $x$} \}.
  \]
  Wir zeigen, dass $U$ gleichzeitig offen und abgeschlossen ist. 
  Aus der Voraussetzung folgt dann $U=D$, also \desc{a}.
  {\setlength{\labelsep}{4mm}
    \begin{itemize}
    \item[\desc{i}] Sei $u\in U$ und $K(u)$ eine beliebige offene Kugel um $u$. 
      Jedes $b\in K(u)$ lässt sich durch einen Streckenzug mit dem 
      Mittelpunkt $u$ verbinden, und für diesen wiederum existiert 
      wegen $u\in U$ ein Streckenzug nach $a$. Die beiden Streckenzüge 
      lassen sich dann zu einem Streckenzug von $a$ nach $b$ verbinden, \sieheAbbildung\ref{fig:09_zusammenhang}. 
      Das zeigt $b\in U$ für jedes $b\in K(u)$. 

      \begin{center}
        \includegraphics{mp/09_zusammenhang}
        \captionof{figure}{$a$ und $b$ lassen sich durch einen Polygonzug 
          verbinden.}
        \label{fig:09_zusammenhang}
      \end{center}

      Also ist $K(u)$ in $U$ enthalten. Das zeigt die Offenheit von $U$. \\[-2mm]
    \item[\desc{ii}] 
      \picskip{1}{\noindent}Für jedes Element $v\in D\setminus U$ liegt jede 
      Kugel $K(v)$ um $v$ ebenfalls in $D\setminus U$, denn für einen 
      Punkt $b \in K(v)\cap U$ existierten Polygonzüge nach $v$ und nach $a$, 
      und durch Verknüpfung erhielte man einen von $a$ nach $v$, woraus $v\in U$ 
      folgen würde, im Widerspruch zu $v\in D\mengeminus U$.
    \end{itemize}}
  \noindent
  $U$ und $D\mengeminus U$ sind also beide offen, $U$ folglich  
  offen \slanted{und} abgeschlossen in $D$. 
  Nach Voraussetzung folgt $D=U$. Für jeden 
  Punkt aus $D$ existiert also ein Streckenzug nach $a$. Da $a$ beliebig 
  gewählt war, folgt, dass $D$ polygonzusammenhängend ist.
  \AntEnd
\end{antwort}


%% --- 128 --- %%
\begin{frage}\index{Zusammenhang}
  Wann nennt man einen metrischen Raum im allgemeinen 
  Sinne \bold{zusammenhängend}?
\end{frage}   


\begin{antwort}
  Ein metrischer Raum $X$ heißt \slanted{zusammenhängend}, wenn es 
  \slanted{keine} Zerlegung 
  $X=U \cup V$ gibt, in welcher $U$ und $V$ beide offen, disjunkt und 
  nicht leer sind.
  \AntEnd
\end{antwort}



%% --- 129 --- %%
\begin{frage}\label{09_intzus}\index{Zusammenhang}
  Können Sie zeigen, dass das Intervall $[0,1]$ zusammenhängend ist?
\end{frage}

\begin{antwort}
  Angenommen, es gäbe eine Zerlegung von $[0,1]$ in zwei disjunkte, 
  offene und nichtleere Umgebungen $A$ und $B$. Man wähle irgendein 
  $a\in A$ und ein $b\in B$. Ohne Beschränkung der Allgemeinheit sei 
  $b>a$. Dann kann 
  \[
  t := \inf\{ x\in B \sets x>a \}  
  \]
  jedenfalls nicht in $B$ enthalten sein, denn $B$ ist offen. 
  Andererseits liegen in jeder Umgebung von $t$ auch Punkte aus $a$. 
  Denn wenn nicht schon $t=a$ gilt, dann ist $a<t$, und das Intervall 
  $\open{a,t}$ ist vollständig in $A$ enthalten. Da $a$ offen ist, folgt 
  $t\not\in A$. Es ist also $t \subset [0,1]$ 
  weder in $A$ noch in $B$ enthalten, im Widerspruch zu $A\cup B=[0,1]$. 
  \AntEnd  
\end{antwort}

%% --- 130 --- %%
\begin{frage}
  Warum ist jeder bogenweise zusammenhängende Raum zusammenhängend?
\end{frage}

\begin{antwort}
  Angenommen, für einen bogenweise zusammenhängenden Raum $X$ gibt es eine 
  Zerlegung $X=U\cup V$ in offene, disjunkte und nichtleere Mengen 
  $U$ und $V$. Für einen stetigen Weg $\gamma\fd [0,1]\to X$ von 
  $u \in U$ nach $v\in V$ sind dann die Mengen 
  $\gamma^{-1}(U)$ und $\gamma^{-1}(V)$ beide offen, disjunkt, nichtleer 
  und es gilt $\gamma^{-1}(U) \cup \gamma^{-1}(V)=[0,1]$. Eine solche 
  Zerlegung von $[0,1]$ kann es aber nach Frage \ref{09_intzus} 
  nicht geben.  
  \AntEnd 
\end{antwort}

%% --- 131 --- %%
\begin{frage}
  Ist jeder zusammenhängende Raum auch wegzusammenhängend?
\end{frage}

\begin{antwort}
  Nein. Ein Gegenbeispiel liefert etwa die Menge  
  \[
  \left\{  \left(x, \sin (1/x) \right)\sets 
    x\in \lopen{0,1} \right\} \cup 
  \left\{ ( 0, 0 ) \right\}.
  \]

  \begin{center}
    \includegraphics{mp/09_sineinsdurchx}
    \captionof{figure}{Der Graph der Funktion $x\mapsto \sin\frac1x$ ist 
      zusammenhängend, aber nicht wegzusammenhängend.}
    \label{fig:09_sineinsdurchx}
  \end{center}

  Die Menge lässt sich nicht in disjunkte offene 
  und nichtleere Teilmengen zerlegen, aber für jeden Punkt auf 
  dem Graphen gibt es keinen stetigen Weg zum Nullpunkt, \sieheAbbildung 
  \ref{fig:09_sineinsdurchx}.
  \AntEnd
\end{antwort}

%% --- 132 --- %%
\begin{frage}\index{Gebiet}
  Was versteht man unter einem \bold{Gebiet} in einem normierten Raum?
\end{frage}

\begin{antwort}
  Ein Gebiet ist eine zusammenhängende offene Teilmenge von $X$. 
  (Manchmal wird in der Definition der \slanted{bogenweise Zusammenhang} 
  gefordert.)
  \AntEnd
\end{antwort}

%% --- 133 --- %%
\begin{frage}\heavy\index{Homöomorphie}
  Können Sie begründen, 
  warum für $n\ge 2$ $\RR^n$ und $\RR$ nicht homöomorph sind?
\end{frage}  

\begin{antwort}
  
  Das hängt damit zusammen, dass die Bilder 
  bogenweise zusammenhängender Räume unter stetigen Abbildungen 
  $f\fd X\to Y$ wieder bogenzusammenhängend sind (denn für einen Weg 
  $\gamma$ zwischen $a$ und $b$ ist $f \circ \gamma$ ein Weg 
  zwischen $f(a)$ und $f(b)$). 

  Gäbe es eine stetige bijektive  
  Abbildung $f \fd \RR^n \to \RR$, so wäre auch 
  die Abbildung $\RR^n \mengeminus \{ 0 \} 
  \to \RR \mengeminus \{ f(0) \}$ stetig und bijektiv. 
  Aber $\RR^n \mengeminus \{ 0 \}$ ist bogenweise zusammenhängend, 
  $\RR \mengeminus \{ f(0) \}$ aber nicht, da es kein Intervall ist.
  \AntEnd    
\end{antwort}



\section{Der Satz von Stone-Weierstraß}
\index{Weierstrass@\textsc{Weierstrass}, Karl Theodor Wilhelm (1815-1897)}

Der klassische Approximationssatz von Weierstraß besagt, dass man jede 
stetige Funktion $f\fd [a,b]\to \RR$ \slanted{gleichmäßig} durch Polynome 
approximieren kann, dass also die Funktionenalgebra der Polynome auf 
$[a,b]$ dicht im Raum $\calli{C}( [a,b] )$ liegt. Der Satz besitzt 
wichtige Verallgemeinerungen für \slanted{kompakte} metrische 
Räume $X$, mit dem man für geeignete Funktionenräume 
$\calli{F}(X)\subset \calli{C}(X)$ 
deren Dichtheit in $\calli{C}(X)$ nachweisen kann. 

%% --- 134 --- %%
\begin{frage}\label{09_appro}
  Welche Voraussetzungen an einen Unterraum 
  $\calli{F} \subset \calli{C}(X,\RR)$ ($X$ kompakter metrischer 
  Raum) muss man {\zB} stellen, damit $\calli{F}$ dicht in $\calli{C}(X,\RR)$ 
  ist?
\end{frage}

\begin{antwort}
  Es gilt: \satz{Besitzt der 
    Unterraum $\calli{F}\subset \calli{C}(X,\RR)$ die Eigenschaften 
    \setlength{\labelsep}{4mm}
    \begin{enumerate}
    \item[\desc{i}] $f\in \calli{F} \Ra |f| \in \calli{F}$,\\[-3.5mm]
    \item[\desc{ii}] $\calli{F}$ trennt die Punkte von $X$, {\dasheisst},  
      zu $x,y\in X$ mit $x\not=y$ gibt es ein $f\in\calli{F}$ mit $f(x)\not=f(y)$,
    \end{enumerate} 
    dann ist $\calli{F}$ dicht in $\RR$. Das heißt, zu jedem 
    $\varphi \in \calli{C}(X,\RR)$ und jedem $\eps>0$ gibt es ein $f\in \calli{F}$ 
    mit $
    \left| \varphi(x) - f(x) \right| < \eps \quad\text{für alle $x\in X$}.$
    \AntEnd
  }

  \medskip
  Wir geben hier als Zugabe einen Beweis dieses Approximationssatzes, 
  weil er dazu beiträgt, den Nutzen der in diesem Kapitel entwickelten 
  topologischen Methoden an einem interessanten Beispiel zu illustrieren. 

  \medskip\noindent
  Zwei Bemerkungen vorneweg. Die Eigenschaften \desc{i} bzw. \desc{ii} 
  implizieren: 
  {\setlength{\labelsep}{5mm}
    \begin{itemize}
    \item[\desc{i'}] Mit $f$ und $g$ gehören auch die Funktionen 
      $f \wedge g $ und $f\vee g$, die durch 
      \[
      (f \vee g ) (x ) := \max\{ f(x) , g(x) \}, \quad
      (f \wedge g ) (x ) := \min\{ f(x) , g(x) \}
      \]
      definiert sind, zu $\calli{F}$. 
    \item[\desc{ii'}] Zu je zwei Punkten $x_1,x_2 \in $ mit $x_1\not=x_2$ und zwei 
      Zahlen $a,b \in \RR$ gibt es eine Funktion $f\in \calli{F}$ mit 
      $f(x_1)=a$ und $f(x_2)=b$. 
    \end{itemize}}
  Die Eigenschaft \desc{i'} folgt aus \desc{i} wegen 
  \[
  f\vee g= \frac{f+g}{2}+ \left| \frac{f-g}{2} \right|,\quad\text{und}\quad
  f\wedge g= \frac{f+g}{2}- \left| \frac{f-g}{2} \right|.
  \] 
  Eigenschaft \desc{ii'} wird von der Funktion 
  \[
  \dis f(x) = a \frac{g(x)-g(x_2)}{g(x_1)-g(x_2)} +
  b \frac{g(x)-g(x_1)}{g(x_2)-g(x_1)}
  \]
  erfüllt, wobei $g$ eine Funktion 
  aus $\calli{F}$ ist, die die Punkte $x_1$ und $x_2$ trennt.

  \medskip
  Zum eigentlichen Beweis: Sei $\varphi \in \calli{C}(X,\RR)$ 
  und $\eps>0$ gegeben. Zu je zwei Punkten $x,y \in X$ gibt es aufgrund 
  von \desc{2'} eine Funktion $f_{x,y}\in \calli{F}$ mit 
  $f_{x,y}(x)=\varphi(x)$ und $f_{x,y}(y)=\varphi(y)$. Das weitere Vorgehen 
  besteht nun darin, aus "`Teilstücken"' der Funktionen $f_{x,y}$ mit 
  den Operationen $\vee$ und $\wedge$ eine stetige Funktion zu konstruieren, 
  deren Graph vollkommen im $\eps$-Streifen von $\varphi$ verläuft. Dass das 
  mit endlich vielen Teilstücken funktioniert, hängt mit der Kompaktheit von 
  $X$ zusammen.   
  Man betrachte zu festgehaltenem $x$ zu jedem $y \in X$ die Mengen 
  \[
  U_{x,y} := \{ z\in X\sets \varphi(z)-\eps < f_{x,y}(z) \}.
  \]
  Diese Mengen 
  sind jeweils offene Umgebungen von $y$ und überdecken zusammen $X$. 
  Wegen der Kompaktheit von $X$ existiert dazu eine endliche Teilüberdeckung 
  $\{ U_{x,y_1(x)}, \ldots , U_{x,y_{r(x)}(x)} \}$ (wobei die $y_i$ genauso 
  wie deren Anzahl von $x$ abhängt). Die Funktionen $f_{x,y_i}$ verlaufen  
  nun für alle $i\in\{ 1,\ldots, n(x) \}$ 
  auf $U_{x,{y_i}}$ oberhalb von $\varphi-\eps$, die Funktion 
  $g_x := f_{x,y_1(x)}\vee \cdots \vee f_{x,y_{r(x)}(x)}$ hat aus 
  diesem Grund die Eigenschaft 
  \[
  g_x(z) > \varphi(z)-\eps \quad\text{für alle $z\in X$.}
  \]
  Jetzt wenden wir auf die Funktionen 
  $g_x$ dasselbe Verfahren "`von oben"' an. 
  Die Mengen 
  \[
  V_x = \{ z\in X\sets g_x(z)< \varphi+\eps \}
  \] 
  sind jeweils offene Umgebungen von $x$ und überdecken zusammen $X$. 
  Es gibt also eine endliche Teilüberdeckung $\{ V_{x_1}, \ldots, V_{x_s} \}$. 
  Auf $U_{x_i}$ verläuft $g_{x_i}$ dann für alle $i=1,\ldots,s$ 
  unterhalb von $\varphi+\eps$. Man bilde also 
  $h := g_{x_1} \wedge \cdots \wedge g_{x_s} $. 
  Dann gilt $\varphi(z)-\eps < h(z) < \varphi(z)+\eps$ für alle 
  $z\in X$, also $\left| h(z)-\varphi(z) \right| < \eps$ für alle 
  $z\in X$. Die Funktion $h(z)$ ist damit die gesuchte Approximation 
  zu $\varphi$. \AntEnd
\end{antwort}

%% --- 135 --- %%
\begin{frage}
  Was besagt der \bold{Satz von Stone-Weierstraß}?
\end{frage}

\begin{antwort}
  Der Satz besagt: \satz{Ist $X$ ein kompakter metrischer Raum und 
    $\calli{A} \subset \calli{C}( X, \KK)$ ($\KK=\RR$ oder $\KK=\CC$) eine 
    Funktionenalgebra mit
    \setlength{\labelsep}{4mm}
    \begin{enumerate}
    \item[\desc{1}] $1\in \calli{A}$,\\[-3.5mm]
    \item[\desc{2}] $f\in \calli{A} \Ra \overline{f} \in \calli{A}$,\\[-3.5mm]
    \item[\desc{3}] $\calli{A}$ trennt die Punkte von $X$,
    \end{enumerate} 
    dann liegt $\calli{A}$ dicht in $X$.}

  \medskip\noindent
  Es gibt viele verschiedene Wege, den Satz von Stone-Weierstraß zu beweisen. 
  Eine Möglichkeit besteht darin, ihn aus dem vorhergehenden Approximationssatz 
  abzuleiten. Dabei sind aber erst einige Hindernisse zu überwinden, denn 
  die Voraussetzung $f\in \calli{A} \Ra | f | \in \calli{A}$ ist 
  für eine Funktionenalgebra im allgemeinen nicht erfüllt. Man kann aber 
  zeigen, dass für eine reelle Funktionenalgebra $\calli{A}$ die 
  die oberen drei Voraussetzungen erfüllt, $|f|$ beliebig genau durch 
  Elemente aus $\calli{A}$ approximiert werden kann. Mit diesem Ergebnis 
  greift dann der Satz aus Frage \ref{09_appro} und liefert den Satz 
  von Stone-Weierstraß zunächst für reelle Funktionenalgebren. 

  Für komplexe Funktionenalgebren, die die Eigenschaft \desc{2} erfüllen, 
  lässt er sich auf den reellen Fall zurückführen. Denn diese Eigenschaft 
  gilt genau dann, wenn mit $f$ auch $\Re f$ und $\Im f$ in $\calli{A}$ 
  enthalten sind. 
  \AntEnd
\end{antwort}

%% --- 136 --- %%
\begin{frage}\label{09_stonekonsequenz}
  Können Sie aus dem allgemeinen Satz von Stone-Weierstraß folgern: 
  {\setlength{\labelsep}{4mm}
    \begin{itemize}
    \item[\desc{a}] Ist $X\subset \RR^n$ kompakt, dann ist die Algebra der 
      Polynome mit Koeffizienten in $\KK$ dicht in $\calli{C}(X,\KK)$.\\[-3.5mm]
    \item[\desc{b}] Ist $S^1 := \{ z\in \CC\sets |z|=1 \}$ die Einheitskreislinie, 
      dann ist die Algebra der trigonometrischen Polynome 
      \[
      \calli{T} := \left\{ \sum_{k=-n}^n c_k z^k \sets n\in\NN_0; c_k 
        \in \CC \right\} 
      \]
      dicht in $\calli{C}( S^1, \CC )$. (Dies folgt auch aus dem Satz 
      von Fej\'er (vgl. Frage \ref{07_fejer}).\\[-3.5mm]
    \item[\desc{c}] Sind $X,Y$ kompakte metrische Räume, dann ist die 
      Algebra
      \[
      \calli{C}(X) \otimes \calli{C}(Y) := \left\{ \sum_{k=1}^n f_k(x) g_k(y) \sets
        f_k \in \calli{C}(X), \, g_k\in \calli{C}(Y) \right\}
      \]
      dicht in $\calli{C}( X\times Y)$. (Das ist für die Integrationstheorie sehr 
      nützlich, vgl. Frage \ref{11_unabreihe}.)
    \end{itemize}}
\end{frage}

\begin{antwort}
  Es muss bei allen drei Antworten nur gezeigt werden, dass die 
  jeweiligen Funktionenalgebren die Voraussetzungen des Satzes von 
  Stone-Weierstraß erfüllen. Das ist aber offensichtlich. Alle enthalten 
  nach Definition das Einselement und mit $f$ auch $\overline{f}$. 
  In \desc{b} hat die identische Funktion die Eigenschaft, 
  die Punkte zu trennen. Im Fall der Polynome $\RR^n \to \KK$ beachte man, 
  dass sich zwei verschiedene Punkte des $\RR^n$ in mindestens einer 
  Koordinate -- sagen wir der $k$-ten -- unterscheiden. 
  Dann trennt das Polynom $p$ mit $p(x_1,\ldots,x_n):=x_k$ die beiden Punkte. 

  Dass auch die Algebra unter \desc{c} die Eigenschaft besitzt, die 
  Punkte in $X\times Y$ zu trennen, folgt daraus, dass es zu zwei verschiedenen 
  Punkten $x_1,x_2$ eines metrischen Raumes stets eine stetige Funktion 
  mit $\varphi(x_1)=0$ und $\varphi(x_2)=1$ gibt (eine Konsequenz des  
  \slanted{Urysohn\sch en Lemmas}). Zu zwei verschiedenen Punkten 
  $(x_1,y_1)$ und $(x_2,y_2)$ aus $X\times Y$, die sich etwa in der 
  ersten Koordinate unterscheiden, hat dann die Funktion 
  $\varphi(x) \cdot 1 \in \calli{C}(X) \otimes \calli{C}(Y)$ 
  die Eigenschaft, die beiden Punkte zu trennen (im Fall $x_1=x_2$ 
  funktioniert dasselbe natürlich mit $Y$ statt $X$). \AntEnd
\end{antwort}





%%% Local Variables: 
%%% mode: latex
%%% TeX-master: "master"
%%% End: 
