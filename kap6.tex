\chapter{Elementare (transzendente) Funktionen}

Wie schon in früheren Kapiteln angedeutet wurde, sind 
\slanted{Potenzreihen} ein äußerst wichtiges und leistungsfähiges 
Konstruktionsprinzip der Analysis. Eine besonders wichtige Rolle 
spielt dabei die \slanted{komplexe Exponentialfunktion} 
$\exp\fd \CC\to\CC$. Erst durch den Übergang zum Komplexen wird eine 
enge Verwandtschaft dieser Funktion 
mit den \slanted{trigonometrischen Funktionen} 
(\slanted{Winkelfunktionen}) oder den  
\slanted{hyperbolischen Funktionen} sichtbar, die im Reellen 
überhaupt nicht erkennbar ist, und mit deren Hilfe sich zahlreiche 
wichtige Eigenschaften dieser Funktionen auf sehr einfache Weise ergeben. 
Wir betrachten die Funktionen daher gleich im Komplexen, bei ihren 
Umkehrungen konzentrieren wir uns allerdings auf den reellen Fall.

\section{Die komplexe Exponentialfunktion}

%% Question 1
\begin{frage}\label{05_exei}\label{05_exko}\index{Exponentialreihe}
  \index{Exponentialfunktion}
  Durch welche Reihendarstellung ist die Exponentialfunkion 
  $\exp \fd \CC\to\CC$ definiert? Wieso konvergiert diese Reihe 
  für alle $z\in\CC$ absolut?
\end{frage}

\begin{antwort}
  Für alle $z\in\CC$ ist $\exp (z)$ definiert durch die Reihe  
  \[
  \boxed{
    \exp z  := \sum_{k=0}^\infty \frac{z^k}{k!} = 
    1 + z + \frac{z^2}{2!} + \frac{z^3}{3!}+ \cdots . 
  }
  \]
  Die absolute Konvergenz für alle 
  $z\in\CC$ erhält man am schnellsten 
  mit dem Quotientenkriterium. Das wurde in der Antwort zu Frage 
  \ref{02_expd} schon vorgeführt.

  Da die Reihendarstellung 
  damit für alle $z\in\CC$ existiert, ist durch sie
  eine stetige Funktion $\exp\fd\CC\to\CC$ gegeben.
  \AntEnd
\end{antwort} 

%% Question 2
\begin{frage}\label{05_exp_funktionalgleichung}
  \index{Additionstheorem!der Exponentialfunktion}
  \index{Funktionalgleichung!der Exponentialfunktion}
  Wie lautet die \bold{Funktionalgleichung} bzw. 
  das \bold{Additionstheorem} der Exponentialfunktion und 
  woraus folgt es?
\end{frage} 

\begin{antwort}
  Für alle $z,w\in\CC$ gilt die \slanted{Funktionalgleichung}
  \[
  \boxed{\exp( z+w )=\exp(z)\exp(w).}
  \]
  Man erhält die Gleichung durch Auswerten des 
  Reihenprodukts
  \[
  \left( \sum_{k=0}^\infty \frac{z^k}{k!} \right)\cdot 
  \left( \sum_{k=0}^\infty \frac{w^k}{k!} \right)
  \]
  mittels
  Cauchy-Multiplikation. Das wurde in der 
  Antwort zu Frage \ref{02_expfunk} bereits gezeigt. \AntEnd 
\end{antwort}

%% Question 3
\begin{frage}
  Wieso ist $\exp(z)\not=0$ für alle $z\in\CC$?
\end{frage}

\begin{antwort}
  
  Aus $\exp(z)=0$ würde mit der 
  Funktionalgleichung $\exp(z-z)=0$ folgen. Es gilt 
  aber $\exp(z-z)=\exp(0)=1$. \AntEnd
\end{antwort} 

%% Question 4
\begin{frage}
  Wieso gilt $\exp(kz)=( \exp(z) )^k$ für alle $k\in \ZZ$ und alle 
  $z\in\CC$?
\end{frage}

\begin{antwort}
  Für positive $k$ ergibt sich die Formel induktiv mithilfe 
  der Funktionalgleichung. Die Erweiterung auf negative Zahlen folgt 
  daraus dann im Zusammenhang mit $\exp(-z)=\exp(z)^{-1}$, was 
  sich wiederum aus $1=\exp(z-z)=\exp(z)\exp(-z)$ ergibt.\AntEnd
\end{antwort}  

%% Question 5
\begin{frage}
  Woraus folgt $\overline{\exp(z)} = \exp(\overline{z})$?
\end{frage}

\begin{antwort}
  Für alle $n\in\NN$ und alle $z\in\CC$ gilt 
  $\overline{z}^n = \overline{z^n}$. Da die Konjugation konvergenter 
  Folgen mit der Limesbildung verträglich ist 
  (vgl. Frage \ref{02_freg}), folgt die Gleichung 
  damit aus der Reihendarstellung von $\exp$. \AntEnd
\end{antwort} 

%% Question 6
\begin{frage}
  Wie kann man zeigen, dass für alle $t\in \RR$ gilt: 
  $\big| \exp(\i t) \big|=1$?
\end{frage}

\begin{antwort}
  Wegen $\overline{\i t}=-\i t$ und $\exp(-z)=1/\exp(z)$ 
  folgt zusammen mit dem vorhergehenden Ergebnis 
  \[
  \left|\exp(\i t)\right| = \exp(\i t)\overline{\exp(\i t)} = 
  \exp(\i t)\exp(\overline{\i t}) = 
  \exp(\i t)\exp(-\i t) = 1. \EndTag
  \]
\end{antwort} 

%% Question 7
\begin{frage}\index{Exponentialfunktion}
  Wieso gilt $\exp'(z)=\exp(z)$ für alle $z\in\CC$?
\end{frage}

\begin{antwort}
  Gliedweises Differenzieren der Exponentialreihe liefert
  \begin{equation}
    \exp'(z)=\sum_{k=1}^\infty \frac{kz^{k-1}}{k!} =
    \sum_{k=1}^\infty \frac{z^{k-1}}{(k-1)!}  =
    \sum_{k=0}^\infty \frac{z^k}{k!} = \exp(z).\EndTag
  \end{equation} 
\end{antwort}

%% Question 8
\begin{frage}\label{05_exid} 
  Können Sie $\lim\limits_{n\to\infty}\dis\left( 1+\frac{z}{n}\right)^n = \exp z$ 
  für alle $z\in\CC$ zeigen?
\end{frage}

\begin{antwort}
  Im Wesentlichen ergibt sich der Zusammenhang, 
  indem man die binomische Formel auf den Ausdruck 
  $\left( 1+\frac{z}{n} \right)^n$ anwendet. Damit erhält man
  \[
  \left( 1+\frac{z}{n} \right )^n = 
  \sum_{k=0}^n \binom{n}{k} \frac{z^k}{n^k} = 
  \sum_{k=0}^n 
  \left(1-\frac{1}{n}\right) \left(1-\frac{2}{n}\right) \cdots
  \left(1-\frac{k-1}{n}\right) \frac{z^k}{k!}.
  \]
  Die Summanden in der rechten Summe 
  konvergieren für $n\to\infty$ gegen $\frac{z^k}{k!}$, 
  und das macht es zumindest plausibel, dass dann auch 
  $\lim\limits_{n\to\infty} \left( 1+\frac{z}{n} \right)^n= 
  \lim\limits_{n\to\infty}\sum_{k=0}^n\frac{1}{k!}$ gilt.  
  
  Für einen sauberen Beweis schreibe man 
  \begin{equation}
    \left| \left( 1+\frac{z}{n} \right)^n 
      -\sum_{k=0}^n \frac{z^k}{k!} \right| 
    \le 
    \sum_{k=0}^N \left| \binom{n}{k} \frac{z^k}{n^k} - \frac{z^k}{k!} \right|
    + 
    \sum_{k=N+1}^n \left| \frac{z^k}{k!} \right| +
    \sum_{k=N+1}^n \left|\binom{n}{k} \frac{z^k}{n^k} \right|
    \tag{$\ast$}
  \end{equation}
  und wähle $N$ so groß, dass 
  $\sum_{k=N+1}^m \left| \frac{z^k}{k!} \right| < \eps$ für alle $m>N$ 
  gilt. Wegen $\binom{n}{k} \frac{1}{n^k} < \frac{1}{k!}$   
  ist dann auch die hintere Summe in $(\ast)$ für alle 
  $n>N$ kleiner als $\eps$. Ferner gilt 
  $\lim\limits_{n\to\infty}\binom{n}{k} \frac{1}{n^k} = \frac{1}{k!}$, 
  und daher gibt es ein $M\in \NN$, sodass die erste Summe 
  für alle $n>M$ ebenfalls kleiner als $\eps$ ist. Es gilt also 
  \[
  \left| \left( 1+\frac{z}{n} \right)^n 
    -\sum_{k=0}^n \frac{z^k}{k!} \right| 
  \le 3\eps,\qquad\text{für alle $n>\max\{M,N\}$},
  \] 
  womit die Identität der beiden Grenzwerte gezeigt ist. \AntEnd
\end{antwort}

%% Question 9
\begin{frage}\label{05_verallg}
  Kennen Sie eine Verallgemeinerung des 
  in der vorhergehenden Antwort gezeigten Zusammenhangs?
\end{frage}

\begin{antwort}
  Es gilt sogar 
  \[ \exp z = \lim_{n\to\infty} \left( 1+\frac{z_n}{n} \right)^n \]
  für eine beliebige gegen $z$ konvergierende Folge $z_n$,   
  was man mit einem ähnlichen Argument wie oben zeigen kann. \AntEnd
\end{antwort}

%% Question 10
\begin{frage}\label{05_exwa}
  \index{Wachstumsgeschwindigkeit der Exponentialfunktion}
  Was lässt sich für reelle Argumente $x$ über die 
  \slanted{Wachstumsgeschwindigkeit} der Exponentialfunktion 
  aussagen?
\end{frage}

\begin{antwort}
  \satz{Die Exponentialfunktion $\exp x$ 
    wächst schneller als jede natürliche Potenz von $x$, formal:
    \[
    \lim_{x\to\infty}\frac{e^x}{x^n} = \infty, \quad\text{bzw}\quad
    \lim_{x\to\infty}\frac{x^n}{e^x} = 0\quad
    \text{für alle $x>0$ und alle $n\in\NN$}.
    \]}
  \noindent   
  Den Zusammenhang erkennt man an der 
  Potenzreihenentwicklung der Exponentialfunktion. 
  Für $x>0$ gilt:
  \[
  e^x > \frac{x^{n+1}}{(n+1)!} \Ra 
  \frac{e^x}{x^n} > \frac{x}{(n+1)!}.
  \]
  Für $x\to\infty$ folgt daraus die Grenzwertaussage. 

  \begin{center}
    \includegraphics{mp/05_exp}
    \captionof{figure}{Graph der reellen Exponentialfunktion.}
    \label{fig:05_exp}
  \end{center}

  Abbildung \ref{fig:05_exp} gibt einen visuellen Eindruck von der rasanten 
  Wachstumsgeschwindigkeit der reellen Exponentialfunktion.
  \AntEnd
\end{antwort}

%% Question 11
\begin{frage}\index{Exponentialfunktion}
  Worin besteht der Hauptunterschied zwischen der 
  komplexen Exponentialfunktion 
  und ihrer Einschränkung auf die reelle Achse?
\end{frage}

\begin{antwort}
  Die reelle Exponentialfunktion ist eine streng monoton 
  wachsende Funktion, die $\RR$ bijektiv auf $\RR^*_+$ abbildet. 
  Die komplexe Exponentialfunktion dagegen ist eine 
  \slanted{periodische} Funktion mit der Periode $2\pi \i$. 
  Da die Periode rein imaginär ist, deutet im Reellen nichts auf 
  diese versteckte Eigenschaft der Exponentialfunktion hin, die 
  erst im Komplexen ans Tageslicht tritt. 

  Die Abbildungen \ref{fig:06_exp1} und \ref{fig:06_exp2} 
  zeigen den Graphen des 
  Realteils bzw. Imaginärteils von $\exp\fd \CC\to\CC$. 
  Die deutlich zu erkennende Wellenbewegung in Richtung der imaginären 
  Achse ist ein Ausdruck der Periodizität der komplexen Exponentialfunktion. 
  \AntEnd

  \begin{minipage}{60mm}
    \includegraphics[width=60mm]{povray/06_exp1}
    \captionof{figure}{Realteil der komplexen Exponentialfunktion. 
      Die entlang der Gerade $\Re z=a$ verlaufende Kurve ist der Graph der 
      Funktion $y\mapsto \exp(a)\cos(y)$ mit $y=\Im z$.}
    \label{fig:06_exp1}
  \end{minipage}
  \qquad
  \begin{minipage}{60mm}
    \includegraphics[width=60mm]{povray/06_exp2}
    \captionof{figure}{Imaginärteil der komplexen Exponentialfunktion. 
      Die entlang der Gerade $\Re z=a$ verlaufende Kurve ist der Graph der 
      Funktion $y\mapsto \exp(a)\sin(y)$ mit $y=\Im z$.}
    \label{fig:06_exp2}
  \end{minipage}
\end{antwort}

%% Question 12
\begin{frage}\index{p@$\pi$}
  Wie ist die Zahl $\pi$ definiert?
\end{frage}

\begin{antwort}
  Die Exponentialfunktion besitzt eine
  rein imaginäre Periode $\i p$ mit $p\in\RR_+$. 
  Die Zahl $\pi\in\RR$ lässt sich nun einfach als 
  \slanted{halbe Länge dieser Periode} definieren, 
  also als $ \pi := p/2$. Die die Zahl 
  $\pi$ charakterisierende Gleichung lautet somit 
  \[
  \exp( z+2\pi \i)=\exp(z) \quad\text{und\quad $\exp(z+\i t)\not=\exp(z)$ 
    für alle $t\in\RR_+$ mit $t\le 2\pi$.}
  \]
  Es ist vielleicht etwas praktischer -- und in aller Regel wird es auch 
  so gehandhabt --, sich der Periodizitätseigenschaften der Exponentialfunktion 
  durch die Untersuchung der reellen Funktionen $\cos t$ und $\sin t$ als 
  Real- bzw. Imaginärteil der Funktion $t\mapsto \exp(\i t)$ anzunähern. 
  In diesem Zusammenhang kann man $\pi/2$ dann auch als erste 
  positive Nullstelle des Cosinus definieren. Wegen des fundamentalen 
  Zusammenhangs mit der Exponentialfunktion läuft das am Ende aber auf genau  
  dasselbe hinaus. \AntEnd
\end{antwort}

%% Question 13
\begin{frage}\label{05_exho}
  \index{komplexe Exponentialfunktion}
  Wie kann man zeigen, dass die Abbildung 
  $\exp\fd\CC\to\CC^{\ast} := \CC \mengeminus \{ 0\}$
  surjektiv ist? 
\end{frage}

\begin{antwort}
  Wegen $\exp(x+\i y)=\exp(x)\exp (\i y)$ folgt die 
  Surjektivität der Exponentialfunktion zusammen aus
  \satz{\setlength{\labelsep}{4mm}
    \begin{enumerate}
    \item[\desc{i}] Die reelle Exponentialfunktion 
      bildet $\RR$ surjektiv auf $\RR_+^*$ ab. \\[-3.5mm]
    \item[\desc{ii}] Die Funktion $\RR \to S^1 := \{ z\in\CC;\; |z|=1 \}$ mit  
      $t\mapsto \exp(it)$ ist surjektiv (\sieheAbbildung\ref{fig:05_exp2}). 
    \end{enumerate}}
  Für jedes $w\in\CC^*$ gibt es unter diesen Voraussetzungen
  nämlich $x,y\in\RR$ mit $\exp(x)=|w|$ und $\exp(\i y)=w/|w|$. 
  Mit $z:=x+\i y$ folgt dann $\exp z=w$, also die Surjektivität der 
  Exponentialfunktion. 

  \begin{center}
    \includegraphics{mp/05_exp2}
    \captionof{figure}{Die komplexe Exponentialfunktion bildet 
      die imaginäre Achse surjektiv auf $S^1$ ab.}
    \label{fig:05_exp2}
  \end{center}

  Die Eigenschaft \desc{i} ist relativ klar und 
  wird in Frage \ref{05_exsu} beantwortet. 
  Die Eigenschaft \desc{ii} erhält man leichter nach einer 
  eingängigeren Untersuchung der Real- und Imaginärteile der Funktion 
  $t \mapsto \exp(\i t)$, was im nächsten Abschnitt \ref{sincos}  
  geschehen wird (Der endgültige Beweis wird in Frage \ref{05_trei}\,\desc{12} 
  erbracht). Prinzipiell lässt sich die Surjektivität der 
  Exponentialfunktion auch aus den Eigenschaften herleiten, 
  die bisher zur Verfügung stehen. Allerdings erfordert das 
  einige Tricks und gehört wohl 
  nicht unbedingt zum Standardwissen (für einen Beweis siehe 
  \citep{Ebbinghaus}).
\end{antwort}

%% Question 14
\begin{frage}\label{05_exin}
  \index{komplexe Exponentialfunktion}
  Wieso kann die Abbildung $\exp\fd\CC\to\CC^{\ast}$ nicht 
  injektiv sein?
\end{frage} 

\begin{antwort}
  Die Abbildung $\exp \fd \CC \to \CC^*$ ist -- 
  algebraisch beschrieben -- 
  ein Homomorphismus von der additiven Gruppe $(\CC, +)$ in die 
  multiplikative Gruppe $( \CC^*, \cdot)$. 
  Diese Gruppen sind nicht isomorph, da die multiplikative 
  Gruppe im Gegensatz zur additiven endliche Untergruppen enthält 
  (etwa $\{-1,1\}$). Somit kann die 
  die Abbildung kein Isomorphismus, also nicht bijektiv sein. 
  Ihre Surjektivität impliziert damit, dass sie 
  nicht injektiv ist. 
  \AntEnd
\end{antwort}

%% Question 15
\begin{frage}\label{05_kern}\index{Exponentialhomomorphismus}
  Was ist der Kern des Homomorphismus $\exp\fd(\CC,+)\to(\CC^*,\cdot)$?
\end{frage}

\begin{antwort}
  Der Kern des Exponentialhomomorphismus 
  ist die durch $2\pi \i$ erzeugte 
  zyklische additive Untergruppe von $\i \RR$, also 
  \[
  \boxed{
    \ker(\exp)=\big\{ 2 \pi \i k \sets k\in \ZZ \big\}.
  }
  \]
  Dies liefert ebenfalls eine Definition der Zahl $\pi$ 
  \AntEnd  
\end{antwort} 

%% Question 16
\begin{frage}\index{Stetigkeit!der Exponentialfunktion}
  Können Sie direkt begründen, warum aus der Stetigkeit von $\exp$ an der 
  Stelle $a=0$ die Stetigkeit an einer beliebigen Stelle $z\in\CC$ folgt?
\end{frage}

\begin{antwort}
  Zu $\eps>0$ sei ein $\delta$ entsprechend der Stetigkeit der 
  Exponentialfunktion im Nullpunkt gewählt. Für beliebiges 
  $z\in \CC$ und $w\in U_\delta( z )$ ist dann 
  $w-z \in U_{\delta}(0)$, und damit gilt  
  \[
  |\exp( w-z ) - \exp(0) | = 
  \left| \frac{\exp w}{\exp z} -1 \right| = 
  \left| \frac{\exp w - \exp z}{\exp z} \right| < \eps,
  \]
  also $|\exp w - \exp z | <  |\exp z| \cdot \eps$. 
  Daraus folgt die Stetigkeit der Exponentialfunktion im Punkt $z$.
  \AntEnd
\end{antwort}

%% Question 17
\begin{frage}\index{Differenzialgleichung!der Exponentialfunktion}
  Ist $f\fd \CC\to\CC$ eine Funktion mit
  \[
  \begin{array}{lp{1mm}lp{4mm}lp{1mm}l}
    \text{\desc{i}} & & f(z+w)=f(z)f(w)\quad\text{für alle $z,w\in\CC$,} & &
    \text{\desc{ii}} & & \dis \lim_{z\to 0} \frac{f(z)-1}{z}=1,
  \end{array}
  \]
  warum gilt dann $f(z)=\exp(z)$ für alle $z\in\CC$?
\end{frage}

\begin{antwort}
  Wegen \desc{i} gilt $f(z)=\left(f\left(\frac{z}{n}\right)\right)^n$ für alle $n\in\NN$ 
  und damit insbesondere 
  $\lim\limits_{n\to\infty}\left(f\left(\frac{z}{n}\right)\right)^n = f(z)$. 
  Durch $f\left( \frac{z}{n} \right) =: 1+\frac{z_n}{n}$ 
  sei nun eine Folge $(z_n)$ komplexer Zahlen definiert. 
  Für diese gilt mit \desc{ii} 
  \[
  \lim_{n\to\infty} z_n 
  = \lim_{n\to\infty} \frac{f(z/n)-1}{1/n} = z. 
  \]
  Die Folge $(z_n)$ konvergiert also gegen $z$, und damit gilt 
  nach der Antwort zu Frage \ref{05_verallg}
  \begin{equation}
    f(z)=\lim_{n\to\infty} \left(1+\frac{z_n}{n} \right)^n = \exp z
    \EndTag
  \end{equation} 
\end{antwort}


\section{Die trigonometrischen Funktionen und die Hyperbelfunktionen}
\label{sincos}



%% Question 18
\begin{frage}\index{Cosinus}\index{Sinus}\label{05_sinus}
  Definiert man für alle $z\in\CC$
  \begin{equation}
    \cos z := \frac{\exp(\i z) + \exp(-\i z)}{2} \qquad\text{und}\qquad
    \sin z := \frac{\exp(\i z) - \exp(-\i z)}{2\i}, \tag{$\ast$}
  \end{equation}
  warum gilt dann für alle $z\in\CC$:
  \begin{equation}
    \cos z := \sum_{k=0}^\infty (-1)^k \frac{z^{2k}}{(2k)!}
    \qquad\text{und}\qquad
    \sin z := \sum_{k=0}^\infty (-1)^k \frac{z^{2k+1}}{(2k+1)!}.
    \tag{$\ast\ast$}
  \end{equation}
\end{frage}

\begin{antwort}
  Die beiden Reihendarstellungen ergeben sich rein formal aus der 
  Potenzreihenentwicklung der Exponentialfunktion, konkret aus den 
  beiden Formeln

  \begin{align}
    \exp(\i z) &= 
    \sum_{k=0}^\infty \frac{(\i z)^{2k}}{(2k)!} + 
    \sum_{k=0}^\infty \frac{(\i z)^{2k+1}}{(2k+1)!} 
    = \sum_{k=0}^\infty (-1)^k \frac{z^{2k}}{(2k)!} + 
    \i \sum_{k=0}^\infty (-1)^k \frac{z^{2k+1}}{(2k+1)!}, \notag \\[1mm]
    \exp(-\i z) &= 
    \sum_{k=0}^\infty \frac{(-\i z)^{2k}}{(2k)!} + 
    \sum_{k=0}^\infty \frac{(-\i z)^{2k+1}}{(2k+1)!} 
    = 
    \sum_{k=0}^\infty (-1)^k \frac{z^{2k}}{(2k)!} - 
    \i \sum_{k=0}^\infty (-1)^k \frac{z^{2k+1}}{(2k+1)!}. \notag  
  \end{align}

  \medskip\noindent
  Für \slanted{reelle} Argumente $x$ sind $\cos x$ und $\sin x$ die 
  Real- bzw. Imaginärteile der Funktion $x\mapsto \exp(\i t)$, 
  was sich an der Definition ($\ast$) ebenso wie an der Reihenentwicklung 
  von $\exp(\i t)$ unmittelbar ablesen lässt. Für komplexe Argumente 
  gilt dieser Zusammenhang allerdings nicht mehr. \AntEnd
\end{antwort}

%% Question 19
\begin{frage}\label{05_trei}
  \index{Cosinus!Haupteigenschaften}\index{Sinus!Haupteigenschaften}
  Welche Haupteigenschaften von $\cos$ und $\sin$ sind Ihnen bekannt?
\end{frage}

\begin{antwort}
  \desc{1} Aus ($\ast$) folgt genauso wie aus ($\ast\ast$) unmittelbar, dass 
  $\cos$ eine \slanted{gerade} und $\sin$ eine \slanted{ungerade} 
  Funktion ist.

  \medskip\noindent
  \desc{2} Ebenso direkt folgt aus der 
  Definition ($\ast$) die \slanted{Euler'sche Formel}
  \index{Eulersche Formel@Euler'sche Formel}
  \[
  \boxed{
    \exp( \i z ) = \cos z + \i \sin z \qquad\text{für alle $z\in\CC$.}
  }
  \]
  Speziell gilt die Euler'sche Formel für $z\in\RR$. 
  In diesem Fall liefert sie eine einfache geometrische 
  Veranschaulichung der komplexen Exponentialfunktion (\sieheAbbildung\ref{fig:05_exp3}). Die Zahl 
  $w=\exp( x+\i y)$ liegt auf dem Kreis um den Nullpunkt mit Radius 
  $\exp x$, und $y \mod 2\pi$ gibt den Winkel an, den die Gerade 
  durch $0$ und $w$ mit der $x$-Achse einschließt, es ist also  
  $y=\arg w$ (vgl. dazu Frage \ref{polar} und speziell zum tieferen 
  Verständnis der Funktionalgleichung auch \ref{polarmult}). 

  \begin{center}
    \includegraphics{mp/05_exp3}
    \captionof{figure}{Geometrische Veranschaulichung der Euler'schen Formel.}
    \label{fig:05_exp3}
  \end{center}

  \medskip\noindent
  \desc{3} Durch Einsetzen der Euler'schen Formel in die 
  Gleichung $\exp(\i z)\exp(-\i z)=1$ erhält man die Identität 
  \[
  \boxed{
    \cos ^2 z + \sin^2 z = 1 \qquad\text{für alle $z\in\CC$.}
  }
  \]

  \medskip\noindent
  \desc{4} Die Euler'sche Formel liefert zusammen mit 
  $\exp(\i nz)=\exp(\i z)^n$ die \slanted{Moivre'sche Formel}
  \index{Moivresche Formel@Moivre'sche Formel}
  \[
  \boxed{
    ( \cos z +\i\sin z)^n = \cos nz +\i \sin nz.
  }
  \]

  \index{Additionstheorem!von Sinus und Cosinus}
  \medskip\noindent
  \desc{5} Für alle $z,w\in \CC$ gelten die \slanted{Additionstheoreme} 
  \[
  \boxed{
    \begin{array}{rcl}
      \cos( z+w ) &=& \cos z\cos w - \sin z\sin w, \\
      \sin( z+w ) &=& \sin z\cos w + \cos z\sin w.
    \end{array}
  }
  \]
  Diese folgen jeweils durch Addition bzw. Subtraktion der beiden 
  Gleichungen
  \begin{align}
    e^{\i(z+w)} &= e^{\i z}\cdot e^{\i w} \hspace*{-2mm} &=
    \cos z\cos w-\sin z\sin w +\i (\sin z\cos w + \cos z\sin w ) \notag \\
    e^{-\i (z+w)} &= e^{-\i z}\cdot e^{-\i w} \hspace*{-2mm} &=
    \cos z\cos w-\sin z\sin w - \i (\sin z\cos w + \cos z\sin w ), \notag
  \end{align}
  bei deren Herleitung von $\cos (-z)=\cos z$ und $\sin(-z)=-\sin z$ 
  Gebrauch gemacht wird.

  \medskip
  \noindent
  \desc{6} Eine Folge der Additionstheoreme sind die beiden Formeln
  \[
  \boxed{
    \begin{array}{rcl}
      \cos z - \cos w &=& -2\sin \frac{z+w}{2} \sin\frac{z-w}{2} \\
      \sin z - \sin w &=& 2\cos \frac{z+w}{2} \sin\frac{z-w}{2}.
    \end{array}
  }
  \]
  Um die beiden Formeln herzuleiten, schreibe man 
  $z=\frac{z+w}{2}+\frac{z-w}{2}$ bzw. 
  $w=\frac{w+z}{2}+\frac{w-z}{2}$ und wende darauf die 
  Additionstheoreme an. Das ergibt im ersten Fall
  \[
  \begin{array}{rcl}
    \cos z &=& \cos( \frac{z+w}{2}+\frac{z-w}{2} ) =
    \cos \frac{z+w}{2} \cos \frac{z-w}{2} - \sin \frac{z+w}{2}\sin \frac{z-w}{2} \\
    \cos w &=& \cos( \frac{z+w}{2}+\frac{w-z}{2} ) =
    \cos \frac{z+w}{2} \cos\frac{z-w}{2} + \sin\frac{z+w}{2}\sin\frac{z-w}{2},
  \end{array}
  \]
  wobei bei der Umformung der zweiten Gleichung wieder die Tatsache 
  verwendet wurde, dass $\cos$ eine gerade und $\sin$ eine ungerade Funktion 
  ist. Durch Subtraktion der beiden Gleichungen erhält man daraus 
  schließlich die Formel für $\cos z-\cos w$. 
  Die Darstellung für $\sin z-\sin w$ folgt nach demselben Schema.

  \medskip\noindent
  \desc{7} Um eine erste Informationen über die 
  \slanted{Nullstellen} der Cosinusfunktion zu gewinnen, 
  lässt sich die Eigenschaft ausnutzen, dass es sich bei deren 
  Reihenentwicklung für positive reelle Argumente 
  um eine \slanted{alternierende} Reihen handelt, 
  deren Summandenbeträge für alle $x\in\open{0,2}$ streng monoton fallende 
  Nullfolgen bilden. Man kann also das Leibniz-Kriterium aus Frage \ref{02_leib} 
  anwenden, und damit erhält man für alle $x\in \open{0,2}$ die Abschätzung
  \[
  1-\frac{x^2}{2} < \cos x < 1-\frac{x^2}{2}+\frac{x^4}{24} .
  \]
  Daraus folgt $\cos (2)<-1/4$. Wegen $\cos (0)=1$ besitzt 
  die Cosinusfunktion im Intervall $\open{0,2}$ also 
  \slanted{mindestens} eine Nullstelle. 

  Wir zeigen weiter, dass $\cos$ auf diesem Intervall streng monoton 
  fällt, woraus dann insgesamt folgt, dass es \slanted{genau eine} 
  Nullstelle in $\open{0,2}$ gibt. Sei dazu $x,y \in \open{0,2}$ mit $x>y$.   
  Die Differenzendarstellung aus Teilantwort \desc{6} liefert 
  für alle $x,y\in\RR$ 
  \begin{equation}
    \cos x -\cos y =  -2\sin \frac{x-y}{2} \sin\frac{x-y}{2}. \tag{$\dagger$}  
  \end{equation}
  Durch eine nochmalige Anwendung des Leibnizkriteriums --  
  diesmal auf die Sinus-Reihe -- erhält man für alle $x\in\open{0,2}$ 
  die Abschätzung 
  \[
  x-\frac{x^3}{6} < \sin x  < x, 
  \]
  also insbesondere $\sin x>0$ für alle $x\in\open{0,2}$. Eingesetzt 
  in die Gleichung ($\dagger$) liefert das $\cos x-\cos y<0$ für 
  alle $x,y\in\open{0,2}$ mit $x>y$. Der Cosinus fällt also streng monoton 
  auf dem Intervall $\ropen{0,2}$ und besitzt dort somit genau eine 
  Nullstelle $p$. Die Zahl $\pi$ lässt sich über diese Nullstelle definieren, 
  indem man $\pi/2 := p$ festlegt.

  \medskip
  \noindent
  \desc{8} Aus der Formel 
  \[
  e^{\i\pi/2} = \cos \frac{\pi}{2}+\i\sin\frac{\pi}{2} = \i 
  \]
  gewinnt man durch Potenzieren die Werte 
  \[
  \boxed{
    e^{\i\pi/2}=\i,\quad e^{\i\pi}=-1, \quad e^{3\pi/2} = -\i,\quad e^{2\pi \i}=1.
  }
  \]
  Mithilfe der Euler'schen Formel erhält man daraus die 
  Sinus- und Cosinuswerte an den entsprechenden Stellen 
  \[
  \boxed{
    \begin{array}{lp{2mm}lp{2mm}lp{2mm}l}
      \dis \cos \frac{\pi}{2}=0, & &  
      \cos \pi=-1,  & &  
      \dis \cos \frac{3\pi}{2}=0, & & 
      \cos 2\pi = 1, \\[3mm]
      \dis \sin \frac{\pi}{2}=1, & &  
      \sin \pi=0,  &  &
      \dis \sin \frac{3\pi}{2}=-1, & & 
      \sin 2\pi =0.
    \end{array}
  }
  \]

  \medskip
  \noindent
  \desc{9} Aus den Formeln
  \[
  e^{\i z+\i\pi/2} = ie^{\i z}, \qquad 
  e^{\i z+\i\pi} = -e^{\i z}, \qquad
  e^{\i z+2\i \pi} = e^{\i z}
  \]
  erhält man die Zusammenhänge
  \[
  \boxed{
    \begin{array}{ccc}
      \cos \left( z+\frac{\pi}{2} \right) = \sin z, & 
      \cos \left( z+\pi \right) = -\cos z, & 
      \cos \left( z+ 2\pi \right) = \cos z, \\
      \sin \left( z+\frac{\pi}{2} \right) = \cos z, & 
      \sin \left( z+\pi \right) = -\sin z, & 
      \sin \left( z+ 2\pi \right) = \sin z,
    \end{array}
  }
  \]
  aus denen insbesondere folgt, dass Sinus und Cosinus beide die reelle 
  Periode $2\pi$ haben. \index{Sinus!Periodizität}\index{Cosinus!Periodizität}
  \index{p@$\pi$!als Nullstelle von $\cos$}
  
  \medskip\noindent
  \desc{10} Aus der vorhergehenden Teilantwort erhält man nun auch sofort 
  \slanted{sämtliche} Nullstellen von Sinus und Cosinus. 
  Nach Teilfrage (7) 
  ist $\frac{\pi}{2}$ die kleinste positive Nullstelle von $\cos$.
  Wegen $\cos(x+\pi)=-\cos x$ sind damit 
  $\frac{\pi}{2}$ und $\frac{3\pi}{2}$ die beiden 
  einzigen Nullstellen von $\cos$ im Intervall 
  $\blopen{ -\frac{\pi}{2}, \frac{3\pi}{2}}$. Da dieses Intervall die 
  Länge $2\pi$ hat, erhält man aus diesen Nullstellen alle weiteren 
  durch Addition eines ganzzahligen Vielfachen von $2\pi$. 
  Schließlich bekommt man aus den Nullstellen des Cosinus diejenigen 
  von Sinus durch Anwendung der Formel 
  $\sin( z+\frac{\pi}{2} )=\cos z$. 
  Zusammenfassend gilt also
  \[
  \boxed{
    \cos z=0 \LLa z = \frac{\pi}{2} + k\pi, \quad
    \sin z=0 \LLa z = k\pi. \quad{ k\in\ZZ }.  
  }
  \]

  \medskip
  \noindent
  \desc{11} Jetzt kann noch gezeigt werden, dass $2\pi$ 
  auch tatsächlich die \slanted{kleinste} Periode von Sinus und Cosinus 
  ist. Aufgrund der Nullstellenverteilung käme als kleinere Periode 
  nur die Zahl $\pi$ infrage, die aber wegen $\cos(z+\pi) = -\cos z$ 
  keine Periode sein kann. 

  \medskip
  \noindent
  \desc{12} Die Funktionen $\cos$ und $\sin$ 
  bilden die reellen Zahlen surjektiv auf das Intervall $[-1,1]$ ab. 
  Damit folgt zu guter Letzt, dass die Funktion
  \[
  \RR \to S^1 := \{ z\in\CC;\; |z|=1 \}, \qquad t\mapsto \exp(\i t)
  \]
  tatsächlich surjektiv ist, wie es in Frage \ref{05_exho} 
  bereits vorweggenommen wurde. 
  Sei $w\in S^1$. Dann gilt $\Re w \le 1$, $\Im w \le 1$ und 
  $(\Re w)^2+ (\Im w)^2=1$. Es gibt also ein $ t\in \RR$ mit 
  $\cos t=\Re w$. Wegen $\cos^2 t+\sin^2 t=1$ folgt $\sin t=\Im w$ oder 
  $-\sin t=\Im w$. Im ersten Fall gilt dann bereits 
  $\exp(\i t)=\cos t+ \i\sin t=w$, 
  im zweiten hat man $\exp(-\i t)=\cos t - \i\sin t=w$. 
  Die Funktion $\exp(\i t)$ bildet $\RR$ also tatsächlich surjektiv auf $S^1$ ab.
  \AntEnd 
\end{antwort} 

%% Question 20
\begin{frage}
  Können Sie im reellen Fall die Graphen von $\cos$ und $\sin$ 
  im Intervall $[-4\pi,4\pi]$ skizzieren?
\end{frage}

\begin{antwort}
  Die durchgezogene Kurve in Abbildung \ref{fig:05_sincos} zeigt den Graphen 
  von $\cos$, die gestrichelte den von $\sin$.\AntEnd
  \begin{center}
    \includegraphics{mp/05_sincos}
    \captionof{figure}{Graphen der reellen Sinus- und Cosinusfunktion.}
    \label{fig:05_sincos}
  \end{center}
\end{antwort}

%% Question 21
\begin{frage}\label{05_tangens}\index{Tangens}\index{Cotangens}
  Wie sind die Funktionen Tangens ($\tan$) und Cotangens ($\cot$) im Komplexen 
  definiert? Skizzieren Sie für reelle Argumente die Graphen dieser Funktionen 
  im Intervall $\left[-\frac{3\pi}{2}, \frac{3\pi}{2} \right]$. 
\end{frage}

\begin{antwort}
  Seien $N_{\cos} := \{ (k+1/2)\pi\sets k\in\ZZ \}$ und 
  $N_{\sin} := \{ k\pi\sets k\in\ZZ \}$ die Nullstellenmengen von $\cos$ 
  bzw. $\sin$. Tangens und Cotangens sind außerhalb dieser Mengen  
  definiert durch
  \[
  \boxed{
    \tan z := \frac{\sin z}{\cos z},\quad z\in 
    \CC\mengeminus N_{\cos},
    \qquad
    \cot z := \frac{\cos z}{\sin z},\quad z\in 
    \CC\mengeminus N_{\sin}. }
  \]
  Durch Einsetzen der Formeln $\cos z=\frac{1}{2}( e^{\i z}+e^{-\i z} )$ 
  und $\sin z=\frac{1}{2\i} ( e^{\i z}-e^{-\i z} )$ erhält man daraus auch die 
  Darstellungen 
  \[
  \tan z := \frac{1}{\i} \frac{e^{\i z}-e^{-\i z}}{e^{\i z}+e^{-\i z}}, \quad
  \cot z := \i \frac{e^{\i z}+e^{-\i z}}{e^{\i z}-e^{-\i z}}.
  \]
  Die Eigenschaften des Tangens folgen aus denen von Sinus und 
  Cosinus. Demnach ist $\tan$ eine ungerade, $\pi$-periodische Funktion. 
  Im Intervall $\bropen{ 0,\frac{\pi}{2} }$ ist Sinus streng monoton 
  wachsend und Cosinus streng monoton fallend. 
  Deswegen und aufgrund seiner Symmetrie- 
  und Periodizitätseigenschaften 
  ist der Tangens auf den Intervallen 
  $\open{ (k-1/2)\pi, (k+1/2)\pi }$ jeweils 
  streng monoton wachsend, und es gilt:
  \[
  \lim\limits_{x\uparrow (k+1/2)\pi }=\infty \quad\text{ und }\quad
  \lim\limits_{x\downarrow (k+1/2)\pi }=-\infty.
  \] 
  Der Tangens bildet somit jedes der Intervalle 
  $\open{ (k-1/2)\pi, (k+1/2)\pi }$ bijektiv auf $\RR$ ab.  

  \begin{center}
    \includegraphics{mp/05_tan}
    \captionof{figure}{Graphen von $\tan$ und $\cot$.} 
    \label{fig:05_tan}
  \end{center}

  Die Eigenschaften von $\cot$ ergeben sich aus denen von $\tan$ 
  über den Zusammenhang $\cot x=\frac{1}{\tan x}$ bzw. 
  $\cot x=-\tan(x+\pi/2)$, \sieheAbbildung\ref{fig:05_tan}.
  \AntEnd
\end{antwort}

%% Question 22
\begin{frage}\label{05_sinh}\index{Sinus Hyperbolicus}
  Wie sind die \bold{hyperbolischen Funktionen} $\sinh$ und $\cosh$ 
  definiert? Welche Haupteigenschaften haben sie? Können Sie im reellen 
  Fall ihren Graphen skizzieren? 
\end{frage}

\begin{antwort}
  Die Funktionen $\cosh$ bzw. $\sinh$ sind für alle $z\in\CC$ definiert durch
  \[
  \boxed{
    \cosh z= \frac{e^z + e^{-z}}{2}, \quad\qquad \sinh z= \frac{e^{z}-e^{-z}}{2}.
  }
  \]
  Damit gilt offensichtlich
  \[
  \cosh(\i z)=\cos z, \quad 
  \cos (\i z)=\cosh z, \quad
  \sinh(\i z)=\i\sin z, \quad
  \sin(\i z)= -\i\sinh z.
  \]
  Mit $\cos^2 z+\sin^2 z=1$ folgt hieraus
  \[
  \boxed{\cosh^2 z - \sinh^2 z = 1.}
  \]
  Weiter sieht man anhand der Definition sofort, dass
  $\cosh$ eine gerade und $\sinh$ eine ungerade Funktion ist. 
  Aus den Additionstheoremen für Cosinus und Sinus folgt ferner
  \begin{align*}
    \cosh( z+w ) &= \cosh z\cosh w+\sinh z\sinh w \\
    \sinh( z+w ) &= \sinh z\cosh w+\cosh z\sinh w,
  \end{align*}
  Aus der Potenzreihenentwicklung der Exponentialfunktion 
  erhält man außerdem 
  \[
  \cosh z = \sum_{k=0}^n \frac{z^{2k}}{(2k)!}, \qquad
  \sinh z = \sum_{k=0}^n \frac{z^{2k+1}}{(2k+1)!}.
  \]
  Aus der Potenzreihenentwicklung folgt, dass und $\cosh x$ 
  und $\sinh x$ beide auf $\ropen{0,\infty}$ streng monoton wachsen. 
  Da $\sinh x$ ungerade und $\cosh x$ gerade ist, 
  ist $\sinh x$ sogar streng monton wachsend 
  auf ganz $\RR$, 
  $\cosh x$ dagegen ist auf $\lopen{-\infty,0}$ streng monoton fallend, 
  \sieheAbbildung\ref{fig:05_sinh}.
  \AntEnd

  \begin{center}
    \includegraphics{mp/05_sinh}
    \captionof{figure}{Graphen $\sinh$ und $\cosh$.} 
    \label{fig:05_sinh}
  \end{center}

\end{antwort}

%% Question 23
\begin{frage}\index{Tangens Hyperbolicus}\index{Cotangens Hyperbolicus}
  Wie sind die Funktionen $\tanh$ und $\coth$ definiert und welche 
  Haupteigenschaften haben sie? Können Sie im reellen Fall ihren 
  Graphen skizzieren?
\end{frage}


\begin{antwort}
  Die Funktionen $\tanh$ und $\coth$ sind definiert durch: 
  \[\boxed{
    \begin{array}{cp{2mm}l}
      \dis \tanh z := \dis \frac{\sinh z}{\cosh z}, & & \dis {z\in\CC},\\[3mm]
      \dis \coth z := \dis \frac{\cosh z}{\sinh z}, & & \dis {z\in\CC}\mengeminus\{0\}.
    \end{array}}
  \]
  \noindent
  Aus den Eigenschaften von $\cosh$ und $\sinh$ folgt unmittelbar, dass 
  es sich bei $\tanh$ und $\coth$ um ungerade Funktionen handelt. Während 
  $\coth$ nullstellenfrei ist, besitzt $\tanh$ eine einzige Nullstelle bei 
  $z=0$. Weiter kann man die Darstellungen 
  \[
  \tanh z = \frac{e^z-e^{-z}}{e^z+e^{-z}} = 1-\frac{2}{e^{2z}+1},
  \qquad
  \coth z = \frac{e^z+e^{-z}}{e^z-e^{-z}} = 1+\frac{2}{e^{2z}-1}
  \]
  \picskip{0}\noindent
  nutzen, um zu sehen, dass $\tanh x$ auf ganz $\RR$ streng 
  monoton steigt und $\coth x$ auf $\open{-\infty,0}$ 
  und $\open{0,\infty}$ jeweils streng monoton fällt, \sieheAbbildung\ref{fig:05_coth}. In beiden Fällen 
  folgt das aus dem streng monotonen Wachstum der Exponentialfunktion. 

  \begin{center}
    \includegraphics{mp/05_coth}
    \captionof{figure}{Graphen von $\tanh$ und $\coth$.}
    \label{fig:05_coth}
  \end{center}

  Aus den Darstellungen folgt außerdem
  \[
  \lim_{x\to\infty} \tanh x=1,\quad
  \lim_{x\to-\infty} \tanh x=-1\quad
  \]
  sowie
  \begin{equation}
    \lim_{x\to\infty} \coth x=1,\quad
    \lim_{x\to-\infty} \coth x=-1,\quad
    \lim_{x\uparrow 0} =-\infty\quad
    \lim_{x\downarrow 0} = \infty.
    \EndTag
  \end{equation}
\end{antwort}

%% Question 24
\begin{frage}\index{hyperbolische Funktionen}
  Woher kommt die Namensgebung bei den hyperbolischen Funktionen.
\end{frage}

\begin{antwort}
  Der jeweilige "`Vorname"' der hyperbolischen Funktionen erklärt sich 
  durch die offensichtlichen Analogien zu den Winkelfunktionen. 
  Der Hyperbel-Aspekt kommt daher, dass die Punkte 
  $(x,y)=(\cosh t, \sinh t)$ wegen $\cosh^2 t-\sinh^2 t=1$ alle auf der Hyperbel 
  $x^2-y^2=1$ liegen. Mittels $t \mapsto (a\cosh t, b\sinh t)$ lässt sich 
  also die durch die Gleichung $x^2/a^2-y^2/b^2=1$ definierte 
  Hyperbel parametrisieren.  
  \AntEnd
\end{antwort}


\section{Natürlicher Logarithmus und allgemeine Potenzen}

%% Question 25
\begin{frage}\label{05_exsu}\index{Exponentialfunktion!Umkehrfunktion der}
  \index{Logarithmus}
  Warum besitzt die reelle Exponentialfunktion $\exp\fd\RR\to\RR$ eine 
  Umkehrfunktion
  \[
  \log\fd \RR_+^{\ast} \to \RR \text{?}
  \]
\end{frage}

\begin{antwort}
  Für alle $x>0$ ist $\exp x>1$, was man an der 
  Potenzreihenentwicklung der Exponentialfunktion unmittelbar erkennen 
  kann. Für alle $x,y\in\RR$ mit $x>y$ folgt daraus mit der Funktionalgleichung 
  $\exp (x)/\exp (y)=\exp(x-y) > 1,$
  also $\exp x>\exp y$ und damit das streng monotone 
  Wachstum der reellen Exponentialfunktion.

  Ferner gilt $\lim\limits_{x\to \infty} \exp x=\infty$, 
  da die Summanden in der 
  Potenzreihenentwicklung von $\exp$ für $x\to\infty$ alle 
  positiv und unbeschränkt sind. 
  Wegen $\exp( -x )= 1/\exp x$ folgt daraus 
  $\lim\limits_{x\to-\infty}=0$. 
  Die Exponentialfunktion bildet $\RR$ damit surjektiv auf 
  $\RR_+^{\ast}$ 
  ab. Aufgrund ihres streng monotonen Wachstums ist die 
  Funktion $\exp\fd \RR\to\RR^*_+$ 
  aber auch injektiv und damit insgesamt bijektiv. 
  Es existiert also eine Umkehrfunktion 
  $\log \fd \RR_+^*\to \RR$.\AntEnd
\end{antwort} 

%% Question 26
\begin{frage}\index{Logarithmus}
  Wieso ist $\log\fd\RR_+^* \to\RR$ stetig und streng monoton wachsend?
\end{frage} 

\begin{antwort}
  Die Stetigkeit und das streng monotone Wachstum von $\log$ ergibt sich 
  aus dem in Frage \ref{03_umke} gezeigten allgemeinen Zusammenhang, demzufolge 
  die Umkehrfunktion einer streng monoton wachsenden stetigen Funktion 
  ebenfalls stetig und streng monoton wachsend ist, 
  \sieheAbbildung\ref{fig:05_log}. \AntEnd

  \begin{center}
    \includegraphics{mp/05_log}
    \captionof{figure}{Graph der reellen Logarithmus-Funktion.}
    \label{fig:05_log}
  \end{center}

\end{antwort}

%% Question 27
\begin{frage}\index{Funktionalgleichung!des Logarithmus}
  Wie lautet die \bold{Funktionalgleichung} für den Logarithmus?
\end{frage}

\begin{antwort}
  Aus $\exp( \log x+\log y )= 
  \exp(\log x)\exp( \log y)=xy= \exp( \log(xy) )$ folgt die 
  \slanted{Funktionalgleichung des Logarithmus}
  \[
  \boxed{
    \log(xy) = \log x+ \log y, \qquad x,y\in\RR_+^*.
  }\EndTag
  \]
\end{antwort}

%% Question 28
\begin{frage}\label{05_lnwa}\index{Logarithmus}
  Warum gilt $\lim\limits_{x\to\infty} \dis \frac{\log x}{\sqrt[n]{x}}=0$ 
  für alle $n\in\NN$?
\end{frage}

\begin{antwort}
  Durch die Substitution $x\mapsto e^{nt}$ reduziert sich 
  die Behauptung auf den in Frage \ref{05_exwa} 
  gezeigten Zusammenhang über die Wachstumsgeschwindigkeit der 
  Exponentialfunktion. Aufgeschrieben sieht das so aus:
  \begin{equation}
    \lim_{x\to\infty} \frac{\log x}{\sqrt[n]{x}} =
    \lim_{t\to\infty} \frac{\log e^{nt}}{\sqrt[n]{e^{nt}}} = 
    \lim_{t\to\infty} \frac{nt}{e^t} = 0. \notag
  \end{equation}
  Der Grenzwert charakterisiert $\log$ als eine in großen Bereichen 
  extrem langsam wachsende Funktion. \AntEnd
\end{antwort}

%% Question 29
\begin{frage}\index{Potenz!allgemeine}
  Wenn man für $a>0$ und $x\in\RR$ $a^x := \exp( x\log a )$, 
  definiert, warum gelten dann für $a,b\in\RR_+^*$ und $x,y\in\RR$ die 
  Rechenregeln
  \[
  a^xa^y = a^{x+y},\qquad
  (a^x)^y = a^{xy},\qquad
  a^x b^x = (ab)^x,\qquad
  \left( \frac{1}{a} \right)^x =a^{-x}\, \text{?}
  \]
\end{frage}

\begin{antwort}
  Alle vier Identitäten erhält man leicht aus den Funktionalgleichungen 
  von $\exp$ und $\log$: 
  \begin{align*}
    a^xa^y &= e^{ x\log a } e^{ x\log b } = e^{(x+y)\log a} = a^{x+y}, \\ 
    (a^x)^y &= (e^{x\log a})^y = e^{ y \log( e^{x\log a} )} = 
    e^{xy \log a} =a^{xy}, \\
    a^x b^y &= e^{x\log a}e^{x\log b} = 
    e^{x\log a + x\log b}=e^{x\log(ab)}=(ab)^x,\\ 
    \left(\frac{1}{a}\right)^x &= 
    e^{x( \log(1)-\log a )} = e^{-x\log a} = a^{-x}.\EndTag
  \end{align*}
\end{antwort}

%% Question 30
\begin{frage}
  Können Sie begründen, warum für eine stetige Funktion 
  $F\fd \RR\to\RR$ mit der Eigenschaft $F(x+y)=F(x)F(y)$ für 
  alle $x,y\in\RR$ notwendig $F(x)=0$ für alle $x\in\RR$ oder 
  $F(x)=a^x$ mit $a:=F(1)>0$ für alle $x\in\RR$ gilt?
\end{frage}

\begin{antwort}
  Sei $F$ nicht die Nullfunktion. 
  Dann sind die Werte $F(n)$ für alle natürlichen Zahlen $n$ wegen 
  $F(n)=F(1+\cdots+1)=F(1)^n = a^n$ 
  bereits festgelegt und stimmen an diesen Stellen mit 
  der Funktion $a^x$ überein. 
  Wegen $F(-n)=F(n)^{-1}$ überträgt sich diese Übereinstimmung 
  auch noch auf die ganzen Zahlen. Für eine 
  rationale Zahl $\frac{p}{q}$ mit $p,q\in\ZZ$ 
  gilt damit aber wegen 
  $F( 1 )= F\left( q \cdot \frac{p}{q} \right ) = 
  F\left( \frac{p}{q} \right)^q$ auch
  \[ 
  F\left(  \frac{p}{q} \right) = \sqrt[q]{ F(p) } =  \sqrt[q]{ a^p }= a^{p/q}
  \]
  und somit $F(x)=a^x$ für alle $x\in\QQ$. 
  Daraus folgt nun aus Stetigkeitsgründen 
  auch die Übereinstimmung von $F(x)$ und $a^x$ für alle $x\in\RR$.  
  \AntEnd 
\end{antwort}

%% Question 31
\begin{frage}\index{Potenzfunktion}
  Definiert man für $a\in\RR, a\not=0$ und $x\in\RR_+$ als 
  \slanted{allgemeine Potenzfunktion}
  \[
  \boxed{ x^a := \exp( a\log x),}
  \]
  warum gilt dann
  \[
  \begin{array}{llp{4mm}ll}
    \desc{1} & 
    \lim\limits_{x\to\infty} x^a \left\{ 
      \begin{array}{ll} 
        \infty & \text{falls $a>0$,} \\
        0 & \text{falls $a<0$,} 
      \end{array} \right.
    & &
    \desc{2} & 
    \lim\limits_{x\downarrow 0} x^a \left\{ 
      \begin{array}{ll} 
        0 & \text{falls $a>0$,} \\
        \infty, & \text{falls $a<0$,} 
      \end{array} \right.
    \\[5mm]
    \desc{3} & 
    \lim\limits_{x\to\infty} \dis\frac{\log x}{x^a} = 0\quad \text{für $a>0$,} 
    & &
    \desc{4} &
    \lim\limits_{x\downarrow 0} x^a \log x =0 \quad\text{für $a>0$?}
  \end{array}
  \] 
\end{frage}

\begin{antwort}
  \desc{1} Die Gleichungen 
  $\lim\limits_{x\to\infty} e^x=\infty$, 
  $\lim\limits_{x\to\infty} e^x=-0$ und 
  $\lim\limits_{x\to\infty} \log x=\infty$ zusammen legen dieses 
  Grenzwertverhalten fest.

  \medskip
  \noindent
  \desc{2} Das folgt aus $\dis 
  \lim_{x\downarrow 0} \log x =-\infty,\quad
  \lim_{x\to -\infty} \exp x = 0, \quad
  \lim_{x\to\infty} \exp x = \infty.$

  \medskip
  \noindent
  \desc{3} Man wähle ein $n\in \NN$ mit $a > \frac{1}{n}$. Dann folgt 
  die Grenzwertbeziehung aus der Antwort zu Frage \ref{05_lnwa}. 

  \medskip
  \noindent
  \desc{4} Mit dem Ergebnis \desc{3} erhält man  
  \begin{equation}
    \lim_{x\downarrow 0} x^a \log x = 
    \lim_{x\to\infty} \frac{\log(1/x)}{x^a} 
    = - \lim_{x\to\infty} \frac{\log x}{x^a} = 0.
    \EndTag
  \end{equation}
\end{antwort}


\section{Die Umkehrfunktionen der trigonometrischen und 
  hyperbolischen Funktionen}

%% Question 32
\begin{frage}
  \index{Arcus-Cosinus}\index{Arcus-Funktionen}
  \index{Arcus-Tangens}
  \index{Arcus-Sinus}
  \index{Arcus-Cotangens}
  Auf welchen Intervallen verlaufen die Winkelfunktionen 
  $\cos$, $\sin$, $\tan$ und $\cot$ jeweils streng monoton, 
  sodass für die Einschränkungen dieser 
  Funktionen auf die entsprechenden Intervalle die 
  Umkehrfunktionen Arcus-Cosinus, Arcus-Sinus, Arcus-Tangens 
  und Arcus-Cotangens existieren?
\end{frage}

\begin{antwort}
  Der Cosinus verläuft für alle $k\in\ZZ$ 
  auf den Intervallen $[k\pi, (k+1)\pi]$ streng monoton, 
  der Sinus entsprechend auf den Intervallen 
  $\left[\left( k-\frac{1}{2} \right)\pi,\left(k+\frac{1}{2}\right)\pi\right]$. 
  Die Einschränkungen von Cosinus bzw. Sinus auf diese Intervalle 
  besitzen für alle $k\in\ZZ$ also jeweils eine Umkehrfunktion
  \[
  \arccos_k\fd [-1,1] \to [k\pi, (k+1)\pi],\qquad
  \arcsin_k\fd [-1,1] \to [(k-1/)\pi, (k+1/2)\pi].
  \]
  Der Tangens verläuft auf den Intervallen 
  $\bopen{\left( k-\frac{1}{2} \right)\pi,\left(k+\frac{1}{2}\right)\pi}$
  streng monoton, der Cotangens auf den Intervallen 
  $\open{ k\pi,(k+1)\pi }$. Somit existieren für 
  alle $k\in \ZZ$ Umkehrfunktionen
  \[
  \arctan_k\fd 
  \text{$\left(\left( k-\frac{1}{2} \right)\pi,
      \left(k+\frac{1}{2}\right)\pi\right)$} \to \RR,\qquad
  \arccot_k\fd (k\pi, (k+1)\pi) \to \RR.\EndTag
  \]
\end{antwort}



%% Question 33
\begin{frage}\index{Hauptzweig der Arcus-Funktionen}
  Wie sind die Hauptzweige dieser Funktionen definiert, und was versteht 
  man allgemein für $n\in\ZZ$ mit $n\not=0$ 
  unter dem $n$-ten Nebenzweig dieser Funktionen? 
\end{frage}

\begin{antwort}
  Der Hauptzweig der Arcus-Funktionen ist jeweils die für $k=0$ gegebene 
  Funktion aus der obigen Definition, entsprechend der $n$-te Nebenzweig 
  die für $k=n$ gegebene Funktion. 

  \begin{center}
    \includegraphics{mp/05_arccos}
    \captionof{figure}{Hauptzweige der Arcus-Funktionen 
      (durchgezogene Linien) und 
      benachbarte Nebenzweige (gestrichelt).} 
    \label{fig:05_arccos}
  \end{center}

  Abbildung \ref{fig:05_arccos} zeigt die Graphen der Hauptzweige 
  der Arcus-Funktionen als durchgezogene Linien. Die gestrichelten Graphen 
  deuten benachbarte Nebenzweige an. 
  \AntEnd
\end{antwort}

%% Question 34
\begin{frage}
  \index{Arcus-Sinus-Hyperbolicus}
  Warum besitzt $\sinh\fd \RR\to \RR$ eine Umkehrfunktion $\asinh$, 
  und warum gilt für diese 
  \[
  \asinh x = \log(x+\sqrt{x^2+1}) \text{?}
  \] 
  Können Sie den Graphen von $\asinh$ skizzieren?
\end{frage}


\begin{antwort}
  Die Funktion $\sinh x$ wächst streng monoton auf ganz $\RR$, 
  woraus die Existenz der Umkehrfunktion folgt. 
  \begin{center}
    \includegraphics{mp/05_arsinh}
    \captionof{figure}{Graph von $\asinh$}
    \label{fig:05_arsinh}
  \end{center}
  Wegen $\sinh y+\cosh y=e^y$ und $\cosh^2 y- \sinh^2 y=1$ gilt
  \[
  \log ( \sinh y + \sqrt{ \sinh^2 y+ 1} )= y.
  \]
  \picskip{2}\noindent
  Mit $\sinh y=x$ bzw. $\asinh(x)=y$ folgt daraus die Formel 
  für die Umkehrfunktion. Der Graph ist in \Abb\ref{fig:05_arsinh} dargestellt.
  \AntEnd
\end{antwort}

%% Question 35
\begin{frage}
  \index{Arcus-Cosinus-Hyperbolicus}
  Die Funktion $\cosh \fd \RR_+ \to \RR$ ist nach Frage 
  \ref{05_sinh} ebenfalls streng 
  monton und damit umkehrbar. Können Sie die Umkehrfunktion $\acosh$ mithilfe 
  des natürlichen Logarithmus ausdrücken?
\end{frage}

\begin{antwort}
  Die Identitäten $\cosh y+\sinh y=e^y$ und 
  $\cosh^2 y-\sinh^2 y=1$ liefern 
  \[
  y=\log( e^y )= \log( \cosh y + \sqrt{ \cosh^2 -1 } ),
  \]
  und mit $x=\cosh y$ für $x\in\ropen{1,\infty}$ erhält man damit
  \begin{equation}
    \acosh x = \log( x + \sqrt{ x^2 -1 } ).\EndTag
  \end{equation}

  \begin{center}
    \includegraphics{mp/05_arcosh}
    \captionof{figure}{Graph von $\acosh$.}
    \label{fig:05_arcosh}
  \end{center}

  Abbildung \ref{fig:05_arcosh} zeigt den Graphen von $\acosh$.
\end{antwort}

%% Question 36
\begin{frage}
  \index{Arcus-Tangens-Hyperbolicus}
  Die Funktion $\tanh \fd \RR\to \RR$ ist ebenfalls streng monoton 
  wachsend und damit umkehrbar. Warum gilt für die Umkehrfunktion 
  \[
  \atanh (x)= \frac{1}{2} \log \frac{1+x}{1-x}, \qquad\text{für $-1<x<1$?}
  \]
  Wie sieht der Graph dieser Funktion aus?
\end{frage}

\begin{antwort}
  Aus 
  \[
  \atanh y = \frac{e^{y}-e^{-y}}{e^y+e^{-y}} = 1-\frac{2}{e^{2y}+1} =x 
  \]
  folgt 
  \[
  e^{2y} = \frac{2}{1-x} -1 \Ra 
  e^{2y} = \frac{1+x}{1-x} \Ra
  y = \frac{1}{2} \log\frac{1+x}{1-x}, 
  \]
  \picskip{1}\noindent
  und damit die gesuchte 
  Funktionsgleichung.\AntEnd

  \begin{center}
    \includegraphics{mp/05_artanh}
    \captionof{figure}{Graph von $\atanh$.}
    \label{fig:05_atanh}
  \end{center}

  Abbildung \ref{fig:05_atanh} zeigt den Graphen von $\atanh$.

\end{antwort}

%% Question 37
\begin{frage}
  \index{Arcus-Cotangens-Hyperbolicus}
  Die Funktion $\coth\fd \RR_+ \to \RR$ ist streng monton 
  fallend und damit umkehrbar. Warum gilt für die Umkehrfunktion
  \[
  \acoth = \frac{1}{2}\log \frac{x+1}{x-1} \qquad\text{für $|x|>1$?}
  \]
\end{frage}

\begin{antwort}
  Der Schlüssel für die Herleitung der Umkehrfunktion liegt in der 
  Darstellung 
  \[
  \coth y = \frac{e^y+e^{-y}}{e^y-e^{-y}} = 1+\frac{2}{e^{2y}-1}.
  \]
  Durch Umformung der Gleichung und anschließendes Logarithmieren 
  erhält man wie im Fall von $\artanh$ die gewünschte Funktionsgleichung. 

  \begin{center}
    \includegraphics{mp/05_arcoth}
    \captionof{figure}{Graph von $\acoth$.}
    \label{fig:05_arcoth}
  \end{center}

  Abbildung \ref{fig:05_arcoth} zeigt den Graphen von $\acoth$.
  \AntEnd
\end{antwort}


























%%% Local Variables: 
%%% mode: latex
%%% TeX-master: "\""
%%% End: 
