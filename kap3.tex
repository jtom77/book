\chapter{(Unendliche) Reihen}

Die Fragen zu diesem Kapitel beschäftigen sich mit dem 
\slanted{Reihenbegriff}, speziell mit solchen Reihen, 
deren Summanden reelle oder komplexe Zahlen sind. Da Reihen nach 
Definition Folgen einer speziellen Bauart sind, gelten für sie dieselben 
Konvergenzkriterien wie für Folgen, darüber hinaus gibt es für Reihen  
aber noch eine Fülle spezifischer Konvergenzkriterien. Eine besonders 
wichtige Klasse von Reihen sind die \slanted{absolut konvergenten} Reihen, 
deren Summandenfolgen beliebig umgeordnet werden dürfen, ohne dass sich 
dadurch das Konvergenzverhalten der Reihe ändert. Diese 
Eigenschaft spielt eine große Rolle bei der Multiplikation konvergenter Reihen. 

\slanted{Potenzreihen} als spezielle \slanted{Funktionenreihen} werden 
in diesem Kapitel nur vorgestellt und erst in Kapitel 5 
systematischer behandelt. 


\section{Definitionen und erste Beispiele}


%% --- 1 --- %%
\begin{frage}\label{02_rdef}
  Was versteht man unter der einer reellen oder komplexen Zahlenfolge 
  $\sans{(a_k)}$ zu\-ge\-ord\-ne\-ten \bold{Reihe}\index{Reihe}?
\end{frage}

\begin{antwort}
  \index{Partialsumme}
  Unter der einer Folge $(a_k)$ zugeordneten Reihe versteht man die Folge 
  $(s_n)$ der (Partial-)Summen
  \[
  s_n := a_0+ a_1+a_2+\cdots + a_n.
  \]
  Eine Reihe ist damit einfach eine Folge spezieller Bauart und kann mit 
  denselben begrifflichen Mitteln beschrieben werden. 

  Man sollte nicht auf die Idee kommen, eine Reihe als eine 
  "`Summe mit unendlich vielen Summanden"' zu definieren. 
  Das ist kein sauberes mathematisches Konzept und wird, 
  sobald es ernst wird, auch nur Verwirrung stiften. 
  \AntEnd
\end{antwort}

%% --- 2 --- %%
\begin{frage}\label{02_rsymb}
  Welche Symbolik (Schreibweisen) verwendet man im Zusammenhang mit Reihen?
\end{frage}

\begin{antwort}
  Die einer Folge $(a_k)$ zugeordnete Reihe bezeichnet man 
  üblicherweise mit  
  \[
  \sum_{k=0}^\infty a_k.
  \]
  Dieser Ausdruck ist als Abkürzung des präzisen Ausdrucks 
  $\left( \sum_{k=0}^n a_k \right )_{n\in\NN_0}$ zu verstehen. 

  Je nachdem, mit welchem Index die Folge $(a_k)$ beginnt, kann die Reihe 
  auch bei einem anderen Wert als $k=0$ anfangen, 
  etwa bei $k=1$ oder $k=2$ etc. 
  
  Bei allgemeinen Konvergenzuntersuchungen, bei denen es nicht 
  auf den Startwert einer Reihe ankommt, schreibt man abkürzend 
  auch oft $\sum_k a_k$ und lässt 
  dabei offen, bei welchem Index die Reihe beginnt. 
  \AntEnd 
\end{antwort}

%% --- 3 --- %%
\begin{frage}\label{02_partial}
  Was versteht man unter der $\sans{n}$-ten \bold{Partialsumme} einer Reihe?
  \index{Partialsumme}
\end{frage}

\begin{antwort}
  Die $n$-te \slanted{Partialsumme} der einer Folge $(a_k)$ zugeordneten 
  Reihe ist die Zahl
  \[
  s_n := \sum_{k=0}^n a_k = a_0+ a_1+ \cdots + a_n.
  \]
  Eine Reihe ist somit \slanted{die Folge $(s_n)$ ihrer Partialsummen}.
  \AntEnd
\end{antwort}

%% --- 4 --- %%
\begin{frage}\label{02_rkonv}
  Wann heißt eine Reihe \bold{konvergent}?
  \index{Konvergenz!einer Reihe}
  \index{Reihe!Konvergenz}
\end{frage}

\begin{antwort}
  Eine Reihe ist konvergent genau dann, wenn die Folge $(s_n)$ 
  ihrer Partialsummen im Sinne des für Folgen eingeführten 
  Konvergenzbegriffs (vgl. Frage \ref{02_fkon}) 
  konvergiert. Speziell im Bezug auf Reihen lautet die Definition dann:

  \medskip\noindent\slanted{Eine Reihe $S=\sum_k a_k$ konvergiert 
    genau dann gegen 
    den Grenzwert $a$, wenn für jedes $\eps>0$ ein $N\in\NN$ 
    mit der Eigenschaft existiert, dass für alle $n>\NN$ gilt}: 
  \[
  \left| \sum_{k=0}^n a_k -a \right| < \eps \EndTag
  \] 
\end{antwort}

%% --- 5 --- %%
\begin{frage}\label{02_rwert}
  Was ist der Unterschied zwischen einer Reihe und ihrem Wert (ihrer Summe)?
\end{frage}

\begin{antwort}
  Da eine Reihe nichts anderes als 
  eine Folge ist, entspricht der Unterschied 
  dem zwischen einer Folge und ihrem Grenzwert.

  Der Wert einer Reihe wird allerdings häufig ebenfalls mit dem Symbol 
  $\sum_{k=0}^\infty a_k$ bezeichnet, der in \slanted{diesem} Zusammenhang 
  gleichbedeutend ist mit $\limm \sum_{k=0}^n a_k$. 
  Das Symbol besitzt somit eine gewisse Zweideutigkeit. 
  So bedeutet die Gleichung $\sum_{k=1}^\infty \frac{1}{k^2} = \frac{\pi^2}{6}$, 
  dass die Reihe $\sum_{k=1}^\infty \frac{1}{k^2}$ konvergiert und die 
  Summe $\frac{\pi^2}{6}$ hat.  
  \AntEnd
\end{antwort}


%% --- 6 --- %%
\begin{frage}\label{02_regeln}
  \index{Reihe!Rechenregeln}
  Welche Rechenregeln für (konvergente) Folgen lassen sich unmittelbar 
  auf (konvergente) Reihen übertragen (Permanenzeigenschaften)?
\end{frage}

\begin{antwort}
  Seien $\sum_{k=0}^\infty a_k$ und $\sum_{k=0}^{\infty} b_k$ zwei 
  (konvergente) Reihen ($a_k,b_k$ aus $\RR$ oder $\CC$). 
  Dann gelten die folgenden beiden Rechenregeln
  \[
  \dis \sum_{k=0}^\infty (a_k + b_k) = \sum_{k=0}^\infty a_k 
  + \sum_{k=0}^\infty b_k, \qquad\qquad
  \dis \sum_{k=0}^\infty \alpha a_k = 
  \alpha 
  \sum_{k=0}^\infty a_k \quad( \alpha\in\KK), 
  \]
  die sich unmittelbar aus den entsprechenden Regeln für Folgen reeller 
  oder komplexer Zahlen ergeben (vgl. Frage \ref{02_freg}). 

  Bezüglich der Produktbildung zweier Reihen lassen sich nicht ohne Weiteres 
  ähnliche Regeln formulieren, da eine Produktreihe nach verschiedenen 
  Prinzipien gebildet werden kann, die auf eine unterschiedliche 
  Anordnung der hinauslaufen. Daher muss zunächst geklärt werden, 
  unter welchen Bedingungen das Konvergenzverhalten einer  
  Produktreihe unabhängig von der Anordnung ihrer Summanden ist 
  (s. Abschnitt \ref{umordnung}).\AntEnd
\end{antwort}

%% --- 7 --- %%
\begin{frage}\label{02_cauchy}\index{Cauchy-Kriterium!fuer Reihen@für Reihen}
  Wie lautet das \bold{Cauchy-Kriterium} für die Konvergenz einer Reihe? 
\end{frage}

\begin{antwort}
  Es handelt sich dabei um Cauchy-Kriterium für Folgen aus 
  Frage~\ref{02_cf}, bezogen auf die Folge der Partialsummen 
  einer Reihe. Es lautet demnach: 

  \medskip\noindent
  \slanted{Eine Reihe $\sum_{k=0}^\infty a_k$ konvergiert genau dann, 
    wenn sie eine Cauchy-Reihe ist, {\dasheisst} wenn 
    zu jedem $\eps>0$ ein $N\in\NN$ existiert, sodass für alle $m,n>N$, $n>m$  
    die Ungleichung 
    \[
    \left| \sum_{k=0}^n a_k - \sum_{k=0}^m b_k \right| = 
    \left|\sum_{k=m+1}^n a_k \right| < \eps 
    \]
    erfüllt ist.}\AntEnd
\end{antwort}

\begin{frage}
  Können Sie mit dem Cauchy-Kriterium zeigen, dass die 
  in Frage \ref{02_edef} eingeführte Exponentialreihe 
  \[
  \sum_{k=0}^\infty \frac{1}{k!}
  \]
  konvergiert?
\end{frage}

\begin{antwort}
  Wegen $k!\ge 2^{k-1}$ für alle $k\in \NN$ gilt für natürliche Zahlen 
  $n,m$ mit $n>m$
  \[
  \sum_{k=m}^n \frac{1}{k!}  \le 
  \sum_{k=m}^n \frac{1}{2^{k-1}} = 
  \frac{1}{2^{m-1}} \sum_{k=0}^{n-m} \frac{1}{2^k} =
  \frac{1}{2^{m-1}} \cdot \frac{ 1-1/2^{n-m+1} }{ 1-1/2 } 
  \le \frac{1}{2^{m-2}}.
  \]
  Wegen $2^{m-2} > m-2$ für alle $m>4$ wird der Wert jeder derartigen 
  Teilsumme kleiner als $\eps$, sofern $n>1/\eps+2$ ist. 
\AntEnd
\end{antwort}

\begin{frage}
  Wie ist die \bold{harmonische Reihe} definiert?
\end{frage}

\begin{antwort}
  \index{harmonische Reihe}
  Die harmonische Reihe summiert die Kehrwerte der natürlichen Zahlen.
  \[
  \sum_{n=1}^\infty \frac1n. \EndTag
  \]
\end{antwort}

\begin{frage}
  Divergiert oder konvergiert die harmonische Reihe? 
  Können Sie ihre Behauptung beweisen?
\end{frage}

\begin{antwort}
  Die harmonische Reihe divergiert. 
  Das lässt sich mit dem Cauchy-Kriterium nachweisen. 
  Für ein beliebiges $N\in\NN$ wähle man $n$ so groß, 
  dass $2^n \ge N$ ist. Dann gilt 
  \[
  \sum_{k=2^n+1}^{2^{n+1}} = 
  \underbrace{\frac{1}{2^n+1}+\cdots + \frac{1}{2^{n+1}}}_
  {\text{$2^n$ Summanden}} > 2^n\cdot \frac{1}{2^{n+1}} 
  =\frac{1}{2}
  \]
  Zu jedem $N$ gibt es also Teilstücke $a_{N+p}+\cdots+a_{N+q}$ 
  mit $q>p$, deren Summe größer als $\frac{1}{2}$ ist. Daher kann die 
  harmonische Reihe nicht konvergieren. 
  \AntEnd
\end{antwort}

%% --- 9 --- %%
\begin{frage}\label{02_abb}
  Sei $\KK=\RR$ oder $\KK=\CC$. Warum ist die 
  folgende Abbildung bijektiv:
  \[
  \mathrm{Abb}\,(\NN_0,\KK) \to \mathrm{Abb}\,(\NN_0,\KK),\quad 
  (a_k)_{k\in\NN}  \mapsto (A_n)_{n\in\NN} = 
  \left( \sum_{k=0}^n a_k \right)_{n\in\NN}.
  \]
\end{frage}

\begin{antwort}
  Die Abbildung ist surjektiv. Ist nämlich $(A_n)$ 
  gegeben, dann ist die Folge $(a_k)$ mit $a_0 := A_0$ und 
  $a_n:=A_n-A_{n-1}$ ein Urbild von $(A_n)$.  

  Sie ist ferner injektiv, denn aus 
  $(a_k)\not= (b_k)$ folgt $a_{k_0} \not= b_{k_0}$ 
  für einen kleinsten Index $k_0$ ({\dasheisst} $a_k = b_k$ für alle $k<k_0$), 
  und daraus 
  folgt für die entsprechende Partialsumme 
  $A_{k_0} \not= B_{k_0}$, also $(A_n) \not= (B_n)$. 
  Damit ist insgesamt die Bijektivität der Abbildung gezeigt.
  \AntEnd
\end{antwort}

\begin{frage}
  Was versteht man unter einer 
  \bold{geometrischen Reihe} in $\RR$ oder $\CC$? 
\end{frage}

\begin{antwort}
  \label{geom_reihe_def}
  Für eine reelle oder komplexe Zahl $q$ heißt 
  \[ 
  \sum_{k=0}^\infty q^k \asttag
  \]
  geometrische Reihe von $q$.
  \AntEnd
\end{antwort}

%% --- 10 --- %%
\begin{frage}
  \label{02_rgeom}\index{geometrische Reihe}
  Für welche reellen oder komplexen Zahlen $q$ konvergiert die 
  geometrische Reihe~{\astref} aus Frage~\ref{geom_reihe_def}? 
  
  Was ist gegebenenfalls ihr Grenzwert?
  Warum ist die geometrische Reihe für alle $\sans{q\in\KK}$ 
  mit $\sans{|q|\ge 1}$ divergent? 
\end{frage}

\begin{antwort}
  Die Werte der Partialsummen $s_n = \sum_{k=0}^n q^k$ lassen sich mit dem 
  folgenden einfachen Trick leicht berechnen. Es ist 
  \[
  s_n - qs_n  = 
  (1+q+q^2+\cdots+q^n)-(q+q^2+\cdots + q^n + q^{n+1}) = 1-q^{n+1}.
  \] 
  Hieraus folgt für $q\not=1$
  \[
  s_n=\frac{ 1-q^{n+1} }{ 1-q }.
  \]
  Die Folge $(q^{n+1})_{n\in\NN}$ konvergiert 
  für $|q| < 1$ gegen $0$ und ist divergent für $|q|\ge 1$ und $q\not=1$ 
  (vgl. Frage \ref{q:folge-geom-absch}). 
  Im letzteren Fall ist somit auch die geometrische Reihe divergent.  
  Für $|q| < 1$ konvergiert sie, und es gilt
  \[
  \boxed{\sum_{k=0}^\infty q^k = \frac{1}{1-q}.}
  \]
  Allein der Fall $q=1$ ist damit noch unentschieden. In diesem Fall ist aber 
  $s_n =n+1$ für alle $n\in\NN$, und somit divergiert die Reihe. \AntEnd
\end{antwort}

%% --- 11 --- %%
\begin{frage}\label{02_nullf}
  Warum ist für eine konvergente Reihe $\sans{\sum_n a_n}$ die 
  (Summanden-)Folge $\sans{(a_n)}$ stets eine Nullfolge? 
\end{frage}

\begin{antwort}
  \label{q:190}
  Ist die Folge $(a_n)$ der Summanden keine Nullfolge, 
  dann gibt es ein $\eps_0>0$, sodass $|a_n|>\eps_0$ für unendlich viele 
  $n\in \NN$ gilt. Also gibt es zu jedem $N\in\NN$ ein $n>N+1$ mit der 
  Eigenschaft
  \[
  \left| \sum_{k=0}^n a_k -\sum_{k=0}^{n-1} a_k \right| = |a_n| > \eps_0. 
  \]
  Die Reihe erfüllt damit nicht das Cauchy-Kriterium, ist also nicht 
  konvergent.
  \AntEnd    
\end{antwort}

%% --- 12 --- %%
\begin{frage}
  Gilt für den Zusammenhang aus der Frage \ref{q:190} auch die Umkehrung?
\end{frage}

\begin{antwort}
  Die Umkehrung gilt nicht, wie das Beispiel der harmonischen 
  Reihe zeigt. Deren Summanden bilden zwar eine Nullfolge, die Reihe 
  selbst divergiert aber trotzdem.
  \AntEnd
\end{antwort}

%% --- 13 --- %%
\begin{frage}\label{02_trivial}\index{Reihe!Trivialkriterium für Konvergenz}
  Welches "`Trivialkriterium"' für die Divergenz einer Reihe erhält man aus 
  den Ergebnissen der Frage \ref{02_nullf}?
\end{frage}

\begin{antwort}
  Eine Reihe divergiert auf jeden Fall dann, wenn ihre Summandenfolge 
  keine Nullfolge ist.
  \AntEnd
\end{antwort}

%% --- 14 --- %%
\begin{frage}\label{02_reste}\index{Reihenrest}
  Was versteht man bei einer konvergenten Reihe $\sans{\sum_k a_k}$ 
  unter der Folge der \bold{Reihenreste} und warum ist diese Folge stets 
  eine Nullfolge?
\end{frage}

\begin{antwort}
  Unter der Folge der Reihenreste versteht man die Folge $(R_n)$ mit 
  \[
  R_n := \sum_{k=n+1}^{\infty} a_k.
  \]
  Wegen der Konvergenz der Reihe gilt für alle $\eps>0$ und genügend große 
  $n$ die Abschätzung $
  |R_n| =  \left |\sum_{k=0}^\infty a_k-\sum_{k=0}^{n} a_k \right| < \eps$, 
  also ist $(R_n)$ eine Nullfolge.
  \AntEnd
\end{antwort}

%% --- 15 --- %%
\begin{frage}\label{02_tele}\index{Teleskopreihe}
  Was versteht man unter einer \bold{teleskopischen Reihe}?
  Was lässt sich über die Partialsummen einer teleskopischen Reihe 
  aussagen?
\end{frage}

\begin{antwort}
  \label{q:193}
  Eine \slanted{teleskopische Reihe} oder \slanted{Teleskopreihe} 
  ist eine Reihe der Bauart
  \[
  \sum_{k=0}^\infty a_k = \sum_{k=0}^\infty (x_{k}-x_{k+1}). 
  \]
  Die spezielle Eigenschaft einer Teleskopreihe 
  besteht darin, dass die Terme zweier 
  aufeinander folgender Summanden sich durch geeignete Zusammenfassung 
  gegenseitig "`neutralisieren"'. 
  
  Für die Partialsummen einer teleskopischen Reihe gilt:
  \begin{align*}
    \sum_{k=0}^n a_k &= (x_0 - x_1)+(x_1-x_2)+\ldots + (x_n-x_{n+1}) \\
    &=
    x_0 - (x_1-x_1) -\cdots - (x_n - x_n ) -x_{n+1}= x_0-x_{n+1}.  
  \end{align*} 
  Die Partialsumme $s_n$ hängt also nur von $x_0$ und $x_{n+1}$ ab.
\end{antwort}

%% --- 16 --- %%
\begin{frage}
  \label{02_tkonv}\index{Teleskopreihe}
  Was wissen Sie über die \slanted{Konvergenz teleskopischer Reihen}?
\end{frage}

\begin{antwort}
  Die Partialsummen $s_n$ der Teleskopreihen besitzen nach 
  der Antwort zu Frage~\ref{q:193} die Gestalt $s_n = x_0 - x_{n+1}$. 
  Daraus folgt (mit den Bezeichnungen aus der Frage \ref{q:193}): 

  \medskip\noindent
  \slanted{Eine Teleskopreihe konvergiert genau dann, wenn 
    die Folge $(x_n)$ konvergiert. In diesem Fall gilt} 
  \begin{equation}
    \sum_{k=0}^\infty (x_{k}-x_{k+1}) 
    = x_0 - \limm x_n. \EndTag
  \end{equation}
\end{antwort}

%% --- 17 --- %%
\begin{frage}\label{02_telea}\index{Teleskopreihe}
  Können Sie 
  $\sans{\dis \sum_{k=1}^\infty \frac{1}{k(k+1)} =1}$ beweisen?
\end{frage}

\begin{antwort}
  Man kann die Summanden in den 
  Partialsummen so aufspalten, dass man eine Teleskopreihe erhält 
  \[
  \sum_{k=1}^n \frac{1}{k(k+1)} = 
  \sum_{k=1}^n \frac{k+1-k}{k(k+1)}=
  \sum_{k=1}^n \left( \frac{1}{k}-\frac{1}{k+1} \right)=1-\frac{1}{n+1}.
  \]
  Nach der Antwort zu Frage~\ref{02_tkonv} ist der Grenzwert 
  dieser Reihe gleich $1-\lim_{k\to\infty} \frac{1}{k+1}=1$. \AntEnd
\end{antwort}


\section{Konvergenzkriterien für reelle Reihen}

\begin{frage}
  \label{02_besc}
  \index{Beschränktheitskriterium!für Reihen}
  Wie lautet das \bold{Beschränkt\-heits\-kri\-te\-rium} 
  für die Konvergenz einer Reihe?
\end{frage}

\begin{antwort}
  Die Reihe $\sans{\sum_{k=0}^\infty a_k}$ mit $a_k \in \RR_+$  
  ist genau dann konvergent, wenn 
  die Folge $(s_n):=\left(  \sum_{k=0}^n a_k \right )$ 
  der Partialsummen nach oben beschränkt ist.
\end{antwort}

\begin{frage}
   Können Sie das 
   \bold{Beschränkt\-heits\-kri\-te\-rium} beweisen?
\end{frage}

\begin{antwort}
  \Ant Das Beschränktheitskriterium ist eine unmittelbare Folge 
  des Monotoniekriteriums für reelle Folgen 
  (vgl. Frage \ref{02_mokr}), 
  da die Partialsummen $s_n$ wegen $a_k\ge 0$ monoton wachsen.
  \AntEnd  
\end{antwort}


%% --- 19 --- %%
\begin{frage}\label{02_rexp}\index{Beschränktheitskriterium}
  Können Sie mit dem Beschränktheitskriterium beweisen, dass die Reihe 
  $\sans{\sum_{k=0}^\infty \frac{1}{k!}}$ konvergiert?
\end{frage}

\begin{antwort}
  Die Summanden $1/k!$ sind alle positiv, 
  also ist nur noch die Beschränktheit der Folge der Partialsummen zu zeigen. 
  
  Diese erhält man zusammen mit der für alle $k\in\NN$ gültigen 
  Abschätzung $k!\ge 2^{k-1}$ aus der Konvergenz der geometrischen Reihe 
  $\sum_{k=0}^\infty 2^{-k}$. Damit gilt: 
  \begin{equation}
    \sum_{k=0}^\infty \frac{1}{k!} \le 
    1 + \sum_{k=1}^\infty \frac{1}{2^{k-1}} = 
    1 + \sum_{k=0}^\infty 2^{-k} = 3.
    \EndTag
  \end{equation}
\end{antwort}


%% --- 20 --- %%
\begin{frage}
  \label{02_verd}
  \index{Verdichtungskriterium}
  Wie lautet das \bold{Verdichtungskriterium}? Können Sie dieses 
  Kriterium beweisen?
\end{frage}

\begin{antwort}
  Das Verdichtungskriterium lautet: 
  Ist $(a_k)$ eine monoton fallende Folge positiver reeller Zahlen, 
  dann ist die Reihe $\sum_k a_k$ genau dann 
  konvergent, wenn die "`verdichtete"' Reihe 
  $\sum_k 2^k a_{2^k}$ konvergiert.

  Zum Beweis: Wenn man die Reihe $\sum_k a_k$ in 
  endliche Teilsummen $T_0, T_1, T_2,\ldots$ 
  mit jeweils $2^0, 2^1, 2^2, \ldots$ 
  Summanden aufspaltet, erhält man die Darstellung
  \[
  \sum_{k=0}^\infty a_k = 
  a_0 + 
  \sum_{k=1}^{2-1} a_k +  
  \sum_{k=2}^{2^2-1} a_k + \cdots + 
  \sum_{k=2^n}^{2^{n+1}-1} a_k + \cdots. 
  \]     
  Die Werte der Teilsummen lassen sich leicht abschätzen. Diese 
  besitzen jeweils $2^n$ Summanden, die 
  (weil $(a_k)$ monoton fällt) alle kleiner oder gleich 
  dem ersten und größer oder gleich 
  dem letzten sind. Also gilt 
  \[
  \sum_{k=0}^\infty a_k \le \sum_{n=0}^\infty 2^n a_{2^n},
  \qquad\qquad 
  \sum_{k=0}^\infty a_k 
  \ge \sum_{n=0}^\infty 2^n a_{2^{n+1}} = \frac{1}{2}\label{02_besc}
  \sum_{n=1}^\infty a_{2^n}.
  \]
  Die Reihe $\sum_k a_k$ ist also genau dann beschränkt, 
  wenn die Reihe $\sum_k 2^k a_{2^k}$ beschränkt ist. 
  Daraus folgt das Verdichtungskriterium.

  Das Verdichtungskriterium ist im Übrigen nicht an die Zahl $2$ gebunden, 
  sondern gilt sinngemäß für alle anderen natürlichen Zahlen $>1$.
  \AntEnd
\end{antwort}

%% --- 21 --- %%
\begin{frage}\label{03_harm}\index{harmonische Reihe}
  Können Sie zeigen, 
  dass die \bold{allgemeine harmonische Reihe}
  \[
  \sum_{n=1}^\infty \frac{1}{n^s}\qquad  \text{mit $s \in \RR$ und 
    $\dis n^s := \exp( s\log n )$}  
  \]
  für $s>1$ konvergiert und für $s\le 1$ divergiert?
\end{frage}

\begin{antwort}
  Die Divergenz der Reihe für $s=1$ wurde bereits 
  in Frage \ref{02_cauchy} gezeigt. Daraus folgt sofort die 
  Divergenz für $s\le 1$. 

  Für $s>1$ kann die Konvergenz mit dem \slanted{Verdichtungskriterium} aus 
  Frage \ref{02_verd} gezeigt werden. Mit $a_k := \frac{1}{k^s}$ gilt
  für die "`verdichtete Reihe"' 
  \[
  \sum_{k=0}^\infty 2^k a_{2^k} = \sum_{k=0}^\infty \frac{2^k}{{(2^k)}^s}
  =\sum_{k=0}^\infty 2^{(1-s)k}. \asttag
  \]
  Für $s>1$ ist $2^{1-s}<1$. Die Konvergenz 
  der rechten Reihe in {\astref} folgt daraus durch Vergleich mit 
  der geometrischen Reihe. \AntEnd
\end{antwort}


%% --- 22 --- %%
\begin{frage}\label{02_verda}\index{Verdichtungskriterium}
  Können Sie mit dem Verdichtungskriterium zeigen, dass 
  für $N\in \NN$, $N>1$ die Reihen der Art
  \begin{equation}
    \sum_{k=2}^\infty \frac{1}{k\log_N(k)}, \quad 
    \sum_{k=3}^\infty \frac{1}{k\log_N(\log_N(k))}, \quad 
    \sum_{k=4}^\infty \frac{1}{k\log_N(\log_N(\log_N(k)))}. \ldots
    \tag{$\ast$}
  \end{equation} 
  alle divergieren?
\end{frage}

\begin{antwort}
  Mit dem Verdichtungskriterium lässt sich direkt zeigen, 
  dass die erste dieser Reihen divergiert, da die Reihe 
  \[
  \sum_{k=2}^\infty \frac{ N^k }{ N^k \log_N( N^k ) } = 
  \sum_{k=2}^\infty \frac{1}{k}.
  \]
  divergent ist. (Hier wurde die allgemeine Version des Kriteriums 
  mit einer beliebigen Zahl $N\ge2$ benutzt.) 
  Weiter lässt sich mit dem Verdichtungskriterium 
  die Divergenz einer beliebigen Reihe der Serie ($\ast$) 
  auf die Divergenz der jeweils vorhergehenden zurückführen, 
  wie etwa im nächsten Schritt 
  \[
  \sum_{k=3}^\infty \frac{1}{ k \log_N (\log_N (k) ) } \to\infty
  \LLa 
  \sum_{k=3}^\infty \frac{N^k}{ N^k \log_N (\log_N (N^k) ) } =
  \sum_{k=3}^\infty \frac{1}{ \log_N (k) }\to\infty.
  \]
  Die Divergenz der hinteren Reihe ergibt sich mit dem Minorantenkriterium 
  nun unmittelbar aus dem bereits bewiesenen. Induktiv folgt daraus  
  die Divergenz sämtlicher Reihen der Form $(\ast)$. 
  \AntEnd  
\end{antwort}

%% --- 23 --- %%
\begin{frage}\label{02_gal}
  \index{g-adischer Bruch@$g$-adischer Bruch}
  \index{Ziffer}
  Ist $\sans{g}$ eine natürliche Zahl $\sans{\ge 2}$. 
  Was versteht man unter dem 
  $\sans{g}$-adischen Bruch mit den Ziffern 
  $\sans{z_1, z_2, z_3, \ldots, z_n}$? 
\end{frage}

\begin{antwort}
  Eine Ziffer ist in diesem Fall eine Zahl aus $\{ 1, 2, \ldots, g-1 \}$. 
  Der den Ziffern $z_1, z_2, z_3 , \ldots, z_n $ 
  zugeordnete $g$-adische Bruch ist die Zahl  
  \[
  a := \frac{z_1}{g}+\frac{z_2}{g^2}+ \cdots + \frac{z_n}{g^n} + 
  \ 
  \]
  Für einen solchen $g$-adischen Bruch benutzt man 
  auch die abkürzende Schreibweise $0,z_1z_2z_3\ldots$. 
  \AntEnd
\end{antwort} 

%% --- 24 --- %%
\begin{frage}
  Wieso konvergiert jeder $g$-adische Bruch?
\end{frage}

\begin{antwort}
  Die Folge $(a_n)$ ist monoton wachsend und nach oben beschränkt wegen
  \[
  \sum_{k=1}^n \frac{z_k}{g^k}
  \le \frac{g-1}{g} \left(1+\frac{1}{g}+\frac{1}{g^2}+\cdots +
    \frac{1}{g^{n-1}} \right) = \frac{g-1}{g}\cdot 
  \frac{1-1/g^n}{1-1/g}=1-\frac{1}{g^{n}} \le 1. 
  \]
  Ein $g$-adischer Bruch konvergiert also in jedem Fall und repräsentiert 
  damit stets eine reelle Zahl $x$.\AntEnd 
\end{antwort}

%% --- 25 --- %%
\begin{frage}\label{02_gala}
  Warum lässt sich jede reelle Zahl mit 
  $\sans{0<x\le 1}$ stets in einen $\sans{g}$-adischen 
  Bruch entwickeln. 
\end{frage}

\begin{antwort}
  Um die Existenz einer $g$-adischen Entwicklung zu zeigen, kann man ein 
  Intervallschachtelungsverfahren anwenden. Dazu unterteile 
  man in einem ersten Schritt das Intervall 
  $\lopen{0,1}$ in die $g$ gleichlangen 
  halboffenen Intervalle 
  \[
  I^k_1 := \blopen{ \tfrac{k}{g}, \tfrac{k+1}{g}  }, 
  \quad k \in \{ 0,\ldots, g-1 \}. 
  \]
  Die Zahl $x$ liegt dann in genau einem dieser Intervalle 
  $I^{z_1}_1$ mit $z_1\in\{1,\ldots,g-1\}$. 
  Man unterteile nun dieses Intervall 
  in einem zweiten Schritt 
  wiederum in $g$ gleichlange halboffene Intervalle  
  \[
  I^k_2:=\blopen{\tfrac{z_1}{g}+\tfrac{k}{g^2}, 
    \tfrac{z_1}{g}+\tfrac{k+1}{g^2} },
  \quad k \in \{ 0,\ldots, g-1 \}. 
  \]
  Die Zahl $x$ liegt jetzt wieder in genau einem Intervall 
  $I^{z_2}_2$ mit $z_2\in\{1,\ldots,g-1\}$.  
  
  Auf diese Weise fortfahrend erhält man eine Intervallschachtelung 
  $\lopen{0,1}\supset I^{z_1}_1\supset I^{z_2}_2 \supset \ldots$, 
  die genau die Zahl $x$ erfasst. 
  Die Folge $(a_n)$ der unteren Intervallgrenzen mit 
  \[
  a_n=\frac{z_1}{g}+\frac{z_2}{g^2}+\cdots+\frac{z_n}{g^n}\asttag
  \]
  konvergiert damit gegen $x$ und liefert eine $g$-adische Bruchdarstellung 
  für $x$. Damit ist die Existenz gezeigt. \AntEnd
\end{antwort}

%% --- 26 --- %%
\begin{frage}
  Inwiefern ist die Darstellung durch $g$-adische Brüche eindeutig?
\end{frage}

\begin{antwort}
  Die $g$-adische Entwicklung {\astref} kann 
  nicht abbrechen, da stets $a_n<x$ gilt. 
  Ist die Zahl $x$ für ein $k\in\NN$  
  identisch mit der oberen Grenze des Intervalls 
  $I^{z_k}_k$, 
  so führt dies auf die $g$-adische Darstellung 
  $x=0,z_1z_2 \ldots$ mit $z_\nu=g-1$ für alle $\nu>k$. 
  Andererseits repräsentiert in 
  diesem Fall auch die \textit{abbrechende} $g$-adische 
  Entwicklung $0,z_1\ldots z_{k-1}(z_k+1)$ 
  die Zahl $x$. Es gilt dann also  
  \[
  0,z_1\ldots z_{n-1}z_k (g-1)(g-1)(g-1)\ldots = 0,z_1\ldots 
  z_{k-1}(z_k+1).
  \]
  Beispielsweise sind $0,4999\ldots$ und $0,5$ beides 
  Dezimalbruchdarstellungen der Zahl $\frac{1}{2}$. 
  Unter der Voraussetzung, dass $x$ mit der oberen Grenze 
  eines der Intervalle $I_n$ zusammenfällt, 
  ist die Darstellung durch einen $g$-adischen Bruch also 
  nicht eindeutig, in diesem Fall existieren zwei mögliche 
  Darstellungen.   

  Benutzt man jedoch die Konvention, dass die Folge der Ziffern 
  in der $g$-adischen Entwicklung einer reellen 
  Zahl stets durch die Folge der \slanted{unteren} 
  Intervallgrenzen gegeben ist, dann ist die Darstellung eindeutig. 

  Denn angenommen $0,z_1z_2\ldots $ und $0,\zeta_1\zeta_2\ldots $ 
  wären zwei verschiedene $g$-adische Darstellungen von $x$. Dann gibt es eine 
  kleinste Zahl $m\in\NN$ mit $z_m \not= \zeta_m$. 
  {\OBdA} kann man $z_m < \zeta_m$ annehmen. Daraus folgt dann 
  (mit einer ähnlichen Berechnung wie in ($\ast$)) der 
  Widerspruch
  \begin{eqnarray*}
    x &=& 0,z_1z_2\ldots z_mz_{m+1} \ldots \le 
    0,z_1z_2\ldots z_m(g-1)(g-1)(g-1) \ldots \\
    &=&
    0,z_1z_2\ldots z_{m-1}+\frac{z_m}{g^m}+
    \sum_{k=m+1}^\infty \frac{g-1}{g^k}
    = 
    0,z_1z_2\ldots z_{m-1}+\frac{z_m+1}{g^m}
    \\
    &\le& 
    0,\zeta_1\zeta_2\ldots \zeta_{m-1}\zeta_m 
    <0,\zeta_1\zeta_2\ldots \zeta_{m-1}\zeta_m\zeta_{m+1}\ldots = x.
  \end{eqnarray*}
  Die letzte Ungleichung gilt, weil aufgrund der Voraussetzung 
  nicht alle $z_k$ mit $k>m$ gleich null sein können. 
  \AntEnd  
\end{antwort}

%% --- 27 --- %%
\begin{frage}\index{alternierende Reihe}
  \index{Reihe!alternierende}
  Was ist eine \bold{alternierende Reihe}?
\end{frage}

\begin{antwort}
  Eine alternierende Reihe ist eine Reihe der Gestalt
  \[
  \sum_{n=0}^\infty c_n = \sum_{n=0}^\infty (-1)^n a_n=a_0-a_1+a_2 - a_3+\cdots, 
  \]
  wobei für alle $n\in\NN$ entweder $a_n\ge0$ oder $a_n\le0$ gilt.\AntEnd
\end{antwort}

%% --- 28 --- %%
\begin{frage}\label{02_leib}\index{Leibnizkriterium}
  \index{Leibniz@\textsc{Leibniz}, Gottfried Wilhelm (1646-1716)}
  Was besagt das \bold{Leibniz-Kriterium} für die Konvergenz einer 
  \bold{alternierenden Reihe}?
\end{frage}

\begin{antwort}
  Das Leibniz-Kriterium besagt: 

  \medskip\noindent
  \satz{Ist $(a_n)$ eine monoton 
    fallende Nullfolge (und damit $a_n\ge 0$ für alle $n\in\NN$), 
    dann konvergiert die Reihe $\sum_n (-1)^n a_n$.  }
\end{antwort}

\begin{frage}
  Können Sie das Leibniz-Kriterium (Frage \ref{02_leib}) beweisen?
\end{frage}

\begin{antwort}
  Für eine monoton fallende Nullfolge $(a_n)$ liegt 
  für jedes $n>2$ die Partialsumme $s_n= \sum_{k=0}^n (-1)^k a_k$
  zwischen $s_{n-2}$ und $s_{n-1}$. Um das einzusehen, betrachte man
  \begin{eqnarray*}
    s_n - s_{n-1} &=& (-1)^n a_n,  \\
    s_n - s_{n-2} &=& (-1)^{n-1} a_{n-1} +(-1)^n a_n 
    = (-1)^{n-1} ( a_{n-1}-a_n ).
  \end{eqnarray*} 
  Ist $n$ gerade, dann ist die erste Differenz positiv und die zweite 
  wegen $a_{n-1} > a_n$ negativ, während sich für ungerade $n$ genau das 
  Gegenteil ergibt. Für jedes $n>k$ ist $s_n$ also in dem Intervall mit den 
  Endpunkten $s_{k-2}$ und $s_{k-1}$ enthalten, und dessen Länge 
  $|s_{n-1}-s_{n-2}|=a_{n-1}$ wird 
  für genügend große $n$ beliebig klein. Daraus folgt, 
  dass $(s_n)$ eine Cauchy-Folge ist und somit konvergiert.
  \AntEnd
\end{antwort}

%% --- 29 --- %%
\begin{frage}\label{02_leiba}\index{Leibnizkriterium}
  Welche Fehlerabschätzung erhält man beim Leibniz-Kriterium?
\end{frage}

\begin{antwort} Der Grenzwert $s$ einer alternierenden Reihe $\sum_k (-1)^ka_k$ 
  mit $a_k\ge 0$ liegt für jedes $n\in\NN$ zwischen $s_n$ und $s_{n+1}$. 
  Wegen  $|s_{n+1}-s_n| = a_{n+1}$ folgt damit für den Fehler 
  \begin{equation}
    R_n := \left| s- s_n \right| \le a_{n+1}. \EndTag
  \end{equation}
  
\end{antwort}




\section{Reihen mit beliebigen Gliedern, absolute Konvergenz}

%% --- 30 --- %%
\begin{frage}\label{02_abs}\index{absolute Konvergenz}
  Wann heißt eine Reihe \bold{absolut konvergent}?
\end{frage}

\begin{antwort}
  Eine Reihe $\sum_n a_n$ heißt \bold{absolut konvergent}, 
  wenn die Reihe $\sum_n |a_n|$ konvergiert. \AntEnd
\end{antwort}

%% --- 31 --- %%
\begin{frage}\label{02_gew}
  Warum folgt aus der absoluten Konvergenz einer Reihe die gewöhnliche 
  Konvergenz? 
\end{frage}

\begin{antwort}
  Aus $\sum_{k=n}^m |a_k| < \eps$ für $n,m>N$ folgt mit 
  der Dreiecksungleichung $\left| \sum_{k=n}^m a_k \right| < \eps$ für $n,m>N$.   
  Somit ist die Folge der Partialsummen $\sum_{k=n}^m a_k$ 
  eine Cauchy-Folge und folglich konvergent. 
  \AntEnd
\end{antwort}

%% --- 32 --- %%
\begin{frage}
  Gilt für den Zusammenhang aus Frage \ref{02_gew} auch die Umkehrung?
\end{frage}

\begin{antwort}
  Die Umkehrung hiervon ("`alle konvergenten Reihen sind auch absolut 
  konvergent"') gilt nicht, 
  wie das Beispiel der alternierenden harmonischen 
  Reihe zeigt (vgl. Frage \ref{02_cauchy} und 
  \ref{02_leib}).  \AntEnd
\end{antwort}

%% --- 33 --- %%
\begin{frage}\index{Dreiecksungleichung!fuer absolut@für absolut 
    konvergente Reihen}
  Wie lautet die 
  \bold{Dreiecksungleichung für absolut konvergente Reihen}?
\end{frage}

\begin{antwort}
  Für eine absolut konvergente Reihe $\sum_k a_k$ lautet die 
  Dreiecksungleichung 
  \[
  \left| \sum_k a_k \right| \le \sum_k |a_k|.
  \]
  Diese ergibt sich aus der 
  Gültigkeit der entsprechenden Ungleichung für alle endlichen Summen 
  aufgrund der Monotonie des Grenzwerts.
  \AntEnd
\end{antwort}

%% --- 34 --- %%
\begin{frage}\label{02_maj}
  \index{Majorantenkriterium!fuer Reihen@für Reihen}
  Wie lautet das \bold{Majorantenkriterium} 
  für die Konvergenz einer Reihe?
\end{frage}

\begin{antwort}
  Das Majorantenkriterium besagt: 

  \medskip
  \noindent
  \slanted{Sind $\sum_k a_k$ und $\sum_k b_k$ Reihen mit der Eigenschaft, 
    dass ab einem bestimmten Index $p$ die Abschätzung $|a_k| \le |b_k|$ gilt, 
    so folgt aus der Konvergenz von $\sum_k |b_k|$ diejenige 
    von $\sum_k |a_k|$ und damit erst recht die Konvergenz von $\sum_k a_k$. 
    Ferner gilt 
    $\left| \sum_{k=p}^\infty a_k \right|  \le \sum_{k=p}^\infty |b_k|$.}

  \medskip\noindent
  Aus $\sum_{k=m}^n |b_k| < \eps$ für alle $n>m>N$ 
  und $n,m>p$ folgt nämlich $\sum_{k=m}^n |a_k| < \eps$. Die Reihe 
  $\sum_k a_k$ konvergiert nach dem Cauchy'schen Konvergenzkriterium damit 
  absolut und folglich auch im gewöhnlichen Sinne. Ferner folgt 
  aus der Dreiecksungleichung 
  $\left| \sum_{k=p}^n a_k \right|  \le \sum_{k=p}^n |b_k|$ für alle 
  $n>p$, und diese Ungleichung bleibt auch noch dann gültig, wenn man 
  den Grenzübergang $n\to \infty$ durchführt. Das ist eine Konsequenz
  der Monotonie des Grenzwerts reeller Zahlenfolgen, angewandt auf die 
  Folge der Partialsummen $s_n:=\sum_{k=p}^n$.  
  \AntEnd 
\end{antwort}

%% --- 35 --- %%
\begin{frage}\index{Minorantenkriterium!fuer die Divergenz@für die 
    Divergenz einer Reihe}
  Wie lautet das \bold{Minorantenkriterium} für die Divergenz einer 
  Reihe?
\end{frage}

\begin{antwort}
  Das \slanted{Minorantenkriterium} 
  ist die logische Kontraposition des Majorantenkriteriums. 
  Es lautet demzufolge:  

  \medskip\noindent
  \slanted{Divergiert $\sum |a_n|$ unter den 
    Voraussetzungen aus Frage \ref{02_maj}, so auch 
    $\sum_n |b_n|$.}
  \AntEnd  
\end{antwort} 

%% --- 36 --- %%
\begin{frage}\label{02_bsp1}
  Warum ist die Reihe 
  $\sans{\sum_{k=1}^\infty \frac{k!}{k^k}}$ konvergent und die 
  Reihe $\sans{\sum_{k=1}^\infty \frac{1}{2k}}$ divergent?
\end{frage}

\begin{antwort}
  Wegen 
  \[
  \frac{k!}{k^k} = \frac{1\cdot 2 \cdots k }{k\cdot k\cdots k} \le 
  \frac{2}{k^2}\qquad\text{für $k\ge 2$}
  \]
  besitzt die Reihe $\sum_{k=1}^\infty \frac{k!}{k^k}$ mit 
  $\sum_k \frac{2}{k^2}$ eine konvergente Majorante und ist damit aufgrund 
  des Majorantenkriteriums konvergent.

  Die Divergenz von 
  \[
  \sum_{k=1}^N \frac{1}{2k} 
  = \frac{1}{2} \sum_{k=1}^N \frac{1}{k} \nodpagebreak
  \]
  für $N\to\infty$ ergibt sich aus der Divergenz 
  der harmonischen Reihe. 
  \AntEnd
\end{antwort}

%% --- 37 --- %%
\begin{frage}\label{02_quot}\index{Quotientenkriterium}
  Was besagt das \bold{Quotientenkriterium}? 
\end{frage}

\begin{antwort}
  Das Quotientenkriterium ermöglicht eine Aussage über die 
  Konvergenz einer Reihe $\sum_n a_n$ durch die Untersuchung der 
  Folge der Quotienten $|a_{n+1}|/|a_n|$. Es lässt sich auf mehrere Weisen 
  formulieren. Die erste Version lautet:

  \medskip
  \noindent
  \desc{Q1} 
  \slanted{Sei $q<1$ eine reelle Zahl. Gilt dann für fast alle $n$}
  \[
  \left| \frac{a_{n+1}}{a_n} \right|\le q \qquad\text{bzw.}\qquad
  \left| \frac{a_{n+1}}{a_n}\right| \ge 1,
  \]
  \slanted{so konvergiert bzw. divergiert die Reihe $\sum_n a_n$. 
    Im Falle der Konvergenz konvergiert sie sogar absolut. }

  \medskip\noindent 
  Den Zusammenhang erhält man mit dem Majorantenkriterium. 
  Gilt $|a_{k+1}|/|a_k| \le q$ für alle $k>N$, dann folgt für $n>m>N$
  \[
  \frac{ |a_n| }{ |a_m| } = 
  \frac{ |a_{m+1}| }{ |a_m| } \cdot \frac{ |a_{m+2}| }{ |a_{m+1}| }
  \cdots \frac{ |a_{n}| }{ |a_{n-1}| } \le q^{n-m} \Ra 
  |a_n| \le \frac{|a_m|}{q^m} q^n.
  \]
  Die Reihe $\sum_n |a_n|$ besitzt in der Reihe 
  $\frac{|a_m|}{ q^m} \sum _n q^{-n}$ wegen $q<1$ 
  also eine konvergente Majorante und ist somit selbst konvergent. 
  Gilt auf der anderen Seite $|a_{n+1}|/|a_n| \ge 1$ für fast alle $n$, 
  dann ist die Folge $(|a_n|)$ monoton wachsend und kann keine Nullfolge 
  sein. In diesem Fall divergiert die Reihe.
\end{antwort}


\begin{frage}
  Für die harmonische Reihe gilt $|a_{n+1}|/|a_n|=n/(n+1)<1$ für alle $n$, 
  dennoch divergiert sie. Wie passt das mit dem Quotientenkriterium 
  zusammen?
\end{frage}

\begin{antwort}
  Das Quotientenkriterium fordert die 
  Bestimmung einer \slanted{festen} Zahl $q<1$. 
  Für die harmonische Reihe gilt zwar $|a_{n+1}|/|a_n|<1$ für alle 
  $n$, aber die Quotienten werden  
  mit wachsendem $n$ größer als jede vorgegebene Zahl $q<1$. \AntEnd
\end{antwort}

%% --- 38 --- %%
\begin{frage}\index{Quotientenkriterium}
  Wie lautet die \bold{Limesform} des Quotientenkriteriums?
\end{frage}

\begin{antwort}
  Eine \slanted{Limesform} des Quotientenkriteriums lässt sich auf 
  zwei Arten formulieren. Die erste ist auf eine stärkere Voraussetzung 
  angewiesen und lautet: 

  \medskip
  \noindent
  \desc{Q2} 
  \slanted{Konvergiert die Folge $\left(|a_{n+1}|/|a_n|\right)$, so 
    konvergiert die Reihe $\sum_n a_n$ absolut bzw. divergiert sie, 
    je nachdem, ob gilt:} 
  \[ 
  \limm \left| \frac{a_{n+1}}{a_n} \right| < 1 \qquad\text{bzw.}\qquad
  \limm \left| \frac{a_{n+1}}{a_n}\right| > 1.
  \]
  \slanted{
    Im Fall $\limm |a_{n+1}|/|a_n|=1$ bleibt die Konvergenz unentschieden.
  }

  \medskip
  Dass das Kriterium unter den gegebenen Voraussetzungen dasselbe leistet 
  wie $(Q1)$, ist unschwer zu erkennen. Man beachte allerdings, 
  dass die Limesform in diesem Fall auf einer stärkeren Voraussetzung, 
  nämlich der Existenz des Grenzwerts $\limm |a_{n+1}|/|a_n|$, beruht.

  Um diese Voraussetzung zu umgehen, lässt sich eine 
  Limesform des Quotientenkriterium auch 
  mit den Begriffen $\lim\sup$ und $\lim\inf$ formulieren. 
  Alles, was man dazu braucht, ist die \slanted{Beschränktheit} der 
  Folge $( |a_{n+1}|/|a_n |)$, die aber trivialerweise erfüllt sein muss, 
  wenn $\sum_n a_n$ überhaupt ein Kandidat für Konvergenz sein soll. In 
  diesem Fall lautet das Kriterium

  \medskip
  \noindent
  \desc{Q3} 
  \slanted{Die Reihe $\sum_n a_n$ konvergiert bzw. divergiert, 
    je nachdem ob gilt:}
  \[
  \lim \sup \left| \frac{a_{n+1}}{a_n} \right | < 1 \qquad\text{bzw.}\qquad
  \lim \inf \left| \frac{a_{n+1}}{a_n} \right | > 1. 
  \]
  \slanted{Im Falle der Konvergenz konvergiert die Reihe dann sogar absolut.}

  \medskip\noindent
  Man beachte, dass durch die beiden Alternativen in dieser Formulierung 
  des Kriteriums, selbst wenn die Voraussetzungen 
  erfüllt sind, noch lange nicht alle möglichen Fälle ({\zB} 
  $\lim\sup>1$ und $\lim\inf <1$) abgedeckt sind. Das Quotientenkriterium 
  liefert in diesen Fällen keine Entscheidung (für ein Beispiel dazu siehe 
  Frage \ref{02_wurzb}). 
\end{antwort}

%% --- 39 --- %%
\begin{frage}\label{02_quota}\index{Quotientenkriterium}
  Können Sie mithilfe des Quotientenkriteriums zeigen, dass die 
  Reihe 
  \[
  \sans{
    B_\alpha ( z ) : = \sum_{k=0}^\infty \binom{\alpha}{k} z^k, \quad 
    z \in \CC, \, \alpha\in \RR \mengeminus \NN_0
  }
  \]
  für $\sans{|z|<1}$ (absolut) konvergiert?
  Dabei sind die verallgemeinerten Binomialkoeffizienten 
  $\sans{\binom{\alpha}{k}}$ 
  für $\sans{\alpha\in\RR}$ und $\sans{k\in\NN_0}$ definiert 
  durch $\sans{\binom{\alpha}{0} := 1}$ und 
  $\sans{\binom{\alpha }{k}:= \frac{\alpha(\alpha-1)\cdots (\alpha-k+1)}{k!}}$.
\end{frage}

\begin{antwort}
  Für den Quotienten zweier aufeinanderfolgender Summanden gilt
  \[
  \left| \binom{\alpha}{k+1}z^{k+1} \Bigg{/} \binom{\alpha}{k} z^{k} \right| 
  =\left| \frac{\alpha-k}{k+1} \right| |z| 
  \le \left( 1+ \left| \frac{\alpha+1}{k+1} \right| \right) 
  |z|.
  \]
  Der erste Faktor im letzten Ausdruck konvergiert 
  für $k\to\infty$ gegen $1$, während der 
  hintere konstant kleiner als $1$ ist. 
  Hieraus folgt: es gibt ein $q<1$, sodass  
  $\left| \binom{\alpha}{k+1}z^{k+1} \big{/} \binom{\alpha}{k} z^k \right|\le q$ 
  für fast alle $k$ gilt. Mit dem Quotientenkriterium 
  schließt man auf die Konvergenz der Reihe.
  \AntEnd 
\end{antwort}

%% --- 40 --- %%
\begin{frage}\label{02_wurz}
  \index{Wurzelkriterium}
  Was besagt das Wurzelkriterium? 
\end{frage}

\begin{antwort}
  Ähnlich wie bei Quotientenkriterium gibt es 
  auch mehrere Formulierungen des 
  Wurzelkriteriums. Die erste Version lautet

  \medskip
  \noindent
  \desc{W1}  
  \slanted{
    Sei $q<1$ eine feste positive Zahl. Gilt dann}
  \[ 
  \sqrt[n]{|a_n|}\le q \quad \text{für fast alle $n$} \qquad\text{bzw.}\qquad
  \sqrt[n]{|a_n|} \ge 1 \quad\text{für unendlich viele $n$},
  \]
  \slanted{so konvergiert bzw. divergiert die Reihe $\sum_n a_n$. 
    Im Falle der Konvergenz konvergiert sie sogar absolut.}

  \medskip
  Gilt nämlich die erste Aussage, dann ist $|a_n| \le q^n$ für fast alle
  $n$, und durch Vergleich mit der geometrischen Reihe $\sum_n q^n$ ergibt 
  sich die Konvergenz von $\sum_n a_n$ aus dem Majorantenkriterium. 
  Aus der zweiten Aussage folgt hingegen, 
  dass $|a_n|$ für unendlich oft größer als $1$ ist 
  und $(a_n)$ damit keine Nullfolge sein kann.

  Im Vergleich zum Quotientenkriterium gibt es einen kleinen Unterschied in 
  den Voraussetzungen. Mit dem Quotientenkriterium kann man die Divergenz 
  der Reihe $\sum_n a_n$ nur dann folgern, wenn $|a_{n+1}|/|a_n|\ge1$ 
  für \slanted{fast alle} $n$ gilt, während die äquivalente 
  Divergenzbedingung beim Wurzelkriterium nur verlangt, dass 
  $\sqrt[n]{|a_n|}\ge 1$ für \slanted{unendlich} 
  viele $n$ gilt. Das ist eine schwächere 
  Bedingung. Die beiden im Wurzelkriterium unterschiedenen Fälle decken 
  somit einen größeren Bereich aller möglichen Fälle ab als die entsprechenden 
  Formulierungen im Quotientenkriterium.  \AntEnd
\end{antwort}

%% --- 41 --- %%
\begin{frage}\index{Wurzelkriterium}
  Wie lautet die Limesform des Wurzelkriteriums?
\end{frage}

\begin{antwort}
  \medskip
  Wie das Quotientenkriterium lässt sich das Wurzelkriteriums auch 
  auf zwei Arten in eine Limesformen "`übersetzen"'. 
  In der ersten Version wird dabei die Existenz des 
  Grenzwerts der Folge $(\sqrt[n]{|a_n|}$ vorausgesetzt. 
  Das führt auf die etwas schwächere Formulierung 

  \medskip\noindent
  \desc{W2} 
  \slanted{Falls $\limm \sqrt[n]{|a_n|}$ existiert, so 
    konvergiert die Reihe $\sum_n a_n$ absolut bzw. divergiert sie, 
    je nachdem, ob gilt:}
  \[
  \limm \sqrt[n]{|a_n|} < 1 \qquad\text{bzw.} \qquad \limm\sqrt[n]{|a_n|} >1.
  \]
  Die Gleichwertigkeit beider Formulierungen des Wurzelkriteriums unter den 
  angegebenen Bedingungen ist wiederum leicht einzusehen. 

  Ohne die Existenz des Grenzwerts $(\sqrt[n]{|a_n|})$ vorauszusetzen, kann man 
  auch nur von der Beschränktheit dieser Folge ausgehen und mit den Konzepten 
  $\lim\sup$ und $\lim\inf$ das Wurzelkriterium folgendermaßen formulieren.

  \medskip\noindent
  \desc{W3} 
  \slanted{Ist $(\sqrt[n]{|a_n|})$ beschränkt, so 
    konvergiert die Reihe $\sum_n a_n$ absolut bzw. divergiert sie, 
    je nachdem, ob gilt:} 
  \[
  \lim \sup \sqrt[n]{|a_n|} < 1 \qquad\text{bzw.} 
  \qquad \lim\sup \sqrt[n]{|a_n|} >1.
  \]
  Man beachte auch in dieser Formulierung wieder  
  den Unterschied zur entsprechenden Version des Quotientenkriteriums. 
  Beim Quotientenkriterium kann man nur dann auf die Divergenz der Reihe 
  schließen, wenn der Limes \slanted{Inferior} der Folge $(|a_{n+1}|/|a_n|)$ 
  größer als $1$ ist.   
  \AntEnd
\end{antwort} 

%% --- 42 --- %%
\begin{frage}\label{02_wurza}
  \index{Wurzelkriterium}
  \index{Quotientenkriterium}
  Welches Kriterium ist leistungsfähiger: Quotienten- oder Wurzelkriterium?
\end{frage}

\begin{antwort}
  Das Wurzelkriterium ist leistungsfähiger insofern, als 
  die im Quotientenkriterium auftretenden Annahmen diejenigen 
  des Wurzelkriteriums implizieren. Gilt nämlich $|a_{n+1}|/|a_n| \le q$ 
  für ein $q<1$, dann folgt daraus ähnlich wie im Beweis des 
  Quotientenkriteriums  
  \[
  |a_n| \le \frac{|a_m|}{q^m} q^n, \qquad\text{also}\qquad 
  \sqrt[n]{|a_n|} \le q \sqrt[n]{ \frac{|a_m|}{q^m} }.
  \]
  Wegen $\sqrt[n]{ |a_m| / q^m } \to 1$ für $n\to\infty$ folgt hieraus 
  $\sqrt[n]{|a_n|} \le q'$ für fast alle $n$ und ein geeignetes $q'<1$. 
  Wenn sich die Konvergenz einer Reihe mit dem 
  Quotientenkriterium zeigen lässt, dann also auch mit dem Wurzelkriterium. 
  Die Umkehrung davon gilt im Allgemeinen nicht, wie die Antwort zur 
  nächsten Frage zeigt. 
  \AntEnd
\end{antwort}

%% --- 43 --- %%
\begin{frage}\label{02_wurzb}
  \index{Wurzelkriterium}
  \index{Quotientenkriterium}
  Kennen Sie ein Beispiel einer Reihe, für die das Wurzelkriterium Konvergenz 
  anzeigt, während das Quotientenkriterium keine Entscheidung über die 
  Konvergenz liefert?
\end{frage}

\begin{antwort}
  Die Reihe 
  \[
  \sum_{n=0}^\infty a_n = \sum_{n=0}^\infty \frac{2+(-1)^{n+1}}{2^n} 
  = 1+\frac{3}{2} +\frac{1}{2^2} +\frac{3}{2^3} +\frac{1}{2^4} + \cdots  
  \]
  liefert ein solches Beispiel. Aus 
  $\limm \sqrt[n]{|a_n|}=\frac{1}{2}$ 
  kann man mit dem Wurzelkriterium (W2)
  auf die Konvergenz der Reihe schließen. Wegen 
  \[
  \frac{|a_{n+1}|}{|a_n|} = \left\{ \begin{array}{ll}
      \frac{1}{2} & \text{für ungerade $n$,} \\
      \frac{3}{2} & \text{für gerade $n$}
    \end{array} \right.
  \]
  kann das Quotientenkriterium aber nicht angewendet werden. \AntEnd
\end{antwort}


\section{Umordnung von Reihen, Reihenprodukte}\label{umordnung}

%% --- 44 --- %%
\begin{frage}\label{02_umo}
  \index{Reihe!Umordnung}
  Was versteht man unter einer \bold{Umordung} einer Reihe?
\end{frage}

\begin{antwort}
  Locker formuliert ist die Umordnung einer Reihe eine weitere 
  Reihe, in der dieselben 
  Summanden in einer vertauschten Reihenfolge vorkommen. Um den Begriff 
  präzise zu definieren, redet man von einer bijektiven Abbildung 
  der Indexmenge $\NN$ auf sich selbst. Die Definition lautet dann: 

  \medskip\noindent
  Sei $\NN\to \NN,\, k\mapsto n_k$ eine bijektive Abbildung. 
  Dann ist die Reihe $\sum_k a_{n_k}$ eine Umordnung der Reihe 
  $\sum_n a_n$.

  \medskip\noindent
  Zum Beispiel ist die Reihe 
  \[
  \frac{1}{10}+\frac{1}{9}+\cdots + 1 + \frac{1}{20}+\frac{1}{19} + 
  \cdots +\frac{1}{11}+\frac{1}{30}+\cdots 
  \]
  eine Umordung der harmonischen Reihe $\sum_n 1/n$.
  \AntEnd
\end{antwort}

%% --- 45 --- %%
\begin{frage}\label{02_unb}
  \index{Reihe!bedingt konvergente}
  Was ist eine \bold{bedingt bzw. unbedingt konvergente Reihe}?
\end{frage}

\begin{antwort}
  Eine Reihe heißt \slanted{unbedingt konvergent}, wenn sie konvergent und 
  invariant bezüglich Umordnungen in dem Sinne ist, dass jede ihrer 
  Umordnungen gegen denselben Grenzwert konvergiert.

  Eine Reihe heißt dagegen \slanted{bedingt konvergent}, wenn sie zwar 
  konvergent (mit Grenzwert $S$), 
  aber nicht unbedingt konvergent ist, {\dasheisst} wenn sie eine 
  Umordnung besitzt, die nicht gegen $S$ konvergiert. Die Umordnung 
  kann in diesem Fall entweder divergieren oder gegen einen anderen 
  Grenzwert $S'\not= S$ konvergieren. 
  \AntEnd
\end{antwort}

%% --- 46 --- %%
\begin{frage}\label{02_klum}
  \index{Umordnungssatz!kleiner}
  Was besagt der sogenannte \bold{kleine Umordnungssatz} für Reihen?
\end{frage}

\begin{antwort}
  Der Satz lautet:

  \medskip\noindent 
  \slanted{Eine beliebige Umordnung einer absolut konvergenten Reihe ist 
    wieder absolut konvergent und hat dieselbe 
    Summe wie die Ausgangsreihe.}

  \medskip\noindent
  Der Beweis des Umordnungssatzes 
  ist im Detail etwas anstrengend zu führen, 
  aber er beruht im Wesentlichen auf der einfachen Tatsache, 
  dass zu jedem vorgegebenen $N\in\NN$ für jede Umordung 
  $k\to n_k$ der natürlichen Zahlen ein $k_0$ existiert, sodass die Menge 
  $\{ n_1, n_2, \ldots , n_{k_0} \}$ die ersten $N$ natürlichen 
  Zahlen enthält, dass also gilt
  \[
  \{ 1, 2, \ldots, N \} \subset \{ n_1, n_2, \ldots , n_{k_0} \}.  
  \]
  Trivialerweise ist dann $k_0\ge N$. Für alle $m>k_0$ folgt daraus, dass 
  in der Summe $\left|  \sum_{n=0}^m a_n - \sum_{k=0}^m a_{n_k} \right|$ 
  ausschließlich Summanden mit einem größeren Index als $N$ vorkommen. 
  Mit $M:=\max\{ n_{k_0},\ldots , n_m \}$ folgt daraus zusammen 
  mit der Dreiecksungleichung
  \[
  \left|  \sum_{n=0}^m a_n - \sum_{k=0}^m a_{n_k} \right| \le 
  \sum_{n=N+1}^M |a_n|.
  \] 
  Die rechte Summe wird wegen der absoluten Konvergenz der Reihe $\sum_n a_n$ 
  beliebig klein, und zwar unabhängig von $M$, wenn nur $N$ 
  genügend groß gewählt ist. Daraus folgt, dass sich mit dem vorgeführten 
  Verfahren stets ein $m$ bestimmen 
  lässt, sodass für alle $m'>m$ die Ungleichung  
  \[
  \left|  \sum_{n=0}^{m'} a_n - \sum_{k=0}^{m'} a_{n_k} \right| < \eps
  \] 
  erfüllt ist. Aus dieser folgt der kleine Umordnungssatz.
  \AntEnd
\end{antwort}

%% --- 47 --- %%
\begin{frage}\label{02_diri}
  \index{Umordnungssatz!von Dirichlet-Riemann}
  \index{Riemann@\textsc{Riemann}, Bernhard (1826-1866)}
  Was besagt der \bold{Umordnungssatz von Dirichlet-Riemann} für konvergente, 
  aber nicht absolut konvergente Reihen?
\end{frage}

\begin{antwort}
  Der Satz besagt: 

  \medskip\noindent
  \slanted{Jede konvergente, aber nicht absolut 
    konvergente Reihe besitzt eine Umordnung, die gegen einen beliebig 
    vorgegebenen Wert $S$ konvergiert oder divergiert.}

  \medskip\noindent
  Der Satz folgt daraus, dass für eine konvergente, aber nicht absolut 
  konvergente Reihe die Reihen $\sum_n a^-_n$ und $\sum_n a^+_n$ mit 
  \[
  a^-_n = \left\{\begin{array}{ll} 
      a_n & \text{falls $a_n < 0$} \\ 
      0   & \text{sonst}
    \end{array}\right.
  \quad\text{und}\quad 
  a^+_n = \left\{\begin{array}{ll} 
      a_n & \text{falls $a_n > 0$} \\ 
      0   & \text{sonst}
    \end{array}\right.
  \]
  beide divergieren müssen, denn andernfalls wären 
  $\sum_n |a^-_n| $ und $\sum_n |a^+_n| $ konvergent und somit auch 
  $\sum_n |a_n|=\sum_n |a^-_n|+\sum_n |a^+_n|$. 

  Da diese Reihen divergieren, sind sie unbeschränkt -- im einen Fall nach oben, 
  im anderen nach unten. Um eine Umordnung zu konstruieren, die gegen 
  einen willkürlich vorgegebenen Grenzwert $S$ konvergiert, kann man also  
  so viele Summanden der positiven Reihe aufsummieren, 
  bis $S$ zum ersten Mal überschritten wird, anschließend 
  so viele der negativen Reihe, bis $S$ unterschritten wird. 
  Wegen der Unbeschränktheit der negativen und positiven Reihen kann dieses 
  Verfahren mit den übrigen Summanden beliebig oft wiederholt werden. 
  Wird $S$ derart durch die Addition eines Glieds $a_k$ 
  unter- bzw. überschritten, 
  dann durch einen Wert kleiner als $|a_k|$. 
  Da die Summanden $a_n$ wegen der 
  Konvergenz von $\sum_n a_n$ eine Nullfolge bilden, 
  kann dieser Differenzbetrag 
  beliebig klein gemacht werden. 
  \AntEnd
\end{antwort}

%% --- 48 --- %%
\begin{frage}\label{02_cprod}
  \index{Cauchy-Produkt}
  Was versteht man unter dem \bold{Cauchy-Produkt (Faltung)} zweier Reihen?
\end{frage}


\begin{antwort}[]%
  \Ant Das Cauchy-Produkt zweier Reihen $\sum_{n=0}^\infty a_n$ 
  und $\sum_{k=0}^\infty b_k$ ist die Reihe
  \[
  \sum_{n=0}^\infty ( a_0 b_n + a_1 b_{n-1} + \cdots + a_n b_0 ) = 
  \sum_{n=0}^\infty \sum_{k=0}^n a_k b_{n-k}.
  \]
  Die Summanden des Cauchy-Produkts sind gerade die  
  Produkte $a_n b_k$ mit $k,n\in \NN$, die nach dem Diagonalschema 
  in \Abb\ref{fig:02_cauchy_prod} angeordnet werden.
  \AntEnd

  \begin{center}
    \includegraphics{mp/03_cauchyprodukt}
    \captionof{figure}{Reihenfolge der Summanden im Cauchy-Produkt.}
    \label{fig:02_cauchy_prod}
  \end{center}
\end{antwort}

%% --- 49 --- %%
\begin{frage}\label{02_beab}
  \index{Konvergenz!bedingte}
  \index{Konvergenz!absolute}
  Welcher Zusammenhang besteht zwischen bedingter und absoluter Konvergenz?
\end{frage}

\begin{antwort}
  Nach dem kleinen Umordnungssatz ist jede absolut konvergente Reihe 
  unbedingt konvergent. Aus dem Umordnungssatz von Dirichlet-Riemann 
  folgt die Umkehrung dieses Zusammenhangs. Also gilt:

  \medskip\noindent
  \slanted{Eine Reihe ist genau dann unbedingt konvergent, wenn sie 
    absolut konvergiert}. 
  \AntEnd

  \smallskip
\end{antwort}

\smallskip
%% --- 50 --- %%
\begin{frage}\label{02_pr}
  \index{Produktreihe}
  Was versteht man allgemein unter einer \bold{Produktreihe} zweier Reihen 
  $\sans{\sum_n a_n}$ und $\sans{\sum_k b_k}$?
\end{frage}

\begin{antwort}[]
  \Ant Ordnet man alle möglichen Produkte $a_n b_k$, $n,k\in\NN$ in einer 
  Folge $c_\ell$ an, dann heißt die Reihe $\sum_\ell c_\ell$ eine Produktreihe 
  der Reihen $\sum_n a_n$ und $\sum_k b_k$.  

  Der Zuordnung $c_\ell \mapsto a_n b_k$ 
  liegt eine \slanted{bijektive} Abbildung $\NN\to \NN \times \NN, \;
  \ell \mapsto (n,k)$ zugrunde. Mit jeder derartigen Abbildung 
  lässt sich zu zwei Reihen eine entsprechende Produktreihe angeben. 
  Das Cauchy-Produkt ist beispielsweise eine spezielle 
  Produktreihe, für die die Abbildung $\NN\to \NN \times \NN$ mit dem 
  Diagonalschema konstruiert ist. Die \Abb\ref{fig:02_produktreihe} zeigt eine 
  weitere Möglichkeit, die Menge $\NN \times \NN$ 
  bijektiv auf $\NN$ abzubilden.\AntEnd

  \begin{center}
    \includegraphics{mp/03_produktreihe}
    \captionof{figure}{
      Mögliche Anordnung der Summanden in einer Produktreihe.
    }
    \label{fig:02_produktreihe}
  \end{center}
\end{antwort}

\smallskip
%% --- 51 --- %%
\begin{frage}\label{02_prk}
  \index{Produktreihe}
  Warum ist eine Produkt-Reihe zweier absolut konvergenter Reihen 
  $\sans{\sum_n a_n}$ und $\sans{\sum_k b_k}$ wieder absolut konvergent?
\end{frage}

\begin{antwort}
  Der Zusammenhang ergibt sich aus der Tatsache, dass für 
  eine Produktreihe $\sum_\ell c_\ell$ für alle $n\in\NN$ die Ungleichung 
  \[
  |c_0|+ |c_1| + \cdots + |c_n| \le (|a_0|+|a_1|+\cdots + |a_m| )
  (|a_0|+|a_1|+\cdots + |a_m| )
  \]
  gilt, sofern $m$ nur groß genug gewählt ist. Also gilt erst recht 
  \[
  |c_0|+ |c_1| + \cdots + |c_n| \le 
  \left( \sum_{j=0}^\infty |a_j| \right) 
  \left( \sum_{k=0}^\infty |b_k| \right)  
  \quad\text{für alle $n\in \NN$.}
  \]
  Wegen der absoluten Konvergenz der Reihen $\sum_j a_j$ und 
  $\sum_k b_k$ steht auf der rechten Seite dieser Ungleichung eine 
  reelle Zahl $S$, die eine obere Schranke für die Folge der 
  Partialsummen $\left(\sum_{\ell=1}^n |c_\ell|\right)_{n\in\NN}$ ist. 
  Da diese Folge monoton wächst, 
  folgt die absolute Konvergenz der Produktreihe 
  aus dem Monotoniekriterium für reelle Folgen. \AntEnd 
\end{antwort}

\smallskip
%% --- 52 --- %%
\begin{frage}\label{02_cpk}
  \index{Produktreihe}
  \index{Konvergenz!einer Produktreihe}
  Was ergibt sich aus Frage \ref{02_prk} 
  für das Cauchy-Produkt zweier absolut konvergenter Reihen?
\end{frage}

\begin{antwort}
  Das Cauchy-Produkt ist nur eine Produktreihe spezieller Bauart. 
  Also kann man aus Frage \ref{02_prk}
  folgern, dass das Cauchy-Produkt zweier absolut konvergenter 
  Reihen ebenfalls konvergiert, und zwar sogar absolut.
  \AntEnd
\end{antwort}

%% --- 53 --- %%
\begin{frage}\label{02_cpd}
  \index{Produktreihe}
  \index{Konvergenz!einer Produktreihe}
  \index{Cauchy-Produkt}
  Kennen Sie ein Beispiel für zwei konvergente Reihen, deren Cauchy-Produkt 
  divergiert?
\end{frage}

\begin{antwort}
  Ein Beispiel liefert die Reihe $\sum_{n=0}^\infty a_n$ mit 
  $a_n := (-1)^n/ \sqrt{n+1}$. Diese Folge konvergiert (Leibniz-Kriterium), 
  allerdings nicht absolut (sie besitzt in der 
  harmonischen Reihe eine divergente Minorante). 
  Wir zeigen, dass das Cauchy-Produkt dieser Reihe mit sich selbst 
  divergiert. Dieses besitzt die Form 
  \begin{equation}
    \sum_{n=0}^{\infty} \left( 
      \sum_{k=0}^n \frac{(-1)^k}{\sqrt{k+1}} \, 
      \frac{(-1)^{(n-k)}}{\sqrt{(n-k)+1}} \right) 
    =
    \sum_{n=0}^{\infty} (-1)^n \sum_{k=0}^n \frac{1}{\sqrt{ (k+1)(n-k+1)}}
    \tag{$\ast$}
  \end{equation}
  Für die hinteren endlichen Summen gilt die Abschätzung  
  \[
  \left| \sum_{k=0}^n \frac{1}{\sqrt{(k+1)(n-k+1)}} \right| \ge 
  \left| \sum_{k=0}^n \frac{1}{n+1} \right| = \frac{n+1}{n+1} = 1.
  \]
  Die Summanden des Cauchy-Produkts ($\ast$) bilden also 
  überhaupt keine Nullfolge, daher kann das Cauchy-Produkt auch nicht 
  konvergieren. \AntEnd  
\end{antwort}

%% --- 54 --- %%
\begin{frage}\label{02_expd}
  \index{Exponentialfunktion!Reihendefinition}
  Wie lautet die 
  \bold{Reihendefinition der Exponentialfunktion} (in $\RR$ oder $\CC$)? 
  Warum ist die betreffende Reihe absolut konvergent?
\end{frage}

\begin{antwort}
  Der Wert der Exponentialfunktion $\exp\fd \KK\to \KK$ für 
  $\KK=\CC$ oder $\KK=\RR$ ist für alle 
  $z \in \KK$ gegeben durch den Wert der Reihe
  \[
  \exp( z ) := \sum_{k=0}^\infty \frac{z^k}{k!}.
  \]
  Die Reihe konvergiert trivialerweise für $z=0$. 
  Für $z\in \KK$ mit $z\not=0$ folgt die absolute 
  Konvergenz unmittelbar aus dem Quotientenkriterium, da 
  \[
  \frac{|z|^{k+1}}{(k+1)} \Big{/} \frac{|z|^{k}}{k!}
  = \frac{|z|}{(k+1)!} \le \frac{1}{2}
  \]
  für alle $z\not=0$ und alle $k\ge 2|z|$ gilt.\AntEnd
\end{antwort}

%% --- 55 --- %%
\begin{frage}\label{02_expfunk}\index{Exponentialfunktion!Funktionalgleichung}
  Warum gilt die \bold{Funktionalgleichung} $\exp( z+w)=\exp( z )\exp(w)$?
\end{frage}

\begin{antwort}
  Mit der Binomialentwicklung erhält man zunächst
  \[
  \exp( z+w )= \sum_{n=0}^\infty  \frac{1}{n!} (z+w)^n = 
  \sum_{n=0}^\infty \frac{1}{n!} \sum_{k=0}^n \binom{n}{k} z^k w^{n-k} =
  \sum_{n=0}^\infty \sum_{k=0}^n \frac{z^k}{k!} \frac{w^{n-k}}{(n-k)!}.
  \]
  Hier steht in der rechten Seite nun gerade das Cauchy-Produkt der 
  beiden Reihen für $\exp(z)$ und $\exp(w)$. Wegen der absoluten Konvergenz 
  der beiden Reihen konvergiert das Cauchy-Produkt 
  gemäß Frage \ref{02_cpk} gegen den Wert $\exp(z)\exp(w)$. Das beweist die Funktionalgleichung.\AntEnd 
\end{antwort} 


\section{Elementares über Potenzreihen}
\label{elmenentare_potenzreihen}

Besonders wichtige Hilfsmittel der Analysis sind \slanted{Potenzreihen}. 
Wir formulieren hier nur Fragen zu einigen elementaren Eigenschaften und 
entwickeln das Thema in 
Kapitel \ref{funktionenfolgen} in einem systematischeren 
Rahmen weiter.   

%% --- 56 --- %%
\begin{frage}\label{02_potr}
  \index{Potenzreihe}
  Was versteht man unter einer \bold{Potenzreihe} mit Entwicklungspunkt 
  $a$ und Koeffizienten $\sans{a_k}$ ($\sans{a_k\in\KK}$)?
\end{frage}

\begin{antwort}
  Unter einer Potenzreihe mit Entwicklungspunkt $a$ und Koeffizienten $a_k$ 
  versteht man die Reihe 
  \begin{equation}
    \sum_{k=0}^\infty a_k ( z-a)^k = a_0 + a_1(z-a)+a_2(z-a)^2 + \ldots \EndTag
  \end{equation}
\end{antwort}

%% --- 57 --- %%
\begin{frage}\label{02_kok}
  \index{Konvergenzkreis}
  \index{Konvergenzradius}
  Was gilt f\"ur die Menge der Punkte, für die eine Potenzreihe 
  konvergiert?
\end{frage}
\begin{antwort}[]%
  \Ant Für jede Potenzreihe um den Entwicklungspunkt $a$ 
  gibt es eine Kreisscheibe $K$ 
  mit Mittelpunkt $a$ und Radius 
  $R$ (wobei die entarteten Fälle $R=0$ und $R=\infty$ zugelassen sind), 
  sodass die Reihe für alle Punkte \slanted{innerhalb} von $K$ absolut 
  konvergiert und für alle Punkte \slanted{außerhalb} von $K$ divergiert 
  ({\dasheisst}, sie konvergiert für alle $z$ mit $|z-a|<R$ und divergiert 
  für alle $z$ mit $|z-a|>R$, \sieheAbbildung\ref{fig:02_konvergenzkreis}). 
  
  Wie sich die Punkte auf dem Rand 
  der Kreisscheibe bezüglich Konvergenz verhalten, darüber kann 
  man ohne eine speziellere Untersuchung der jeweiligen Potenzreihe 
  keine allgemeinen Aussagen treffen. 
  \AntEnd

  \begin{center}
    \includegraphics{mp/02_konvergenzkreis}
    \captionof{figure}{
      Eine Potenzreihe konvergiert für alle Punkte innerhalb 
      und divergiert für alle 
      Punkte außerhalb ihres Konvergenzkreises.
    }
    \label{fig:02_konvergenzkreis}
  \end{center}
\end{antwort}

%% --- 58 --- %%
\begin{frage}
  Können Sie begründen, 
  warum eine Potenzreihe innerhalb ihres Konvergenzkreises 
  konvergiert und außerhalb davon divergiert?
\end{frage}

\begin{antwort}
  Sei $w_n := z_n-a$. Zunächst zeigt man, dass 
  aus der Konvergenz von $\sum_n a_n w_0^n$ für ein $w_0\in\KK$ 
  die Konvergenz von $\sum_n a_n |w|^n$ für alle $w$ mit $|w|<|w_0|$ folgt. 
  Dazu beachte man, dass in diesem Fall $|a_n w_0^n| < S$ für ein $S\in\RR$ 
  und alle $n\in\NN$ gilt. Das impliziert 
  $|a_n w^n| < S \left|\frac{w}{w_0}\right|^n = : Sq^n$ mit $q<1$. Die Reihe 
  $\sum_n |a_n w^n|$ hat also in $S \sum_n q^n$ eine konvergente Majorante. Das 
  beweist den ersten Teil der Behauptung.

  Man setze nun 
  \[
  R := \sup\, \left\{ x\in\RR\sets 
    \sum_{n=0}^\infty a_n x^n \text{ konvergiert}\right\}.
  \]
  Die Reihe $\sum_n  a_n w^n$
  konvergiert dann aufgrund des vorigen für jedes $w$ mit 
  $|w| < R$. Würde sie für ein $w$ mit $|w|>R$ ebenfalls 
  konvergieren, dann (wiederum aufgrund des 
  vorigen Arguments) auch für jedes $x \in \RR$ mit 
  $|w|>x>R$, im Widerspruch zur Supremumseigenschaft 
  von $R$.
  \AntEnd
\end{antwort}

%% --- 59 --- %%
\begin{frage}\label{02_prbsp}
  \index{Potenzreihe}
  \index{Potenzreihe}
  Kennen Sie ein Beispiel einer Potenzreihe, die genau in der 
  offenen Kreisscheibe 
  $\sans{\mathbb{E} = \{ z\in\CC;\; z <1 \}}$ konvergiert?
\end{frage}

\begin{antwort}
  \index{geometrische Reihe}
  Ein Beispiel dafür liefert die geometrische Reihe 
  \[
  \sum_{n=0}^\infty z^n.
  \]
  Diese konvergiert nach Frage \ref{02_rgeom} für genau die 
  Zahlen $z$ mit $|z| < 1$, während sie sonst divergent ist.
  \AntEnd
\end{antwort}

%% --- 60 --- %%
\begin{frage}\label{02_prbspa}
  \index{Potenzreihe}
  \index{Potenzreihe}
  \index{Konvergenzkreis}
  \index{Konvergenzradius}
  Kennen Sie ein Beispiel einer Potenzreihe, die genau in der 
  abgeschlossenen Kreisscheibe 
  $\overline{\mathbb{E}} = \{ z\in\CC;\; z  \le 1 \}$ 
  konvergiert?
\end{frage}



\begin{antwort}
  Die Reihe 
  \[
  \sum_{n=0}^\infty \frac{z^n}{n^2}
  \] 
  besitzt die 
  gesuchte Eigenschaft. Für den Fall $|z| < 1$ besitzt sie in 
  der geometrischen Reihe 
  $\sum_n z^n$ eine konvergente Majorante, während sie für $|z|>1$ 
  nicht konvergent sein kann, da ihre Summanden in diesem Fall 
  wegen $\limm (|z|^n/n^2) = \infty$ (vgl. Frage 
  \ref{q:folge_absch}) überhaupt keine Nullfolge bilden. 

  Sie konvergiert allerdings noch für $|z|=1$. 
  In diesem Fall gilt nämlich  
  $\sum_n |z|^n / n^2=\sum_n 1/n^2$, und dass diese Reihe konvergiert, 
  wurde in Frage \ref{03_harm} bereits gezeigt.
  \AntEnd
\end{antwort}

%% --- 61 --- %%
\begin{frage}\label{02_kokf}
  \index{Konvergenzradius}
  \index{Cauchy-Hadamard Formel}
  Welche Formeln für den Konvergenzradius einer Potenzreihe sind Ihnen geläufig?
\end{frage}

\begin{antwort}
  Das Wurzel- und Quotientenkriterium liefern 
  im Zusammenspiel mit der speziellen Bauart der 
  Summanden einer Potenzreihe $\sum_n a_n z^n$  
  unmittelbar zwei Formeln für deren Konvergenzradius $R$. 
  Für den Fall, dass $\lim \sup \sqrt[n]{|a_n|}$ bzw. 
  $\limm |a_{n+1}|/|a_n|$ existieren, gilt 
  \[\boxed{
    \begin{array}{rclp{1mm}cp{1mm}rclp{2mm}l}
      R  &=& \dis \frac{1}{W} & & \text{mit} & & 
      W  &:=& \dis \lim_{n\to\infty} 
      \sup \sqrt[n]{|a_n|} & & \text{(Cauchy-Hadamard)}\\[2mm]
      R  &=& \dis \frac{1}{Q} & & \text{mit} & & 
      Q  &:=& \dis \limm \left| \frac{a_{n+1}}{a_n}\right|
      & & \text{(Euler).}
    \end{array}}
  \]
  Hier darf mit den Festsetzungen 
  $1/0 := \infty$ und $1/\infty :=0$ gerechnet werden.  
  
  Der Beweis ist in beiden Fällen sehr einfach und folgt unmittelbar 
  aus dem Wurzel- bzw. Quotientenkriterium (hier in den 
  Formulierungen \desc{W3} und \desc{Q2}). So gilt nach dem Wurzelkriterium, 
  dass die Potenzreihe $\sum_n a_n z^n$ konvergiert bzw. divergiert, je 
  nachdem, ob für $\lim\sup\sqrt[n]{|a_n z^n|}=|z| W$ gilt
  \[
  |z| W < 1 \LLa |z| < \frac{1}{W} \qquad\text{bzw.}\qquad 
  |z| > \frac{1}{W}. 
  \]
  Daraus folgt schon unmittelbar die Formel von Cauchy-Hadamard, und diejenige 
  von Euler erhält man mit einem vollkommen analogen Argument.   
  \AntEnd
\end{antwort}

%% --- 62 --- %%
\begin{frage}\label{02_chkok}
  \index{Cauchy-Hadamard Formel}
  Warum ist die Formel von Cauchy-Hadamard zur Bestimmung des Konvergenzradius 
  für die Anwendungen schwerfällig, aber ab und zu doch nützlich?
\end{frage}

\begin{antwort}
  Die "`Schwerfälligkeit"' des Kriteriums hängt damit zusammen,
  dass hier zum einen Folgen von Wurzelausdrücken untersucht werden müssen, 
  deren Eigenschaften in der Regel nicht besonders deutlich an der Oberfläche 
  zu erkennen sind. Wollte man etwa den Konvergenzradius der 
  Exponentialreihe mit dem Kriterium von Cauchy-Hadamard bestimmen, 
  so führt das auf das Problem, den Grenzwert bzw. 
  Limes Superior der Folge $(\sqrt[n]{1/n!})$ zu bestimmen, was weitergehende 
  Argumente erfordert. Dagegen erhält man den Konvergenzradius mit dem 
  Quotientenkriterium unmittelbar. 
  Hinzu kommt, dass der Begriff des Limes Superior die Untersuchung 
  einer oftmals unendlichen Menge von Häufungspunkten erfordert, was ein 
  schwieriges Unterfangen sein kann. 

  Ein Vorteil besteht jedoch darin, dass das Kriterium von Cauchy-Hadamard 
  öfter anwendbar ist als das Kriterium von Euler, 
  da es schwächere Voraussetzungen benutzt, nämlich nur 
  die Existenz eines Limes Superior und nicht die Existenz eines Grenzwerts 
  schlechthin. 

  Als Beispiel betrachte man die Potenzreihe 
  \[
  \sum_{n=0}^\infty  a_n z^n \qquad\text{mit}\quad 
  a_n := \left\{ \begin{array}{ll} 
      2^{-n} & \text{für gerade $n$} \\
      3^{-n} & \text{für ungerade $n$}
    \end{array}\right.
  \]
  Der Grenzwert $\limm |a_{n+1}|/|a_n|$ existiert nicht, 
  das Euler-Kriterium ist also nicht anwendbar. Dagegen existiert der Limes 
  Superior der Folge $(\sqrt[n]{|a_n|})$ sehr wohl und besitzt den Wert $1/2$. 
  Daraus kann man schließen, dass der Konvergenzradius der Reihe den 
  Wert $2$ hat. \AntEnd 
\end{antwort}

%% --- 63 --- %%
\begin{frage}\label{02_kok1}
  \index{Potenzreihe!formale Ableitung}
  \index{Potenzreihe!formale Stammfunktion}
  Hat eine Potenzreihe $\sum_{n=0}^\infty c_n (z-a)^n$ 
  den Konvergenzradius $\sans{R}$, dann haben die 
  \[
  \begin{array}{lp{2mm}l}
    \text{"`formale"' Ableitung}  & &
    \dis \sans{\sum_{n=1}^\infty  n c_n  (z-a)^{n-1}} 
    \qquad\text{und die} \\[3mm]
    \text{"`formale"' Stammfunktion} & &  
    \sans{\dis \sum_{n=0}^\infty \frac{c_n}{n+1}  (z-a)^{n+1}} 
  \end{array}
  \]
  ebenfalls den 
  Konvergenzradius $R$. Können Sie das begründen?
\end{frage}

\begin{antwort}
  Wegen $\lim\limits_{n\to\infty}\sqrt[n]{n} = 1$ und 
  $\lim\limits_{n\to\infty} \sqrt[n]{1/(n+1)} = 1$ gilt  
  \[
  \lim_{n\to\infty} \sup\sqrt[n]{n|c_n|}= 
  \lim_{n\to\infty} \sup\sqrt[n]{|c_n|} =
  \lim_{n\to\infty} \sup\sqrt[n]{\frac{|c_n|}{n+1}} 
  \]
  Hieraus folgt die Gleichheit der Konvergenzradien aus der Formel 
  von Cauchy-Hadamard. \AntEnd
\end{antwort}

\begin{frage}
  Was ist der Konvergenzradius der Reihe 
  \[
  \sum_{n=0}^\infty (\cos n) z^n\,?
  \]
\end{frage}

\begin{antwort}
  Es ist $\lim_{n\to\infty} \sup \sqrt[n]{|\cos n|}=1$. Mit der Formel von Cauchy-Hadamard erhält man den Konvergenzradius $R = 1$
\end{antwort}

\section{Der Große Umordnungssatz}\label{umordungssatz}

In vielen Situationen treten Reihen der Gestalt 
$\sum_{s\in S} a_s$
auf, in denen $S$ eine abzählbare Indexmenge ist und 
$a \fd S \to \KK$ ($\KK=\CC$ oder $\KK=\RR$) eine gegebene Abbildung 
mit $s\mapsto a(s)=:a_s$. In diesem Fall ist die Summation über eine 
\slanted{Schar} oder reeller oder komplexer Zahlen zu führen.
\index{Familie komplexer oder reeller Zahlen}
\index{Schar komplexer oder reeller Zahlen} 

Da $S$ abzählbar ist, gibt es in diesem Fall 
eine Bijektion $\varphi \fd \NN_0\to S$, 
$j\mapsto \varphi(j)$. Um die obige Reihe mit den bekannten Techniken zu 
untersuchen ist es daher naheliegend, den Wert der Summe $\sum_{s\in S} a_s$ 
durch
\[
\sum_{j=0}^\infty a_{\varphi(j)} 
\]
zu definieren. Man nennt $\sum a_{\varphi(j)}$ dann eine 
\slanted{Realisierung} der Reihe. Die Definition ist allerdings nur dann 
sinnvoll, wenn alle Realisierungen denselben Wert ergeben. Das ist 
genau dann der Fall, wenn die Reihe $\sum a_{\varphi(j)}$ 
\slanted{absolut} konvergiert. Nach dem 
kleinen Umordnungssatz konvergiert dann nämlich jede Umordnung 
gegen denselben Wert.

%% --- 64 --- %%
\begin{frage}\index{summierbar}
  Sei $S$ eine abzählbare Menge und $(a_s)_{s\in S}$ eine Schar. Wann heißt 
  $(a_s)$ \bold{summierbar}? 
\end{frage}

\begin{antwort}
  Die Familie $(a_s)_{s\in S}$ heißt \slanted{summierbar}, wenn 
  eine Bijektion $\varphi\fd \NN_0 \to S$ existiert, für die die 
  Reihe $
  \sum_{j=0}^\infty a_{\varphi(j)}$ 
  absolut konvergiert. Nach dem kleinen Umordnungssatz konvergiert dann 
  jede Realisierung von $\sum_{s\in S} a_s$, und zwar gegen denselben Wert 
  wie $\sum_{j=0}^\infty a_{\varphi(j)}$. 
  \AntEnd 
  
\end{antwort} 

%% --- 65 --- %%
\begin{frage}
  Wie ist für eine summierbare Schar $(a_s)_{s\in S}$ der Wert der Summe 
  $\sum_{s\in S} a_s$ definiert?
\end{frage}

\begin{antwort}
  Ist $\sum_{j=0}^\infty a_{\varphi(j)}$ eine absolut 
  konvergente Realisierung, so setzt man 
  \[
  \sum_{s\in S} a_s:=  \sum_{j=0}^\infty a_{\varphi(j)}.
  \]
  Da jede Realisierung im Fall der Summierbarkeit der Schar denselben 
  Grenzwert hat, ist die Summe damit wohldefiniert. \AntEnd
\end{antwort} 

%% --- 66 --- %%
\begin{frage}\index{Indexmenge}\label{03_schema}
  Können Sie einige Beispiele für Indexmengen nennen, die in Anwendungen 
  häufig vorkommen?
\end{frage}

\begin{antwort}
  Das einfachste Beispiel ist natürlich $\NN_0$ selbst. 
  In vielen Anwendungen ({\zB} Fourierreihen) 
  wird über die Indexmenge $\ZZ$ summiert. Für die Reihen 
  $\sum_{k\in \ZZ} a_k $ schreibt man meist auch 
  \[
  \sum_{k=-\infty}^\infty a_k.
  \]
  Eine solche Reihe konvergiert genau dann absolut, wenn die 
  Reihe $a_0+a_1+a_{-1}+ a_2+a_{-2}+\cdots$ absolut konvergiert und 
  folglich genau dann, wenn die beiden Reihen 
  \[
  \sum_{k=0}^\infty a_k,\qquad\quad 
  \sum_{k=1}^\infty a_{-k}
  \]
  absolut konvergieren. 

  Ein weiteres Beispiel ist die Indexmenge 
  $S=\NN_0\times \NN_0$. Die Elemente 
  einer Familie $(a_s)_{s\in S}$  
  lassen sich in diesem Fall übersichtlich in einem 
  quadratischen Schema anordnen 
  \[
  \begin{array}{cccc} 
    a_{00}, & a_{01} & a_{02} & \cdots \\
    a_{10}, & a_{11} & a_{12} & \cdots \\
    a_{20}, & a_{21} & a_{22} & \cdots \\
    \vdots & \vdots & \vdots & \ddots 
  \end{array}
  \]
  Dieses Beispiel lässt sich auf Indexmengen 
  $S=\NN_0\times \cdots \times \NN_0$ 
  mit beliebig vielen Faktoren verallgemeinern. 
  Diese Mengen sind auch abzählbar. 
  \AntEnd
\end{antwort}

%% --- 67 --- %%
\begin{frage}\index{summierbar}
  Durch was kann man die Voraussetzung 
  $\sum_{j=0}^\infty | a_{\varphi(j)} |<\infty$ für Summierbarkeit 
  ersetzen?
\end{frage}

\begin{antwort}
  Eine Familie $(a_s)_{s\in S}$ ist genau dann summierbar, wenn es eine 
  Zahl $C\in\RR$ gibt, sodass für jede endliche Teilmenge $\calli{E}\subset S$ 
  gilt
  \[
  \sum_{s\in \calli{E}} |a_s| \le C.
  \]
  Ist diese Bedingung nämlich erfüllt, dann gilt auch für 
  die endlichen Summen $\sum_{j=0}^N |a_{\varphi(j)}| \le C$, und zwar 
  für jedes $N\in\NN_0$, daraus folgt die Konvergenz. \AntEnd
\end{antwort} 

%% --- 68 --- %%
\begin{frage}\index{Zerlegung einer abzählbaren Menge}
  Was versteht man unter einer \bold{Zerlegung} einer abzählbaren 
  Menge $S$?
\end{frage}

\begin{antwort}
  Eine \slanted{Zerlegung} von $S$ ist eine Folge $S_1, S_2, S_3, \ldots$ von 
  Teilmengen mit der Eigenschaft, dass jedes Element von $S$ in 
  \slanted{genau einer} der Mengen $S_k$ enthalten ist, also    
  \[
  S=S_1 \cup S_2 \cup S_3 \cup \ldots \qquad\text{und}\qquad
  S_k \cap S_\ell=\emptyset\quad\text{für $k\not=\ell$}. \EndTag
  \]
\end{antwort}

%% --- 69 --- %%
\begin{frage}\index{Grosser Umordnungssatz@Großer Umordnungssatz}
  \index{Umordnungssatz!kleiner}
  Was besagt der \bold{Große Umordnungssatz} für Reihen?
\end{frage}

\begin{antwort}
  Der Satz besagt: 

  \medskip\noindent
  \satz{Sei $S=S_1 \cup  S_2 \cup S_3 \cup \ldots$ eine Zerlegung einer 
    abzählbaren Menge $S$. Sei ferner $(a_s)_{s\in S}$ eine summierbare 
    Familie. Dann gilt:
    {\setlength{\labelsep}{4mm}
      \begin{itemize}
      \item[\desc{i}] Die Teilscharen $(a_s)_{s\in S_n}$
        sind summierbar, die Zahlen 
        \[
        A_n := \sum_{s\in S_n} a_s \qquad\text{für $n=1,2,3,\ldots$}
        \]
        also wohldefiniert.\\[-3.5mm]
      \item[\desc{ii}] Die Reihe $\sum_{n=1}^\infty A_n$ ist absolut konvergent 
        und es gilt 
        \[
        \sum_{s\in S} a_s = \sum_{n=1}^\infty A_n 
        = \sum_{n=1}^\infty \left( \sum_{s\in S_n} a_s \right).
        \]
      \end{itemize}}}
  Für einen Beweis dieses Satzes verweisen wir {\zB} auf \citep{Kaballo}. \AntEnd
\end{antwort} 

%% --- 70 --- %%
\begin{frage}\index{Cauchy-Produkt}
  Wie kann man den Großen Umordnungssatz auf Doppelreihen, 
  insbesondere die Multiplikation von Reihen und speziell das 
  Cauchy-Produkt anwenden?
\end{frage}

\begin{antwort}
  Für die Anwendung des Großen Umordnungssatzes auf Doppelreihen 
  $\sum_{i,j=0}^\infty a_{ij}$ benötigt man die Existenz einer Konstanten 
  $C\ge 0$ mit $\sum_{i,j=0}^m |a_{ij}|\le C$ für alle $m\in\NN_0$ bzw. die 
  (äquivalente) Voraussetzung 
  \[
  \sum_{i=0}^\infty \left( \sum_{j=0}^\infty |a_{ij}| \right) < \infty.
  \]
  Man kann das Schema aus Frage \ref{03_schema} zerlegen 
  \[
  \begin{array}{lp{3mm}l}
    \text{nach Zeilen} & & Z_i = \{ i \} \times \NN_0, \\
    \text{nach Spalten} & & S_j = \NN_0 \times \{ j \}, \\
    \text{nach Diagonalen} & & D_k = \big\{ (i,j) \in \NN_0\times \NN_0 \sets 
    i+j=k \big\}
  \end{array}
  \]
  und erhält den \slanted{Doppelreihensatz}:\index{Doppelreihensatz}
  \[
  \sum_{i,j=0}^\infty =  
  \sum_{i=0}^\infty \left( \sum_{j=0}^\infty a_{ij} \right) =
  \sum_{j=0}^\infty \left( \sum_{i=0}^\infty a_{ij} \right) =
  \sum_{k=0}^\infty \left( \sum_{i+j=k} a_{ij} \right) = 
  \sum_{k=0}^\infty \sum_{i=0}^k a_{i,k-i}.
  \]
  Für die Anwendung auf die Multiplikation zweier absolut konvergenter Reihen 
  $\sum_i b_i $ und $\sum_j c_j $ setzt man $a_{ij} := b_i c_j$. \AntEnd 
\end{antwort}