\chapter{Vektorfelder, Kurvenintegrale, Integrals\"atze}

Die Fragen in diesem Kapitel betreffen \slanted{Vektorfelder}, 
\slanted{Kurvenintegrale}, 
\slanted{Integration auf $\calli{C}^1$-Untermannigfaltigkeiten} 
und schließlich \slanted{Integralsätze}, 
speziell den Gauß\sch en Integralsatz 
in seiner klassischen Form. 
Um den allgemeinen Stokes\sch en Integralsatz 
mithilfe des Differentialformenkalküls formulieren zu können, 
ist erheblich größerer Aufwand nötig, {\zB} 
\begin{itemize}[2mm]\index{Grassmann@Graßmann-Algebra}
  \index{berandete Mannigfaltigkeit}\index{Cartan-Ableitung}
\item[\desc{i}] die Graßmann-Algebra (alternierende Multilinearformen),
\item[\desc{ii}] Integration und Differenziation (Cartan-Ableitung) von 
  Differenzialformen,
\item[\desc{iii}] der Begriff der berandeten Mannigfaltigkeit,
\item[\desc{iv}] der Begriff der \slanted{Orientierung} von Mannigfaltigkeiten.
\end{itemize}
Der Beweis des Stokes\sch en Integralsatzes ist nach der Entwicklung 
dieses Begriffsapparates relativ einfach. Wir werden diese Themen im letzten 
Abschnitt anreißen, jedoch ohne die Theorie systematisch zu entwickeln 
und vollständige Beweise zu geben. Wir verweisen dafür 
(insbesondere hinsichtlich der Rückübersetzung des Cartan-Kalküls in die 
klassische Sprache der \index{Vektoranalysis}Vektoranalysis) auf 
\citep{JaenichPhy}.

\section{Vektorfelder, Kurvenintegrale, Pfaff'sche Formen}

%% Question 1
\begin{frage}\index{Vektorfeld}
  Was versteht man unter einem \bold{Vektorfeld} auf einem Gebiet $D\subset\RR^n$? 
\end{frage}

\begin{antwort}
  Unter einem \slanted{Vektorfeld} $F$ auf einem Gebiet 
  $D\subset \RR^n$ versteht man eine Abbildung, die jedem Punkt $x\in D$ 
  einen Vektor $F(x)\in\RR^n$ zuordnet. 

  Eine Vekorfeld heißt ($s$-mal stetig) differenzierbar, wenn die 
  Komponentenfunktionen die entsprechende Eigenschaft haben.

  Für $D\subset\RR^2$ bzw. $D\subset\RR^3$ lässt sich ein Vektorfeld 
  $F$ auf $D$ visualisieren, indem man an ausgewählten 
  Punkten $x\in D$ den Vektor $F(x)$ anträgt (genauer den Pfeil mit der 
  Länge und der Richtung des Vektors $F(x)$.)

  \begin{center}
    \includegraphics{mp/12_vektorfeld1}
    \captionof{figure}{Beispiele von Vektorfeldern.}
    \label{fig:12_vektorfeld1}
  \end{center}

  Abbildung~\ref{fig:12_vektorfeld1} zeigt 
  einige Beispiele für Vektorfelder. Die zweite 
  Abbildung zeigt ein \slanted{Zentralfeld}, die dritte ein 
  \slanted{Rotationsfeld}.\AntEnd
\end{antwort}

%% Question 2
\begin{frage}\index{Nabla@$\nabla$ (Nabla-Operator)}\label{12_nabla}
  Wie ist der \bold{Nabla-Operator} erklärt? Welcher Zusammenhang besteht 
  mit dem Gradienten einer Funktion $f\fd \RR^n\to\RR$. 
\end{frage}

\begin{antwort}
  Der Nabla-Operator ist ein \slanted{vektorwertiger Differenzialoperator}, 
  symbolisch schreibt man $\nabla := ( \partial_1, \ldots, \partial_n )^T$. 
  Für eine partiell differenzierbare Funktion ist in einem Punkt 
  $x\in D$ per Definition \nomenclature{$\nabla$}{Nabla-Operator}
  \[
  \nabla f(x) = \big(\partial_1 f(x), \ldots, \partial_n f(x) \big) =
  \grad f(x).
  \]  
  Durch 
  \[
  D \to \RR^n; \qquad x \mapsto \nabla f(x) = \grad f(x),
  \]
  wird auf $D$ ein spezielles \slanted{Vektorfeld}, das sogenannte 
  \slanted{Gradientenfeld}\index{Gradientenfeld} definiert. \AntEnd 
\end{antwort}

%% Question 3
\begin{frage}\index{Divergenz}
  Was versteht man unter der \slanted{Divergenz} eines Vektorfeldes 
  $F=(F_1,\ldots,F_n)^T \fd D\to\RR^n$?
\end{frage}

\begin{antwort}
  Die \slanted{Divergenz von $F$} ist ein Skalarenfeld auf $D$, definiert 
  durch
  \[\boxed{
    \Div F(x) := \frac{\partial F_1}{\partial x_1} + 
    \cdots + \frac{\partial F_n}{\partial x_n}.
  }
  \]
  In der Antwort zur Frage \ref{12_quelldichte}\index{Quelldichte}
  wird die $\Div F(x)$  
  physikalisch als Maß der \slanted{Quelldichte} von $F$ im Punkt $x$ 
  interpretiert. \AntEnd 
\end{antwort}

%% Question 4
\begin{frage}\index{Rotation eines Vektorfelds}
  Wie ist im Fall $n=3$ die \bold{Rotation} eines 
  $\calli{C}^1$-Vektorfeldes $F$ erklärt?
\end{frage}

\begin{antwort}
  Die \slanted{Rotation von $F$} ist ein Vektorfeld 
  $\rot F \fd \RR^3\to\RR^3$, \nomenclature{$\rot F$}{
    Rotation des Vektorfelds $F$}
  definiert durch 
  \[
  \boxed{
    \rot F :=\big( 
    \partial_2F_3-\partial_3 F_2,\, 
    \partial_3F_1-\partial_1 F_3,\,
    \partial_1F_2-\partial_2 F_1 
    \big)^T.
  }
  \]
  Die Rotation lässt sich als \slanted{Wirbeldichte} eines Vektorfelds 
  interpretieren, anschaulich als Maß dafür, wie stark die 
  Richtungsänderung des Felds in einem Punkt ist. 
  Die Rotation spielt in der Formulierung 
  des klassischen Integralsatzes von Stokes eine große Rolle.
\end{antwort} 

%% Question 5
\begin{frage}
  Wie lassen sich $\grad f$, $\rot v$ und $\Div v$ mithilfe des 
  Nabla-Operators symbolisch ausdrücken?
\end{frage}

\begin{antwort}
  Man schreibt symbolisch 
  \[
  \boxed{
    \grad f = \nabla f, \qquad 
    \rot F  = \nabla \times F, \qquad 
    \Div F = \nabla \cdot F.}
  \]
  Dabei ist "`$\cdot$"' bzw. "`$\times$"' in Analogie zum 
  Skalarprodukt im $\RR^n$ bzw. Vektorprodukt im $\RR^3$ zu verstehen. 

  Man muss bei der Verwendung dieser Schreibweise allerdings etwas aufpassen. 
  Sie suggeriert, dass sich mit $\nabla$ wie mit einem Vektor rechnen ließe. 
  Das ist aber aufgrund der Ableitungsregeln für Produkte und Quotienten 
  nicht der Fall. 
  \AntEnd
\end{antwort} 

%% Question 6
\begin{frage}\index{Laplace-Operator}
  \index{Laplace@\textsc{Laplace}, Pierre-Simon (1749-1827)}
  Wie lautet die Definition des \bold{Laplace-Operators}? 
\end{frage}

\begin{antwort}
  Der \slanted{Laplace-Operator} $\Delta$ ist ein 
  Differenzialoperator, für $U\subset\RR^n$ ist durch 
  ihn eine lineare Abbildung 
  \[
  \calli{C}^2(U) \to \calli{C}(U), \qquad
  \Delta f := \partial_1^2 f + \cdots + \partial_n^2 f
  \] 
  definiert. Man benutzt die Schreibweise
  \[
  \boxed{\Delta := \partial_1^2 + \cdots + \partial_n^2.} 
  \]
  Mit dem Nabla-Operator hat man auch die Darstellung 
  $\Delta = \nabla \cdot \nabla.$ \AntEnd
\end{antwort} 

%% Question 7
\begin{frage}
  Was bedeutet die \bold{Drehinvarianz} des Laplace-Operators?
\end{frage}

\begin{antwort}
  Der Laplace-Operator ist \slanted{drehinvariant} 
  in folgendem Sinn: Ist $\{ v_1,\ldots,v_n \}$ eine 
  Orthonormalbasis, dann gilt für $f\in\calli{C}^2$: 
  \[
  \Delta f = \partial_{v_1}^2 f + \cdots + \partial_{v_n}^2 f. 
  \]
  Für jeden Basisvektor $v_i$ gilt nämlich
  \[
  \partial_{v_i} \big(\partial_{v_i} f(x)\big) = 
  \sum_{i,j=1}^n \partial_{ij} f(a) v_i v_j = v_i^T H_f(x) v_i, 
  \]
  wobei $H_f(x):=H$ die Hesse-Matrix von $f$ in $x$ ist. 
  Bezeichnet $V$ die Matrix mit den Basisvektoren $v_i$ als Spalten, so  
  folgt daraus $\partial_{v_i}\partial{v_i} f = e_i^T V^T H_f(x) V e_i$. 
  Da $V$ orthogonal ist, haben $V^T H_f(x) V$ und $H_f(x)$ dieselbe Spur. 
  Diese ist aber nach Definition des 
  Laplace-Operators gleich $\Delta f(x)$. Es gilt 
  also tatsächlich 
  \[
  \partial_{v_1}^2 f + \cdots + \partial_{v_n}^2 = 
  \mathrm{Spur}\, V^T H_f(x) V = \mathrm{Spur}\,H_f(x)=\Delta f(x).
  \EndTag
  \]
\end{antwort} 

%% Question 8
\begin{frage}\index{harmonische Funktion}\index{holomorphe Funktion}
  Was versteht man unter einer \bold{harmonischen Funktion} 
  $f\fd \RR^n\to\RR$? Kennen Sie eine wichtige Klasse 
  von Funktionen, deren Elemente harmonische Funktionen sind?
\end{frage}

\begin{antwort}
  Eine Funktion $f$ heißt \slanted{harmonisch}, wenn 
  $\Delta f=0$ gilt. 

  Wichtige harmonische Funktionen sind die 
  \slanted{komplex differenzierbaren} bzw. \slanted{holomorphen Funktionen}.  
  Das ist eine unmittelbare Folge der 
  \slanted{Cauchy-Riemann\sch en Differenzialgleichungen}
  \index{Cauchy-Riemann'sch e Differenzialgleichungen} 
  (s. Frage \ref{10_cauchy_riemann}).    
  \AntEnd 
\end{antwort} 

%% Question 9
\begin{frage}\index{rotationssymmetrische Funktionen}
  Wie lassen sich die \bold{rotationssymmetrischen 
    harmonischen Funktionen} bestimmen?
\end{frage}

\begin{antwort}
  Sei $\varphi$ eine $\calli{C}^2$-Funktion auf einem 
  Intervall $I\subset\open{0,\infty}$ und sei 
  \[
  f(x)=\varphi(\n{x}_2)
  \]
  für $x\in K(I) := \{ x\in\RR^n\sets \n{x}_2 \in I\}$. 
  Die Funktion $f$ heißt in diesem Fall \slanted{rotationssymmetrisch}

  Mit $r:=\n{x}_2$ gilt $\partial_\nu f(x) = F'( r )\cdot \frac{x_\nu}{r}$ 
  und damit
  \[
  \partial_\nu^2 f(x) = \varphi''( r ) \cdot \frac{x_\nu^2}{ r^2 } 
  + \varphi'(r) \left( \frac{1}{r} - \frac{x_\nu^2}{r^3} \right), 
  \]
  also 
  \[
  \Delta f(x) = \varphi''(r) + \frac{n-1}{r} \varphi'(r), \qquad r=\n{x}_2. 
  \]
  Diese lineare Differenzialgleichung erster Ordnung in $\varphi'$ 
  besitzt die Lösungen $ar^{n-1}$ mit $a\in\CC$. Also ist 
  $\varphi(r) = c \log r + b$, falls $n=2$ und 
  $\varphi(r) =  cr^{2-n}+b$ im Fall $n>2$ mit $c,b\in\CC$.
  Speziell sind also die in 
  $\RR\mengeminus\{0\}$ definierten 
  Funktionen
  \[
  h(x) := \log\n{x}_2\quad\text{für $n=2$ und}, \qquad
  g(x) := 1/\n{x}_2^{2-n}\quad \text{für $n>2$}
  \]
  harmonisch, erfüllen also die Potenzialgleichung $\Delta f=0$.\AntEnd
\end{antwort} 

%% Question 10
\begin{frage}\index{Kurvenintegral!eines Vektorfelds}
  Sei $\alpha\fd [a,b]\to D$ eine in einem Gebiet $D$ 
  verlaufende stetig differenzierbare Kurve. Wie ist dann für ein 
  stetig differenzierbares Vektorfeld $F\fd D\to\RR^n$ das 
  \bold{Kurvenintegral} von $F$ längs $\alpha$ erklärt?
\end{frage}

\begin{antwort}
  Man definiert 
  \nomenclature{$\int_\alpha F$}{Kurvenintegral des Vektorfelds $F$ längs $\alpha$} 
  \[
  \boxed{
    \int_\alpha F := \int_a^b F \big( \alpha(t) \big) \cdot 
    \dot{\alpha}(t)\dift 
    = 
    \int_a^b \langle F ( \alpha(t) \big), \dot{\alpha}(t) \rangle \dift. 
  }
  \]

  \begin{center}
    \includegraphics{mp/12_kurvenintegral}
    \captionof{figure}{Tangentieller Anteil des Vektorfeldes $F$ im Bezug auf $\alpha$}
    \label{fig:12_kurvenintegral}
  \end{center}

  Das Integral $\int_\alpha F$ ist ein Maß für den 
  \slanted{tangentiellen Anteil} des Vektorfeldes $F$ im Bezug 
  auf $\alpha$ im Mittel (\sieheAbbildung\ref{fig:12_kurvenintegral}). 
  Physikalisch wird dadurch etwa die Arbeit 
  angegeben, die ein Probekörper bei Durchlaufen der Kurve $\alpha$ 
  in einem Kraftfeld $F$ verrichtet. \AntEnd   
\end{antwort}

%% Question 11
\begin{frage}\index{Permanenzeigenschaften!des Kurvenintegrals}
  Welche \bold{Permanenzeigenschaften} hat das Kurvenintegral?
\end{frage}

\begin{antwort}
  \desc{i}\, Für zwei Vektorfelder $F$ und $G$ auf $D\subset\RR^n$ 
  und $a , b\in \RR$ gilt:
  \[
  \int_{\alpha} a F + b G   = 
  a\int_\alpha F   + b\int_\alpha G.
  \]
  \noindent
  \desc{ii}\, Für jede Umparametrisierung 
  $\beta\fd [c,d] \to \RR^2$ von $\alpha$, 
  also jede reguläre Kurve $\beta$, 
  die dieselbe Spur wie $\alpha$ beschreibt,  
  folgt aus der Substitutionsregel für Integrale einer reellen Veränderlichen
  \[
  \int_\beta F  = \left\{ \begin{array}{rr} 
      \int_\alpha F , &\text{falls $\beta(c)=\alpha(a)$ gilt,}\\
      -\int_\alpha F , &\text{falls $\beta(c)=\alpha(b)$ gilt}.
    \end{array}\right.
  \]
  Insbesondere ändert das Integral sein Vorzeichen, wenn der Integrationsweg 
  in entgegengesetzter Richtung durchlaufen wird.   

  \noindent
  \desc{iii}\, Ist $\alpha=\gamma_1 \oplus \gamma_2$ die Zusammensetzung 
  zweier regulärer Kurven, dann gilt für das Kurvenintegral 
  längs $\alpha$: 
  \[
  \int_\alpha F = \int_{\gamma_1} F + 
  \int_{\gamma_2} F.
  \]

  \noindent
  \desc{iv}\, Es gilt die Abschätzung 
  \[
  \int_\alpha F \le \nb{ F\circ \alpha }_{[a,b]} \cdot \ell( \alpha ), 
  \]
  wobei $\ell(\alpha)$ die \slanted{Kurvenlänge} angibt. 
  \AntEnd
\end{antwort} 

%% Question 12
\begin{frage}\label{12_hauptsatz}\index{Hauptsatz für Kurvenintegrale}
  Wie lautet der \bold{Hauptsatz über Kurvenintegrale}?
\end{frage}

\begin{antwort}
  In völliger Analogie zum Hauptsatz der Differenzial- und 
  Integralrechnung liefert der Hauptsatz für Kurvenintegrale eine Aussage 
  über die Beziehung zwischen Integration (längs einer Kurve) und 
  Differenziation. Er lautet 

  \medskip\noindent
  \satz{\noindent
    Ist $F=\grad f\fd D\to\RR^n$ ein stetiges Gradientenfeld auf dem Gebiet 
    $D\subset\RR^n$, dann gilt für jede stückweise reguläre Kurve 
    $\alpha$ in $D$ mit Anfangspunkt $\alpha(a)$ und Endpunkt $\alpha(b)$: 
    \[
    \boxed{ 
      \int_\alpha F = f \big( \alpha(b) \big)-f \big( \alpha(a) \big).
    } 
    \]
  }

  \medskip\noindent
  Der Zusammenhang ergibt sich daraus, dass die Funktion 
  $t\mapsto \grad f\big( \alpha(t) \big) \cdot\dot{\alpha}(t)$ 
  nach der Kettenregel eine Stammfunktion der Funktion 
  $t\mapsto f\big( \alpha(t) \big)$ ist. Damit folgt der Hauptsatz 
  für Kurvenintegrale aus dem Hauptsatz der Differenzial- und 
  Integralrechnung einer Veränderlichen
  \[
  \int_\alpha F  = 
  \int_a^b \grad f\big(\alpha(t)\big)\cdot\dot{\alpha}(t) \dift = 
  f \big( \alpha(b) \big)-f \big( \alpha(a) \big).\EndTag
  \]
\end{antwort}

%% Question 13
\begin{frage}\index{konservatives Vektorfeld}
  \index{Stammfunktion!eines Vektorfelds}
  \index{Potential}
  Wann nennt man ein Vektorfeld \bold{konservativ}?
\end{frage}

\begin{antwort}
  Ein Vektorfeld $F\fd D\to\RR^n$ heißt \slanted{konservativ}, 
  wenn es ein Gradientenfeld ist, {\dasheisst} wenn eine Funktion 
  $f\fd D\to\RR^n$ existiert, so dass $F=\grad f$ gilt. 
  In diesem Fall heißt $f$ \slanted{Stammfunktion} von $F$  
  und $u:=-f$ \slanted{Potenzial} von $F$.  
  \AntEnd  
\end{antwort} 

%% Question 14
\begin{frage}\label{12_integrab}
  \index{Integrabilitätsbedingung!für die Existenz eines Potentials}
  Welche notwendige Bedingung muss ein stetig 
  differenzierbares Vektorfeld $F$ erfüllen, damit es 
  ein Potenzial besitzt? 
\end{frage}

\begin{antwort}
  Aus $F=\grad f$ folgt  
  $\partial_i F_k = \partial_i \partial_k f$ für alle 
  $i,k \in\{1,\ldots,n\}$.  
  Nach dem Satz von Schwarz ist die Reihenfolge der 
  partiellen Ableitungen in $\partial_i \partial_k f$ 
  vertauschbar, und es folgt
  \[
  \partial_i F_k = \partial_k \partial_i f = \partial_k F_i, 
  \qquad i,k \in{1,\ldots,n}.
  \]
  Das ist gleichbedeutend damit, dass die Jacobi-Matrix 
  von $F$ in jedem Punkt $x\in D$ symmetrisch ist: 
  \[
  \calli{J}(F;x)=\calli{J}(F;x)^T.\EndTag
  \] 
\end{antwort}

%% Question 15
\begin{frage}
  Wie lässt sich die Integrabilitätsbedingung aus der vorigen Frage 
  für ein Vektorfeld $F\fd \RR^3 \to \RR^3$ ausdrücken?
\end{frage} 

\begin{antwort}
  Im Fall $n=3$ ist die Bedingung gleichbedeutend mit $\rot F = 0$. \AntEnd 
\end{antwort} 

%% Question 16
\begin{frage}
  Kennen Sie ein Beispiel eines Vektorfeldes, welches die 
  Integrabilitätsbedingung erfüllt, 
  die auf ihrem Definitionsbereich aber keine Stammfunktion besitzt?
\end{frage} 

\begin{antwort}
  Sei $D=\RR^2 \mengeminus\{ 0 \}$ und $F \fd D \to \RR^2$ definiert durch 
  \[
  F(x,y) := \frac{1}{x^2+y^2} (-y,x) .
  \]
  Für $F$ gilt $\partial_1 F_2 = \partial_2 F_1 = \frac{-x+y^2}{(x^2+y^2)^2}$, 
  die Integrabilitätsbedingung ist also erfüllt. 

  Wir zeigen, dass $F$ auf $D$ keine Stammfunktion 
  besitzt. Dazu integrieren wir $F$ längs einer geschlossenen Kurve $\alpha$, 
  deren Spur die Einheitskreislinie im $\RR^2$ ist. Da Anfangs- und Endpunkt 
  von $\alpha$ identisch sind, müsste das Kurvenintegral $\int_\alpha F$ 
  nach dem Hauptsatz gleich null sein, falls $F$ in $D$ eine Stammfunktion 
  besitzt. Mit der Parametrisierung $\alpha(t)=(\cos t, \sin t)$ 
  erhält man jedoch 
  \[
  \int_\alpha F(x) \difx = F\big( \alpha(t) \big) \cdot \dot{\alpha}(t) \dift 
  =
  \int_0^{2\pi} \begin{pmatrix} -\sin t \\ \cos t \end{pmatrix} 
  \cdot  
  \begin{pmatrix} -\sin t \\ \cos t \end{pmatrix} \dift = 
  \int_0^{2\pi} 1 \dift =2\pi. 
  \]
  $F$ kann also auf $D$ also keine Stammfunktion besitzen. 
  Der Grund liegt darin, dass $D$ kein einfach zusammenhängendes 
  Gebiet ist (vgl. Frage \ref{12_poincare} und \ref{12_einfachzus}). \AntEnd 
\end{antwort} 

%% Question 17
\begin{frage}\label{12_poincare}\index{Lemma von Poincar\'e}
  \index{Poincare@\textsc{Poincar\'e}, Jules-Henri (1854-1912)}
  Wenn $D\subset\RR^n$ ein Sterngebiet ist und $F\fd D\to\RR^n$ ein stetig 
  differenzierbares Vektorfeld, das die Integrabilitätsbedingung erfüllt, 
  wie kann man dann eine stetig differenzierbare Funktion 
  $f\fd\RR^n\to\RR$ mit $\grad f=F$ konstruieren?
\end{frage}

\begin{antwort}
  Sei $x_0$ das Zentrum des Sterngebiets. {\OBdA} können 
  wir $x_0=0$ annehmen und integrieren $F$ längs des Streckenzugs 
  $\alpha$ von $0$ nach $x\in D$, also der Kurve 
  $\alpha\fd [0,1]\to D$ mit $t\mapsto xt$.  
  Für eine eventuelle Stammfunktion $f$ von $F$ 
  mit $f(0)=0$ müsste nach Frage \ref{12_hauptsatz} gelten: 
  \[
  f(x)=\int_{x_0}^x  F (u) \dd u = 
  \int_0^1 \sum_{i=1}^n F_i (xt) x_i \dift 
  \]
  Wir zeigen, dass tatsächlich $\grad f = F$, also 
  $\partial_k f=F_k$ für $k=1,\ldots,n$ gilt. Differen\-ziation 
  unter dem 
  Integral und die Voraussetzung $\partial_k F_i = \partial_i F_k$
  ergibt zunächst  
  \begin{align*}
    \partial_k f(x) &= 
    \int_0^1 \left( \sum_{i=1}^n t\cdot \partial_k F_i (xt) \cdot x_i + 
      F_k(xt) \right) \dift \\
    &=
    \int_0^1 \left( 
      t\cdot \Big( \sum_{i=1}^n  \partial_i F_k (xt) \cdot x_i \Big) + 
      F_k(xt) \right)  \dift.
  \end{align*}
  Nach der Kettenregel ist 
  $\sum_{i=1}^n \partial_i F_k (xt) \cdot x_i = 
  \frac{\dd}{\dift} F_k(xt)$, also 
  folgt zusammen mit Produktregel und dem Hauptsatz 
  der Differenzialrechnung einer Veränderlichen 
  \[
  \partial_k f(x) = 
  \int_0^1 \left(  t\cdot \Big( \frac{\dd}{\dift} F_k(xt) \Big) + 
    F_k(xt) \right) \dift = 
  \int_0^1 \Big( \frac{\dd}{\dift} t F_k(xt) \Big) \dift = 
  F_k(x).\EndTag
  \]
  
\end{antwort} 

%% Question 18
\begin{frage}\label{12_einfachzus}\index{einfach zusammenhängend}
  Kennen Sie eine größere Klasse von Gebieten $D\subset\RR^n$, für welche die 
  Integrabilitätsbedingung \slanted{hinreichend} für die Existenz eines 
  Potenzials ist?
\end{frage}

\begin{antwort}
  Der Satz gilt allgemein für \slanted{einfach zusammenhängende Gebiete}. 
  Ein Gebiet $D\subset \RR^n$ heißt \slanted{einfach zusammenhängend}, wenn 
  jede geschlossene Kurve in $D$ stetig auf einen Punkt in $D$ 
  zusammengezogen werden kann, ohne dass $D$ verlassen wird. 
  Speziell im Zweidimensionalen sind einfach zusammenhängende Gebiete  
  als zusammenhängende Teilmengen dadurch ausgezeichnet, dass 
  sie keine "`Löcher"' besitzen. 

  Aus dem vorhergehenden Satz über Sterngebiete folgt zunächst, dass 
  jedes Vektorfeld, das auf einer beliebigen offenen Menge 
  $D\subset \RR^n$ die Integrabilitätsbedingung erfüllt, 
  dort \slanted{lokal} eine Stammfunktion 
  besitzt in dem Sinne, dass für jedes $x\in D$ eine Umgebung 
  $U_x$ existiert, auf der $F$ eine Stammfunktion besitzt. 
  Denn $D$ enthält zu jedem Punkt $x\in D$ eine $\eps$-Umgebung, 
  und das ist ein Sterngebiet. 

  Besitzt $F$ lokal eine Stammfunktion auf $D$, dann existiert eine globale 
  Stammfunktion genau dann, wenn $\int_\alpha F=0$ für jede in $D$ 
  verlaufende geschlossene Kurve $\alpha$ gilt. Das ist eine Konsequenz 
  aus dem Hauptsatz für Kurvenintegrale. Diese Bedingung ist nicht für 
  beliebige 
  Gebiete $D$ erfüllt, wohl aber für die einfach zusammenhängenden (Beweis 
  s. \citep{Koenig}).
  \AntEnd  
\end{antwort} 

%% Question 19
\begin{frage}\index{Pfaffsche Form@Pfaff\sch e Form}\index{Eins@$1$-Form}
  \index{Pfaff@\textsc{Pfaff}, Johann Friedrich (1765-1932)}
  Was versteht man unter einer \bold{Pfaff'schen Form ($1$-Form)} auf einer 
  offenen Menge $D\subset\RR^n$? 
\end{frage}

\begin{antwort}
  Eine Pfaff\sch e Form oder $1$-Form auf einer offenen Menge $D\subset\RR^n$ 
  ist nach Definition eine Abbildung $\omega$, die jedem $x\in D$ 
  eine lineare Abbildung $\omega(x)\fd \RR^n\to \RR$ zuordnet, also 
  eine Abbildung 
  \[
  \omega\fd D \to \mathrm{L}(\RR^n,\RR).
  \]
  Eines der wichtigsten Beispiele einer Pfaff\sch en Form ist das 
  Differenzial $\dd f$ einer differenzierbaren 
  Funktion $f\fd \RR^n\to \RR$, welches  
  jedem Punkt $x\in D$ die Linearform $\dd f(x)$ zuordnet. \AntEnd
\end{antwort} 

%% Question 20
\begin{frage}
  Gibt es eine Bijektion zwischen $1$-Formen und Vektorfeldern?
\end{frage}

\begin{antwort}
  Pfaff\sch e Formen lassen sich als die zu Vektorfeldern 
  $F\fd D\to\RR^n$ \slanted{dualen Objekte} verstehen. 
  Der Zusammenhang wird wie in der linearen Algebra durch das 
  Skalarprodukt hergestellt. Ist $F$ ein Vektorfeld auf $D$, dann ist  
  durch 
  \[
  \omega(x) v := \langle F(x), v \rangle \qquad\text{für alle $v\in \RR^n$}
  \asttag
  \]     
  für jeden Punkt $x\in D$ eine Linearform $\omega(x)$ definiert. 

  Ist umgekehrt eine $1$-Form $\omega$ auf $D$ gegeben, dann folgt mit 
  linearer Algebra, dass durch die Gleichung {\astref}
  der Vektor $F(x)$ für jedes $x\in D$ eindeutig bestimmt ist. 
  \AntEnd
\end{antwort}

%% Question 21
\begin{frage}\index{Kurvenintegral!einer $1$-Form}
  Wie ist das \bold{Kurvenintegral für eine $1$-Form $\omega$} erklärt?
\end{frage}

\begin{antwort}
  \nomenclature{$\int_\alpha \omega$}{Integral 
    der $1$-Form $\omega$ längs der Kurve $\alpha$}
  Aufgrund der Beziehung {\astref} definiert man das Kurvenintegral 
  von $\omega$ längs der Kurve $\alpha\fd [a,b]\to D$ durch 
  \[
  \int_\alpha \omega := 
  \int_a^b \omega\big( \alpha(t) \big) \dot{\alpha}(t) \dift.
  \]
  Ist $F_\omega$ das der $1$-Form $\omega$ assozierte Vektorfeld, 
  so gilt also $\int_\alpha \omega = \int_\alpha F_\omega$. \AntEnd
\end{antwort}  

%% Question 22
\begin{frage}\index{Koordinatendifferenzial}\label{12_koorddiff}
  \nomenclature{$\difx_i$}{Differenzial ($1$-Form) zur 
    Koordinatenfunktion $(\xi_1,\ldots, \xi_n) \mapsto \xi_i$}
  Wie sind die $1$-Formen $\dd x_i$ definiert und wie 
  lassen sich allgemeine $1$-Formen durch diese ausdrücken? 
\end{frage}

\begin{antwort}
  Die $1$-Formen $\dd x_i$ sind die Differenziale der 
  Koordinatenfunktion $(\xi_1,\ldots, \xi_n) \mapsto \xi_i$. 
  Es handelt sich also bei $\dd x_i$ um eine konstante $1$-Form, 
  für die $\dd x_i (\xi ) v =v_i$ an jeder Stelle $\xi\in D$ und 
  jeden Vektor $v\in \RR^n$ gilt.    
  Ferner gilt aufgrund der Linearität von $\omega(\xi)$    
  \[
  \omega( \xi ) v = \sum_{i=1}^n \omega (\xi ) e_i \cdot v_i = 
  \sum_{i=1}^n \omega (\xi ) e_i \cdot \dd x_i (\xi) v. 
  \]
  Durch $a(\xi) := \omega (\xi ) e_i$ sind eindeutig $n$ Funktionen 
  $a_i \fd \RR^n \to \RR$ definiert. Mit diesen besitzt 
  jede $1$-Form eine Darstellung der Gestalt 
  \[
  \boxed{
    \omega = a_1 \difx_1 + \cdots + a_n \difx_n. 
  } \EndTag
  \]
\end{antwort}

%% Question 23
\begin{frage}
  Wie lautet die Darstellung aus der letzten Frage für den Fall, dass 
  $\omega=\dd f$ das Differenzial einer 
  differenzierbaren Funktion $f$ ist, wie lautet sie, 
  wenn $\omega=\omega_F$ die dem Vektorfeld $F$ zugeordnete $1$-Form ist? 
\end{frage}

\begin{antwort}
  Für $\omega=\dd f$ ist $a_i(\xi)=\dd f(\xi) e_i=\partial_i f(\xi)$, 
  und damit besitzt $\dd f$ die Darstellung
  \[
  \dd f = \partial_1 f \difx_1 + \cdots +\partial_n f \difx_n.
  \]

  Ist $\omega$ die einem Vektorfeld $F=(F_1,\ldots,F_n)$ zugeordnete 
  Form, so gilt $a_i(\xi)=\langle F(\xi), e_i \rangle =F_i(x)$ und 
  damit 
  \[
  \omega_F = F_1 \difx_1 + \cdots + F_n \difx_n.
  \EndTag
  \]
  
\end{antwort}

%% Question 24
\begin{frage}\index{Stammfunktion!einer $1$-Form}
  Was versteht man unter einer \slanted{Stammfunktion} einer 
  $1$-Form $\omega$ auf einer offenen Teilmenge $U\subset\RR^n$?
\end{frage}

\begin{antwort}
  Eine \slanted{Stammfunktion} einer 
  $1$-Form $\omega = f_1 \difx_1 + \cdots + f_n \difx_n$ auf $U$ ist eine 
  differenzierbare Funktion $f\fd U\to\RR$ mit 
  $\omega = \dd f$, also $f_1 = \partial_1 f, \ldots, f_n=\partial_n f$. 

  Eine $1$-Form heißt \slanted{exakt} auf $U$, wenn sie auf $U$ eine 
  Stammfunktion besitzt. \AntEnd
\end{antwort} 



%% Question 25
\begin{frage}\index{Integrabilitätsbedingung!fuer 1-Formen@für $1$-Formen}
  Wie kann man die Integrabilitätsbedingung für eine $1$-Form formulieren?
\end{frage}

\begin{antwort}
  Besitzt die $1$-Form $\omega= f_1 \dd x_1 +\cdots+ f_n\dd x_n$ auf $U$ eine 
  Stammfunktion, dann gilt $f_k= \partial_k \varphi$ für $k=1,\ldots,n$ 
  und einer Funktion $\varphi\fd U\to\RR$. Ist ferner $\omega$ stetig 
  differenzierbar, dann auch die Komponentenfunktionen $\partial_k \varphi$. 
  Wegen des Satzes von Schwarz gilt dann 
  $\partial_k \partial_j \varphi=\partial_j\partial_k \varphi$ 
  und folglich wegen $f_k = \partial_k\varphi$
  \[
  \boxed{ \partial_j f_k - \partial_k f_j = 0, 
    \qquad\text{für alle $j,k=1,\ldots,n$.}}\asttag
  \]  
  Die Integrabilitätsbedingung für $1$-Formen ist also vergleichbar mit 
  der für Vektorfelder, hat dieselbe Gestalt.\AntEnd 
\end{antwort} 

%% Question 26
\begin{frage}\index{Lemma von Poincar\'e}
  Was besagt das Poincar\'e\sch e Lemma im Bezug auf $1$-Formen?
\end{frage} 

\begin{antwort}
  Das Poincar\'e\sch e Lemma besagt: 

  \medskip%
  \noindent%
  \slanted{Erfüllt eine stetig differenzierbare 
    $1$-Form $\omega$ auf einem 
    Sterngebiet $D$ die Integrabilitätsbedingung {\astref}, dann 
    besitzt sie auf $D$ eine Stammfunktion.}

  \medskip
  \noindent% 
  Der Beweis geht wörtlich wie in Frage \ref{12_poincare}. 
  Genauso treffen die Verallgemeinerungen aus Frage 
  \ref{12_einfachzus} ebenso auf $1$-Formen zu.\AntEnd 
\end{antwort}


\section{Die Integralsätze von Gauß und Stokes}


Mithilfe des Gauß'schen Integralsatzes kann man ein 
Volumenintegral über die Divergenz eines Vektorfeldes 
durch ein Oberflächenintegral ausdrücken.

%% Question 27
\begin{frage}\index{Orientierung!einer regulären Hyperfläche}
  \index{regulaere Hyperflaeche@reguläre Hyperfläche} 
  Wie definiert man für eine reguläre Hyperfläche $X \subset \RR^n$ den 
  Begriff der \bold{Orientierung}? 

  Dabei heißt eine Teilmenge $X\subset\RR^n$ reguläre Hyperfläche, wenn eine 
  $(n-1)$-dimensionale $\calli{C}^1$-Untermannigfaltigkeit $M$ existiert, sodass 
  $M$ offen und dicht in $X$ ist und $X\mengeminus M$ eine Nullmenge zur Dimension $n-1$ ist.
\end{frage}

\begin{antwort}
  Den Orientierungsbegriff definiert man 
  mithilfe eines \slanted{Einheitsnormalenfelds}\index{Einheitsnormalenfeld} 
  auf $X$. Darunter versteht man ein stetiges Vektorfeld 
  $\eta\fd X\to\RR^n$ derart, dass für jedes $x\in X\cap M$ der Vektor 
  $\eta(x)$ senkrecht auf dem Tangentialraum 
  $T_x M$ steht und $\n{x}=1$ gilt.

  Eine reguläre Hyperfläche $X$ heißt \slanted{orientierbar}, falls 
  sie ein Einheitsnormalenfeld besitzt. 
  Mit $\eta$ ist stets auch $-\eta$ ein 
  Einheitsnormalenfeld, die Orientierung 
  einer Hyperfläche muss also in jedem speziellen Fall explizit angegeben werden.  

  Das Standardbeispiel einer nicht orientierbaren regulären Hyperfläche 
  im $\RR^3$ ist das \slanted{Möbius-Band}.\AntEnd
\end{antwort}



%% Question 28
\begin{frage}\index{regulaerer Randpunkt@regulärer Randpunkt}
  \index{singulaerer Randpunkt@singulärer Randpunkt}
  Was versteht man unter einem \bold{regulären Randpunkt} einer 
  offenen Teilmenge $G\subset\RR^n$, was unter einem \bold{singulären 
    Randpunkt}? 
\end{frage}

\begin{antwort}
  Ein Punkt $a\in\partial G$ heißt \slanted{regulärer Randpunkt} von $G$, 
  wenn eine Umgebung $U\subset \RR^n$ von $a$ und eine $\calli{C}^1$-Funktion 
  $g\fd U\to\RR$ mit $g'\not=0$ existiert, sodass gilt: 
  $G\cap U = \{ x\in U\sets g(x) < 0 \}$ (\sieheAbbildung\ref{fig:12_reg}).

  \begin{center}
    \includegraphics{mp/12_reg}
    \captionof{figure}{Der Punkt $a$ ist ein regulärer Randpunkt von $G$.}
    \label{fig:12_reg}
  \end{center}

  Ein Punkt aus $\partial G$ heißt \slanted{singulärer Randpunkt}, wenn 
  er nicht regulär ist. 
  \AntEnd
\end{antwort} 

%% Question 29
\begin{frage}\index{Rand!einer Teilmenge des $\RR^n$}\index{glatt berandet}
  Wann heißt eine offene Teilmenge $G\subset\RR^n$ \bold{glatt berandet}, 
  was versteht man unter einem \bold{$\calli{C}^1$-Polyeder}?
\end{frage}

\begin{antwort}
  $G\subset\RR^n$ heißt 
  \begin{itemize}
  \item \slanted{glatt berandet}, wenn jeder Randpunkt von $G$ 
    regulär ist, 
  \item \slanted{$\calli{C}^1$-Polyeder}, wenn die Menge der singulären 
    Randpunkte von $G$ eine $(n-1)$-Nullmenge ist. 
  \end{itemize}
  Es gilt, dass der glatte Rand einer Teilmenge $G\subset\RR^n$ eine 
  reguläre orientierbare $\calli{C}^1$-Hyperfläche ist. 
  \AntEnd  
\end{antwort} 

%% Question 30
\begin{frage}\label{12_flint}\index{Integral!über eine $\calli{C}^1$-Fläche}
  Ist $(M,\eta)$ eine orientierbare reguläre Hyperfläche im $\RR^n$ und 
  $F\fd M \to \RR^n$ ein Vektorfeld. Wann heißt dann 
  $F$ über $M$ \slanted{integrierbar} und wie ist gegebenenfalls 
  das Integral definiert?
\end{frage}

\begin{antwort}
  $F$ heißt über $M$ integrierbar, wenn die Funktion 
  $x\mapsto \langle F(x), n(x) \rangle$ über $M$ integrierbar ist. 
  In diesem Fall ist das Integral von $F$ über $M$ definiert durch 
  \[
  \boxed{
    \int_M F \overrightarrow{\dd S} = 
    \int_M \langle F, \eta \rangle \dd S. 
  }
  \]
  $\overrightarrow{\dd S}=\eta\dd S$ nennt man 
  \slanted{vektorielles Flächenelement}.\index{vektorielles Flächenelement}
  \nomenclature{$\int_{\partial G} F \overrightarrow{\dd S}$}{Integral über den Rand des 
    $\calli{C}^1$-Polyeders $G$} 
  \AntEnd
\end{antwort} 

%% Question 31
\begin{frage}
  Wie lässt sich diese Integraldefinition physikalisch deuten?
\end{frage}

\begin{antwort}
  Man stelle sich $F$ als Geschwindigkeitsfeld\index{Geschwindigkeitsfeld}
  einer stationären Strömung\index{stationäre Strömung} 
  vor. Der Wert $\langle F(x), n(x) \rangle$ ist die Komponente des Vektors 
  $F(x)$ in Richtung der Normalen an $M$ im Punkt $x$, und somit  
  beschreibt $\langle F(x), \eta(x) \rangle \dd S$ die Menge  
  an Flüssigkeit, die pro Zeiteinheit durch das Flächenelement $\dd S$ 
  fließt, folglich $\int_M \langle F, n \rangle \dd S$ 
  die Gesamtmenge der pro Zeiteinheit durch $M$ strömenden Flüssigkeit 
  (\sieheAbbildung\ref{fig:12_physik1}). 
  \AntEnd
  
  \begin{center}
    \includegraphics{mp/12_physik1}
    \captionof{figure}{$\langle F(x), \eta(x) \rangle \dd S$ beschreibt die Menge  
      an Flüssigkeit, die pro Zeiteinheit durch das Flächenelement $\dd S$ fließt.}
    \label{fig:12_physik1}
  \end{center}

\end{antwort}



%% Question 32
\begin{frage}\index{Gausscher Integralsatz@Gauß\sch er Integralsatz}
  \index{Integralsatz!von Gauß}\index{Gauss@\textsc{Gauss}, Carl Friedrich (1777-1855)}
  Wie lautet der \bold{Gauß\sch e Integralsatz} 
  \begin{itemize}[2mm]
  \item[\desc{a}] für ein Kompaktum $A\subset\RR^n$ mit glattem Rand,\\[-3mm]
  \item[\desc{b}] ein beschränktes $\calli{C}^1$-Polyeder?
  \end{itemize}
\end{frage}

\begin{antwort}
  Der Fall $\desc{b}$ ist eine Verallgemeinerung von $\desc{a}$, 
  deswegen genügt es, den Gauß'schen Integralsatz für $\calli{C}^1$-Polyeder 
  zu formulieren. Der Satz lautet in diesem Fall 

  \medskip\noindent
  \satz{Sei $G\subset\RR^n$ ein beschränktes $\calli{C}^1$-Polyeder und 
    $F\fd D\to \RR^n$ mit $\overline{G}\subset D$ ein stetig 
    differenzierbares Vektorfeld. Ist dann $\Div F$ über $G$ und 
    $F$ über $\partial G$ integrierbar, dann gilt 
    \[
    \boxed{ 
      \int_G \Div F \difx = 
      \int_{\partial G} F \overrightarrow{\dd S}.
    } \EndTag\] 
  }
\end{antwort} 

%% Question 33
\begin{frage}
  Können Sie den Gauß\sch en Integralsatz in dem Spezialfall beweisen, 
  dass es sich bei $G$ um einen offenen Quader 
  $Q=\open{a_1,b_1}\times \cdots \times \open{a_n,b_n}$ handelt?
\end{frage}

\begin{antwort}
  Es genügt zu zeigen, dass für alle Komponentenfunktionen $F_k$ 
  von $F$ und alle Komponenten $\eta_k$ des Einheitsnormalenfeldes 
  $\eta$ die Gleichung 
  \[
  \int_{\partial Q} F_k \eta_k \dd S = \int_Q \partial_k F_k \difx
  \]
  gilt. Die Formel im Gauß'schen Integralsatz folgt daraus durch 
  Summation über $k$.  

  Nach einer eventuellen Umnummerierung der Variablen können wir 
  $k=n$ annehmen (das vereinfacht die Notationen im Beweis). 
  Sei $Q'\subset \RR^{n-1}$ der $(n-1)$-dimensionale Quader mit 
  $Q= Q'\times \open{a_n,b_n}$. Die $n$-te Komponente 
  $\eta_n$ des Einheitsnormalenfeldes verschwindet 
  dann auf den Randstücken $\partial_{Q'} \times \open{a,b}$, 
  $F_n \eta_n$ muss also nur über die "`oberen"' und "`unteren"'  
  Randstücke $Q'\times \{ b \} $ und $Q'\times \{ a \}$ integriert 
  werden (\sieheAbbildung\ref{fig:12_quader}). 
  Auf dem oberen gilt $\eta_n(x)=1$ und auf dem unteren 
  $\eta_n(x)=-1$. Mit der Notation $x':=(x_1,\ldots,x_{n-1})$ folgt dann 
  \begin{align*}
    \int_{\partial Q} F_n \eta_n \dd S &= 
    \int_{Q'} F_n (x',b)  \difx' - 
    \int_{Q'} F_n( x',a ) \difx' \\
    &=
    \int_{Q'}\left(\int_a^b\partial_n F_n( x',x_n) \difx_n \right) \difx' 
    = \int_{Q'} \partial_n F_n \difx.
  \end{align*}
  Das beweist den Gauß'schen Integralsatz im Spezialfall eines offenen 
  Quaders $Q$. Für einen Beweis der allgemeinen Version siehe 
  \citep{Koenig}.\AntEnd

  \begin{center}
    \includegraphics{mp/12_quader}
    \captionof{figure}{Zum Beweis des Gauß'schen Integralsatzes.}
    \label{fig:12_quader}
  \end{center}

\end{antwort} 



%% Question 34
\begin{frage}\label{12_quelldichte}\index{Quelldichte}
  Können Sie eine physikalische Interpretation des 
  Gauß\sch en Integralsatzes geben?
\end{frage}

\begin{antwort}
  Man stelle sich $F$ als das stationäre Geschwindigkeitsfeld einer 
  Flüssigkeit vor, die den $\calli{C}^1$-Polyeder 
  $G$ durchströmt (\sieheAbbildung\ref{fig:12_physik}). 
  Dann misst $F \overrightarrow {\dd S}$ die Masse 
  die in einer Zeiteinheit über das Flächenelement ${\dd S}$ strömt. 
  Entsprechend gibt $\int_{\partial G} F\overrightarrow{\dd S}$ 
  die Bilanz der Massen an, die in einer 
  Zeiteinheit ins Innere des Polyeders bzw. aus ihm herausfließen. 
  Diese Bilanz ist (bei einer inkompressiblen Flüssigkeit) gleich der 
  Masse, die im Inneren von $G$ in einer Zeiteinheit in Senken 
  verschwindet bzw. durch Quellen zugeführt wird. Entsprechend ist 
  $\frac{1}{\vol(G)} \int_{\partial G} F \overrightarrow{\dd S}$ ein Maß 
  für die \slanted{mittlere Quelldichte} des Vektorfeldes $F$ in $G$. 

  \begin{center}
    \includegraphics{mp/12_physik}
    \captionof{figure}{Zur physikalischen Interpretation des Gauß'schen 
      Integralsatzes.}
    \label{fig:12_physik}
  \end{center}


  Im Grenzfall, in dem $G$ zu einem Punkt $x$ zusammenschrumpft, 
  entspricht die mittlere Quelldichte von $F$ in $G$ gleich der Divergenz 
  von $F$ in in $x$. Dies sieht man folgendermaßen: Sei $(Q_k)\subset G$ 
  eine Folge abgeschlossener Quader mit $\lim Q_k = x$. Der Gauß'sche 
  Integralsatz impliziert dann  
  \[
  \min_{x\in Q_k} \Div F(x) \le 
  \frac{1}{v(Q_k)} \int_{\partial G} F\overrightarrow{\dd S} \le 
  \max_{x\in Q_k} \Div F(x).
  \] 
  Wegen der Stetigkeit von $\Div F$ auf $G$ muss also  
  \[
  \lim_{k\to\infty} \frac{1}{v(Q_k)}\int_{\partial G} 
  F \overline{\dd S}=\Div x
  \]
  gelten. 
  Damit kann man $\Div F(x)$ als \slanted{Quelldichte} von $F$ im Punkt $x$ 
  interpretieren. Das Volumenintegral über diese "`Dichten"' misst die 
  in einem Zeitintervall in $G$ entstehende Flüssigkeitsmasse. 

  Locker formuliert beinhaltet der Gauß'sche 
  Integralsatz also eine Aussage der Art
  \[
  \Big\{
  \begin{array}{l} \text{Bilanz der Mengen, die über den} \\
    \text{Rand von $G$ ein- und ausfließen} \end{array}
  \Big\}
  = 
  \Big\{
  \begin{array}{l} \text{Bilanz der Mengen, die in $G$} \\
    \text{erzeugt und vernichtet werden} \end{array}
  \Big\} \EndTag
  \] 
\end{antwort} 



%% Question 35
\begin{frage}\index{Volumen!der Sphäre}\index{Volumen!der Einheitskugel}
  Welchen Zusammenhang zwischen dem Volumen der $n$-dimensionalen 
  Einheitskugel $K_1(0)$ im $\RR^n$ 
  und des $(n-1)$-dimensionalen Volumens $\omega_n$ der 
  Sphäre $S^{n-1}$ erhält man durch Anwendung des Gauß'schen Integralsatzes 
  auf das Vektorfeld $F\fd \RR^n\to\RR^n$ mit $x\mapsto x$?
\end{frage}

\begin{antwort}
  $F$ ist auch das Einheitsnormalenfeld der Sphäre $S^{n-1}$. Damit 
  gilt 
  \[
  \int_{S^{n-1}} F\overrightarrow{\dd S} = 
  \int_{S^{n-1}} \langle F, F \rangle \dd S = 
  \int_{S^{n-1}} \n{x}_2 \dd S = 
  \int_{S^{n-1}} 1 \dd S = \omega_n.
  \]
  Auf der anderen Seite erhält man wegen $\Div F=1+\cdots+1=n$ 
  \[
  \int_{K_1(0)} \Div F \difx =  
  \int_{K_1(0)} n \difx = n\kappa_n,
  \]
  mit dem Gauß\sch en Integralsatz folgt also $\omega_n = n \kappa_n$. \AntEnd 
\end{antwort} 



%% Question 36
\begin{frage}\index{Kontinuitaetsgleichung@Kontinuitätsgleichung}
  Wie erhält man aus dem Gauß'schen Integralsatz die 
  \bold{Kontinuitätsgleichung}?
\end{frage}

\begin{antwort}
  Sei $D\subset\RR^n$ offen und 
  $G\subset \RR^n$ ein $\calli{C}^1$-Polyeder 
  mit $\overline{G}\subset D$. 
  Wir betrachten eine $\calli{C}^1$-Abbildung und eine 
  $\calli{C}^1$-Funktion 
  \[
  v \fd R\times D \to \RR^3, \qquad
  \varrho \fd R\times D \to \RR.
  \] 
  Dabei können wir $v$ etwa als das zeitabhängige Geschwindigkeitsfeld 
  einer strömenden Flüssigkeit interpretieren und $\varrho$ als 
  ebenfalls zeitabhängige Funktion der Massendichte auf $D$. 

  Die Änderung des durch das 
  Oberflächenelement $\dd S$ fließenden Flüssigkeitsvolumens zum 
  Zeitpunkt $t$ wird durch $\int_{\partial G} v(t,x) \overrightarrow{\dd S}$ 
  beschrieben, die Gesamtänderung der Masse zum Zeitpunkt $t$ also durch 
  $\int_{\partial G} \varrho(t,x) v(t,x) \overrightarrow{\dd S}$. 

  Wird in $G$ keine Masse erzeugt oder vernichtet, dann ist die zeitliche 
  Änderung der Masse auch durch 
  $-\frac{\dd}{\dift} \int_G \varrho \dd V$ gegeben (das Minuszeichen 
  kommt daher, dass das nach außen weisende Einheitsnormalenfeld zugrunde 
  gelegt wurde). Es gilt also 
  $\int_{\partial G} \varrho v \overrightarrow{\dd S} + 
  \int_G \frac{\partial \varrho}{\partial t} \dd V =0$ 
  (Differenziation unter dem Integralzeichen), 
  und mit dem Gauß\sch en Integralsatz folgt daraus 
  $
  \int_{G} \left( 
    \Div \varrho v + 
    \frac{\partial \varrho}{\partial t} \right) \dd V  =0.
  $

  Da diese Gleichung (unter der Voraussetzung, dass 
  die Gesamtmasse in $D$ erhalten bleibt)
  für beliebige $\calli{C}^1$-Polyeder $G\subset D$ gilt und der 
  Integrand stetig ist, folgt die \slanted{Kontinuitätsgleichung}
  \[
  \boxed{
    \Div \varrho v + 
    \frac{\partial \varrho}{\partial t} =0,
  }\EndTag
  \]
  
\end{antwort} 

%% Question 37
\begin{frage}
  \index{Integralsatz!von Green}
  \index{Green@\textsc{Green}, George (1793-1841)}
  Können Sie die den folgenden (häufig \bold{Satz von Green} genannten) 
  Satz aus dem Gauß\sch en Integralsatz herleiten:

  Ist $G\subset\RR^2$ ein Gebiet mit stückweise glattem Rand und 
  $U\subset\RR^2$ eine offene Menge, die den Abschluss $\overline{D}$ 
  enthält und $F=(f,g)\fd U\to\RR^n$ ein stetig differenzierbares Vektorfeld. 
  Dann ist 
  \[
  \boxed{
    \int_G ( \partial_1 g - \partial_2 f ) \difx_1 \difx_2 = 
    \int_{\partial G} f\difx_1 +g\difx_2,}
  \]
  wobei $\partial D$ so orientiert ist, dass das Gebiet $D$ links vom Rand liegt.
\end{frage}

\begin{antwort}
  Sei $\Phi:=(g,-f)$ das "`um den Winkel $-\pi/2$ gedrehte"' 
  Vektorfeld $F$. Für dieses gilt $\Div \Phi = \partial_1 g - \partial_2 f$ 
  und 
  \[
  \langle \Phi, \eta \rangle = g\eta_1 - g\eta_2 = 
  \langle F, \tau \rangle \quad\text{mit}\quad \tau := (-\eta_2,\eta_1).
  \]
  Für alle $x\in\partial G$ ist also $\tau(x)$ der um den Winkel 
  $\pi/2$ gedrehte Einheitsnormalenvektor $\eta(x)$. 

  \begin{center}
    \includegraphics{mp/12_green}
    \captionof{figure}{Die Menge $G$ liegt "`links"' von der Kurve $\alpha$.}
    \label{fig:12_green}
  \end{center}
  

  Man betrachte nun eine Kurve $\alpha \fd \open{0,1} \to \partial G$, 
  mit $\n{\dot{\alpha}(t)}=1$, die ein glattes Teilstück von 
  $\partial G$ so durchläuft, dass $G$ "`links"' von $\alpha$ liegt, 
  \sieheAbbildung\ref{fig:12_green}. 
  Für jedes $t\in\open{0,1}$ gilt dann 
  $\dot{\alpha}(t)=\tau\big(\alpha(t)\big)$ und folglich für das 
  Kurvenintegral von $F$ längs $\alpha$ 
  \[
  \int_\alpha F = 
  \int_0^1 \langle F\big( \alpha(t) ), 
  \tau\big( \alpha(t) \big) \rangle \dift = 
  \int_0^1 \langle \Phi \big( \alpha(t) ), 
  \eta\big( \alpha(t) \big) \rangle \dift
  . \asttag
  \]
  Diese Gleichung gilt für beliebige Kurven, die die Spur von $\alpha$ 
  in derselben Richtung wie $\alpha$ durchlaufen. Dies sind aber genau 
  diejenigen Kurven, die so verlaufen, dass $G$ "`links"' von ihnen liegt, 
  bzw. deren Einheitstangentialvektor im Punkt $x\in\partial G$ 
  gleich $(-\eta_2(x), \eta(x))$ ist. 
  Ist $\gamma$ eine beliege stückweise reguläre Kurve, die $\partial G$ 
  in diesem Sinn umrundet, dann kann man 
  durch $\int_{\partial G} F:=\int_{\gamma} F$ das Kurvenintegral längs 
  $\partial G$ eindeutig definieren. 

  Mit dieser Vereinbarung lautet {\astref} $
  \int_{\partial G} F = \int_{\partial G} f\difx_1+g\difx_2 = 
  \int_{\partial G} \Phi \overrightarrow{\dd S}.$
  Das hintere Integral ist gleich 
  $\int_G \Div \Phi \difx = \int_G (\partial_1 g - \partial_2 f)\difx_1\difx_2$ 
  nach dem Gauß\sch en Integralsatz.
  \nomenclature{$\int_{\partial G} F$}{Integral des Vektorfeldes $F$ längs des 
    orientierten Randes von $G$}
  \AntEnd
\end{antwort} 

%% Question 38
\begin{frage}\index{k-Form@$k$-Form}
  Was versteht man unter einer \bold{alternierenden $\mathbf{k}$}-Form auf 
  einem reellen Vektorraum $V$?
\end{frage}

\begin{antwort}
  Unter einer alternierenden $k$-Form $\omega$ auf $V$ 
  versteht man eine $k$-fach lineare Abbildung 
  \[ 
  \omega \fd \underbrace{V\times \cdots \times V}_k \to \RR
  \]
  mit der Eigenschaft, dass für linear abhängige Vektoren 
  $v_1,\ldots,v_n$ gilt: $\omega(v_1,\ldots,v_n)=0$. 

  Den Vektorraum der alternierenden $k$-Formen auf $V$ bezeichnet man mit 
  $\Alt^k(V)$. 
  \AntEnd  
\end{antwort}

%% Question 39
\begin{frage}\label{12_zuruck}\index{Zurückholen@einer alternierenden $k$-Form}
  Ist $L\fd V \to W$ eine lineare Abbildung zwischen $\RR$-Vektorräumen 
  $V$ und $W$ und $\omega$ eine alternierende 
  $k$-Form auf $W$, wie ist dann die durch $L$ von $W$ 
  \bold{zurückgeholte $k$-Form} $L^*\omega$ auf $V$ definiert? 
\end{frage}

\nomenclature{$L^*\omega$}{auf den Parameterraum 
  zurückgeholte Differenzialform}
\begin{antwort}
  Die $k$-Form $L^*\omega \in \Alt^k(W)$ ist gegeben durch  
  \[
  L^*\omega (v_1,\ldots,v_n ):=\omega( Lv_1,\ldots,Lv_n).\EndTag
  \]
\end{antwort} 

%% Question 40
\begin{frage}\label{12_zuruckdet}\index{Zurückholen@einer alternierenden $k$-Form}
  Ist $L\fd \RR^n\to \RR^n$ eine lineare Abbildung, die durch die 
  Matrix $A$ beschrieben ist, wie stehen dann 
  $\omega\in \Alt^k(V)$ und die zurückgeholte $k$-Form $L^*\omega$ 
  zueinander in Beziehung?   
\end{frage}

\begin{antwort}
  In diesem Fall gilt 
  \[
  \boxed{ L^* \omega = \det A \cdot \omega. } \EndTag
  \]
\end{antwort} 

%% Question 41
\begin{frage}\index{Dachprodukt}
  Wie ist für $\omega\in\Alt^r(V)$ und $\eta\in\Alt^s(V)$ das 
  \bold{Dachprodukt} oder 
  \bold{äußere Produkt} $\omega\wedge\eta$ definiert?
\end{frage}

\begin{antwort}
  \nomenclature{$\omega \wedge \eta$}{Dachprodukt} 
  Das Dachprodukt ist die durch
  \[
  \omega\wedge\eta (v_1,\ldots,v_{r+s}) := 
  \frac{1}{s!t!} \sum_{\tau\in\mathfrak{S}_{r+s}} \sign \tau 
  \cdot \omega( v_{\tau(1)},\ldots,v_{\tau(r)} )\cdot 
  \eta( v_{\tau(r+1)},\ldots,v_{\tau(r+s)} )
  \]
  gegebene $(k+s)$-Form auf $V$. 
  Dabei bezeichnet $\mathfrak{S}_p$ die symmetrische Gruppe, also die 
  Gruppe der Permutationen der Zahlen $1,\ldots,p$ 
  \AntEnd 
\end{antwort} 

%% Question 42
\begin{frage}\index{Basisvektoren von $\Alt^k(n)$}
  Ist $e_1,\ldots,e_n$ die Standardbasis des $\RR^n$, 
  wie lauten dann die dazugehörigen Basisvektoren von $\Alt^k(n)$?
\end{frage}

\begin{antwort}
  \nomenclature{$\Alt^k(n)$}{Vektorraum der alternierenden $k$-Formen im $\RR^n$}
  Man betrachte die in Frage \ref{12_koorddiff} eingeführten 
  Koordinatendifferenziale $\dd x_i$ ($i=1,\ldots,n$). 
  Bei diesen handelt es sich um $1$-Formen, daher sind alle äußeren 
  Produkte $\dd x_{i_1}\wedge \cdots \wedge \dd x_{i_k}$ 
  mit $k$ Faktoren Elemente aus $\Alt^k(\RR^n)$. 
  Wegen $\dd x_i(e_j)=\delta_{ij}$ folgt (Induktion nach $k$):
  \[    
  \dd x_{i_1}\wedge \cdots \wedge 
  \dd x_{i_k}(e_{j_1},\ldots,e_{j_k}) = 
  \left\{ \begin{array}{ll} \sign \tau, & \text{falls 
        $\sigma( \{i_1,\ldots,i_k \} )=\{ j_1,\ldots,j_k \}$} \\
      & \text{für ein $\sigma\in\mathfrak{S}_k$} \\
      0 & \text{sonst}. \end{array}\right.
  \]
  Ähnlich wie in der linearen Algebra bezüglich des 
  Dualraums zeigt man, dass die $\tbinom{n}{k}$ 
  $k$-Formen $\dd x_{i_1} \wedge \cdots \wedge \dd x_{i_k}$ mit 
  $i_1 < i_2 < \cdots < i_n$ eine Basis von $\Alt^k(\RR^n)$ bilden. 
  Beispielsweise ist 
  \[
  \begin{array}{lp{6mm}l}
    \dd x_1,\; \dd x_2, \; \dd x_3 & & \text{eine Basis von $\Alt^1(\RR^3)$,} \\
    \dd x_1 \wedge \dd x_2, \; \dd x_1 \wedge 
    \dd x_3, \; \dd x_2 \wedge \dd x_3 & & \text{eine Basis von $\Alt^2(\RR^3)$,} \\
    \dd x_1 \wedge \dd x_2 \wedge \dd x_3 & & 
    \text{eine Basis von $\Alt^3(\RR^3)$.} 
  \end{array}
  \] 
  Jede $k$-Form $\omega$ auf $\RR^n$ besitzt damit genau eine 
  Darstellung 
  \[
  \omega = \sum_{i_1 < \cdots < i_k } a_{i_1 \cdots i_k} 
  \dd x_{i_1} \wedge \cdots \wedge \dd x_{i_k},
  \]
  mit 
  \[
  a_{i_1 \cdots i_k} = \omega( e_{i_1}, \ldots, e_{i_k} ). \EndTag
  \]
\end{antwort} 

%% Question 43
\begin{frage}\index{Differenzialform}
  Was ist eine 
  \bold{Differenzialform vom Grad $\mathbf{k}$} 
  auf einer offenen Teilmenge $U\subset\RR^n$?
\end{frage}

\begin{antwort}
  Eine \slanted{Differenzialform vom Grad $\mathbf{k}$} oder 
  kurz \slanted{$k$-Form} auf $U$ ist eine Abbildung, die jedem 
  $x\in U$ eine alternierende $k$-Form $\omega(x)$ zuordnet, also 
  eine Abbildung $U \to \Alt^k(\RR^n)$. Ist diese 
  Abbildung differenzierbar, so heißt $\omega$ differenzierbar. Der 
  Raum der differenzierbaren $k$-Formen auf $U$ wird mit 
  $\Omega^k (U)$ bezeichnet.
  \nomenclature{$\Omega^k(U)$}{Raum der differenzierbaren $k$-Formen auf $U$}
  \AntEnd
\end{antwort} 

%% Question 44
\begin{frage}\index{Null-Form}
  Welches sind die $0$-Formen auf $\RR^n$?
\end{frage}

\begin{antwort}
  Eine Form $\omega\in\Alt^0(\RR^n)$ ordnet jedem $x\in\RR^n$ eine 
  Abbildung $\omega(x) \fd \RR^0 \to \RR$ zu, also einfach ein Element 
  aus $\RR$. Die $0$-Formen sind 
  somit gerade die Funktionen $\RR^n\to\RR$. \AntEnd 
\end{antwort} 

%% Question 45
\begin{frage}\label{zuruck}\index{Zurückholen!einer Differenzialform}
  Seien $V\subset\RR^m$ und $U\subset\RR^n$ offen und $\gamma \fd V \to U$ 
  eine $\calli{C}^1$-Abbildung. Weiter sei $\omega$ eine Differenzialform 
  vom Grad $k$ auf $U$. Wie ist dann die 
  mittels $\gamma$ \bold{zurückgeholte} Differenzialform 
  $\gamma^*\omega \in \Alt^k(\RR^m)$ definiert?
\end{frage}

\begin{antwort}
  Die Definition der auf $V$ \slanted{zurückgeholte} 
  Differenzialform $\gamma^*\omega$ wird über das Differenzial 
  $\dd \gamma$ mit der linearen Version aus Frage \ref{12_zuruck} 
  definiert: 
  \[
  \big(\gamma^*\omega \big)(x):=  
  \big( \dd \gamma (a) \big)^* \omega\big( \gamma(x) \big).{\asttag}
  \]
  Für Vektoren $v_1,\ldots,v_k \in \RR^m$ gilt also 
  \[
  \boxed{\big( \gamma^* \omega )_x (v_1,\ldots,v_k)=
    \omega_{\gamma(x)} \big( \dd\gamma (x) v_1, \ldots, \dd\gamma (x) v_k \big).}
  \EndTag
  \]
\end{antwort} 


%% Question 46
\begin{frage}\index{Null-Form}
  Was bedeutet diese Definition für eine $0$-Form, 
  also eine Funktion $g\fd V\to U$?
\end{frage} 

\begin{antwort}
  Für eine Funktion bedeutet das gerade $\gamma^* f=f\circ \gamma$. 
  Die Definition stimmt in diesem Fall also mit der 
  üblichen Methode überein, "`woanders"' definierte 
  Funktionen via einer $\calli{C}^1$-Abbildung 
  "`zurückzuholen"'. \AntEnd
\end{antwort} 

%% Question 47
\begin{frage}
  Ist speziell $\gamma\fd \RR^n\to\RR^n$, wie lautet dann die Gleichung 
  {\astref}? Woran erinnert dieses Transformationsverhalten? 
\end{frage}

\begin{antwort}
  Wegen dem Zusammenhang aus Frage \ref{12_zuruckdet} gilt 
  in diesem Fall
  \[
  \big( \gamma^*\omega \big) (x)=
  \det \calli{J}(x)\cdot \omega\big( \gamma(x) \big).{\asttag}
  \]
  Dieser Zusammenhang erinnert an die Transformationsformel. 
  Diese stellt ja bei der Integration bezüglich verschiedener 
  Koordinaten den Faktor $\big|\det \calli{x}\big|$ als Maß 
  für die dabei auftretende infinitesimale Volumenverzerrung 
  in Rechnung. Die Gleichung {\astref} deutet darauf hin, dass 
  Differenzialformen diese bei der Integration zu berücksichtigende 
  Invarianz gegenüber Parametertransformationen (bis auf das Vorzeichen) 
  bereits von Natur aus besitzen. Diese Eigenschaft 
  qualifiziert sie als die natürlichen Integranden 
  bei einer Integration über Mannigfaltigkeiten. 
  \AntEnd 
\end{antwort} 

%% Question 48
\begin{frage}\label{12_integral1}\index{Integral!einer $n$-Form}
  Wie definiert man das \bold{Integral} 
  einer $n$-Form über eine Menge im $\RR^n$?
\end{frage}

\begin{antwort}
  Eine $n$-Form 
  $\omega=a \difx_1\wedge\cdots\wedge \difx_n$ 
  ist genau dann integrierbar über $U\subset\RR^n$, wenn ihre 
  Koeffizientenfunktion $a\fd \RR^n\to\RR$ über $U$ 
  integrierbar ist. In diesem Fall definiert man das 
  \slanted{Integral von $\omega$ über $U$} durch
  \[
  \int_U \omega := \int_U a(x) \difx.\EndTag
  \]
\end{antwort} 

%% Question 49
\begin{frage}\index{Differenzalform}
  Wie definiert man eine Differenzialform 
  auf einer $\calli{C}^1$-Untermannigfaltigkeit?
\end{frage}

\begin{antwort}
  Eine Differenzialform vom Grad $k$ auf einer differenzierbaren 
  Untermannigfaltigkeit $M$ ist eine Abbildung, die jedem 
  $x\in M$ eine $k$-fach alternierende Abbildung auf dem Tangentialraum 
  $T_xM$ zuordnet. \AntEnd
\end{antwort} 

%% Question 50
\begin{frage}\index{Orientierung!einer Mannigfaltigkeit}
  Können Sie die kurz erläutern, wie man für Mannigfaltigkeit 
  $M$ den Begriff der 
  \bold{Orientierung} definiert?
\end{frage}

\begin{antwort}
  Mithilfe der lokalen Parameterdarstellungen führt man den 
  Orientierungsbegriff für $M$ auf denjenigen des $\RR^k$ zurück. 
  Der $\RR^k$ besitzt genau zwei Orientierungen, die definiert sind 
  als die \slanted{Wegzusammenhangskomponenten} der Menge $\calli{B}(\RR^k)$ 
  der geordneten Basen von $\RR^k$. Das heißt, zwei geordnete Basen 
  $B=(b_1,\ldots,b_k)$ und $B'=(b_1',\ldots,b_k')$ besitzen genau dann 
  dieselbe Orientierung, wenn sie stetig ineinander deformiert werden können, 
  wenn also eine stetige Kurve $\beta\fd [0,1] \to \calli{B}(\RR^k)$ 
  mit $\beta(0)=B$ und $\beta(1)=B'$ existiert. Das ist gleichbedeutend 
  damit, dass der Automorphismus $\RR^k \to \RR^k$, der $B$ in $B'$ überführt, 
  eine positive Determinante besitzt. Als die \slanted{positive} Orientierung 
  von $\RR^k$ definiert man diejenige, die die Basis $(e_1,\ldots,e_k)$ 
  enthält. 

  Für jeden Punkt $a\in M$ besitzt damit auch der Tangentialraum 
  $T_a M$ genau zwei Orientierungen. 
  Ist $\alpha \fd V\to U$ eine Einbettung mit 
  $\alpha( v_0 ) = a\in U \subset M$ und $V\subset\RR^k$, 
  dann ordnet der Isomorphismus $\dd \alpha (v_0)  \RR^k \to T_a M$ 
  der positiven Orientierung von $\RR^k$ genau eine 
  der beiden Orientierungen von $T_a M$ zu. 

  Eine Untermannigfaltigkeit $M$ heißt nun 
  \slanted{orientierbar}, wenn sich dem Tangentialraum 
  $T_a M$ für jedes $a\in M$ 
  je eine der beiden Orientierungen \slanted{in stetiger Weise} 
  zuordnen lässt. Das heißt, dass für jedes $v\in V$ 
  der Isomorphismus 
  \[
  \dd \alpha (v) \fd \RR^k \to T_{\alpha(v)} M 
  \]
  der positiven Orientierung von $\RR^k$ 
  die vorgeschriebene Orientierung 
  von $T_{\alpha(v)}$ zuordnet. 

  \index{Mobiusband@Möbiusband}
  \index{Mobius@\textsc{Möbius}, August Ferdinand (1790-1868)}

  \begin{center}
    \includegraphics{mp/12_moebius}
    \captionof{figure}{Das Möbiusband ist nicht orientierbar.}
    \label{fig:12_moebius}
  \end{center}
  

  Demnach ist zum Beispiel das Möbius-Band, das in Abbildung~\ref{fig:12_moebius} dargestellt ist, 
  \slanted{nicht} orientierbar, da das plötzliche "`Umschlagen"' der Orientierung an den "`Nahtstelle"' 
  jedenfalls nicht mit der Forderung in Einklang zu bringen ist, dass 
  in einer Umgebung des Urbilds eines Punktes der Nahtstelle das 
  Differenzial $\dd \alpha$ stetig \slanted{und} orientierungstreu ist. 
  \AntEnd
\end{antwort} 

%% Question 51
\begin{frage}
  \label{12_integral2}
  \index{Integral!einer Differenzialform über ein Kartengebiet}
  \nomenclature{$int_U \omega$}{Integral einer Differenzialform über 
    ein Kartengebiet}
  Sei $M$ eine $k$-dimensionale differenzierbare 
  Untermannigfaltigkeit, $U\subset M$ ein Kartengebiet 
  und $\alpha \fd V\to U$ eine orientierungstreue 
  lokale Parametrisierung von $U$. Ferner sei $\omega$ eine 
  $k$-Form auf $U$. Wie definiert man dann das Integral 
  von $\omega$ über $U$? Welche Methode benutzt man, um ausgehend davon das 
  Integral über die gesamte Untermannigfaltigkeit? 
\end{frage}

\begin{antwort}
  $\omega$ heißt bezüglich über $U$ 
  \slanted{integrierbar}, wenn die auf den Parameterraum 
  $V$ zurückgeholte Differenzialform 
  $\alpha^* \omega = a \difx_1 \wedge \cdots \wedge \difx_n$ 
  dort integrierbar im Sinne von Frage 
  \ref{12_integral1} ist. In diesem Fall ist das 
  \slanted{Integral von $\omega$ über $U$} definiert durch: 
  \[
  \boxed{
    \int_U \omega := \int_V \alpha^* \omega = \int_V a(x)\difx.
  }  \asttag
  \]
  Das Integral über die gesamte Untermannigfaltigkeit wird mithilfe einer 
  Zerlegung der Eins auf die Integration über Kartengebiete zurückgeführt, 
  also mit derselben Methode, die in Antwort \index{Zerlegung der Eins} 
  \ref{ZerlegungderEins} beschrieben wurde. \AntEnd
\end{antwort}

%% Question 52
\begin{frage}\index{Integral!einer $k$-Form über ein Kartengebiet}
  Können Sie der Definition {\astref} aus Frage \ref{12_integral2} einen 
  anschaulichen Sinn geben?
\end{frage}

\begin{antwort}
  \index{Differenzialform}
  \index{Integral!einer Differenzialform über ein Kartengebiet}
  Um eine anschauliche Vorstellung von Differenzialformen zu gewinnen, 
  sieht man deren Aufgabe am besten darin, so etwas wie 
  "`Dichteverteilungen"' im $\RR^n$ zu beschreiben -- mit der 
  Besonderheit allerdings, dass diese aufgrund des alternierenden 
  Verhaltens der Differenzialformen mit einen "'Richtungssinn"' 
  ausgestattet sind. 

  Im Fall einer $(n-1)$-Form hat man dafür das adäquate Bild 
  einer Strömungsdichte, und an diesem speziellen Bild wollen wir die 
  folgenden Überlegungen ausrichten. Sei $M$ eine 
  $k$-dimensionale Untermannigfaltigkeit des $\RR^n$ und in einer Umgebung 
  von $M$ sei eine $k$-Form $\omega$ gegeben, die man sich etwa  
  als Geschwindigkeitsfeld einer durch $M$ strömenden Flüssigkeit 
  vorstellen kann. Die Integration von $\omega$ über $M$ zielt 
  natürlich darauf ab, die Strömungsbilanz zu erfassen, also die Menge an 
  Flüssigkeit, die pro Zeiteinheit durch $M$ hindurchfließt.   

  \begin{center}
    \includegraphics{mp/12_spat}
    \captionof{figure}{In einer infinitesimalen Umgebung von $a$ 
      lässt sich die Differentialform als konstant auffassen und durch ihre 
      Wirkung auf die $k$-Spate beschreiben.}
    \label{fig:12_spat}
  \end{center}
  
  Gemäß dem üblichen Vorgehen der Analysis betrachtet man 
  zu dieser Situation zunächst das lokale Modell in einer 
  Umgebung $U$ eines Punktes $a\in M$. Im Bezug auf diese Umgebung 
  lässt sich die Differenzialform (die Strömungsdichte) als annähernd 
  konstant betrachten, und augenscheinlich lässt sie sich  
  durch ihre Wirkung auf die $k$-Spate charakterisieren, 
  die von je $k$ Vektoren, die man sich am Punkt $a$ angeheftet denkt, 
  aufgespannt werden, \sieheAbbildung\ref{fig:12_spat}. 
  Die in einer kleinen Umgebung von $a$ durch 
  $M$ strömende Flüssigkeitsmenge wird damit annähernd durch eine 
  alternierende $k$-Form auf dem Tangentialraum $T_a M$ beschrieben.    

  
  \begin{center}
    \includegraphics{mp/12_stokes}
    \captionof{figure}{Das Bild des Quaders $\sigma_p$ unter dem 
      Differenzial $\dd \alpha$ ist ein $k$-Spat im 
      Tangentialraum $T_{\alpha(p)} U$.}
    \label{fig:12_stokes}
  \end{center}

  Man betrachte nun ein Kartengebiet $U$ von $M$ und dazu eine 
  lokale Parametrisierung $\alpha \fd V \to U$ mit 
  $V\subset \RR^k$. In dem Parameterbereich $V$ wähle man einen Quader 
  $Q'=[a_1,b_1]\times\cdots \times [a_k,b_k]$, 
  und unterteile diesen im Sinne der Abbildung~\ref{fig:12_stokes} 
  in kleine Teilquader 
  $\sigma_p := \prod_{i=1}^k [ p, p+\Delta x_i \cdot e_i ]$. 
  Das Bild von $Q'$ unter $\alpha$ bezeichnen wir mit $Q$, und 
  die Menge aller Zerlegungspunkte $p\in Q'$ soll \slanted{Gitter} heißen. 

  \noindent%
  Das Bild des Quaders $\sigma_p$ unter dem 
  Differenzial $\dd \alpha$ ist dann der $k$-Spat im 
  Tangentialraum $T_{\alpha(p)} U$, der durch die Vektoren    
  \[
  \Delta x_1 \partial_1 \alpha(p), \ldots, \Delta x_k \partial \alpha(p)
  \]
  aufgespannt wird. Die Differenzialform $\omega$ ordnet diesem Spat die Zahl  
  \[
  \omega_{\alpha(p)} 
  \big( \Delta x_1 \partial_1\alpha(p),\ldots, 
  \Delta x_n \partial_k \alpha(p) \big) = 
  \omega_{\alpha(p)} 
  \big(\partial_1\alpha(p),\ldots, 
  \partial_n\alpha(p) \big)  \Delta x_1 \cdots \Delta x_k
  \]
  zu, die sich als lineare Approximation an die durch die Masche 
  $\alpha\big( \sigma_p \big)$ strömende Menge verstehen lässt, also 
  als lineare Approximation an den Wert $\int_{\alpha(\sigma_p)} 
  \omega$. Natürlich soll 
  \[
  \int_Q \omega = \sum_{p \in \text{Gitter} } \int_{\alpha(\sigma_p)} \omega 
  \]
  gelten. Die Gesamtbilanz der insgesamt durch $Q$ strömenden Menge 
  wird damit annähernd beschrieben durch
  \[
  \sum_{p\in \text{Gitter}} \omega_{\alpha(p)}  
  \left( \partial_1 \alpha(p), \ldots, \partial_k \alpha(p) \right) 
  \Delta x_1 \cdots  \Delta x_k.
  \]
  Nach Frage \ref{zuruck} ist 
  \[
  \omega_{\alpha(p)}  
  \big( \partial_1 \alpha(p), \ldots, \partial_n \alpha(p) \big) = 
  \big( \alpha^* \omega )_p ( e_1,\ldots, e_n ) = : a(x),
  \]
  wobei $a(x)$ die Koeffezientenfunktion der 
  aus $\RR^k$ zurückgeholten $k$-Form 
  $\big( \alpha^* \omega )_p= a(x) \dd x_1 \wedge \cdots \wedge \dd x_k$ ist. 
  Insgesamt erhält man als lineare Annäherung an $\int_Q \omega$ also 
  \[
  \sum_{p\in\text{Gitter}} a(p) \Delta x_1\cdots \Delta x_n.
  \]
  Wählt man nun eine Folge immer feiner werdender Rasterungen 
  von $Q'$, sodass die Kantenlängen $\Delta x_i$ gegen $0$ konvergieren, 
  dann gelangt man auf diesem Wege zu der Formel
  \[
  \int_\alpha \omega = \int_V \alpha^* \omega = \int_V a(x) \difx,
  \]
  die genau der Definition des Integrals aus Frage \ref{12_integral2} 
  entspricht. 
  \AntEnd   
\end{antwort} 

%% Question 53
\begin{frage}\index{Differenzial!einer $k$-Form}\index{außere Ableitung@äußere Ableitung}
  \index{Ableitung!einer Differenzialform}
  \index{Cartan-Ableitung}
  \index{Cartan@\textsc{Cartan}, Elie Joseph (1869-1951)}
  Was für ein Objekt ist das \bold{Differenzial} oder die 
  \bold{äußere Ableitung} $\dd \omega$ einer 
  differenzierbaren $k$-Form $\omega$? 
  Kennen Sie ein elementares Beispiel?  
\end{frage} 

\begin{antwort}
  Ist $\omega$ eine differenzierbaren $k$-Form, dann 
  ist $\dd \omega$ eine $(k+1)$-Form. 

  \nomenclature{$\dd\omega$}{äußere Ableitung einer Differenzialform}
  Ein einfaches Beispiel einer äußeren Ableitung ist das 
  Differenzial $\dd f$ einer differenzierbaren Funktion $f$. 
  $f$ ist eine $0$-Form, $\dd f$ eine $1$-Form.
  \AntEnd 
\end{antwort} 

%% Question 54
\begin{frage}
  Durch welche Eigenschaften ist die Abbildung 
  \[
  \dd \fd \Omega^k(\RR^n) \to \Omega^{k+1}(\RR^n), \qquad 
  \omega \mapsto \dd \omega, \qquad k=0,1,2,3,\ldots
  \]
  eindeutig bestimmt? Welche Rechenregel 
  erhält man für die Ableitung einer Differenzialform 
  \[
  \omega = 
  \sum_{i_1 <\cdots< i_k} a_{i_1\ldots i_k} \dd x_{i_1} \wedge 
  \cdots \wedge \dd x_{i_k}
  \text{?}
  \]
\end{frage}

\begin{antwort}
  Die Abbildung $\dd$ ist durch die folgenden 
  Eigenschaften eindeutig bestimmt:
  \satz{\setlength{\labelsep}{5mm}
    \begin{enumerate}
    \item[\desc{i}] $\dd$ ist linear: $\dd( \omega_1 + \omega_2 )
      =\dd \omega_1 + \dd \omega_2$.\\[-3.5mm]
    \item[\desc{ii}] Für eine Funktion $f$, also im Fall $k=0$, ist 
      $\dd f$ gleich dem Differenzial von $f$: 
      $\dd f = \partial_1 \dd x_1 + \cdots + \partial_n \dd x_n$.\\[-3.5mm] 
    \item[\desc{iii}] Es gilt die Produktregel: $
      \dd( \omega \wedge \eta )=\dd \omega \wedge \eta +(-1)^k \omega 
      \wedge \dd \eta$.\\[-3.5mm]
    \item[\desc{iv}] $\dd$ hat die Komplexeigenschaft: $
      \dd^2 \omega := (\dd \circ \dd) \omega = 0.$ \AntEnd
    \end{enumerate}} 

  Mit diesen Eigenschaften gilt die Regel
  \[
  \boxed{
    \dd \left( \sum_{i_1 <\cdots< i_k} a_{i_1\ldots i_k} \dd x_{i_1} \wedge 
      \cdots \wedge \dd x_{i_k} \right) = 
    \sum_{i_1 <\cdots< i_k} \dd a_{i_1\ldots i_k} \wedge 
    \dd x_{i_1} \wedge 
    \cdots \wedge \dd x_{i_k}  }
  \]
  Für eine stetig differenzierbare $1$-Form 
  $\omega = \sum_{i=1}^n a_i x_i $ erhält man zum Beispiel 
  \[
  \dd \omega = 
  \sum_{i=1} \dd a_i \wedge \dd x_i = 
  \sum_{i=1}^n \left( \sum_{k=1}^n \partial_k a_i \dd x_k \right) \dd x_i 
  = 
  \sum_{i<k} (\partial_i a_k - \partial_k a_i ) \dd x_i \wedge \dd x_k.
  \] 

  \begin{center}
    \includegraphics{mp/12_differenzial}
    \captionof{figure}{Zur Interpretation der Cartan'schen Ableitung.}
    \label{fig:12_differenzial}
  \end{center}
  
  Intuitiv kann man die Cartan'sche Ableitung etwa folgendermaßen 
  interpretieren. Man betrachte eine $k$-dimensionale 
  "`Masche"' $S$ im $\RR^k$, also das Bild eines Quaders 
  $Q\subset \RR^n$ unter einer Immersion $\RR^n \to \RR^k$. 
  Die Masche wird begrenzt von \index{Masche}
  $2k$ Seitenmaschen $s_1,s_2,\ldots, s_{2k}$ 
  der Dimension $(k-1)$, deren 
  Orientierung durch diejenige von $S$ vorgegeben ist. 
  Eine $(k-1)$-Form $\omega$  
  wirkt auf jede dieser Seitenmaschen, ordnet diesen 
  jeweils eine reelle Zahl $\int_{s_i} \omega$ zu, 
  \sieheAbbildung\ref{fig:12_differenzial}. Der Durchfluss 
  durch die Masche ist also gegeben durch
  \[
  \sum_{i=1}^{2k} \int_{s_i} \omega = 
  \int_{\partial S} \omega.
  \]
  Die äußere Ableitung $\dd \omega$ ist nun gerade die $k$-Form, die auf 
  die Masche selbst so wirkt wie die $(k-1)$-Form $\omega$ auf die $2n$ 
  Seitenmaschen, für die also 
  \[
  \int_S \dd \omega = \int_{\partial S} \omega
  \]
  gilt. Im Prinzip ist das schon der Satz von Stokes \index{Satz!von Stokes} 
  für Maschen \AntEnd
\end{antwort}

%% Question 55
\begin{frage}\index{Rand!einer Mannigfaltigkeit}
  \nomenclature{$\partial M$}{Rand der Mannigfaltigkeit $M$}
  Können Sie (informal und ohne Beweise) erklären, 
  wie man den Begriff der \bold{berandeten Mannigfaltigkeit} 
  einführen kann? 
\end{frage}

\begin{antwort}
  Um den Begriff zu definieren, geht man von dem lokalen Modell 
  einer $k$-dimensionalen berandeten Mannigfaltigkeit aus, nämlich 
  dem Halbraum   
  \[
  \RR^k_- := \{ (x_1, \ldots, x_n ) \in \RR^k \sets x_1 \le 0 \}. 
  \]
  Der Rand $\partial \RR^k_-$ dieses Halbraums ist dann die Menge 
  aller Vektoren aus $\RR^k$, deren erste Komponente gleich null ist. 
  Als Parameterbereiche für berandete Mannigfaltigkeiten kommen nun 
  die offenen Mengen $V \subset \RR^k_-$ zum Einsatz. 
  Solche Mengen sind im Allgemeinen nicht offen in $\RR^k$, sondern 
  nur dann, wenn $V \cap \partial \RR^k_- = \emptyset$ gilt. 
  Als \slanted{Rand} von $V$ definiert man 
  $\partial V := \partial\RR^k_- \cap V$. (Man beachte, dass diese 
  Randdefinition  nichts zu tun hat mit dem \slanted{topologischen} 
  Rand von $V$.)  

  Damit $V$ als Parameterbereich einer Einbettung 
  $\alpha$ fungieren kann, muss noch geklärt werden, wie die 
  Differenzierbarkeit von $\alpha$ in den Randpunkten zu verstehen ist. 
  Dazu definiert man: $\alpha$ ist 
  differenzierbar in einem Randpunkt $b\in V$, wenn $\alpha$ 
  eine stetig differenzierbare Fortsetzung auf eine Umgebung 
  $V' \subset \RR^k$ von $b$ besitzt. 
  Die Fortsetzung ist dann zwar auf $V'\mengeminus V$ 
  nicht eindeutig bestimmt, wohl aber das Differenzial $\dd \alpha( b )$.  

  Eine $k$-dimensionale Mannigfaltigkeit heißt nun \slanted{glatt berandet}, 
  wenn es zu jedem Punkt $a\in M$ eine Umgebung $U\subset M$ 
  und eine Einbettung $\alpha \fd V \to U$ von einer in 
  $\RR^k_-$ offenen Menge $V$ gibt, \sieheAbbildung\ref{fig:12_rand1}. 
  
  \begin{center}
    \includegraphics{mp/12_rand1}
    \captionof{figure}{Eine glatte berandete Mannigfaltigkeit.}
    \label{fig:12_rand1}
  \end{center}
  
  Der Rand $\partial M$ ist 
  dann gegeben durch  $
  \alpha( \partial \RR^k_- \cap V ) = \partial M \cap U$, 
  und diese Festlegung ist \slanted{unabhängig} von der Einbettung. 
  Ist nämlich $\beta \fd V' \to U$ mit $V' \in V$ eine weitere 
  Einbettung, so überführt der Übergangsdiffeomorphismus 
  $V \to V'$ die Menge $\partial \RR^k_- \cap V$ in die Menge
  $\partial \RR^k_- \cap V'$.  

  Mit der Einschränkung $\alpha | \partial \RR^k_- \cap V$ wird 
  $\partial M$ damit selbst zu einer Untermannigfaltigkeit. \AntEnd
\end{antwort}

%% Question 56
\begin{frage}\index{Orientierung!des Randes einer orientierbaren Mannigfaltigkeit}
  Inwiefern induziert die Orientierung auf einer 
  berandeten Mannigfaltigkeit $M$ eine Orientierung des Randes 
  $\partial M$?
\end{frage}

\begin{antwort}
  Für jeden Punkt $a\in \partial M$ ist der Tangentialraum 
  $T_a \partial M$ ein Unterraum des Tangentialraums $T_a M$, und 
  für jede Einbettung $\alpha$ ist 
  \[
  \dd \alpha ( \partial \RR^k_- \cap V ) = T_a \partial M. 
  \]
  Bezeichnet 
  $\RR^k_+ = \{ \sum_{i=1}^k \lambda_i e_i \sets \lambda_1 > 0 \}$, 
  dann ist der \slanted{Außenraum}  
  \[
  T^+ _a M := \dd \alpha \RR^k_+ 
  \]
  unabhängig von der Einbettung wohldefiniert. 
  Ist $(v_2,\ldots, v_k)$ eine Basis von $T_ \partial M$, 
  dann gehören für je zwei in den Außenraum $T^+ _a M$ 
  weisende Vektoren $v_1$ und $v_1'$ die geordneten Basen 
  $( v_1, v_2, \ldots, v_k )$ und 
  $( v_1', v_2, \ldots, v_k )$ von $T_a M$ zur selben Orientierung 
  von $T_a M$. Aufgrund dieser Tatsache ist der Rand von $M$ 
  ebenfalls orientierbar. Die Randorientierung legt man durch die 
  Konvention fest, dass eine Basis $(v_2,\ldots, v_k)$ von 
  $T_a \partial M$ genau dann zur ausgezeichneten Orientierung von 
  $T_a\partial M$ gehören soll, wenn für jeden in 
  den Außenraum weisenden \index{Orientierungskonvention}
  Vektor $v_1$ der Vektor $(v_1,v_2,\ldots,v_k)$ zur ausgezeichneten 
  Orientierung von $T_a M$ gehört. \AntEnd

  \begin{center}
    \includegraphics{mp/12_rand}
    \captionof{figure}{Zur Orientierung des Randes 
      einer berandeten orientierbaren Mannigfaltigkeit.}
    \label{fig:12_rand}
  \end{center}
\end{antwort}

%% Question 57
\begin{frage}\index{Stokes@\textsc{Stokes}, George Gabriel (1819-1903)}
  \index{Integralsatz!von Stokes}\index{Stokesscher@Sokes\sch er Integralsatz}
  \index{Differenzialformenkalkül}
  Kennen Sie eine Formulierung des Stokes\sch en Integralsatzes in der 
  Sprache des Differenzialformenkalküls?
\end{frage}

\begin{antwort}
  \satz{Ist $U\subset\RR^n$ offen und $M\subset U$ eine 
    glatt berandete orientierbare Mannigfaltigkeit der Dimension $k\ge 2$ und 
    $\omega$ eine stetig 
    differenzierbare $k-1$-Form auf $U$, dann gilt für jedes Kompaktum 
    $G\subset M$ mit glattem Rand:
    \[
    \boxed{ \int_G \dd \omega = \int_{\partial G} \omega.}
    \]
    Dabei besitzt der Rand die durch die Orientierung von $M$ induzierte 
    Orientierung.
  }

  \medskip\noindent%
  Man beweist den Satz schrittweise unter allgemeiner werdenden Voraussetzungen. 
  Zuerst zeigt man ihn für den Halbraum und eine 
  Differenzialform mit kompaktem Träger, erweitert dieses Ergebnis dann auf 
  allgemeine Untermannigfaltigkeiten und Differenzialformen, deren 
  Träger in einem Kartengebiet liegt. Den allgemeinen Fall führt man dann 
  auf diesen mithilfe einer Zerlegung der Eins zurück. 
  \AntEnd
\end{antwort} 

%% Question 58
\begin{frage}\label{tausend}\index{Vektoranalysis}\index{Linienelement}
  Können Sie den Differenzialformenkalkül in die 
  klassische Sprache der Vektoranalysis zurückübersetzen und 
  die klassische Version des Satzes von Stokes formulieren?
\end{frage}  

\begin{antwort}
  Die klassische Vektoranalysis handelt von Vektorfeldern 
  im $\RR^3$ (oder $\RR^2$) und nicht von Differenzialformen. 
  Die "`Übersetzung"' eines Vektorfeldes in eine Differenzialform 
  geschieht mithilfe des \slanted{vektoriellen Linienelements} 
  \index{Linienelement}\index{Flächenelement}\index{Volumenelement}
  $\overrightarrow{ \dd {s} }$, des \slanted{vektoriellen Flächenelements} 
  $\overrightarrow{ \dd {S} }$ sowie des Volumenelements $\dd V$, 
  die schon an anderer Stelle rein symbolisch verwendet wurden. 
  Diese lassen sich auch präzise als (vektorwertige) Formen im 
  $\RR^3$ einführen, nämlich durch die Festsetzungen
  \[
  \overrightarrow{\dd s} := \begin{pmatrix} 
    \dd x_1 \\ \dd x_2 \\ \dd x_3 \end{pmatrix},
  \qquad 
  \overrightarrow{ \dd {S} } := \begin{pmatrix} 
    \dd x_2 \wedge \dd x_3 \\
    \dd x_3 \wedge \dd x_1 \\
    \dd x_1 \wedge \dd x_2
  \end{pmatrix}, \qquad
  \dd V := \dd x_1 \wedge \dd x_2 \wedge \dd x_3.
  \]
  Damit ist $\overrightarrow{ \dd s }$ also eine $\RR^3$-wertige 
  $1$-Form, $\overrightarrow{ \dd S }$ eine $\RR^3$-wertige $2$-Form 
  und $\dd V$ eine normale $3$-Form in $\RR^3$.
  \nomenclature{$\overrightarrow{\dd S}$}{vektorielles Flächenelement}
  \nomenclature{$\overrightarrow{\dd s}$}{vektorielles Linienelement}
  \nomenclature{${\dd V}$}{Volumenelement}

  Die Bedeutung dieser Formen wird durch ihre geometrischen 
  Abbildungseigenschaften verständlich. Für jedes $x\in\RR^3$ ist nämlich 
  \begin{align*}
    \overrightarrow{ \dd s }_x \fd \RR^3 &\to \RR^3 & \text{die Identität}\\
    \overrightarrow{ \dd S }_x \fd \RR^3 \times \RR^3 &\to \RR^3 & 
    \text{das Kreuzprodukt}\\
    \dd {V}_x \fd \RR^3 \times \RR^3 \times \RR^3 &\to 
    \RR & \text{die Determinante},
  \end{align*}
  wie man leicht nachrechnet. Sei nun 
  $U\subset \RR^3$ und $\calli{V}(U)$ der Vektorraum der 
  stetig differenzierbaren Vektorfelder $U \to\RR^3$. Die Abbildungen 
  \begin{align*}
    \calli{V}( U ) \to \Omega^1 U, &\qquad 
    F \mapsto  \langle F, \overrightarrow{ \dd s } \rangle \\
    \calli{V}( U ) \to \Omega^2 U, &\qquad F \mapsto 
    \langle  F, \overrightarrow{ \dd S } \rangle \\
    \calli{C}^1( U ) \to \Omega^3 U, &\qquad f 
    \mapsto f \dd V 
  \end{align*}
  siften dann eine eineindeutige Zuordnung zwischen 
  den stetig differenzierbaren Vektorfeldern bzw. Funktionen auf $U$ 
  und den Formen auf $U$. 

  Mit dem Cartan-Kalkül lassen sich die äußeren \index{aussere@äußere Ableitung}
  Ableitungen der Differenzialformen 
  $\langle F , \overrightarrow{ \dd s } \rangle$, 
  $\langle  F ,\overrightarrow{ \dd S } \rangle$ bestimmen, 
  und zwar erhält man 
  \begin{align}
    \dd \langle F, \overrightarrow{ \dd s } \rangle &= 
    \sum_{j=1}^3 \dd F_j \wedge \dd x_j = 
    \sum_{j=1}^3 \left( \partial_1 F_j \dd x_1+ 
      \partial_2 F_j \dd x_2  + 
      \partial_3 F_j \dd x_3 \right) \wedge \dd x_j \notag \\
    &=(\partial_2 F_3 - \partial_3 F_2) \dd x_2 \wedge \dd x_3+ 
    (\partial_3 F_1-\partial_1 F_3)\dd x_3 \wedge \dd x_1 \notag \\[2mm] 
    &\qquad +(\partial_1 F_2 - \partial_2 F_1) \dd x_1 \wedge \dd x_2 \notag \\[2mm]
    &=\langle \rot F, \overrightarrow{ \dd S }\rangle. \asttag
  \end{align}
  und 
  \begin{align}
    \dd \langle F, \overrightarrow{ \dd S } \rangle &=
    \dd F_1 \wedge \dd x_2 \wedge \dd x_3 +
    \dd F_2 \wedge \dd x_3 \wedge \dd x_1 + 
    \dd F_3 \wedge \dd x_1 \wedge \dd x_2 \notag \\
    &=
    (\partial_1 F_1 + \partial_2 F_2 + \partial_3 F_3)
    \dd x_1 \wedge \dd x_2 \wedge \dd x_3 = \Div F \dd V. \aasttag
  \end{align}
  Aus der letzten Gleichung folgt zusammen mit dem Stokes'schen 
  Integralsatz noch einmal der Gauß'sche Integralsatz für dreidimensionale 
  Untermannigfaltigkeiten des $\RR^3$: 

  \medskip
  \noindent\slanted{Ist $U\subset\RR^3$ offen 
    und $F$ ein differenzierbares Vektorfeld 
    auf $U$, dann gilt für alle orientierten kompakten berandeten 
    $3$-dimensionalen Untermannigfaltigkeiten $M^3\subset U$:
    \[
    \boxed{
      \int_{M^3} \Div F \dd V   = 
      \int_{\partial M^3} \langle F, \overrightarrow{\dd S} \rangle  
    }
    \]}



  Aus der Gleichung {\astastref} erhält man den \slanted{Stokes'schen 
    Integralsatz} in seiner klassischen Form: 

  \medskip
  \noindent\slanted{Sei $U\subset\RR^3$ offen 
    und $F$ ein differenzierbares Vektorfeld 
    auf $U$. Dann gilt für alle orientierten kompakten berandeten 
    $2$-dimensionalen Untermannigfaltigkeiten $M^2\subset U$:
    \[
    \boxed{
      \int_{M^2} \langle \rot F, \overrightarrow{\dd S}  \rangle  = 
      \int_{\partial M^2} \langle F,\overrightarrow{\dd s}  \rangle } 
    \]}    

  Die letzte Formel kann man auch noch auf eine andere Weise ausdrücken.  
  Mit $\dd s$ bzw. $\dd S$ bezeichnet man die sogenannten   
  \slanted{kanonischen Volumenformen} von $\partial M$ bzw. 
  $M$. Das heißt, $\dd s$ ist die $1$-Form, die jedem positiv 
  orientierten Einheitsvektor aus einem Tangentialraum 
  $T_a \partial M$ den Wert $+1$ zuordnet und entsprechend $\dd S$ diejenige  
  $2$-Form, die jeder orthonormalen Basis von $T_b M$ den Wert $+1$ zuordnet. 
  Ist auf $M$ ein Einheitsnormalenfeld $\eta$ 
  und auf $\partial M$ Einheitstangentialfeld $\tau$ gegeben, 
  dessen Orientierung von der durch $\eta$ auf $M$ gegebenen Orientierung 
  induziert ist, dann kann man sich unter Berücksichtigung der 
  Abbildungseigenschaften der vektoriellen Elemente $\overrightarrow{ \dd s }$ 
  und 
  $\overrightarrow{ \dd S }$ durch eine geometrische Überlegung zumindest 
  plausibel machen, dass sich der klassische Stokes'sche Integralsatz 
  auch in der 
  Form      
  \[
  \boxed{
    \int_{M^2} \langle \rot F, \eta \rangle \dd S = 
    \int_{\partial M^2} \langle F, \tau \rangle \dd s} 
  \]
  schreiben lässt. \AntEnd
  
  \begin{center}
    \includegraphics{mp/12_dS}
    \captionof{figure}{Zur Interpretation von Linien- und Flächenelement.}
    \label{fig:12_stokes}
  \end{center}
\end{antwort} 





%%% Local Variables: 
%%% mode: latex
%%% TeX-master: t
%%% End: 
