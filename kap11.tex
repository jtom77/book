\chapter{Integralrechnung in mehreren Variablen} 

"`Die mehrdimensionale Integration ist wahrscheinlich innerhalb der 
mathematischen Grundvorlesungen das unangenehmste Stoffgebiet"' (Otto 
Forster in \citep{Forster}).

Eine der Schwierigkeiten ist sicher die Tatsache, dass mehrdimensionale 
Integrationsbereiche im allgemeinen eine viel komplexere Gestalt haben können 
als im Eindimensionalen, wo zunächst nur kompakte Intervalle als 
Integrationsbereiche auftreten und das Integral über nicht beschränkte 
Intervalle durch "`Ausschöpfen"' mit kompakten Intervallen auf die 
Integration über diese zurückgeführt wird. 

Ein Ziel der zu entwickelnden Integralrechnung sollte es sein, möglichst 
vielen auf geeigneten Teilmengen $A\subset\RR^n$ erklärten Funktionen 
ein $n$-dimensionales Integral $\int_A f$ so zuzuordnen, dass 
\begin{enumerate}
\item die Abhängigkeit von $A$ und $f$ überschaubaren Gesetzmäßigkeiten 
  genügt, {\zB} dass für zwei Funktionen $f,g \fd A\to \RR$ gilt 
  \[
  \int_A (f+g) = \int_A f + \int_A g.
  \]
\item im Fall $A=[a,b]\in \RR$ $\int_{[a,b]} f$ das gewohnte Integral 
  $\int_a^b f$ ist. 
\item eine geometrische Interpretation des Integrals $\int_A f$ für eine 
  nichtnegative Funktion $f\fd A\to \RR$ als $(n+1)$-dimensionales Volumen 
  $v_{n+1}$ (als \slanted{Maß}) der Menge 
  $
  K := \big\{ (x,t) \in \RR^n \times \RR \sets x\in A, \; 0 \le t \le f(x) \big\}
  $ ermöglicht wird, es soll also gelten 
  \[
  v_{n+1}( K) =\int_A f,
  \]

  insbesondere soll für die konstante 
  Funktion $\eins$ das $(n+1)$-dimensionale Volumen $\int_A \eins$ des Zylinders 
  mit "`Basis $A$"' und "`Höhe $1$"' (\Abb\ref{fig:11_zylinder}) mit dem $n$-dimensionalen Volumen 
  von $A\subset\RR^n$ identisch sein. 

  \begin{center}
    \includegraphics{mp/11_zylinder}
    \captionof{figure}{Zylinder mit "`Basis $A$"' und "`Höhe $1$"'}
    \label{fig:11_zylinder}
  \end{center}

  In den niederen Dimensionen $n=1,2,3$ 
  soll das $n$-dimensionale Volumen von $A$ mit den elementargeometrischen 
  Begriffen von Länge, Flächeninhalt und Rauminhalt übereinstimmen. 
\item möglichst weitreichende Konvergenzsätze des folgenden Typs 
  gelten: Konvergiert eine Folge $(f_n)$ von Funktionen $f_n \fd A\to \RR$ 
  gegen $f\fd A\to\RR$, so konvergiert auch die Folge der Integrale gegen 
  das Integral von $f$:
  \[
  \lim_{n\to\infty} \int_A f_n = \int_A f.
  \]
\end{enumerate}

\index{Daniell@\textsc{Daniell}, P.\, J. (1889-1946)}
\index{Lebesgue@\textsc{Lebesgue}, Henri (1875-1941)}
Die nachfolgenden Fragen beziehen sich auf die Konstruktion des 
\slanted{Lebesgue-Integrals} mithilfe des 
\slanted{Daniell-Lebesgue-Prozesses}. 
Ausgangspunkt ist dabei das Integral für 
\slanted{stetige Funktionen mit kompaktem Träger} (\slanted{Lebesgue'sches 
  Elementarintegral}), das im ersten Schritt auf die 
\slanted{halbstetigen Funktionen} 
fortgesetzt und in einem zweiten Schritt auf die 
\slanted{Lebesgue-integrierbaren}  
Funktionen ausgedehnt wird. Dieser ökonomische und elegante Zugang, der 
auch auf allgemeinere Räume ausdehnbar ist, findet sich {\zB} auch 
bei O. Forster (vgl. \citep{Forster}). Wem dieser Zugang unbekannt ist, 
kann die nächsten drei Abschnitte einfach im Sinne eines Lehrbuchs 
lesen, die Begriffe werden ausführlich erläutert. 

Es gibt alternative Zugänge zum Lebesgue-Integral, bei 
denen {\zB} vom Integral für Treppenfunktionen als 
"`Elementarintegral"' ausgegangen wird. Ein derartiger 
Weg wir etwa in \citep{Koenig} und 
\citep{Fritzsche} beschritten. Bei den Eigenschaften des 
Lebesgue-Integrals treffen sich die verschiedenen Zugänge dann wieder. 

Wir beginnen mit Integralen, die 
von einem Parameter abhängen und fragen nach der Stetigkeit und 
Differenzierbarkeit solcher Integrale.

\section{Parameterabhängige und $n$-fache Integrale}
\label{paramintegrale}

%% --- 1 --- %%
\begin{frage}\label{11_paramstet}
  \index{parameterabhängige Integrale!Stetigkeit}
  \index{Stetigkeit!parameterabhängiger Integrale} 
  Ist $[a,b]\subset\RR$ ein kompaktes Intervall und $M\subset\RR^n$ eine 
  beliebige nichtleere Teilmenge sowie $f\fd [a,b]\times M \to \RR$ 
  eine stetige Funktion (s. \Abb\ref{fig:11_int}), warum ist dann die Funktion 
  \[
  G \fd M \to\RR; \qquad y \mapsto \int_a^b f(x,y) \difx
  \]
  stetig auf $M$? 

  \begin{center}
    \includegraphics[width=80mm]{povray/11_int.pdf}
    \captionof{figure}{Der Wert der Funktion $G$ an einer Stelle $y_0\in M$ ist das 
      Integral $\int_a^b f(x,y_0)\difx$.}
    \label{fig:11_int}
  \end{center}
\end{frage}

\begin{antwort}
  Das folgt im Wesentlichen aus der 
  \slanted{gleichmäßigen} Stetigkeit stetiger Funktionen 
  auf kompakten Mengen (vgl. Frage \ref{09_glmstet}). 

  Sei $(y_k)$ eine Folge in $M$, die gegen $y_0 \in M$ konvergiert. 
  Es ist $\lim G(y_k)=G(y_0)$ zu zeigen, also dass es zu jedem 
  $\eps>0$ ein $k_0\in \NN$ gibt, sodass für alle $k>k_0$ gilt 
  \[
  \bigl| G(y_k) - G(y_0) \bigr| = \left| \int_a^b 
    \bigl( f(x,y_k)- f(x,y_0) \bigr) \difx \right| < \eps.
  \]
  Nun ist die Menge $Y:= \{ y_k \sets k\in \NN \}\cup\{ y_0\}$ 
  eine kompakte Teilmenge von $M$ (vgl. Frage \ref{09_kompaktbsp}), 
  und damit ist auch $K := [a,b]\times Y 
  \subset \RR^1\times \RR^n = \RR^{1+n}$ kompakt, 
  folglich $f$ auf $K$ gleichmäßig stetig. 
  Insbesondere gibt es zu beliebig 
  vorgegebenem $\eps>0$ ein $\delta>0$ mit der Eigenschaft, 
  dass für alle $x\in [a,b]$ gilt
  \[
  \bigl| f(x,y_k)-f(x,y_0) \bigr| < \eps, 
  \quad\text{falls\; $\nb{ (x,y_k) -(x,y_0) } = \nb{ y_k -y_0 } < \delta.$} 
  \]
  Wegen $\lim y_k \to y_0$ ist dies für $k\ge k_0$ erfüllt. Für diese 
  $k$ gilt dann auch
  \[
  \bigl| G(y_k)-G(y_0) \bigr| = \left| \int_a^b 
    \bigl( f(x,y_k)-f(x,y_0) \bigr)  \difx \right| < \eps(b-a). \nodpagebreak
  \]
  Das zeigt die Stetigkeit der Funktion $G$. \AntEnd
\end{antwort}

%% --- 2 --- %%
\begin{frage}\label{11_paramdiff}
  \index{parameterabhängige Integrale!Differenzierbarkeit}
  Ist $[a,b]\subset\RR$ ein kompaktes Intervall und $U\subset \RR^n$ 
  eine nichtleere offene Teilmenge (mit den Koordinaten $y_1,\ldots,y_n$) sowie 
  $f\fd [a,b]\times U\to\RR$ eine stetige Funktion, die eine auf 
  $[a,b]\times U$ stetig partielle Ableitung nach der $j$-ten Variable $y_j$ 
  besitzt, warum ist dann die Funktion 
  \[
  G \fd U \to \RR; \qquad y \mapsto \int_a^b f(x,y_1,\ldots,y_n)\difx 
  \]
  stetig partiell differenzierbar, und es gilt 
  \[
  \partial_j G(y) = \int_a^b \partial_j f(x,y_1,\ldots,y_n) \difx \,\text{?}
  \]
  (Differenziation unter dem Integral, Leibniz'sche Regel)
  \index{Differenziation!unter dem Integral}
  \index{Leibniz'sche Regel}
\end{frage} 

\begin{antwort}
  Wir können uns auf den Fall $n=1$ beschränken. 
  Dann ist zu zeigen, dass für festes 
  $y_0 \in U$ und alle hinreichend nahe bei $y_0$ 
  gelegenen $y\in U \mengeminus{y_0}$ der Betrag von 
  \[
  \frac{G(y)-G(y_0)}{y-y_0} - 
  \int_a^b \frac{\partial f}{\partial y} (x,y_0) \difx =
  \int_a^b \left[ \frac{f(x,y)-f(x,y_0)}{y-y_0} - 
    \frac{\partial f}{\partial y} (x,y_0) \right] \difx 
  \]
  beliebig klein wird. 

  Dazu schreibt man den Integranden auf der rechten Seite in der Form 
  \[
  \frac{1}{y-y_0} \left[ f(x,y)-y
    \frac{\partial f}{\partial y} (x,y_0) - \left( 
      f(x,y_0)- y_0 \frac{\partial f}{\partial y} (x,y_0) \right) \right]
  = \frac{ h(x,y)-h(x,y_0) }{y-y_0}
  \]
  mit $ h(x,y):= f(x,y)-y \frac{\partial f}{\partial y} (x,y_0)$. 
  Es gilt $
  \frac{\partial h}{\partial y} (x,y)=
  \frac{\partial f}{\partial y} (x,y) -
  \frac{\partial f}{\partial y} (x,y_0)$, 
  also ist $\frac{\partial h}{\partial y}$ stetig mit 
  $\frac{\partial h}{\partial y} (x,y_0)=0$. 

  An dieser Stelle kommt wieder die gleichmäßige Stetigkeit ins Spiel. 
  Da $\frac{\partial h}{\partial y}$ auf kompakten Mengen gleichmäßig 
  stetig ist, können wir zu vorgegebenem $\eps>0$ ein $\delta>0$ so wählen, 
  dass für die abgeschlossene Kugel $K := \overline{U_\delta (y_0)}$ 
  gilt 
  \[
  \left| \frac{\partial h}{\partial y} (x,y) \right| < \eps 
  \quad\text{für $(x,y) \in [a,b]\times K$}. 
  \]
  Der Mittelwertsatz liefert dann für $(x,y) \in [a,b]\times K$ 
  für $y\not=y_0$ 
  \[
  \left| \frac{h(x,y)-h(x,y_0)}{y-y_0} \right| 
  \le \sup \left\{ \left| \frac{\partial h}{\partial y} (x,y) \right| \sets
    (x,y) \in [a,b]\times K \right\}
  < \eps,
  \]
  und daraus folgt
  \[
  \left| \frac{G(y)-G(y_0)}{y-y_0} - \int_a^b 
    \frac{\partial f}{\partial y} (x,y) \difx \right| < \eps.
  \]
  $G$ besitzt also die behauptete Ableitung. Deren Stetigkeit 
  folgt aus der Antwort zur vorhergehenden Frage.

  Bemerkung: Die Ableitung parameterabhängiger 
  Integrale spielt eine große Rolle bei der Herleitung der 
  Euler-Lagrange'schen Differenzialgleichungen der Variationsrechnung.
  \index{Variationsrechnung}
  \index{Euler-Lagrange'sche Differenzialgleichungen} 
  Man vergleiche hierzu \citep{Koenig}.
  \AntEnd
\end{antwort} 

%% --- 3 --- %%
\begin{frage}
  Bleibt der Zusammenhang aus Frage \ref{11_paramdiff} auch richtig, wenn 
  $U$ durch einen kompakten achsenparallelen Quader $Q\subset\RR^n$ ersetzt 
  wird?
\end{frage}

\begin{antwort}
  Der Zusammenhang bleibt gültig, da im Fall eines kompakten 
  Quaders auch auf den Randpunkten alle partiellen Ableitungen 
  (zumindest in einer Richtung) existieren. 
  \AntEnd
\end{antwort} 

%% --- 4 --- %%
\begin{frage}\label{11_doppelint}\index{Doppelintegral}
  \index{mehrdimensionales}
  Seien $[a,b]\subset \RR$ und $[c,d] \subset \RR$ kompakte Intervalle und 
  $Q := [a,b]\times [c,d]$. Wie sind für eine stetige Funktion 
  $f\fd Q\to \RR$ dann die "`Doppelintegrale"'
  \[
  A := \int_a^b \left( \int_c^d f(x,y) \difx \right) \dify \quad\text{bzw.}\quad
  B := \int_c^d \left( \int_a^b f(x,y) \dify \right) \difx
  \]
  erklärt, und warum gilt $A=B$?
\end{frage}

\begin{antwort}
  Als partielle Funktion der stetigen Funktion $f$ ist die Funktion 
  $[a,b]\to\RR$ mit $x\mapsto f(x,y)$ bei festgehaltenem $y$ stetig, kann 
  also über $[a,b]$ integriert werden. Sei 
  $G(y) := \int_a^b f(x,y)\difx$. Lässt man $y$ im Intervall $[c,d]$ variieren, 
  dann ist $G\fd [c,d]\to \RR$ nach Frage \ref{11_paramstet} stetig, kann 
  also integriert werden. Damit existiert 
  \[
  A := \int_c^d G(y) \dify = \int_c^d \left( \int_a^b f(x,y) \difx \right) \dify,
  \]
  und aus Symmetriegründen auch 
  \[
  B := \int_a^b \left( \int_c^d f(x,y) \dify \right) \difx.
  \]
  Um $A=B$ zu zeigen, definieren wir $\varphi \fd [c,d]\to\RR$ durch 
  \[
  \varphi(y) = \int_a^b \left( \int_c^y f(x,t) \dift \right) \difx. 
  \]
  Offensichtlich ist $\varphi(c)=0$. Nach Frage \ref{11_paramdiff} 
  ist $\varphi$ stetig differenzierbar, und es gilt
  \[
  \varphi'(y) = \int_a^b \left( 
    \frac{\partial}{\partial y} \int_c^y f(x,t) \dift \right) \difx 
  = \int_a^b f(x,y) \difx 
  \]
  nach dem Hauptsatz der Differenzial- und Integralrechnung. 
  Wiederum aufgrund des Hauptsatzes gilt
  \[
  \varphi(y)=\varphi(y)-\varphi(c) = 
  \int_c^y \varphi'(t) \dift = \int_c^y 
  \left( \int_a^b f(x,t) \difx \right)\dift.
  \]
  Für $y=d$ erhält man einerseits
  \[
  \varphi(d)=\int_c^d \left( \int_a^b f(x,t) \difx \right) \dift,
  \]
  andererseits ist nach Definition 
  \[
  \varphi(d)=\int_a^b \left( \int_c^d f(x,t) \dift \right) \difx.
  \]
  Damit ist die Gleichheit der Doppelintegrale $A$ und $B$ gezeigt. 
  Man nennt diese das \slanted{Doppelintegral von $f$ über den Quader 
    $[a,b]\times[c,d]$.}
  \AntEnd
\end{antwort}

%% --- 5 --- %%
\begin{frage}\label{11_mehrfachint}\index{Mehrfachintegral}
  Sei $Q := [a_1,b_1]\times \cdots \times [a_n,b_n] \subset\RR^n$ ein 
  achsenparalleler kompakter Quader und $f\fd Q\to\RR$ eine 
  stetige Funktion. Wie kann man rekursiv das $n$-fache 
  Integral $I_Q(f)$ von $f$ über den Quader $Q$ erklären?
\end{frage}

\begin{antwort}
  Man denkt sich die Variablen $(x_2,x_3,\ldots,x_n)$ festgehalten und 
  betrachtet die stetige Funktion $[a_1,b_1]\to \RR$ mit 
  $x_1 \mapsto f(x_1,x_2,\ldots,x_n)$. Integriert man diese 
  Funktion, dann hängt das Resultat
  \[
  F_1(x_1,x_2,\ldots,x_n ):= \int_{a_1}^{b_1} f(x_1,x_2,\ldots,x_n)\difx_1 
  \]
  von den Parametern $(x_2,\ldots,x_n)$ ab. Lässt man diese Variablen 
  nun wieder variieren, dann erhält man nach Frage \ref{11_paramstet} 
  eine stetige Funktion
  \[
  F_1 \fd Q' \to \RR; 
  (x_2,\ldots,x_n) \mapsto F_1(x_2,\ldots,x_n) := \int_{a_1}^{b_1} 
  f(x_1,\ldots, x_n ) \difx_1,
  \]
  wobei $Q':=[a_2,b_2]\times \cdots\times [a_n,b_n]$. 

  Somit kann das Mehrfachintegral induktiv definiert 
  werden. Im Fall $n=1$ setzt man  
  \[
  \int_{[a,b]} f(x_1) \difx_1 := \int_a^b f(x_1) \difx_1 ,
  \]
  und für $n\ge 2$ ist $I(Q)$ definiert durch
  \begin{align}
    I_Q(f) &:= \int_Q f(x_1,\ldots,x_n)\difx_1 \cdots \difx_n 
    := \int_{Q'} F_1( x_2,\ldots, x_n ) \difx_2\cdots\difx_n \notag \\
    &=
    \int_{Q'} \left( \int_{a_1}^{b_1} f(x_1,\ldots,x_n ) \difx_1 \right)
    \difx_2 \cdots \difx_n . \EndTag
  \end{align}
\end{antwort}

%% --- 6 --- %%
\begin{frage}
  Ist $Q := [a_1,b_1]\times \cdots \times [a_n,b_n] \subset\RR^n$ 
  und $V := \calli{C}(Q)$ der Vektorraum der stetigen Funktionen auf 
  $Q$, welche Haupteigenschaften hat dann die Abbildung (das Integral) 
  \[
  I_Q \fd V \to \RR; \qquad f\mapsto I_Q(f).
  \]
\end{frage}

\begin{antwort}
  $I_Q$ ist ein \slanted{lineares Funktional} mit den Eigenschaften 
  \satz{\setlength{\labelsep}{3mm}
    \begin{enumerate}
    \item[\desc{a}] 
      $I_Q$ ist \slanted{nichtnegativ}, {\dasheisst}, aus $f\ge 0$ folgt 
      $I_Q(f) \ge 0$. Diese Eigenschaft ist äquivalent zur \slanted{Monotonie}
      \[
      f \le g \Ra I_Q (f) \le I_Q (f),\qquad f,g\in V.
      \]
    \item[\desc{b}] Es gilt die \slanted{Standardabschätzung}
      \[
      \big| I_Q(f) \big| \le \nb{ f }_\infty v_n (Q).
      \]
      Dabei ist $\n{\;\,}_\infty$ die Maximumsnorm von $f$ auf $Q$ und 
      $v_n(Q) := (b_1-a_1)\cdots(b_n-a_n)$ das elementargeometrische 
      Volumen des Quaders $Q$. 
    \item[\desc{c}] Ist $(f_k)\subset V$ eine Folge von Funktionen, 
      die \slanted{gleichmäßig} gegen $f\fd Q\to\RR$ konvergiert 
      (woraus $f\in V$ folgt), dann gilt
      \[
      \lim_{k\to\infty} I(f_k) = I(f) = I(\lim_{k\to\infty} f_k ).
      \]
    \end{enumerate}}

  \noindent
  Die Beweise für die Eigenschaften ergeben sich unmittelbar aus 
  denen in einer Variablen. Ferner erfüllt das Mehrfachintegral 
  nach den Fragen \ref{11_doppelint} und \ref{11_mehrfachint} 
  noch folgende wichtige Eigenschaft

  \satz{\setlength{\labelsep}{3mm}
    \begin{enumerate}
    \item[\desc{$\ast$}] $I_Q(f)$ ist unabhängig von der Integrationsreihenfolge, 
      {\dasheisst}, für jede Permutation 
      $\sigma \fd \{1,\ldots,n\}\to\{1,\ldots,n\}$ gilt 
      \begin{multline}
        \Int_{a_{\sigma(n)}}^{b_{\sigma(n)}} \cdots 
        \Int_{a_{\sigma(2)}}^{b_{\sigma(2)}}  
        \Big( 
        \Int_{a_{\sigma(1)}}^{b_{\sigma(1)}} 
        f(x_1,\ldots,x_n) \difx_{\sigma(1)} \Big) 
        \difx_{\sigma(2)}\cdots\difx_{\sigma(n)} = \\
        \Int_{a_n}^{b_n} \cdots 
        \Int_{a_2}^{b_2}  
        \Big( 
        \Int_{a_1}^{b_1} 
        f(x_1,\ldots,x_n) \difx_1) \Big) 
        \difx_2\cdots\difx_n.
        \EndTag   
      \end{multline}
    \end{enumerate}}
\end{antwort}

%% --- 7 --- %%
\begin{frage}\label{11_unabreihe}\label{11_integrationsreihenfolge}
  Können Sie für die Unabhängigkeit des Integrals $I_Q(f)$ 
  von der Integrationsreihenfolge zwei methodisch verschiedene 
  Beweise geben?
\end{frage}

\begin{antwort}
  1. Beweis: Jede Permutation ist die Hintereinanderausführung 
  von Nachbarschaftsvertauschungen. Damit lässt sich das Problem auf den 
  Fall $n=2$ aus Frage \ref{11_doppelint} reduzieren. 

  2. Beweis: Die Behauptung ist klar für 
  stetige Funktionen $\varphi\fd Q\to\RR$ der 
  Gestalt $\varphi(x)=\varphi_1(x_1)\cdots \varphi_n(x_n)$ 
  mit $\varphi_i \in \calli{C}( [a,b] )$. Nun wurde in Frage 
  \ref{09_stonekonsequenz} mit dem Satz von Stone-Weierstraß gezeigt, dass 
  die Algebra $\calli{C}([a_1,b_1])\otimes \cdots \otimes \calli{C}([a_n,b_n])$ 
  dicht in $\calli{C}(Q)$ liegt. Das heißt, zu jedem $f\in \calli{C}(Q)$ 
  und jedem $\eps>0$ gibt es eine Funktion $\varphi \in \calli{C}([a_1,b_1])
  \otimes \cdots \otimes \calli{C}([a_n,b_n])$ mit 
  \[
  \big| f(x) - \varphi(x) \big| < \eps \quad\text{für alle $x\in Q$.}
  \]
  Bezeichnet man mit $I_{Q}(f;\sigma)$ das Mehrfachintegral von $f$ über 
  $Q$ mit der durch die Permutation $\sigma$ vorgegebenen 
  Integrationsreihenfolge, dann gilt für 
  beliebige Permutationen $\sigma_1$, $\sigma_2$ also 
  \[
  \big| I_{Q}(f;\sigma_1) - I_Q(\varphi;\sigma_1) \big| < v_n(Q) \cdot \eps 
  \quad\text{und}\quad 
  \big| I_{Q}(f;\sigma_2) - I_Q(\varphi;\sigma_2) \big| < v_n(Q) \cdot\eps,
  \]
  und wegen $I_Q(\varphi;\sigma_1) = I_Q(\varphi;\sigma_2)$ also 
  \[
  \big| I_{Q}(f;\sigma_1) - I_Q(f;\sigma_2) \big| < 2
  v_n(Q)\cdot\eps.
  \]
  Da $\eps$ beliebig klein gewählt werden kann, folgt die Behauptung.
  \AntEnd
\end{antwort}


\section{Das Integral f\"ur stetige Funktionen mit kompaktem Tr\"ager}

Wir beginnen jetzt damit, das Lebesgue-Integral mithilfe eines 
Fortsetzungsprozesses in mehreren Schritten zu konstruieren. Dazu wird 
das Integral in diesem Abschnitt zunächst für 
\slanted{stetige Funktionen mit kompaktem Träger} definiert. Ausgehend 
von diesem 
"`Elementarintegral"' 
erweitern wir den Integralbegriff in den darauf folgenden Abschnitten 
auf eine wesentlich größere Klasse von Funktionen. 
Wir konzentrieren uns dabei zunächst auf den $\RR^n$ als Integrationsbereich. 

%% --- 8 --- %%
\begin{frage}\index{Traeger@Träger einer Funktion}
  \nomenclature{$\Tr(f)$}{Träger der Funktion $f$}
  Was versteht man unter dem \bold{Tr\"ager} einer Funktion 
  $f\fd X\to\RR$ ($X$ beliebiger metrischer Raum)?
\end{frage}

\begin{antwort}
  Unter dem \slanted{Tr\"ager von $f$} 
  (englisch $\supp$ wie "`support"') versteht man die 
  \slanted{abgeschlossene Menge} 
  \[
  \Tr (f ) := \overline{ \{ x\in X \sets f(x)\not=0 \} }.
  \]
  Ein Punkt $x\in X$ geh\"ort also genau dann zum Tr\"ager von $f$, 
  wenn $f(x)\not=0$ gilt oder wenn es eine Folge $(x_k)$ gibt 
  mit $x_k\in X$, $\lim x_k = x$ und $f(x_k) \not=0$. 
  Zum Beispiel hat die Funktion $f \fd \RR\times\RR\to\RR$ mit 
  $(x,y) \mapsto \sin x\sin y$ den Träger $\Tr(f) = \RR\times\RR$. 
  \AntEnd
\end{antwort}

%% --- 9 --- %%
\begin{frage}\nomenclature{$\calli{C}_c(\RR^n)$}{Raum der stetigen Funktionen 
    mit kompaktem Träger auf $\RR^n$}
  \index{kompakter Träger}
  Hier und im Folgenden bezeichne
  \[
  \calli{C}_c = 
  \calli{C}_c (\RR^n) := \{f\fd \RR^n \to \RR\sets 
  \text{$f$ stetig, $\Tr(f)$ kompakt}\}.
  \]
  die Menge der stetigen Funktionen auf $\RR^n$ mit kompaktem 
  Träger. 
  Warum ist $\calli{C}_c (\RR^n)$ 
  ein $\RR$-Vektorraum, der mit $f,g$ 
  auch 
  \begin{align*}
    f\wedge g\fd \RR^n \to \RR; &\qquad x\mapsto
    (f \wedge g) (x) := \min \{ f(x), g(x) \},\\
    f\vee g\fd \RR^n \to \RR; &\qquad x\mapsto
    (f \vee g) (x) := \max \{ f(x), g(x) \},
  \end{align*}
  ferner $f_+ := f\vee0$, $f_- := (-f)\vee 0$ und $|f|$ enth\"alt? 
  Warum liegt mit $f$ auch $1\wedge f$ in $\calli{C}_c(\RR^n)$? 
\end{frage}

\begin{antwort}
  
  Da $\calli{C}_c(\RR^n)$ 
  ein Untervektorraum des Vektorraums $\calli{C}( \RR^n )$ der 
  stetigen Funktionen ist, braucht man 
  zum Nachweis der Vektorraumeigenschaft nur die 
  Abgeschlossenheit bez\"uglich 
  der Addition und der Multiplikation 
  mit Skalaren sowie der Bildung von 
  $f\wedge g$, $f\vee g$ und $|f|$ nachzuweisen. Dies folgt aus 
  \begin{align*}
    \Tr (f+g) &\subset \Tr(f)\cup\Tr(g) & \Tr(C \cdot f) &= Tr (f) 
    &\Tr ( | f | ) &= \Tr (f)  \\
    \Tr (f\wedge g) &\subset \Tr(f)\cup\Tr(g) & 
    \Tr (f\vee g) &\subset \Tr(f)\cup\Tr(g) & &
  \end{align*} 
  Ferner gilt $\Tr( 1\wedge f)=\Tr(f)$. Damit gehört auch 
  $1\wedge f$ zu $\calli{C}_c(\RR^n)$. 

  Zusammengefasst besagen diese Eigenschaften, dass $\calli{C}_c(\RR^n)$ 
  ein \slanted{Stone'scher Verband} ist. \index{Stone'scher Verband}
  \AntEnd
\end{antwort}

%% --- 10 --- %%
\begin{frage}
  K\"onnen Sie begr\"unden, warum f\"ur eine stetige 
  Funktion $f\fd \RR^n \to \RR$ gilt: 
  $f$ hat kompakten Tr\"ager genau dann, wenn es einen 
  kompakten W\"urfel $W\subset\RR^n$ gibt mit der 
  Eigenschaft $f(x)=0$ f\"ur alle $x\in\RR^n \mengeminus W$?
\end{frage}

\begin{antwort}
  Als Teilmenge von $\RR^n$ ist die abgeschlossene Menge $\Tr(f)$ genau 
  dann kompakt, wenn sie beschr\"ankt 
  ist. Ist sie beschr\"ankt, dann liegt sie in einem kompakten W\"urfel $W$. 
  Liegt sie umgekehrt in einem kompakten W\"urfel, so ist sie 
  beschr\"ankt. 
  \AntEnd
\end{antwort}

%% --- 11 --- %%
\begin{frage}\index{Integral!stetiger Funktionen mit kompaktem Träger}
  Wie ist das Integral f\"ur eine Funktion $f\in\calli{C}_c (\RR^n)$ 
  erkl\"art?
\end{frage}

\begin{antwort}
  F\"ur eine stetige Funktion $f$ mit kompaktem Träger $\Tr(f)$ 
  definiert man das Integral $I(f)$ als das Mehrfachintegral über einen 
  beliebigen achsenparallelen Quader 
  $Q:= [a_1,b_1]\times\cdots\times [a_n,b_n]$ mit 
  $Q\supset \Tr(f)$, also  
  \[
  I(f) = \Int_{\RR^n} f(x) \difx := \Int_Q f(x) \difx
  =\Int_{a_n}^{b_n}\cdots \Int_{a_1}^{b_1}
  f(x_1,\ldots,x_n) \difx_1 \cdots \difx_n.
  \]
  Offensichtlich ist diese Definition unabhängig von dem gewählten 
  Quader $Q$, sofern nur $\Tr(f)\subset Q$ gilt (Beweis durch Rückführung 
  auf den eindimensionalen Fall).  
  \AntEnd  
\end{antwort} 

%% --- 12 --- %%
\begin{frage}\label{11_haupteigenschaft}
  Welche Permanenzeigenschaften hat die Abbildung (das Integral) 
  \[
  I \fd L \to \RR; \qquad f \mapsto I(f)\; \text{?}
  \]
\end{frage}

\begin{antwort}
  \satz{Für $f,g\in\calli{C}_c$ und $a,b\in\RR$ gilt
    \[
    \begin{array}{rp{2mm}lp{15mm}r}
      \text{\desc{a}} & &
      I(a f+b g)= a \cdot I( f)+b\cdot I(g),  & &
      \text{(Linearität)} \\
      \text{\desc{b}} & &
      f \le g \Ra I(f) \le I(g ), & &
      \text{(Monotonie)}\\
      \text{\desc{c}} & &
      \left| I(f) \right| \le \nb{f}_\infty \cdot v_n(Q). & &
      \text{(Beschränktheit)}
    \end{array}
    \]
    Damit ist $I$ ein \slanted{lineares, monotones, beschränktes} Funktional.} 

  Ferner ist $I$ \slanted{translationsinvariant} in folgendem Sinne: 
  Für einen Vektor $v\in\RR^n$ sei $\tau_v f \fd \RR^n \to\RR$ die durch 
  $\tau_v f(x)=f(x+v)$ gegebene Funktion. Dann gilt
  $I(\tau_v f)=I(f)$. Das folgt wiederum durch Rückführung 
  auf den eindimensionalen Fall und Anwendung der 
  Substitutionsregel. \AntEnd
\end{antwort}

%% --- 13 --- %%
\begin{frage}\index{Satz!von Dini}
  \index{Dini@\textsc{Dini}, Ulisse (1845-1918)}
  Was besagt der Satz von Dini?
\end{frage}

\begin{antwort}
  Der Satz liefert ein entscheidendes Verbindungsglied zwischen 
  monotoner und gleichm\"a{\ss}iger Konvergenz f\"ur Funktionenfolgen 
  auf einer kompakten Menge. Dieser Zusammenhang ist deswegen zentral, 
  weil man sich bei der Konstruktion des Lebesgue-Integrals durch einen 
  Fortsetzungsprozess von der Einschränkung der \slanted{gleichmäßigen 
    Konvergenz} lösen möchte, andernfalls käme man nie über den Raum der 
  stetigen Funktionen hinaus. Der Satz von Dini lautet:

  \medskip\noindent
  \satz{Es sei $K\subset \RR^n$ kompakt und $(f_k)$ eine Folge 
    stetiger Funktionen, die monoton wachsend gegen $f$ konvergiert, 
    {\dasheisst}, es gilt  
    \[
    \begin{array}{lp{3mm}l}
      \text{\desc{i}} & & f_1 \le  f_2 \le f_3 \le \cdots, \\[2mm]
      \text{\desc{ii}} & &  \lim\limits_{k\to\infty} f_k=f,
    \end{array}
    \]
    dann konvergiert $(f_k)$ auf $K$ sogar gleichmäßig gegen $f$. 
  }

  \medskip 
  Für den Beweis zeigt man, dass die Folge $(g_k)$ mit 
  $g_k := f - f_k$ auf $K$ gleichmäßig gegen null konvergiert, das genügt. 
  Sei dazu $\eps>0$ vorgegeben,  dann gibt es zu jedem $\xi\in K$ 
  eine Schranke $N(\xi) \in \NN$, sodass 
  \[
  \left| g_{N(\xi)} (\xi) \right| < \eps
  \]
  gilt. Wegen der Stetigkeit von $g_{N(\xi)}$ gilt die Ungleichung 
  auch noch in einer Umgebung $U(\xi)$ von $\xi$, also hat man
  \[
  \left| g_{N(\xi)} (x) \right| < \eps \qquad\text{für alle $x\in U(\xi)$}. 
  \]
  Da $K$ kompakt ist, wird $K$ von endlich vielen Umgebungen 
  $U(\xi_1),\ldots,U(\xi_r)$ überdeckt. Setzt man 
  $N:=\max\{ N(\xi_1),\ldots, N(\xi_r) \}$, dann gilt 
  \[
  \left| g_N (x) \right| < \eps \qquad\text{für alle $x\in K$}, 
  \]
  und da $(g_k)$ monoton fällt, folgt daraus 
  \[
  \left| g_k (x) \right| < \eps 
  \qquad\text{für alle $x\in K$ und alle $k\ge N$},
  \]
  {\dasheisst}, die Folge $(g_k)$ konvergiert gleichmäßig. 
  \AntEnd
\end{antwort}

%% --- 14 --- %%
\begin{frage}\label{11_sigma}\index{sigma-Stetigkeit@$\sigma$-stetig}
  Was versteht man unter der \bold{$\mathbf{\sigma}$-Stetigkeit} 
  eines linearen, nichtnegativen Funktionals auf einem Unterraum 
  $L \subset \mathrm{Abb}(X,\RR)$?
\end{frage}

\begin{antwort}
  Ein Funktional $I \fd L\to \RR$ hei{\ss}t 
  \slanted{$\sigma$-stetig}, wenn f\"ur jede Folge 
  $(f_k)$ mit $f_k\in\calli{C}_c( \RR^n )$ und 
  \[
  f_1 \ge f_2 \ge f_3 \ge \ldots, 
  \]
  die punktweise gegen $0$ konvergiert, gilt
  \[
  \lim_{k\to\infty} I(f_k) = I (\lim_{k\to\infty} f_k).
  \]
  Für ein $\sigma$-stetiges Funktional sind "`Integration"' und 
  Grenzwertbildung also vertauschbar, falls $(f_k)$ monoton fallend 
  gegen $0$ konvergiert.

  Diese Eigenschaft ist äquivalent dazu, 
  dass für jede Folge $(f_k)$ mit $f_k\in\calli{C}_c(\RR^n)$ mit 
  $f_1 \le f_2 \le f_3 \le \cdots $, die punktweise gegen eine 
  Funktion $f$ konvergiert, der Zusammenhang 
  $\lim\limits_{k\to\infty} I(f_k)=I(f)$ gilt.   
  \AntEnd 
\end{antwort}

%% --- 15 --- %%
\begin{frage}
  Warum ist das Funktional (Integral) $I \fd \calli{C}_c (\RR^n ) 
  \to \RR$ mit $f \mapsto I(f)$ $\sigma$-stetig?
\end{frage}

\begin{antwort}
  Ist $(f_k)$ eine monoton fallende Folge von Funktionen 
  $f_k\in \calli{C}_c(\RR^n)$, die punktweise gegen 
  Null konvergiert, dann ist nach dem Satz von Dini die Konvergenz 
  gleichm\"a{\ss}ig auf 
  dem Kompaktum $K:= \Tr(f_1)$. Aufgrund der Monotonie 
  gilt $\Tr(f_{k}) \subset \Tr(f_1)$ f\"ur alle $k\in \NN$  
  und damit $\lim_{k\to\infty} \nb{f_k}_K=0$. Die 
  Behauptung folgt dann aus der Standardabsch\"atzung \desc{c} aus 
  Frage \ref{11_haupteigenschaft}.
  \AntEnd
\end{antwort}




\section{Fortsetzung des Integrals auf halbstetige Funktionen}


\index{Daniell@\textsc{Daniell}, P.\, J. (1889-1946)}
\index{Lebesgue@\textsc{Lebesgue}, Henri (1875-1941)}
Wir wollen auch Funktionen integrieren, die nicht stetig sind und/oder die 
keinen kompakten Tr\"ager haben. Wir erweitern daher zun\"achst in einem ersten 
Schritt das Integral f\"ur Funktionen aus $\calli{C}_c ( \RR^n )$ 
auf eine gr\"o{\ss}ere 
Funktionenklasse, n\"amlich auf solche 
Funktionen, die sich als \slanted{monotone Limites} 
von Funktionen aus $\calli{C}_c(\RR^n)$ darstellen 
lassen. Im Wesentlichen sind das die von unten bzw. oben \slanted{halbstetigen Funktionen}.

%% --- 16 --- %%
\begin{frage}\index{erweiterte Zahlengerade}
  \nomenclature{$\overline{\RR}$}{erweiterte Zahlengerade 
    $\RR\cup\{ -\infty, \infty \}$} 
  Was versteht man unter der 
  \bold{erweiterten Zahlengeraden $\mathbf{\overline{\RR}}$}?
\end{frage}

\begin{antwort}
  Unter der \slanted{erweiterten Zahlengeraden} versteht man 
  die mit den Elementen $-\infty$ und $+\infty$ erweiterten reellen Zahlen, 
  also 
  \[
  \overline{\RR} = \RR \cup \{ -\infty, \infty \}.
  \]
  Die Elemente $-\infty$ und $\infty$ sind charakterisiert durch
  \[
  -\infty < x <  \infty \qquad\text{f\"ur alle $x\in \RR$}. 
  \asttag
  \]
  Die Erweiterung von $\RR$ auf $\overline{\RR}$ 
  hat sich in der Integrationstheorie als n\"utzlich erwiesen, 
  da man auch Funktionen mit Werten in $\{ -\infty, \infty\}$ 
  zulassen m\"ochte. Ein entscheidender Vorteil besteht darin, dass 
  jede Teilmenge von $\overline{\RR}$ eine größte obere und kleinste 
  untere Schranke besitzt. Überträgt man die Definition von "`Supremum"' und 
  "`Infimum"' auf $\overline{\RR}$, dann besitzt also jede Teilmenge von 
  $\overline{\RR}$ ein Supremum und ein Infimum.  


  Die Menge $\overline{\RR}$ besitzt in dem kompakten Intervall 
  $[-1,1]$ ein topologisches Modell, und zwar {\zB} 
  verm\"oge der bijektiven Abbildung $s\fd \overline{\RR}\to[-1,1]$ 
  (\sieheAbbildung\ref{fig:11_infty}), 
  definiert durch
  \[
  s(x) := \left\{ \begin{array}{cl} 
      \dis \frac{x}{1+|x|} & \text{f\"ur $x\not=\pm\infty$} \\
      \dis 1 & \text{f\"ur $x=\infty$} \\
      \dis -1 & \text{f\"ur $x=-\infty$}. 
    \end{array}\right.
  \]

  \begin{center}
    \includegraphics{mp/11_infty}
    \captionof{figure}{Die Funktion $s$ bildet $\overline{\RR}$ bijektiv auf 
      $[-1,1]$ ab.}
    \label{fig:11_infty}
  \end{center}

  Die Köperstruktur von $\RR$ lässt sich nicht widerspruchsfrei  
  auf $\overline{\RR}$ erweitern. Die folgenden Regeln 
  für das Rechnen mit $\infty$ und $-\infty$ implizieren aber keine 
  Widersprüche:   
  \begin{gather}
    \begin{array}{rlrl}
      \infty + x = \infty & \text{f\"ur alle $x\in\RR\cup\{\infty\}$} & 
      x \cdot \infty = \infty & \text{f\"ur alle $x\in\RR_+$} \\
      -\infty + x = \infty & \text{f\"ur alle $x\in\RR\cup\{-\infty\}$} &
      -x \cdot \infty = -\infty & \text{f\"ur alle $x\in\RR_+$} \\
    \end{array}  \notag \\
    \infty \cdot \infty = \infty, \quad 
    \infty \cdot (-\infty ) = -\infty, \quad
    (-\infty) \cdot(- \infty) = \infty, \quad 
    0\cdot (\pm\infty)=0.\notag
  \end{gather}
  Man beachte, dass $\infty +(-\infty)$ und $(-\infty)+\infty$ nicht 
  definiert sind. \AntEnd
\end{antwort}

%% --- 17 --- %%
\begin{frage}\index{Supremum!in $\overline{\RR}$}
  \nomenclature{$\Sup$}{verallgemeinertes Supremum in $\overline{\RR}$}
  Wieso ist f\"ur eine nichtleere Teilmenge $M\subset\RR$ die Definition
  \[
  \Sup( M) = \left\{ \begin{array}{ll} \sup M, & \text{falls $M$ nach oben beschr\"ankt ist,}\\
      \infty, & \text{falls $M$ nicht nach oben beschr\"ankt ist,}
    \end{array}\right.
  \]
  sinnvoll?
\end{frage}

\begin{antwort}
  Ist $M \subset \RR$ eine nach oben unbeschr\"ankte Menge, dann 
  ist $\infty$ aufgrund von {\astref} eine obere Schranke von $M$, 
  gleichzeitig aber auch die \slanted{kleinste} obere Schranke, hat 
  also dieselben Eigenschaften wie das gew\"ohnliche Supremum einer 
  nichtleeren beschr\"ankten Teilmenge von $\RR$. Insofern macht die 
  Definition Sinn.
  \AntEnd 
\end{antwort}

%% --- 18 --- %%
\begin{frage}\index{Bairesche Klasse@Baire\sch e Klasse}
  \index{Funktion!Baire'sche}
  \index{Baire@\textsc{Baire}, Louis (1874-1932)}
  Wie ist die sogenannte \bold{Baire\sch e Klasse} auf $\RR^n$ definiert?
\end{frage}

\begin{antwort}
  Die Baire\sch e Klasse besteht aus Abbildungen $f\fd \RR^n\to \overline{\RR}$, die 
  die folgende Eigenschaft besitzen: 
  Es gibt eine Folge $(f_k)\subset \calli{C}_c ( \RR^n ) $ mit  
  \[
  \begin{array}{lp{3mm}l}
    \text{\desc{i}} & & 
    f_1(x) \le f_2(x) \le f_3(x) \le \ldots \quad 
    \text{f\"ur alle $x\in\RR^n$}, \\[1mm]
    \text{\desc{ii}} & & 
    \dis f(x)=\lim_{k\to\infty} f_k(x) = 
    \Sup \left\{ f_k (x)\sets k\in\NN \right\}.
  \end{array}
  \]
  Gilt \desc{i} und \desc{ii}, dann schreiben wir $f_k \uparrow f$. 
  Für die Baire'sche Klasse sind unterschiedliche Bezeichnungen gebräuchlich,
  etwa $B^+ ( \RR^n )$. Wir verwenden in Anlehnung an \citep{Forster} 
  die suggestive Schreibweise 
  $\calli{H}^{\uparrow}(\RR^n)$ \nomenclature{$\calli{H}^\uparrow$}
  {Baire\sch e Klasse}
  oder auch einfach nur $\calli{H}^{\uparrow}$. Ferner definieren wir 
  $\calli{H}^{\downarrow}( \RR^n ) := - \calli{H}^{\uparrow}( \RR^n ).$ \AntEnd
\end{antwort}

%% --- 19 --- %%
\begin{frage}\label{11_heig}
  Welche Haupteigenschaften hat die Klasse $\calli{H}^{\uparrow} (\RR^n)$?
\end{frage}

\begin{antwort}
  F\"ur $f,g \in \calli{H}^{\uparrow}$ und $C\in \RR$ mit $C\ge 0$ gilt 
  \[
  f+g \in \calli{H}^{\uparrow},\quad
  Cf \in \calli{H}^{\uparrow},\quad 
  f\wedge g \in \calli{H}^{\uparrow},\quad
  f\vee g \in \calli{H}^{\uparrow}, 
  \]
  wie man leicht nachpr\"uft. 
  Man beachte, dass mit $f\in \calli{H}^\uparrow$ noch lange 
  nicht $-f \in \calli{H}^\uparrow $ zu gelten braucht. 
  $\calli{H}^\uparrow$ ist also kein Vektorraum. 



  Die wichtigste Eigenschaft der Baire\sch en Klasse allerdings ist ihre  
  \slanted{Abgeschlossenheit gegen\"uber monotoner Konvergenz}: Ist 
  $(f_k)$ eine Folge von Funktionen $f_k \in \calli{H}^{\uparrow}$, 
  die monoton wachsend gegen eine Funktion $f\fd \RR^n \to\RR$ konvergiert, 
  dann gilt auch $f\in \calli{H}^{\uparrow}$. Das zeigt man, 
  indem man aus den Folgen 
  \[
  f_{k,1},f_{k,2}, f_{k,3},\ldots 
  \qquad f_{k,j}\in \calli{C}_c(\RR^n), \qquad f_{k,j}\uparrow f_k
  \]
  eine geeignete Folge $(g_\ell)$ mit $g_\ell \in \calli{C}_c (\RR^n)$ und 
  $g_\ell \uparrow f$ konstruiert. Man kann nachprüfen, dass etwa die durch 
  \[
  g_{\ell} = \bigvee_{j+k\le \ell} f_{k,j}, \quad 
  \text{also $g_\ell(x)=\dis\max_{j+k\le \ell} \{ f_{k,j}(x) \}$}
  \]
  gegebene Folge das Gewünschte leistet.
  \AntEnd
\end{antwort}

%% --- 20 --- %%
\begin{frage}\label{11_intdef}
  \index{Integral!fuer H@für halbstetige Funktionen}
  Wie wird das Integral f\"ur eine Funktion $f\in\calli{H}$ 
  definiert?                                             
\end{frage}

\begin{antwort}
  Das Integral $\widetilde{I}(f)$ für eine Funktion 
  $f\in\calli{H}^\uparrow$ wird in naheliegender Weise als der Grenzwert der
  Integrale der approximierenden Folge aus $\calli{C}_c(\RR^n)$ definiert. 
  Da dieser Grenzwert, wie in der nächsten Frage gezeigt wird, unabhängig 
  von der approximierenden Funktionenfolge ist, ist diese Definition sinnvoll. 

  Ist $f\in \calli{H}^{\uparrow}$ und $(f_k)$ eine Folge mit 
  $f_k \in \calli{C}_c(\RR^n)$ und $f_k \uparrow f$, dann definiert man 
  also durch  
  \[
  \boxed{
    \widetilde{I}(f) := \lim_{k\to\infty} I(f_k) = 
    \Sup_k \,\{ I(f_k) \} \in \overline{\RR} 
  }
  \]
  das \slanted{Integral von $f\in \calli{H}^{\uparrow} ( \RR^n )$}. 

  Da die Folge $\big( I(f_k) \big)$ monoton 
  wächst, existiert der Grenzwert immer im 
  eigentlichen oder uneigentlichen Sinn. 
  F\"ur $f\in \calli{C}_c(\RR^n)$ ist offensichtlich $\widetilde{I}(f)=I(f)$, 
  aus diesem Grund 
  verzichten wir im Folgenden zur Vereinfachung der Bezeichnungen auf 
  die Tilde $\widetilde{\quad}$. Ferner verwenden wir die Bezeichnung 
  $\lim_{k\to\infty} a_k$ stets als Synonym für $\Sup_k \{ a_k \}$. 
  \AntEnd
\end{antwort}

%% --- 21 --- %%
\begin{frage}
  Warum ist die Integraldefinition in Frage 
  \ref{11_intdef} unabh\"angig von der approximierenden Folge?
\end{frage}

\begin{antwort}
  F\"ur zwei Folgen $(f_k)$ und $(g_k)$ aus $\calli{C}_c(\RR^n)$ 
  mit $f_k \uparrow f$ und $g_k \uparrow g$ ist zu zeigen:
  \[
  \lim_{k\to\infty} I(f_k) = \lim_{k\to\infty} I(g_k).
  \]
  Zu festem $\ell\in\NN$ und jedem $k\in\NN$ sei
  \[
  h_k := g_k \wedge f_\ell. 
  \]
  Die Folge $(h_k)$ liegt dann in $\calli{C}_c (\RR^n)$, und es gilt 
  $h_k \!\uparrow \! f_\ell$. Wegen $\Tr( h_k) \subset \Tr( f_\ell )$ 
  konvergiert $(h_k)$ nach dem Satz von Dini   
  sogar gleichm\"a{\ss}ig gegen $f_\ell$, und daher gilt:  
  \[
  I (f_\ell) = \lim_{k\to\infty} I(h_k) \qquad\text{f\"ur alle $\ell\in\NN$}.
  \]
  Wegen $h_k \le g_k$ folgt daraus aufgrund der Monotonie des Integrals 
  \[
  I(f_\ell) \le \lim_{k\to\infty} I(g_k) 
  \qquad\text{f\"ur alle $\ell\in\NN$}, 
  \]
  also $\lim\limits_{\ell\to\infty} I(f_\ell) \le
  \lim\limits_{k\to\infty}  I(g_k)$. Aus Symmetriegründen gilt aber 
  auch $\lim\limits_{k\to\infty} I(g_k) \le 
  \lim\limits_{\ell\to\infty} I(f_\ell)$,  
  da die Rollen von $(g_k)$ und $(f_\ell)$ 
  in dem Beweis vertauscht werden können. 
  Insgesamt folgt daraus die Behauptung. 

  Man beachte, dass hier in einem zentralen Argumentationsschritt der 
  Satz von Dini und damit letztendlich die Stabilitätseigenschaften 
  gleichmäßig konvergenter Funktionenfolgen benutzt wurden, 
  um die Vertauschung von Integration und Limesbildung zu rechtfertigen. 
  \AntEnd
\end{antwort}

%% --- 22 --- %%
\begin{frage}\index{Integral!fuer H@für halbstetige Funktionen}
  Welche \bold{Permanenzeigenschaften} besitzt die Abbildung 
  \[
  I \fd \calli{H}^{\uparrow} \to \overline{\RR},\qquad 
  f \mapsto I(f)\, \text{?}
  \]
\end{frage}

\begin{antwort}
  Es gelten die folgenden Eigenschaften: 

  \slanted{
    \setlength{\labelsep}{4mm}
    \begin{itemize}
    \item[\desc{i}] Mit $f,g\in \calli{H}^{\uparrow}$ 
      und $C\ge 0$ 
      liegen nach Frage 
      \ref{11_heig} auch $f+g$ und $C\cdot f$ in 
      $\calli{H}^{\uparrow}$, und es gilt $I(f+g)=I(f)+I(g)$ 
      sowie $I(C\cdot f)=C\cdot I(f)$.\\[-3.5mm]
    \item[\desc{ii}] Aus $f\le g$ folgt $I(f) \le I(g)$.\\[-3.5mm]
    \item[\desc{iii}] F\"ur eine Folge $(f_k)$ von Funktionen 
      $f_k\in\calli{H}^{\uparrow}$ mit $f_k \uparrow f$ 
      ist nach Frage \ref{11_heig} auch $f\in\calli{H}^{\uparrow}$, und 
      es gilt 
      \[
      I \left( \lim\limits_{k\to\infty} f_k \right) =I(f) = 
      \lim\limits_{k\to\infty} I(f_k).
      \]
      Das heißt, dass 
      auch das auf $\calli{H}^\uparrow$ fortgesetzte Integral $\sigma$-stetig ist.  
    \end{itemize}}
  \noindent
  Die Eigenschaften \desc{i} und \desc{ii} sind offensichtlich. 
  Die Eigenschaft \desc{iii} muss man dagegen 
  wirklich \slanted{beweisen}. Dafür benutzt man die Eigenschaften 
  der in Frage \ref{11_heig} angegeben Folge 
  $(g_\ell)\subset \calli{C}_c(\RR^n)$. Für diese 
  gilt wegen $g_\ell \uparrow f$ 
  nach Definition einerseits $\Sup_\ell I(g_\ell)=I(f)$, 
  andererseits gilt nach Konstruktion auch $g_k \le f_k$ für alle 
  $k$ und damit wegen der Monotonie des Integrals 
  $I(g_k)\le I(f_k) \le I(f)$. Daraus folgt 
  zusammen wie gewünscht $\lim\limits_{k\to\infty} (f_k)=I(f)$.     
  \AntEnd
\end{antwort}

%% --- 23 --- %%
\begin{frage}
  Warum hat eine Funktion $f\in\calli{H}^\uparrow$ 
  die folgende Eigenschaft: 
  Ist $a\in\RR^n$ und $c$ eine beliebige reelle Zahl mit 
  $c<f(a)$, dann gibt es eine 
  Umgebung $U(a)$ von $a$ mit der Eigenschaft $c<f(x)$ 
  f\"ur alle $x\in U(a)$.
\end{frage}

\begin{antwort}
  Sei $(f_k)$ eine Folge von Funktionen $f_k \in \calli{C}_c(\RR^n)$ 
  mit $f_k \uparrow f$. Dann gilt insbesondere 
  $f_k(a) \uparrow f(a)$. Wegen $f(a) = \Sup_k\{ f_k(a) \}$ 
  gibt es nach der Definition des Supremums einen Index $N$ mit 
  $c < f_N (a) \le f(a)$. 
  Da $f_N$ stetig ist, gilt $c< f_N(x)$ auch noch in einer 
  Umgebung $U(a)$ von $a$. 
  F\"ur alle $x\in U(a)$ gilt dann aber erst recht 
  $c< f(x)$. \AntEnd
\end{antwort}

%% --- 24 --- %%
\begin{frage}\index{halbstetig}
  Wann heißt eine Funktion $f\fd \RR^n \to\overline{\RR}$ 
  in einem Punkt $a\in \RR^n$ (bzw. in $\RR^n$)
  \bold{von unten bzw. von oben halbstetig?}
\end{frage}

\begin{antwort}
  \desc{a} Die Funktion $f$ hei{\ss}t 
  \slanted{von unten halbstetig} in $a$, wenn zu jedem 
  $c\in\RR$ mit $c<f(a)$ eine Umgebung $U(a)$ existiert, 
  in der die Ungleichung immer noch gilt: 
  \[
  c<f(x) \quad\text{f\"ur alle $x\in U(a)$}.
  \]
  \desc{b} Analog heißt $f$ \slanted{von oben halbstetig} in $a$, 
  wenn zu jedem $c\in\RR$ mit $c>f(a)$ eine Umgebung 
  $U(a)$ existiert mit, in der $c>f(x)$ 
  f\"ur alle $x\in U(a)$ gilt, \sieheAbbildung\ref{fig:11_halbstetig1}

  \begin{center}
    \includegraphics{mp/11_halbstetig1}
    \captionof{figure}{Graph einer in $a$ von unten (links) bzw. oben (rechts) 
      halbstetigen Funktion.}
    \label{fig:11_halbstetig1}
  \end{center}

  $f$ hei{\ss}t von \slanted{unten (von oben) halbstetig 
    auf $\RR^n$}, wenn sie in jedem Punkt $a\in\RR^n$ von unten (von oben) 
  halbstetig ist. 
  Insbesondere sind also alle stetigen Funktionen 
  von unten \slanted{und} oben halbstetig. 
  \AntEnd
\end{antwort} 

%% --- 25 --- %%
\begin{frage}\index{charakteristische Funktion}
  \nomenclature{$\chi_M$}{charakteristische Funktion der Menge $M$}
  Wie ist f\"ur eine nichtleere Teilmenge $M\subset \RR^n$ die 
  \bold{charakteristische Funktion} $\chi_M \fd \RR^n \to \RR$ 
  definiert?
\end{frage}

\begin{antwort}
  Die charakteristische Funktion von $M$ ist definiert durch
  \[
  {\chi}_M (x) := \left\{ \begin{array}{ll} 1 & \text{falls $x\in M$} \\
      0 & \text{falls $x\in \RR^n \mengeminus M$}. 
    \end{array}
  \right.
  \EndTag
  \]
\end{antwort} 

%% --- 26 --- %%
\begin{frage}\label{11_charakhalb}
  K\"onnen Sie zeigen: 
  \begin{itemize}[1mm]
  \item[\desc{a}] 
    ${\chi}_M$ ist von unten halbstetig $\LLa$ $M$ ist offen in $\RR^n$, 
    \\[-3.5mm] 
  \item[\desc{b}] ${\chi}_M$ ist von oben halbstetig $\LLa$ $M$ ist 
    abgeschlossen in $\RR^n$ 
  \end{itemize}
\end{frage}

\begin{antwort}
  \desc{a} Sei $c\in \RR$. Es ist 
  \[
  \mathfrak{M}(c) := \{ x\in \RR^n \sets c < \chi_M(x) \} = \left\{ 
    \begin{array}{lll} \RR^n, & \text{falls $c<0$,} \\
      M, & \text{falls $0\le c < 1$,} \\
      \emptyset, & \text{falls $c \ge 1$.}
    \end{array}\right.
  \]
  Ist $M$ offen, so sind alle drei Mengen offen und enthalten 
  zu jedem ihrer Punkte eine volle Umgebung, $\chi_M$ ist also 
  von unten halbstetig. Ist umgekehrt $\chi_M$ halbstetig 
  von unten, dann ist $\mathfrak{M}(c)$ für alle $c \in \RR$ offen, 
  insbesondere ist $M$ offen. 

  \desc{b} Wegen $\chi_{\RR^n\mengeminus M}=1-\chi_M$ lässt sich dieser 
  Fall auf \desc{a} zurückführen.\AntEnd
  
\end{antwort}

%% --- 27 --- %%
\begin{frage}\label{11_foff}\index{triviale Fortsetzung}
  \nomenclature{$\widetilde{f}$}{triviale Fortsetzung von $f$}
  Ist $U\subset \RR^n$ eine nichtleere offene Menge, 
  $f\fd U\to\RR_+$ eine stetige Funktion und 
  \[
  \widetilde{f} \fd \RR^n \to\RR, \qquad 
  \widetilde{f} = \left\{ \begin{array}{ll}
      f(x) & \text{falls $x\in U$}, \\
      0    & \text{falls $x\in \RR^n \mengeminus U$}
    \end{array} \right.
  \]
  die triviale Fortsetzung von $f$. 
  Warum ist dann $\widetilde{f}$ von unten halbstetig?
\end{frage}

\begin{antwort}
  Die Funktion $\widetilde{f}$ ist in jedem Punkt $a\in U$ stetig, 
  und damit f\"ur alle $a\in U$ erst recht halbstetig 
  (von unten und  von oben). 
  In einem Punkt $b\in \RR^n \mengeminus U$ ist sie aber ebenfalls von unten  
  halbstetig, denn aus $\widetilde{f}(b) =0 > c$ folgt wegen 
  $\widetilde{f} \ge 0$, dass sogar f\"ur alle $x\in \RR^n$ die Ungleichung 
  $\widetilde{f}(x)>c$ gilt. 
  \AntEnd
\end{antwort} 

%% --- 28 --- %%
\begin{frage}\label{11_fkom}
  Ist $K\subset\RR^n$ eine nichtleere kompakte Teilmenge des 
  $\RR^n$ und 
  $f\fd K \to \RR_+$ eine stetige Funktion. 
  Warum ist dann die triviale Fortsetzung 
  $\widetilde{f}$ von $f$ von oben halbstetig?
\end{frage}

\begin{antwort}
  Zu jedem Punkt $b\in\RR^n \mengeminus K$ gibt es wegen der Offenheit 
  von $\RR^n \mengeminus K$ auch eine Umgebung $U(b)$, die vollkommen in 
  $\RR^n \mengeminus K$ enthalten ist. 
  Wegen $\widetilde{f}(x)=0$ f\"ur alle $x\in U(b)$ 
  folgt aus $\widetilde{f}(b)<c$ dann auch $\widetilde{f}(x)<c$ f\"ur 
  alle $x\in U(b)$. Also 
  ist $\widetilde{f}$ von oben halbstetig f\"ur $b\in \RR^n \mengeminus K$. 

  Sei nun $a\in K$ und $c>f(a)$. Dann ist $c > 0$, und 
  wegen der Stetigkeit von $f$ gibt es eine Umgebung 
  $U(a)$ mit $c>\widetilde{f}(x)$ f\"ur alle $x\in U(a)\cap K$. Ist 
  $U(a)$ ganz in $K$ enthalten, dann sind wir fertig. 
  Im anderen Fall gibt es ein $y\in U(a)\mengeminus K$. 
  Dann gilt aber $\tilde{f}(y)=0$ und damit ebenfalls $\tilde{f}(y)<c$. 
  Es folgt $\tilde{f}(x)<c$ f\"ur alle $x\in U(a)$, also ist $\tilde{f}$ 
  von oben halbstetig in $a$. 
  \AntEnd
\end{antwort} 

%% --- 29 --- %%
\begin{frage}\label{11_halbstetig}
  \index{Baire'sche Funktion}
  \index{Funktion!halbstetige}
  Wie kann man mithilfe des Begriffs der Halbstetigkeit die Funktionen aus 
  $\calli{H}^\uparrow$ und $\calli{H}^\downarrow$ charakterisieren?
  Warum gilt 
  $\calli{H}^\uparrow \cap \calli{H}^\downarrow = \calli{C}_c$ ? 
\end{frage}

\begin{antwort}
  \satz{Es ist $f\in\calli{H}^\uparrow$ genau dann, wenn gilt
    \setlength{\labelsep}{3mm}
    \begin{enumerate}
    \item[\desc{i}] $f$ ist von unten halbstetig.\\[-3.5mm]
    \item[\desc{ii}] Es gibt ein Kompaktum $K\subset\RR^n$, sodass für alle 
      $x\in \RR^n\mengeminus K$ gilt: $f(x)\ge 0$ (äquivalent hierzu: 
      es gibt ein $g\in\calli{C}_c(\RR^n)$ mit $g\le f$. 
    \end{enumerate}} 
  \noindent
  Die eine Richtung der Aussage ist klar. 
  Die Umkehrung erfordert einigen Aufwand. Einen Beweis findet man 
  {\zB} bei \citep{Forster}.
  \AntEnd
\end{antwort} 

%% --- 30 --- %%
\begin{frage}
  Zu einer nichtleeren offenen Menge $U\subset \RR^n$ und ihrer 
  charakteristischen Funktion 
  $\chi_U$ kann man direkt eine Folge $(\chi_k)$ mit 
  $\chi_k \in \calli{C}_c$ und $\chi_k \uparrow \chi_U$ konstruieren. 

  Die Existenz einer solchen Folge werde vorausgesetzt. 
  Wie kann man dann zu einer 
  nichtleeren kompakten Menge $K\subset\RR^n$ eine Folge $(\chi_k)$ mit 
  $\chi_k \in \calli{C}_c (\RR^n)$ und $\chi_k \downarrow \chi_{_{\footnotesize K}}$ konstruieren?
\end{frage}

\begin{antwort}
  Da die Menge $\RR^n\mengeminus K$ offen ist, gibt es nach Voraussetzung 
  eine Folge $(\chi_l)$ in $\calli{C}_c$ mit 
  $\chi_l \uparrow \chi_{\RR^n\mengeminus K}$. Wegen 
  $\chi_K = 1-\chi_{\RR^n\mengeminus K}$ und $1-\chi_l \in \calli{C}_c$ 
  gilt dann $1-\chi_l \downarrow \chi_K$. \AntEnd
\end{antwort} 

%% --- 31 --- %%
\begin{frage}\label{11_bairecharak}
  \desc{a} Ist $U\subset \RR^n$ eine nichtleere offene 
  Teilmenge, $f\fd U\to\RR_+$ eine stetige Funktion und 
  $\tilde{f}$ die triviale Fortsetzung von $f$ auf $\RR^n$,  
  warum gilt dann $\tilde{f}\in \calli{H}^\uparrow$? 

  \desc{b} Ist $K\subset\RR^n$ eine nichtleere kompakte Teilmenge und 
  $f\fd K\to\RR_+$ stetig. Warum gilt dann 
  $\widetilde{f}\in \calli{H}^\downarrow$
\end{frage}

\begin{antwort}
  \desc{a} $\widetilde{f}$ erfüllt nach der Antwort zu Frage \ref{11_foff} 
  und wegen $\widetilde{f}\ge 0$ die Voraussetzungen von \desc{ii} aus Frage 
  \ref{11_halbstetig}

  \desc{b} Das folgt aus denselben Gründen wie unter \desc{a} aus 
  Frage \ref{11_fkom} und \ref{11_halbstetig}.\AntEnd 
\end{antwort}

%% --- 32 --- %%
\begin{frage}\label{11_trans}
  \index{Transformationsformel!fuer Baireschen@für Baire\sch e Funktionen}
  Was besagt der Spezialfall der 
  \bold{Transformationsformel} für eine Funktion 
  $f \in \calli{H}^\uparrow (\RR^n)$ und eine affine Abbildung 
  $\varphi \fd \RR^n \to \RR^n$, die durch $\varphi(x) = Ax +b$ 
  mit $A\in \mathrm{GL}(\RR^n)$ gegeben ist?
\end{frage}

\begin{antwort}
  Die Transformationsformel lässt sich als Verallgemeinerung 
  der Substitutionsregel für Regelintegrale auffassen. 
  Sie lautet für den in der Frage formulierten Spezialfall 
  \[
  \boxed{
    \int_{\RR^n} f(y) \dify = 
    \int_{\RR^n} f(Ax+b) \big| \det A \big| \difx.
  }
  \]   
  Um die Transformationsformel in diesem Spezialfall zu beweisen, 
  betrachte man zunächst eine Funktion $g\in \calli{C}_c(\RR^n)$. 
  Für diese lässt sich das Integral $\int_{\RR^n} g(x) \difx$ 
  auf iterierte Regelintegrale über $\RR$ zurückführen. 
  Durch Anwendung der Substitutionsformel für das Regelintegral erhält man 
  \begin{align}
    &\int g(x_1,\ldots, x_{i-1}, x_i+x_j, x_{i+1}, \ldots, x_n ) 
    \difx = \int f(x) \difx. \tag{1}\\
    &\int g(a_1 x_1,\ldots, a_n x_n ) 
    \difx = \big| a_1\cdots a_n \big| \int g(x) \difx. \tag{2}\\
    &\int g(x_1,\ldots, x_n ) \difx_{\sigma(1)} \cdots \difx_{\sigma(n)} 
    = \int g(x) \difx_1\cdots \difx_n. \tag{3}\\
    &\int g(x+b)\difx =\int g(x) \difx.\tag{4}
  \end{align}
  Man betrachte nun $\int g( Ax+b) \difx$. Wegen \desc{4} kann man 
  {\oBdA} $b=0$ annehmen. Nun muss man aus der Linearen Algebra nur wissen, 
  dass sich die Matrix $A$ als Produkt von Matrizen $A_1 \cdots A_\ell$ 
  schreiben lässt, die jeweils eine der Variablentransformationen 
  des Typs \desc{1} bis \desc{3} beschreiben. 
  Wegen $\det(A)=\det(A_1)\cdots \det(A_\ell)$ folgt daraus 
  die Transformationsformel für Funktionen $g\in \calli{C}_c (\RR^n)$. 

  Für $f\in \calli{H}^\uparrow$ folgt die Transformationsformel nun 
  einfach aus der Tatsache, dass mit $f_k \uparrow f$ und 
  $f_k \in \calli{C}_c$ auch die Funktionen $g_k\fd\RR^n\to\RR$ mit 
  \[
  g_k (x) = f_k(Ax+b)
  \]
  in $\calli{C}_c$ liegen und monoton wachsend gegen $f\circ\varphi$ 
  konvergieren. Nach der Definition des Integrals für 
  Funktionen der Baire'schen 
  Klasse gilt also
  \[
  \int f (y) \dify = 
  \lim_{k\to\infty} \int f_k(y) \difx = 
  \lim_{k\to\infty} \int f_k(Ax+b) \big|\det A \big|  \difx  = 
  \int  f (Ax+b) \big|\det A \big|  \difx.
  \]
  Das beweist die Transformationsformel im Fall 
  $f\in \calli{H}^\uparrow$ und $\varphi(x)=Ax+b$. In Frage 
  \ref{11_transformationsformel} 
  formulieren wir eine wesentliche Verallgemeinerung für 
  Lebesgue-integrierbare Funktionen $f$ und Diffeomorphismen $\varphi$. 
  \AntEnd  
\end{antwort} 

%% --- 33 --- %%
\begin{frage}\label{11_kleinerfubini}\index{Satz!von Fubini, kleiner}
  Was besagt der \bold{(kleine) Satz von Fubini} für Funktionen 
  $f\in\calli{H}^\uparrow$?
\end{frage}

\begin{antwort}
  Der Satz lautet in diesem Spezialfall:

  \medskip\noindent
  \satz{Sei $n=k+m$ und 
    $f \fd \RR^n \to \RR$ eine 
    Funktion aus $f\in\calli{H}^\uparrow (\RR^{n})$. 
    Dann liegen für $(x,y) \in \RR^n = 
    \RR^{k+m}=\RR^k\times\RR^m$ die Funktionen  
    \[
    F\fd \RR^k\to\RR,\quad x \mapsto \int_{\RR^m} f(x,y)\dify, \qquad 
    G\fd \RR^m\to\RR,\quad y \mapsto \int_{\RR^k} f(x,y)\difx 
    \]
    in $\calli{H}^\uparrow (\RR^k)$ bzw. $\calli{H}^\uparrow (\RR^m)$ 
    und es gilt die Formel
    \[
    \boxed{
      \int_{\RR^{n}} f(x,y)\dd (x,y) = 
      \int_{\RR^k} F(x) \difx = 
      \int_{\RR^m} G(y) \dify.
    }
    \]
  }
  \noindent
  Der Zusammenhang ist aufgrund von Frage 
  \ref{11_integrationsreihenfolge} richtig für stetige 
  Funktionen mit kompaktem Träger. Für $f\in\calli{H}^\uparrow (\RR^{n})$ 
  wähle man eine Folge $(f_\ell)$ mit $f_\ell\in\calli{C}_c(\RR^{n})$ 
  und $f_\ell\uparrow f$. Dann ist 
  \[
  \int_{\RR^{n}} f(x,y)\dd (x,y) = 
  \lim_{\ell\to\infty} \int_{\RR^{n}} f_\ell(x,y)\dd (x,y) = 
  \lim_{\ell\to\infty} \int_{\RR^k} 
  \left(\int_{\RR^{m}} f_\ell(x,y)\dify \right)\difx. 
  \asttag
  \]
  Bei festgehaltenem $x$ gilt dann
  \[
  F_\ell(x):=\int_{\RR^m} f_\ell(x,y) \dify 
  \;\;\Big{\uparrow}\;\; \int_{\RR^m} f(x,y) \dify = F(x).  
  \]
  Die Funktionen $F_\ell$ verschwinden außerhalb eines Kompaktums und sind 
  nach Frage \ref{11_paramstet} stetig, also 
  ist $F\in\calli{H}^\uparrow (\RR^k)$. Mit der Definition des Integrals 
  für halbstetige Funktionen folgt also
  \[
  \int_{\RR^{n}} f(x,y) \dd (x,y) = 
  \lim_{\ell\to\infty} \int_{\RR^k} F_\ell(x) \difx = 
  \int_{\RR^k} \lim_{\ell\to\infty} F_\ell(x) \difx = :
  \int_{\RR^k} F(x) \difx.
  \]
  Aus Symmetriegründen und der Vertauschbarkeit der Integrationsreihenfolge 
  im letzten Term von {\astref} gilt ebenso 
  \[
  \Int_{\RR^{n}} f(z)\difz = \Int_{\RR^{m}} G(y) \dify. \EndTag
  \]
\end{antwort}




\section{Berechnung von Volumina einiger kompakter Mengen}

Da f\"ur ein nichtleeres Kompaktum $K\subset\RR^n$ gilt 
$\chi_K \in \calli{H}^\downarrow (\RR^n)$, und das 
$n$-dimensionale Volumen durch 
\[
\vol_n (K) := \int_{\RR^n} \chi_K (x) \difx 
\]
\index{Volumen!n-dimensionales@$n$-dimensionales}
\index{Volumen!elementargeometrisches}
\nomenclature{$v_n(G)$}{elementargeometrisches Volumen von $G$}
\noindent%
definiert ist, kann man f\"ur zahlreiche geometrische K\"orper wie 
Quader, Zylinder, Simplices und Kugeln ihre Volumina berechnen. 
Entscheidende Hilfsmittel sind der Satz von Fubini 
und ein Spezialfall des 
Cavalieri-Prinzips.

%% --- 34 --- %%
\begin{frage}\label{11_quader}
  Warum stimmen im Fall eines achsenparallelen Quaders 
  $Q=[a_1,b_1]\times \cdots \times [a_n,b_n]$ das elementargeometrische 
  Volumen $v_n(Q) = (b_1-a_1)\cdots (b_n-a_n)$ und das mithilfe des 
  Integrals 
  \[
  \vol_n(Q) = \int_{\RR^n} \chi_Q (x) \difx 
  \]
  definierte Volumen \"uberein? Ist $\varphi\fd \RR^n\to\RR^n$ die durch 
  $x\mapsto Ax+b$ mit $A\in O(n,\RR)$ und $b\in\RR^n$ gegebene Abbildung. 
  Warum gilt dann $\vol_n\big( \varphi(Q) \big) = \vol_n (Q)$ ?
\end{frage}

\begin{antwort}
  Es ist $\chi_Q(x_1,\ldots,x_n) = 
  \chi_{[a_1,b_1]} (x_1) \cdots \chi_{[a_n,b_n]} (x_n)$, wenn 
  $Q$ ein achsenparalleler Quader ist. Da außerdem nach Frage 
  \ref{11_charakhalb} $\chi_Q\in\calli{H}^\downarrow$ gilt,
  folgt mit dem "`kleinen"' Satz von Fubini 
  \[
  \vol_n (Q) = \Int_\RR \cdots \Int_\RR 
  \chi_Q(x_1,\ldots,x_n) \difx_1\cdots \difx_n = 
  \Int_\RR \chi_{[a_1,b_1]} (x) \difx \cdots 
  \Int_\RR \chi_{[a_1,b_1]} (x) \difx. 
  \]

  Im Sinne der Abbildung lassen sich die Funktionen $\chi_{[a_k,b_k]}$ 
  durch eine Folge von "`Trapezfunktionen"' $f_{k,i}$ 
  (\sieheAbbildung\ref{fig:11_charak}) approximieren.

  \begin{center}
    \includegraphics{mp/11_charak}
    \captionof{figure}{Approximation von $\chi_{[a_k,b_k]}$ durch 
      Trapezfunktionen.}
    \label{fig:11_charak}
  \end{center}


  Es gilt $f_{k,i}\in\calli{C}_c(\RR^n)$, 
  $f_{k,i} \downarrow \chi_{[a_k,b_k]}$ und 
  $\lim_{i\to\infty} I(f_{k,i})= b_k-a_k$. Das beantwortet die 
  erste Frage.     

  Die zweite Behauptung folgt unmittelbar aus der Transformationsformel, 
  da für eine Matrix $A\in O(n,\RR)$ stets $| \det A | =1$ gilt. 
  \AntEnd
\end{antwort}

%% --- 35 --- %%
\begin{frage}\index{Cavalieri-Prinzip}\index{Schnittmenge}
  Was besagt das \bold{Cavalieri-Prinzip} zur Berechnung des $n$-dimensionalen 
  Volumens einer kompakten Teilmenge $K\subset\RR^n$ mithilfe 
  von $(n-1)$-dimensionalen "`Schnittmengen"'? 
  Was besagt das klassische Prinzip von Cavalieri?
\end{frage}


\begin{antwort}[]%
  \Ant F\"ur $t\in \RR$ sei 
  \[
  K(t):= \{ (x_1,\ldots,x_{n-1}) \in \RR^{n-1} \sets 
  (x_1,\ldots,x_{n-1}, t ) \in K\} 
  \]
  die $(n-1)$-dimensionale "`Schnittmenge"' von $K$ zum Wert  
  $x_n=t$ (\sieheAbbildung\ref{fig:11_cavalieri}). 
  Nach dem \slanted{Cavalieri-Prinzip} gilt dann 
  \[
  \boxed{
    \vol_n( K ) = \int_\RR \vol_{n-1} K(t) \dift.}
  \]

  \begin{center}
    \includegraphics{mp/11_cavalieri}
    \captionof{figure}{"`Schnittmenge"' $K(t)$ von $K$ zum Wert $t$}
    \label{fig:11_cavalieri}
  \end{center}

  (Die Bezeichnung "`Schnittmenge"' f\"ur $K(t)$ ist 
  sprachlich nicht ganz richtig. Die 
  Schnittmenge ist eigentlich $K(t) \times \{ t \}$.)

  Das klassische Prinzip von Cavalieri besagt: Sind $K,L\in\RR^n$ zwei 
  vorgegebene Kompakta, f\"ur die 
  $\vol_{n-1}\big( K(t) \big)=\vol_{n-1} \big( L(t) \big)$ f\"ur 
  \slanted{jedes} $t\in\RR$ gilt, dann gilt auch $\vol_n(K)=\vol_n(L)$ 
  (\sieheAbbildung\ref{fig:11_cavalieri2}). 
  \AntEnd

  \begin{center}
    \includegraphics{mp/11_cavalieri2}
    \captionof{figure}{Klassisches Prinzip von Cavalieri: Die horizontalen 
      Schnitte mit den Figuren sind jeweils gleich lang, also haben beide 
      Figuren dasselbe Volumen.}
    \label{fig:11_cavalieri2}
  \end{center}
\end{antwort}

%% --- 36 --- %%
\begin{frage}\label{11_kappa}
  \index{Volumen!der $n$-dimensionalen Einheitskugel}
  \nomenclature{$K_n(1)$}{Einheitskugel im $\RR^n$}
  \nomenclature{$\kappa_n$}{Volumen der $n$-dimensionalen Einheitskugel}
  K\"onnen Sie mithilfe des Cavalieri-Prinzips das Volumen 
  der $n$-dimensionalen Kugel 
  $K_n (r) = \{ x\in\RR^n \sets \nb{ x }_2 \le r\} $ berechnen?
\end{frage}

\begin{antwort}
  Die Kugel $K_n(r)$ ist das Bild der Einheitskugel $K_n(1)$ unter 
  der Abbildung $\RR^n \to \RR^n $ mit $x \mapsto rx$. Diese 
  Abbildung hat die Determinante $r^n$. 
  Damit folgt aus der Transformationsformel zunächst 
  $\vol_n\big( K_n(r)\big) = r^n \vol_n\big( K_n(1)\big)$, und daher genügt es, 
  das Volumen 
  \[
  \kappa_n := \vol_n \big( K_n(1) )
  \]
  zu berechnen. F\"ur $n=1$ gilt $K_1 = [-1,1]$ und man erh\"alt $\kappa_1=2$. 

  \begin{center}
    \includegraphics{mp/11_kugel}
    \captionof{figure}{
      Durch Betrachtung der "`Schnitte"' $K(1,t)$ lässt sich 
      die Volumenberechung der $n$-dimensionalen Einheitskugel auf die Dimension 
      $n-1$ zurückführen.
    }
    \label{fig:11_kugel}
  \end{center}

  Die Berechnung von $\kappa_n$ f\"ur $n>1$ l\"asst sich nun  
  mit dem Cavalieri-Prinzip auf die von $\kappa_{n-1}$ 
  zur\"uckf\"uhren.  
  Die Mengen $K_n(1;t)$ (vgl. Abbildung~\ref{fig:11_kugel}) haben das Volumen 
  \[
  \vol_{n-1}\big( K_n(1;t) \big) = \left\{ \begin{array}{ll}
      \kappa_{n-1} \sqrt{1-t^2}^{n-1} & \text{f\"ur $|t|\le 1$,}  \\
      \emptyset   & \text{f\"ur $|t|>1$.}
    \end{array}\right.
  \]
  Also erh\"alt man mit dem Cavalieri-Prinzip und der Transformationsformel 
  \begin{align*}
    \kappa_n &= \vol_n\big( K_n(1) \big) = 
    \int_{-1}^1 \vol_{n-1}\big( K_{n-1} (1;t) \big) \dift = 
    \kappa_{n-1} \int_{-1}^1 (1-t^2)^{\frac{n-1}{2}} \dift. \\
    &= \kappa_{n-1} \int_\pi^0 \sin^{n-1} x (-\sin x) \difx = 
    \kappa_{n-1} \int_0^\pi \sin^n x \difx = 
    2\cdot \kappa_{n-1} \int_0^{\frac{\pi}{2}} \sin^n x \difx 
  \end{align*}
  \picskip{0}
  F\"ur das Integral $I_n := \int_0^{\pi/2} \sin^n x \difx$
  hatten wir in Frage \ref{06_wallis1} bereits eine Rekursionsformel 
  angegeben und 
  \[
  I_{2n} = \frac{2n-1}{2n}\cdots\frac{3}{4}\cdot\frac{1}{2}\cdot\frac{\pi}{2}, 
  \qquad 
  I_{2n+1} = \frac{2n}{2n+1}\cdots\frac{4}{5}\cdot\frac{2}{3}
  \]
  hergeleitet. Damit gilt 
  $I_n I_{n-1} = \frac{\pi}{2n}$  f\"ur alle $n\in \NN$. 
  F\"ur die Kugelvolumina $\kappa_n$ liefert das die Rekursionsformel
  \[
  \kappa_n = 2 \kappa_{n-1} I_n = 4 \kappa_{n-2} I_n I_{n-1} = 
  \frac{2\pi}{n}\kappa_{n-2}.
  \]
  Damit lassen sich nun alle $\kappa_n$ berechnen. 
  Es gelten die Formeln 
  \[
  \boxed{
    \kappa_{2k} = \frac{1}{k!}\pi^k, \qquad
    \kappa_{2k+1} = \frac{2^{k+1}}{1\cdot 3\cdots (2k+1)}\pi^k.
  }
  \EndTag
  \] 
\end{antwort}

%% --- 37 --- %%
\begin{frage}
  Kennen Sie eine explizite Formel f\"ur das Volumen der 
  $n$-dimensionalen Einheitskugel?
\end{frage}

\begin{antwort}
  Eine einheitliche Formel erh\"alt man mithilfe 
  der $\Gamma$-Funktion. F\"ur gerades $n=2k$ gilt 
  \[
  \Gamma \left( \frac{n}{2}+1 \right) = \Gamma(k+1) = k!, 
  \]
  und f\"ur ungerade $n=2k+1$ hat man
  \begin{align*}
    \Gamma \left(\frac{n}{2} +1 \right) &= \Gamma\left(k+\frac{3}{2}\right) = 
    \left(k+\frac{1}{2}\right)\Gamma\left( k+\frac{1}{2} \right)= 
    \left(k+\frac{1}{2}\right)\left(k-\frac{1}{2}\right)\Gamma\left( k-\frac{1}{2} \right)\\
    &= \cdots 
    = \left( \frac{2k+1}{2} \right)\left( \frac{2k-1}{2} \right) \cdots 
    \left( \frac{3}{2} \right) 
    \left( \frac{1}{2} \right) \Gamma\left(\frac{1}{2}\right).
  \end{align*}
  Wegen $\Gamma(\frac{1}{2})=\sqrt{\pi}$ folgt 
  daraus durch Vergleich mit {\astref} die Formel
  \[
  \boxed{
    \kappa_n = \frac{\pi^{n/2}}{\Gamma( n/2 +1 )} 
  }  \EndTag
  \]
\end{antwort}

%% --- 38 --- %%
\begin{frage}
  Ist $\kappa_n=\vol_n \big( K_n(1) \big)$, f\"ur welches $n$ 
  ist $\kappa_n$ dann maximal?
\end{frage}

\begin{antwort}
  Aus den Rekursionsformeln am Ende von Frage \ref{11_kappa} folgt 
  für $k\in \NN$  
  \begin{align*}
    \frac{\kappa_{2k}}{\kappa_{2k+1}} = 
    \frac{1}{2} \cdot \left( 
      \frac{3}{2}\cdot \frac{5}{4} \cdots \frac{2k+1}{2k} \right) > 1 
    &\LLa  2k+1 > 5, \\[1mm]
    \frac{\kappa_{2k+1}}{\kappa_{2k+2}} = 
    \frac{2}{\pi} \cdot \left(  
      \frac{4}{3}\cdot \frac{6}{5} \cdots \frac{2k+2}{2k+1} \right) > 1 
    &\LLa  2k+1 \ge 5.
  \end{align*}
  Daraus folgt $\kappa_1  \kappa_2 \le \cdots \le \kappa_5$ 
  und $\kappa_{n+1}\le \kappa_n$ für $n\ge 5$. Damit nimmt 
  $\kappa_n$ genau f\"ur $n=5$ ein Maximum an.  
  Die Tabelle listet die ersten zehn Werte von $\kappa_n$ auf. 
  \[ 
  \begin{array}{c|cccccccccc}
    $n$ & 1 & 2 & 3 & 4 & 5 & 6 & 7 & 8 & 9 & 10\\ \hline
    \kappa_n & 2 & \pi & \frac{4}{3}\pi & \frac{\pi^2}{2} & 
    \frac{8}{15}\pi^2 & \frac{\pi^3}{6} & \frac{16}{105}\pi^3 & 
    \frac{1}{24}\pi^4 & 
    \frac{32}{945}\pi^4 & 
    \frac{1}{120}\pi^5
    \\ \approx &
    2 & 3.141 & 4.189 & 4.935 & 5.264 & 5.168 & 4.729 & 4.059 
    & 3.299 & 2.550 
  \end{array}
  \EndTag
  \]
\end{antwort}

%% --- 39 --- %%
\begin{frage}
  Warum gilt $\lim\limits_{n\to\infty}\kappa_n=0$?
\end{frage}

\begin{antwort}
  Aus den Ungleichungen der letzten Antwort folgt, 
  dass die Folge $(\kappa_n)$ f\"ur $n>5$ streng monoton f\"allt. 
  Da alle Glieder positiv sind, besitzt sie also einen Grenzwert. Es gilt 
  \[
  \lim_{n\to\infty} \kappa_n = \lim_{k\to\infty } \kappa_{2k} = 
  \lim_{k\to\infty}\frac{1}{k!}\pi^k = 0.
  \EndTag
  \]
\end{antwort} 

\section{Die Lebesgue-integrierbaren Funktionen}

Wir erweitern nun den Integralbegriff ein weiteres Mal. Dazu definieren 
wir f\"ur beliebige Funktionen $f\fd\RR^n \to\overline{\RR}$ ein Oberintegral 
$I^*(f)$ und ein Unterintegral $I_* (f)$. F\"ur die Funktionen 
$f\in\calli{H}^\uparrow$ bzw. $f\in\calli{H}^\downarrow$ 
wird sich dadurch nichts Neues ergeben. Die Funktionen 
$f\fd \RR^n  \to\overline{\RR}$, f\"ur welche 
$I(f):= I_*(f)=I^*(f)$ gilt und f\"ur welche dieser gemeinsame Wert 
endlich ist, sind genau die Lebesgue-integrierbaren Funktionen. 
Die Lebesgue-integrierbaren Funktionen 
mit Werten in $\RR$ 
bilden einen Vektorraum $\calli{L}^1 ( \RR^n )$, 
auf dem
\nomenclature{$\calli{L^1}(\RR^n)$}{Raum der Lebesgue-integrierbaren 
  Funktionen $\RR^n\to\RR$} 
\index{Lebesgue-integrierbar}\index{LaaL1@$\calli{L}^1$}
\[
I\fd \calli{L}^1 (\RR^n)\to\RR \qquad 
f \mapsto I(f)
\]
ein nichtnegatives lineares Funktional ist und in dem 
starke Konvergenzs\"atze gelten 
(Satz von Beppo Levi, Grenzwertsatz von Lebesgue), die unter 
geeigneten Voraussetzungen 
gestatten, aus der punktweisen Konvergenz 
auf die Integration der Grenzfunktion und 
auf die Vertauschbarkeit von Grenzwertbildung und 
Integration zu schlie{\ss}en.   

%% --- 40 --- %%
\begin{frage}\index{Oberintegral}\index{Unterintegral}
  \nomenclature{$I^*(f),\,I_*(f)$}{Ober- bzw. Unterintegral von $f$}
  Wie sind f\"ur eine Funktion $f\fd \RR^n\to\overline{\RR}$ das 
  \bold{Ober- bzw. Unterintegral} definiert?
\end{frage}

\begin{antwort}
  F\"ur $f\fd\RR^n\to\overline{\RR}$ hei{\ss}t 
  \begin{align*}
    I^*(f) = \Inf\, \{ I(h)\sets h\in\calli{H}^\uparrow,\, h\ge f \}
    & \quad\text{das \slanted{Oberintegral von $f$} und} \\
    I_*(f) = \Sup\, \{ I(g)\sets g\in\calli{H}^\downarrow,\, g\le f \}
    & \quad\text{das \slanted{Unterintegral von $f$}}.
  \end{align*}
  Wegen $\calli{H}^\downarrow = - \calli{H}^\uparrow $ gilt 
  stets $I_*(f)= -I^*(-f)$. \AntEnd
\end{antwort} 

%% --- 41 --- %%
\begin{frage}
  Warum sind die Mengen, \"uber die das Infimum bzw. Supremum gebildet werden, 
  nicht leer?
\end{frage}

\begin{antwort}
  Die konstante Funktion $h\fd \RR^n \to\overline{\RR}$ mit 
  $x\mapsto \infty$ liegt in 
  $\calli{H}^\uparrow$, und die konstante Funktion 
  $g\fd \RR^n \to\overline{\RR}$ mit $x\mapsto -\infty$ 
  liegt in $\calli{H}^\downarrow$. 

  \parpic[r]{}
  Beide Behauptungen zeigt man, indem man explizit eine Folge aus 
  $\calli{C}_c(\RR^n)$ angibt, die monoton wachsend gegen $\infty$ bzw. 
  monoton fallend gegen $-\infty$ konvergiert, wie etwa für $n=1$ 
  die Folge der "`Trapezfunktionen"' in der Abbildung~\ref{fig:11_dreieck}.

  \begin{center}
    \includegraphics{mp/11_dreieck}
    \captionof{figure}{Die Folge der Trapezfunktionen konvergiert 
      monoton wachsend gegen $\infty$.}
    \label{fig:11_dreieck}
  \end{center}


  Die Konstruktion lässt sich leicht auf 
  höhere Dimensionen verallgemeinern, 
  \AntEnd
\end{antwort} 



%% --- 42 --- %%
\begin{frage}\label{11_lebeab}
  Warum gilt f\"ur alle Funktionen $f\fd \RR^n \to \overline{\RR}$ stets $
  I_*(f) \le I^*(f)$,
  und warum gilt f\"ur $f\in \calli{H}^\uparrow (\RR^n)$ stets $
  I(f)= I_*(f)=I^*(f)$?
\end{frage}

\begin{antwort}
  F\"ur den ersten Teil der Frage muss 
  nur gezeigt werden, dass mit $h \in \calli{H}^\uparrow$ 
  und $g\in \calli{H}^\downarrow$ aus $h\ge g$ stets 
  $I(h) \ge I(g)$ folgt. 
  Wegen $\calli{H}^\downarrow = - \calli{H}^\uparrow$ 
  ergibt sich das aus den Monotonieeigenschaften 
  des Integrals f\"ur halbstetige Funktionen:
  \[
  0 \le I(h-g) = I(h+(-g))=I(h)-I(g). \asttag
  \]
  Zum zweiten Teil: Die Identit\"at $I^* (f)=f$ f\"ur 
  $f\in \calli{H}^\uparrow$ folgt direkt aus der Definition des 
  Oberintegrals. Um $I_*(f)=I(f)$ zu zeigen, w\"ahle man eine Folge 
  $(f_k)\subset \calli{C}_c (\RR^n)$ mit $f_k \uparrow f$. Dann gilt 
  nach Definition 
  \[
  I(f) = \Sup_k \int_{\RR^n} f_k (x) \difx. \aasttag
  \]
  Andererseits folgt aus der Definition des Unterintegrals 
  zusammen mit $f_k \le f$, 
  $f_k \in \calli{C}_c \subset \calli{H}^\downarrow$ 
  und {\astref}
  \[
  \Sup_{k} \int_{\RR^n} f_k (x) \difx \le I_*(f) \le I^*(f) = I(f).
  \]
  Wegen {\astastref} muss hier \"uberall ein Gleichheitszeichen 
  stehen.\AntEnd
\end{antwort} 

%% --- 43 --- %%
\begin{frage}\index{Lebesgue-integrierbar}
  Wann hei{\ss}t eine Funktion $f\fd \RR^n\to\overline{\RR}$ 
  \bold{Lebesgue-integrierbar}?
\end{frage}

\begin{antwort}
  
  Eine Funktion $f\fd \RR\to\overline{\RR}$ hei{\ss}t 
  \slanted{Lebesgue-integrierbar} genau dann, wenn 
  \[
  \boxed{ I^*(f)=I_*(f) }
  \]
  gilt und dieser gemeinsame Wert von $\infty$ und $-\infty$ verschieden ist. 
  Der gemeinsame Wert hei{\ss}t in diesem Fall das 
  \slanted{Integral von $f$} und wird wiederum mit $I(f)$ bezeichnet. 
  \AntEnd
\end{antwort} 

%% --- 44 --- %%
\begin{frage}
  Welche Bedingung ist für eine Funktion $f\in \calli{H}^\uparrow$ 
  (bzw. $f\in \calli{H}^\downarrow$) notwendig und hinreichend 
  für ihre Lebesgue-Integrierbarkeit?
\end{frage}

\begin{antwort}
  Nach Frage \ref{11_lebeab} gilt für halbstetige Funktionen 
  $f$ stets $I^*(f)=I_*(f)$. Für ihre Integrierbarkeit ist es 
  daher hinreichend und notwendig, dass $I^*(f)$ einen endlichen Wert 
  hat. \AntEnd
\end{antwort} 

%% --- 45 --- %%
\begin{frage}\label{11_lebekrit}\index{epsilon@$\eps$-Kriterium 
    für Integrierbarkeit}
  Kennen Sie ein Kriterium f\"ur die Lebesgue-Integrierbarkeit einer 
  Funktion $f\fd \RR^n \to \overline{\RR}$, in welcher $\Sup$ und $\Inf$ 
  nicht vorkommen?
\end{frage}

\begin{antwort}
  Durch eine Anwendung der Definition 
  des Supremums erh\"alt man {\zB} das folgende 
  (häufig $\eps$-Kriterium genannte) Kriterium: 

  \medskip\noindent
  \satz{Eine Funktion $f\fd \RR^n\to\overline{\RR}$ ist genau dann 
    Lebesgue-integrierbar, wenn es zu jedem $ \eps>0$ Funktionen 
    $ g\in \calli{H}^\downarrow $ und  
    $ h\in \calli{H}^\uparrow $ mit endlichen Integralen gibt, 
    f\"ur die gilt:
    \[
    g\le f\le h \qquad\text{und}\qquad
    0 \le I(h-g) = I(h)-I(g) \le \eps \EndTag
    \]
  }  
\end{antwort}

%% --- 46 --- %%
\begin{frage}\label{11_lebesgue}
  \index{LaaL1@$\calli{L}^1$!Eigenschaften}
  Welche Haupteigenschaften hat die Menge 
  \[
  \boxed{
    \calli{L}^1 = \calli{L}^1 (\RR^n) := 
    \big\{ f \fd \RR^n \to \RR \sets \text{$f$ Lebesgue-integrierbar} \big\}.
  }
  \]
  (Wir schlie{\ss}en hier die Werte $\infty$ und $-\infty$ noch aus. 
  Wie wir sehen werden, bedeutet das aber keine wesentliche 
  Einschr\"ankung.)
\end{frage}

\begin{antwort}
  $\calli{L}^1$ ist ein $\RR$-Vektorraum, {\dasheisst} 
  \[
  f, g \in \calli{L}^1 \Ra f+g \in \calli{L}^1 \text{ und } 
  c f \in \calli{L}^1 \text{ für $c\in\RR$}.
  \]
  Ferner gilt, dass mit $f$ und $g$ auch 
  $f\wedge g$, $f\vee g$,
  $f^+$ und $f^{-}$ und $|f|$ in $\calli{L}^1$ enthalten sind.
  Ist $g$ zus\"atzlich beschr\"ankt und ist $f(x)\not=\pm \infty$ 
  f\"ur alle $x\in \RR^n$, dann liegt auch 
  $fg$ in $\calli{L}^1$. 

  Diese Eigenschaften sind \slanted{nicht} offensichlich, sondern 
  müssen einzeln nachgewiesen werden. Dies gelingt in jedem 
  einzelnen Fall aber mühelos mit den Ergebnissen aus Frage 
  \ref{11_lebeab} und dem Kriterium aus Frage \ref{11_lebekrit}.

  Speziell für den Nachweis der Linearität benutze man 
  folgenden Zusammenhang: Sind $f,g,h$ Funktionen mit $f+g=h$, so gilt
  \[
  I^*(f)+I^*(g) \ge I^*(h), \qquad
  I_*(f)+I_*(g) \le I_*(h). \asttag
  \]
  Dies ist für Baire\sch e Funktionen offensichtlich und folgt daraus 
  für allgemeine Funktionen aus der Definition des Unter- und Oberintegrals. 
  (Man beachte, dass der Zusammenhang $I^*(f+g)=I^*(f)+I^*(g)$ und 
  $I_*(f+g)=I_*(f)+I_*(g)$ im Allgemeinen nicht gilt.) \AntEnd 
\end{antwort} 

%% --- 47 --- %%
\begin{frage}\index{Permanenzeigenschaften!des Lebesgue-Integrals}
  Welche \bold{Permanenzeigenschaften} hat die Abbildung 
  $I\fd\calli{L}^1(\RR^n)\to \RR$?
\end{frage} 

\begin{antwort}
  Die Abbildung $I$ ist ein \slanted{lineares, monotones Funktional}, 
  für $f,g \in \calli{L}^1$ und $a,b\in \RR$ gilt also
  \[
  \begin{array}{rp{2mm}lp{15mm}r}
    \text{\desc{a}} & &
    I(a f+b g)= a \cdot I( f)+b\cdot I(g),  & &
    \text{(Linearität)} \\
    \text{\desc{c}} & &
    f \le g \Ra I(f) \le I(g ). & &
    \text{(Monotonie)}
  \end{array}
  \]
  Ferner ist $\calli{L}^1(\RR^n)$ $\sigma$-stetig 
  im Sinne von Frage \ref{11_sigma}.  
  Ist also $(f_k)$ eine Folge mit $f_k \in \calli{L}^1$ und 
  $f_k \downarrow 0$, dann gilt stets $\lim_{k\to\infty} I(f_k)=0$.
  \AntEnd
\end{antwort}

%% --- 48 --- %%
\begin{frage}\index{LaaL1@$\calli{L}^1$!Halbnorm}
  \nomenclature{$\n{\,\;}_{\calli{L}^1}$}{$\calli{L}^1$-Halbnorm}
  Wie ist die $\calli{L}^1$-Halbnorm für Funktionen 
  $f\fd \RR^n\to\overline{\RR}$ definiert?
\end{frage}

\begin{antwort}
  Die $\calli{L}^1$-Halbnorm $\n{\,\;}_{\calli{L}^1}$ ist 
  definiert durch
  \[
  \nnb{ f }_{\calli{L}^1} = I^*( |f| ).
  \]
  Die Eigenschaften einer Halbnorm folgen für 
  $\nnb{\;\,}_{\calli{L}^1}$ unmittelbar aus denen des Lebesgue-Integrals. 
  Da aber aus $\nnb{ f }_{\calli{L}^1}=0$ nicht $f=0$ folgt, handelt 
  es sich nicht um eine Norm. \AntEnd  
\end{antwort}

%% --- 49 --- %%
\begin{frage}
  Was besagt die Aussage "`$\calli{C}_c(\RR^n)$ ist dicht 
  in $\calli{L}^1(\RR^n)$"'?
\end{frage}

\begin{antwort}
  Die Aussage bedeutet: \satz{Eine Funktion $f\fd \RR^n \to\overline{\RR}$ 
    liegt genau dann im Raum $\calli{L}^1(\RR^n)$, wenn es zu jedem 
    $\eps>0$ eine Funktion $g\in\calli{C}_c(\RR^n)$ gibt mit 
    \[
    \nnb{ f-g }_{\calli{L}^1} = I^*\big( | f-g | \big)<\eps.
    \asttag\]
  }
  \noindent
  Es ist, wenn man die Konstruktion des 
  Lebesgue-Integrals nachvollzieht, relativ klar, 
  dass die Voraussetzungen \slanted{notwendig} sind. Denn $I(f)$ wird 
  durch das Integral einer Baire'schen Funktion beliebig genau approximiert, 
  und dieses wiederum durch das Integral einer stetigen Funktion mit kompaktem 
  Träger.  
  
  Gilt umgekehrt {\astref}, dann gibt es eine Baire'sche Funktion $h$ 
  mit $|f(x)-g(x)| < h(x)$ und $I^*( h ) < \eps$. Die Behauptung folgt 
  dann durch Anwendung des Kriteriums aus Frage \ref{11_lebekrit} auf 
  die Funktionen $g-h \in \calli{H}^\downarrow $ und 
  $g+h \in \calli{H}^\uparrow$. \AntEnd
\end{antwort}



\section{Die Grenzwerts\"atze von Beppo Levi und Lebesgue}

Die St\"arke des Lebesgue-Integrals liegt in seiner 
fabelhaften Stabilit\"at gegen\"uber Grenzprozessen. 
Unter bestimmten Voraussetzungen erlaubt die nur punktweise 
Konvergenz einer Funktionenfolge bereits, auf die 
Integrierbarkeit der Grenzfunktion und die Vertauschbarkeit von Limesbildung 
und Integration zu schließen. 

%% --- 50 --- %%
\begin{frage}\index{Satz!von Beppo Levi}\index{monotone Konvergenz}
  \index{Satz!von der monotonen Konvergenz}
  Was besagt der \bold{Satz von Beppo Levi}?
\end{frage}

\begin{antwort}
  Der Satz wird auch 
  \slanted{Satz von der monotonen Konvergenz} genannt und besagt 
  kurz formuliert, dass f\"ur eine monoton wachsende Folge 
  Lebesgue-integrierbarer Funktionen mit beschr\"ankter Integralfolge 
  auch die Grenzfunktion Lebesgue-integrierbar ist, und dass deren Integral 
  der Grenzwert der Integralfolge ist, genauer:

  \medskip
  \noindent\slanted{Ist $(f_k)$ eine Folge von Funktionen  
    mit $f_k \in\calli{L}^1(\RR^n)$ und $f_{k+1} \ge f_k$ 
    für die die Folge der Integrale $I(f_k)$ beschränkt ist, 
    dann ist auch die 
    punktweise gebildete Grenzfunktion $f=\lim\limits_{k\to\infty} f_k$ 
    Lebesgue-integrierbar, und es gilt:
    \[
    \boxed{
      I(f)=I \left( \lim_{k\to\infty} f \right)= \lim_{k\to\infty} I(f_k).
    } \notag 
    \]
  }  
  \noindent
  Dieser wichtige Satz lässt sich folgendermaßen beweisen. 
  Zunächst gilt wegen $f_k \le f$ 
  \[
  \lim_{k\to\infty} I(f_k) \le I_*(f).
  \]
  Der Satz von Beppo Levi folgt also, wenn man zusätzlich 
  \[
  \lim_{k\to\infty} I (f_k) \ge I^*(f)
  \asttag
  \]
  zeigen kann. Um das zu beweisen, schreiben wir die Funktionen 
  $f_k$ als Summe
  \[
  f_k = \sum_{\ell=1}^k h_\ell , 
  \qquad h_\ell = f_\ell - f_{\ell-1}, \quad f_0 := 0.
  \]
  Die Funktionen $h_\ell$ sind wegen der Vektorraumeigenschaft 
  von $\calli{L}^1$ integrierbar. Nach Definition 
  des Integrals gibt es daher zu jedem $\ell\in\NN$ 
  eine Funktion $\overline{h}_\ell \in \calli{H}^\uparrow$ 
  mit $\overline{h}_\ell \ge h_\ell$ sowie  
  \[
  I ( \overline{h}_\ell ) \le I(h_\ell) + \frac{\eps}{2^\ell}.
  \]
  Aufgrund der Stabilität der Baire\sch en Funktionen bezüglich 
  monotoner Konvergenz (Frage \ref{11_heig}) gilt 
  $\sum_{k=1}^\infty \overline{h}\in \calli{H}^\uparrow$, also 
  ist Integration und Grenzwertbildung vertauschbar.  
  Daraus folgt 
  \[
  I^*(f) = I^* \left( \sum_{\ell=1}^\infty h_\ell \right) \le 
  I \left( \sum_{\ell=1}^\infty \overline{h}_\ell \right) = 
  \sum_{\ell=1}^\infty I (\overline{h}_\ell ) \le 
  \sum_{\ell=1}^\infty \left( I ( h_\ell ) + \frac{\eps}{2^\ell} \right) =
  \lim_{k\to\infty} I(f_k)+\eps.
  \]
  Da $\eps$ beliebig klein gewählt werden kann, folgt daraus {\astref} 
  und damit der Satz von Beppo Levi. 

  Ein entsprechender Zusammenhang gilt für monoton fallende Folgen $(f_k)$, 
  deren Integralfolge nach unten beschränkt ist. Durch Übergang zur Folge 
  $(-f_k)$ führt man das auf die im Satz formulierten Voraussetzungen zurück.
  \AntEnd
\end{antwort}

%% --- 51 --- %%
\begin{frage}
  \index{Satz!von Lebesgue}\index{Satz!von der majorisierten Konvergenz}
  Was besagt der \bold{Lebesgue\sch e Grenzwertsatz 
    (Satz von der majorisierten Konvergenz)}?
\end{frage}

\begin{antwort}
  Der Satz lautet (in einer Formulierung, die zunächst noch keine 
  Ausnahme-Nullmengen zulässt):

  \medskip
  \noindent\slanted{Sei $(f_k) \subset \calli{L}^1 (\RR_n)$ eine Folge 
    Lebesgue-integrierbarer Funktionen, die punktweise gegen eine 
    Funktion $f\fd \RR^n \to \RR$ konvergiert. Gibt es dann eine 
    Funktion $F\fd \RR^n \to \overline{\RR}$ mit $I^*(F)<\infty$, sodass 
    \[
    | f_k | \le F \quad\text{f\"ur alle $k\in\NN$} 
    \]
    gilt, dann ist auch $f$ integrierbar, und es gilt 
    \[
    \boxed{ I(f)=\lim_{k\to\infty} I(f_k). }
    \]}
  \noindent\bold{Anmerkung und Zusatz:} 
  Aufgrund des \slanted{Modifikationssatzes}, der in Frage 
  \ref{11_modisatz} gezeigt wird, genügt es, 
  die Konvergenz nur \slanted{fast überall} zu fordern, was die Voraussetzungen 
  für die Gültigkeit des Satzes abschwächt. 

  \medskip\noindent
  Man kann den Lebesgue\sch en Grenzwertsatz auf den Satz von Beppo Levi 
  zurückführen, indem man die Folge der Funktionen 
  $(\underline{g}_k)$ und $(\overline{g}_k)$ mit 
  \[
  \underline{g}_k ( x ) = \Inf \,\{ f_\ell (x) \sets \ell\ge k \}, \qquad
  \overline{g}_k ( x ) = \Sup \,\{ f_\ell (x) \sets \ell\ge k \}
  \]
  betrachtet. Es gilt dann offensichtlich 
  $\underline{g}_k \uparrow f$ 
  und $\overline{g}_k \downarrow f$. Ferner lässt sich zeigen, dass die 
  $\underline{g}_k$ und $\overline{g}_k$ integrierbar sind 
  (s. \citep{Forster}). Aus der Ungleichungskette 
  \[
  - F \le \underline{g}_k \le f_k \le \overline{g}_k \le F 
  \]
  folgt dann 
  \[
  - I^*(F) \le I (\underline{g}_k) \le I(f_k) \le I(\overline{g}_k) \le I^*(F). 
  \]
  Insbesondere sind die Folgen der Integrale $I (\underline{g}_k)$ und 
  $I (\overline{g}_k)$ beschränkt. Nach dem Satz von Beppo Levi 
  ist $f$ damit integrierbar und es gilt 
  \[
  \lim_{k\to\infty} I (\underline{g}_k) = 
  \lim_{k\to\infty} I (f_k) = 
  \lim_{k\to\infty} I (\overline{g}_k) = 
  I(f). \EndTag
  \]
\end{antwort} 

%% --- 52 --- %%
\begin{frage}\index{messbar}\index{Volumen!einer endlich messbaren Menge}
  Wann hei{\ss}t eine Teilmenge $A\subset\RR^n$ 
  \bold{endlich messbar} und wie ist ihr Volumen definiert?
\end{frage}

\begin{antwort}
  Eine Teilmenge $A\subset\RR^n$ heißt \slanted{endlich messbar} genau 
  dann, wenn ihre charakteristische Funktion Lebesgue-integrierbar ist.
  \AntEnd
\end{antwort}

%% --- 53 --- %%
\begin{frage}\index{Integrierbarkeit!im Sinne des Lebesgue-Integrals}
  Wann hei{\ss}t eine Funktion $f\fd X\to\overline{\RR}$ 
  mit $X\subset\RR^n$ \bold{integrierbar}?
\end{frage}

\begin{antwort}
  Eine Funktion $f\fd X\to\overline{\RR}$ heißt 
  \slanted{integrierbar} genau dann, wenn die triviale Fortsetzung 
  $\widetilde{f}$ Lebesgue-integrierbar ist. 
  In diesem Fall schreibt man auch $\int_X f(x) \difx$ für 
  $\int_{\RR^n} \widetilde{f}(x) \difx$. \AntEnd
\end{antwort}

%% --- 54 --- %%
\begin{frage}
  Warum ist jede stetige beschr\"ankte Funktion $f\fd D \to\RR$ 
  ($D\subset\RR^n$ offen und beschr\"ankt) integrierbar?
\end{frage}

\begin{antwort}
  Die triviale Fortsetzung $\tilde{f}$ gehört nach Frage 
  \ref{11_bairecharak} \desc{a} 
  unter diesen Voraussetzungen zu $\calli{H}^\uparrow$. 
  Wegen der Beschränktheit ist $\widetilde{f}$ integrierbar.  
  \AntEnd 
\end{antwort} 

%% --- 55 --- %%
\begin{frage}\index{Integrierbarkeit!stetiger Funktionen}
  Ist $K\subset\RR^n$ kompakt, warum ist dann jede stetige Funktion 
  \"uber $K$ integrierbar?
\end{frage}
\begin{antwort}
  
  Die Funktion liegt in diesem Fall in 
  $\calli{H}^\downarrow $ (vgl. Frage 
  \ref{11_bairecharak} \desc{b}). 
  \AntEnd 
\end{antwort} 

%% --- 56 --- %%
\begin{frage}
  Warum ist jede beschr\"ankte offene Menge des $\RR^n$ endlich messbar?
  Warum ist jede kompakte Teilmenge des $\RR^n$ endlich messbar?
\end{frage}

\begin{antwort}
  Nach den vorhergehenden beiden Fragen gehören die charakteristischen 
  Funktionen zu $\calli{H}^\uparrow$ bzw. $\calli{H}^\downarrow$ und sind 
  daher integrierbar, da das Oberintegral (im ersten 
  Fall) bzw. das Unterintegral (im zweiten Fall) in $\RR$ liegen. 
  \AntEnd
\end{antwort} 

%% --- 57 --- %%
\begin{frage}\label{11_riemannlebesgue}
  \index{Integral!Zusammenhang zwischen Regel- und Lebesgue-Integral}
  Welcher Zusammenhang besteht zwischen der Integrierbarkeit einer 
  Funktion $f\fd\to\RR$ ($D\subset\RR$ ein Intervall)
  im Lebesgue\sch en Sinne und der Integrierbarkeit von $f$ im 
  Sinne des Regelintegrals?
\end{frage}

\begin{antwort}
  Es gilt der Zusammenhang: 
  \setlength{\labelsep}{4mm}
  \begin{itemize}
  \item[\desc{i}] \slanted{Ist die Funktion $f\fd [a,b] \to \RR$ 
      integrierbar im Sinne des Regelintegrals, 
      dann ist sie auch Lebesgue-integrierbar, und die beiden 
      Integrale stimmen überein.} \\[-3.5mm]
  \item[\desc{ii}] \slanted{Eine Regelfunktion $f$ auf einem offenen Intervall $]a,b[$ 
      (die Werte $a=-\infty$ und $b=\infty$ sind zugelassen) ist genau dann 
      Lebesgue-integrierbar, wenn das uneigentliche Regelintegral 
      $\int_a^b | f | \difx$ existiert. In diesem Fall stimmen die Werte des 
      uneigentlichen Regelintegrals und des Lebesgue-Integrals überein:
      \[
      \int_{]a,b[} f(x) \difx = \int_a^b f(x) \difx.
      \]
    }
  \end{itemize}
  \noindent
  Die erste Behauptung beweist man, indem man zunächst für 
  Treppenfunktionen zeigt, dass sie zu $\calli{H}^\uparrow$ gehören, 
  wenn man die Werte an den Unstetigkeitsstellen entsprechend festlegt 
  und dass für Treppenfunktionen Riemann- und Lebesgue-Integral übereinstimmen.  
  Anschließend zeigt man, dass jede Regelfunktion auf $[a,b]$ der Grenzwert 
  einer monoton wachsenden Folge $(t_k)$ von Treppenfunktionen mit 
  $\n{t_k-f}_\infty \to 0$ ist. Daraus folgt dann die Übereinstimmung 
  von Regel- und Lebesgue-Integral von $f$ über $[a,b]$. 

  Für den Beweis der zweiten Behauptung schließt man an dieses 
  Ergebnis an und benutzt eine Ausschöpfung von $\open{a,b}$ durch 
  eine Folge kompakter Intervalle $[a_k,b_k]$. Die uneigentliche 
  Regel-Integrierbarkeit von $f$ über $\open{a,b}$ 
  folgt aus der Lebesgue-Integrierbarkeit dann aus der für 
  alle $k$ gültigen Abschätzung 
  \[
  \int_{a_k}^{b_k} \big|f(x)\big| \difx = 
  \int_{a_k}^{b_k} \big|f(x)\big| \difx \le 
  \int_{\open{a,b}} \big|f(x)\big| \difx.
  \]
  (Man beachte, dass nach 
  Frage \ref{11_lebesgue} mit $f$ auch 
  $|f|$ Lebesgue-integrierbar ist). 

  Um die andere Richtung zu zeigen, sei $f_k$ die triviale Fortsetzung 
  von $f$ auf $[a_k,b_k]$. Die Folge $(|f_k|)$ konvergiert dann 
  monoton wachsend gegen $|f|$, und für die Integrale gilt  
  $\int_{[a_k,b_k]}|f_k| = \int_{a_k}^{b_k} f$. Ferner ist die Folge 
  der Integrale beschränkt. Die Behauptung ergibt sich dann aus dem Satz 
  von Beppo Levi.
  \AntEnd
\end{antwort}

%% --- 58 --- %%
\begin{frage}
  Kennen Sie ein Beispiel einer Funktion, deren uneigentliches Regelintegral 
  über $\open{a,b}$ existiert, die aber nicht Lebesgue-integrierbar ist?
\end{frage}

\begin{antwort}
  Das uneigentliche Regelintegral $\int_1^\infty \frac{\sin x}{x} \difx$ 
  konvergiert nach der Antwort zu Frage \ref{07_sinxdurchx}. 
  Wenn man das dortige Argument nachvollzieht, erkennt man schnell, dass 
  die Konvergenz wesentlich mit dem alternierenden Verhalten der Sinus-Funktion 
  zusammenhängt, und dass das Integral nicht absolut konvergiert. 
  Nach der Antwort zur vorigen Frage ist $\frac{\sin x}{x}$ 
  (\sieheAbbildung \ref{fig:11_sinus}) über 
  $\open{a,\infty}$ daher nicht Lebesgue-integrierbar. 

  \begin{center}
    \includegraphics{mp/11_sinus}
    \captionof{figure}{Die Funktion $\frac{\sin x}{x}$ ist nicht 
      Lebesgue-integrierbar.}
    \label{fig:11_sinus}
  \end{center} 

  \vspace*{-3mm}
\end{antwort}




\section{Nullmengen und fast überall geltende Eigenschaften}

%% --- 59 --- %%
\begin{frage}\index{Nullfunktion}
  Wann hei{\ss}t eine Funktion 
  $f\fd \RR^n \to \overline{\RR}$ eine \bold{Nullfunktion}?
\end{frage}

\begin{antwort}
  Eine Funktion $f\fd \RR^n \to \overline{\RR}$ hei{\ss}t 
  \slanted{Nullfunktion} genau dann, wenn das Oberintegral 
  von $|f|$ gleich null ist: $I^*(|f|) = 0$.\AntEnd
\end{antwort}

%% --- 60 --- %%
\begin{frage}\index{Nullmenge}
  Wann hei{\ss}t eine Teilmenge $A\subset\RR^n$ eine \bold{Nullmenge}?
\end{frage}

\begin{antwort}
  Eine Teilmenge $A\subset\RR^n$ heißt Nullmenge, 
  wenn deren charakteristische 
  Funktion $\chi_A$ eine Nullfunktion ist. Da das Unterintegral einer 
  charakteristischen Funktion stets nichtnegativ und kleiner als 
  das Oberintegral ist, ist diese Charakterisierung gleichbedeutend mit 
  $\int \chi_A =0$.
  \AntEnd
\end{antwort}

%% --- 61 --- %%
\begin{frage}\label{11_teilnull}
  Warum ist eine Teilmenge einer Nullmenge wieder eine Nullmenge?
\end{frage}

\begin{antwort}
  Aus $B\subset A$ folgt $0\le \chi_B \le \chi_A$. 
  Gilt $I^*(\chi_A)=0$, dann gilt also auch $I^*(\chi_B)=0$. \AntEnd
\end{antwort}

%% --- 62 --- %%
\begin{frage}
  Warum ist eine abzählbare Vereinigung von Nullmengen ebenfalls wieder 
  eine Nullmenge?
\end{frage}

\begin{antwort}
  Für eine endliche Vereinigung folgt das aus der Linearität des 
  Integrals, da sich die charakteristische Funktion der Vereinigung 
  dann als endliche Summe charakteristischer Funktionen 
  von Nullmengen schreiben lässt.  

  Sind $A_1, A_2, A_3,\ldots$ abzählbar viele Nullmengen, dann ist 
  \[
  \chi_{A_1}, \chi_{A_1\cup A_2}, \chi_{A_1\cup A_2 \cup A_3} ,\ldots 
  \]
  eine monoton wachsende Folge integrierbarer Funktionen, deren Integrale 
  nach der obigen Teilantwort alle gleich null sind. Die Behauptung folgt 
  dann aus dem Satz von Beppo Levi.
  \AntEnd
\end{antwort}

%% --- 63 --- %%
\begin{frage}\index{Q@$\QQ$!ist eine Nullmenge}
  Ist $\QQ^n$ in $\RR^n$ eine Nullmenge?
\end{frage}

\begin{antwort}
  Für $\xi\in \RR^n$ ist $\{ \xi \}$ eine Nullmenge in $\RR^n$. 
  Da $\QQ^n$ die Vereinigung abzählbar vieler derartiger Mengen ist, 
  ist $\QQ^n$ eine Nullmenge.
  \AntEnd  
\end{antwort}



%% --- 64 --- %%
\begin{frage}
  Wie lautet das $\eps$-Kriterium für Nullmengen?
\end{frage}

\begin{antwort}
  Das Kriterium lautet:
  
  \begin{Satz}%
    Eine Menge $M\subset \RR^n$ ist genau dann eine Nullmenge, wenn 
    es zu jedem $\eps>0$ abzählbar viele Quader $Q_1,Q_2,Q_3,\ldots $ gibt mit 
    \[
    M \subset \bigcup_{k=1}^\infty, \qquad 
    \sum_{k=1}^\infty v(Q_n) < \eps. 
    \]
  \end{Satz}%
  Beweis s. etwa\citep{Koenig}. \AntEnd
\end{antwort}


%% --- 65 --- %%
\begin{frage}
  Warum ist jeder affine Unterraum $A=a+U\subset \RR^n$ 
  der Dimension $<n$ eine Nullmenge in $\RR^n$.
\end{frage}

\begin{antwort}
  Dies folgt für achsenparalle Unterräume direkt aus der Antwort 
  zu Frage \ref{11_quader}. Der allgemeine Fall lässt sich darauf 
  mithilfe der Transformationsformel zurückführen.
  \AntEnd
\end{antwort}

%% --- 66 --- %%
\begin{frage}\index{Cantormenge}
  Kennen Sie ein Beispiel einer überabzählbaren Nullmenge?
\end{frage}

\begin{antwort}
  Die Cantormenge $\mathfrak{C}$ liefert ein Beispiel. 
  Das Bildungsprinzip ist in der unteren Grafik gezeigt. 
  Ausgehend vom Intervall $[0,1]$ verdoppelt man in jedem Schritt die 
  Anzahl der Intervalle, indem man aus den bestehenden Intervallen $[a,b]$ 
  das offene mittlere Intervall der Länge $1/3\cdot(b-a)$ 
  herausschneidet. Im Grenzfall erhält man die Cantormenge, 
  \sieheAbbildung \ref{fig:cantormenge}.

  \begin{center}
    \includegraphics{mp/cantormenge}
    \captionof{figure}{Die ersten sechs Rekursionsstufen 
      zur Bildung der Cantormenge.}
    \label{fig:cantormenge}
  \end{center}

  \noindent%
  Jedes der $2^n$ Intervalle, die man nach $n$ Schritten erhält, 
  lässt sich eindeutig durch ein $n$-Tupel $(a_1, \ldots, a_n)$ mit 
  $a_i \in \{0,1\}$ identifizieren 
  (wo "`$0$"' das jeweils linke Intervall, "`$1$"' das jeweils rechte bedeuten möge). 
  Es gibt also eine bijekitve Abbildung der Cantormenge 
  auf die Menge aller Folgen $(a_n)$ mit $a_n \in \{0,1\}$. Da diese überabzählbar ist, 
  ist auch die Cantormenge überabzählbar.  
  
  Nach $n$ Konstruktionsschritten hat man $2^n$ Intervalle der Länge $1/3^n$. 
  Damit gilt 
  \[
  v( \mathfrak{C} ) = \lim_{n \to \infty} \left( \frac{2}{3} \right)^n =0 \EndTag
  \]
\end{antwort}


%% --- 67 --- %%
\begin{frage}\index{fast ueberall@fast überall (f.\,ü.)}
  \nomenclature{f.\,ü.}{"`fast überall"'}
  Was bedeutet die Sprechweise \bold{"`fast \"uberall"' (f.\,\"u)}? 
\end{frage}

\begin{antwort}
  Ist $E$ eine Eigenschaft, die jeder Punkt $x\in\RR^n$ haben kann 
  oder nicht haben kann, dann sagt man 
  "`$E$ gilt \slanted{fast überall}"', wenn die Menge aller Punkte 
  $x\in\RR^n$, für die $E$ nicht gilt, eine Nullmenge ist.  
  \AntEnd 
\end{antwort} 

%% --- 68 --- %%
\begin{frage}
  Warum ist eine Funktion $f\fd \RR^n\to\RR$ 
  genau dann eine Nullfunktion, wenn sie fast überall verschwindet?
\end{frage}

\begin{antwort}
  Man betrachte die monoton wachsende Funktionenfolge 
  $(h_k)$ mit 
  \[
  h_k (x) := \left\{ \begin{array}{ll} k\cdot \big| f(x) \big|, & 
      \text{falls $|f(x)| < \infty$,}\\
      0,& \text{falls $|f(x)| = \infty$}. \end{array}\right. 
  \asttag
  \]
  Sei $h$ deren Grenzwert. 

  Ist $f$ eine Nullfunktion, 
  so folgt daraus $I^*(h_k)=I(h_k)=0$ für alle 
  $k$, und der Satz von Beppo Levi liefert $I(h)=0$. 
  Aus $f(x)=0$ folgt $h(x)=0$, und aus $f(x)\not=0$ folgt $h(x)=\infty$. 
  Mit $A:= \{ x\in\RR^n \sets f(x)\not=0 \}$ gilt also 
  $\chi_A \le h$ und damit $I(\chi_A)=0$. Die Menge $A$ ist somit eine 
  Nullmenge. 

  Ist umgekehrt $A:= \{ x\in\RR^n \sets f(x)\not=0 \}$ eine Nullmenge, 
  dann folgt wegen {\astref}, dass $h$ eine Nullfunktion ist. Wegen 
  $|f|<h$ ist dann auch $f$ eine Nullfunktion. \AntEnd  
\end{antwort}


%% --- 69 --- %%
\begin{frage}\label{11_modisatz}\index{Modifikationssatz} 
  Was besagt der \bold{Modifikationssatz}?
\end{frage}

\begin{antwort}
  Der Modifikationssatz besagt:

  \medskip\noindent
  \satz{Ist $f\fd \RR^n\to \overline{\RR}$ 
    eine integrierbare Funktion und $g$ eine Funktion, 
    die fast überall mit $f$ übereinstimmt, dann ist auch $g$ integrierbar 
    und es gilt $\int f= \int  g $.}

  \medskip
  Gilt nämlich fast überall $f=g$, dann ist 
  $f=g+h$ mit einer Nullfunktion $h$, und der Rest folgt dann aus der 
  Ungleichung {\astref} in Frage \ref{11_lebesgue}. 

  Aus dem Modifikationssatz folgt insbesondere, 
  dass man die Werte einer integrierbaren 
  Funktion auf einer Nullmenge willkürlich verändern darf, ohne  
  die Integrierbarkeit der Funktion 
  zu beeinflussen bzw. den Wert ihres Integrals zu verändern.
  \AntEnd 
\end{antwort} 

%% --- 70 --- %%
\begin{frage}
  Wenn $A_1 \subset A_2 \subset A_3\subset \ldots$ eine aufsteigende 
  Folge endlich messbarer Teilmengen des $\RR^n$ ist und die Folge 
  $\big(\vol_n(A_k)\big)$ beschränkt ist, warum ist dann auch die 
  Vereinigung $A:=\bigcup_{k=1}^\infty A_k$ endlich messbar und warum gilt 
  $\vol_n(A)=\lim\limits_{k\to\infty} \vol_n(A_k)$? 
\end{frage}

\begin{antwort} 
  Man wende den Satz von Beppo Levi auf die Folge 
  $(\chi_{A_k})$ an. \AntEnd
\end{antwort} 

%% --- 71 --- %%
\begin{frage}
  Können Sie zeigen: 
  Ist $B_1, B_n, B_3, \ldots$ eine beliebige Folge endlich messbarer 
  Mengen derart, dass für alle $\nu,\mu$ mit 
  $\nu\not=\mu$ die Menge $B_\nu\cap B_\mu$ 
  eine Nullmenge ist und außerdem 
  $\sum_{k=1}^\infty \vol_n(B_k) < \infty$ gilt, 
  dann ist $ \vol_n \left( \bigcup_{k=1}^\infty \right) B_k = 
  \sum_{k=1}^\infty \vol_n(B_k)$.
\end{frage}

\begin{antwort}
  Wegen der ersten Eigenschaft ist 
  $\chi_{B_\nu \cup B_\mu}=\chi_{B_\nu} + \chi_{B_\mu} + h_{\nu\mu}$ mit einer 
  Nullfunktion $h_{\nu\mu}$. Für jedes $N\in \NN$ gilt daher  
  \[
  \vol_n \left( \bigcup_{n=1}^N B_k \right)= 
  \sum_{k=1}^N \int  \chi_{B_k} = \int \sum_{k=1}^N \chi_{B_k} =
  \sum_{k=1}^N \vol_n ( B_k ).
  \]
  Wegen der zweiten Eigenschaft ist die Folge der Integrale 
  $\int \sum_{k=1}^N \chi_{B_k}$ beschränkt und 
  konvergiert damit nach dem Satz von Beppo Levi gegen  
  $\sum_{k=1}^\infty\int \chi_{B_k} = 
  \sum_{k=1}^\infty \vol_n(B_k).$\AntEnd
\end{antwort} 

%% --- 72 --- %%
\begin{frage}\index{Ausschoepfung@Ausschöpfung}
  \index{Integration!durch Ausschöpfung}
  Was besagt der Satz über die \bold{Integration durch Ausschöpfung}?
\end{frage}

\begin{antwort}
  Der Satz besagt: 

  \medskip
  \slanted{Sei $A_1\subset A_2 \subset A_3 \subset\ldots$ eine aufsteigende 
    Folge von Teilmengen des $\RR^n$ und sei $A := \bigcup_{k=1}^\infty A_k$ 
    (die Folge $(A_k)$ nennt man in diesem Fall eine 
    \slanted{Ausschöpfungsfolge von $A$}). 
    Sei ferner $f\fd A\to\RR$ eine Funktion mit der Eigenschaft, die 
    für alle $k$ über $A_k$ integrierbar ist. Dann ist $f$ über 
    $A$ integrierbar genau dann, wenn die Folge 
    $\int_{A_k} |f|$ beschränkt ist. In diesem Fall 
    gilt $\int_A f = \lim\limits_{k\to\infty} \int_{A_k} f$. }

  \medskip\noindent
  Der Satz folgt wiederum aus dem Satz von Beppo Levi. Man betrachte 
  \[
  f_k(x) := \left\{ \begin{array}{ll} 
      f(x) & \text{für $x\in A_k$},\\
      0    & \text{für $x\in \RR^n\mengeminus A_k$}.
    \end{array}\right.
  \]
  Dann gilt $f(x)=\lim f_k(x)$ für alle $x\in\RR^n$. Insbesondere 
  ist $|f_k|$ eine monoton wachsende Folge integrierbarer 
  Funktionen mit beschränkter Integralfolge. 
  Aus dem Satz von Beppo Levi folgt 
  $\lim\limits_{k\to\infty} \int |f_k| = \int |f|$ und daraus die 
  Behauptung. 
  \AntEnd
\end{antwort} 

%% --- 73 --- %%
\begin{frage}\index{Integration!rotationssymmetrischer Funktionen}
  \index{Funktion!rotationssymmetrische}
  Was besagt der \bold{Satz über rotationssymmetrische Funktionen}? 
\end{frage}

\begin{antwort}
  Der Satz lautet: 

  \slanted{Ist $f\fd\RR_+\to\overline{\RR}$ eine 
    über $\open{a,b}$ integrierbare Funktion, dann ist die durch Funktion 
    $g(x) := f( \n{x}_2 )$ definiert Funktion 
    $g\fd \RR^n\to\overline{\RR}$ genau dann über die Kugelschale 
    $K_{a,b}:=\{x\in\RR^n\sets a< \n{x}_2 < b\}$ integrierbar, 
    wenn $|f(r)|r^{n-1}$ über das Intervall $\open{a,b}$ integrierbar 
    ist, und in diesem Fall gilt 
    \[
    \boxed{
      \int_{K_{a,b}} g(x) \dd^n x = 
      \int_{K_{a,b}} f( \n{x}_2 ) \dd^n x =  
      n\kappa_n \int_a^b f(r) r^{n-1} \dd r.}
    \]
    wobei $\kappa_n$ das Volumen der $n$-dimensionalen 
    Einheitskugel $K_1(0)$ ist. 
  }

  \medskip\noindent
  Dieser Satz über rotationssymmetrische Funktionen 
  wird in der Antwort zu Frage \ref{11_rot} mithilfe 
  der Transformationsformel bewiesen. Für einen Beweis, der nur 
  Konzepte aus dem gegenwärtigen Kontext benutzt s. \citep{Koenig}
  \AntEnd 
\end{antwort} 

%% --- 74 --- %%
\begin{frage}
  Ist $\alpha$ eine reelle Zahl mit $\alpha>n$ und ist 
  $M:=\{ x\in\RR^n\sets \n{x}_2 \ge r \}$ mit $r>0$, 
  warum ist dann die Funktion $M\to\RR$ mit 
  $x\mapsto \n{x}_2^{-\alpha}$ über $M$ integrierbar und warum gilt 
  \[
  \int_M \frac{1}{\n{x}_2^\alpha} \dd^n x = 
  \frac{n\kappa_n}{\alpha-n} \cdot \frac{1}{r^{\alpha-n}}.
  \]
\end{frage}

\begin{antwort}  
  Da die Funktion $r\mapsto r^{n-1-\alpha}$ unter den gegebenen 
  Voraussetzungen über $\ropen{r,\infty}$ integrierbar ist, liefert der 
  Satz über rotationssymmetrische Funktionen 
  \[
  \int_M \frac{1}{\n{x}_2^\alpha}  \dd^n x = 
  n\kappa_n \int_r^\infty \frac{1}{r^\alpha} r^{n-1}\dd r = 
  \frac{n\kappa_n}{\alpha-n} \cdot \frac{1}{r^{\alpha-n}}.\EndTag
  \] 
\end{antwort} 

%% --- 75 --- %%
\begin{frage}\index{sigma@$\sigma$-kompakt}
  Was versteht man unter einer \bold{$\mathbf{\sigma}$-kompakten Teilmenge} 
  im $\RR^n$. Kennen Sie Beispiele?
\end{frage}

\begin{antwort}
  Eine Menge $A\subset\RR^n$ heißt \slanted{$\sigma$-kompakt}, wenn sie eine 
  Vereinigung abzählbar vieler kompakter Mengen ist. Beispiele 
  $\sigma$-kompakter Mengen sind alle offenen Mengen und alle abgeschlossenen 
  Mengen sowie die Vereinigung endlich vieler $\sigma$-kompakter Mengen.
  \AntEnd 
\end{antwort} 

%% --- 76 --- %%
\begin{frage}\index{lokal integrierbar}
  Wann heißt eine Funktion auf einer $\sigma$-kompakten Teilmenge 
  $A\subset\RR^n$ \bold{lokal integrierbar}?
\end{frage}

\begin{antwort}
  Eine Funktion $f\fd A\to \overline{\RR}$ heißt \slanted{lokal integrierbar}, 
  wenn sie über jede kompakte Teilmenge $K\subset A$ integrierbar ist. 
  \AntEnd
\end{antwort} 

%% --- 77 --- %%
\begin{frage}\index{Majorantenkriterium!für Integration über 
    $\sigma$-kompakte Teilmengen}
  Was besagt das \bold{Majorantenkriterium} für eine Funktion auf einer 
  $\sigma$-kompakten Teilmenge $A\subset\RR^n$?
\end{frage}

\begin{antwort}
  Das Kriterium besagt: \satz{Ist $f$ eine lokal integrierbare 
    Funktion auf der $\sigma$-kompakten Teilmenge $A$ und existiert  
    eine integrierbare Funktion $F\fd A\to\RR^n$ mit $|f|\le F$, dann 
    ist $f$ über $A$ integrierbar.}

  \medskip\noindent
  Es gibt in diesem Fall eine Ausschöpfung $(A_k)$ von $A$ durch kompakte 
  Mengen. Das Kriterium folgt dann durch Anwendung des Lebesgue\sch en 
  Grenzwertsatzes auf die Folge der Funktionen $f_{A_k}$. \AntEnd
\end{antwort}  

%% --- 78 --- %%
\begin{frage}\index{Satz!von Fubini}
  \index{Fubini@\textsc{Fubini}, Guido (1879-1943)}
  Was besagt der \bold{Satz von Fubini}?
\end{frage}

\begin{antwort}
  Der Satz von Fubini besagt:

  \medskip
  \noindent\slanted{
    Ist $f \fd \RR^k \times \RR^m \to \overline{\RR}$ eine 
    Lebesgue-integrierbare Funktion. Dann gibt es eine Nullmenge 
    $N\subset\RR^m$, so dass f\"ur jedes feste $y\in \RR^m\mengeminus N$ 
    die Funktion
    \[
    \RR^k \to \overline{\RR}; \qquad x\mapsto f(x,y)
    \]
    über $\RR^k$ integrierbar ist. Definiert man 
    f\"ur $y\in\RR^m\mengeminus N$
    \[
    F(y) := \Int_{\RR^k} f(x,y)  \dd^k x 
    \]
    und definiert $F(y)$ beliebig f\"ur $y\in N$, so ist die 
    Funktion $F\fd \RR^m \to \overline{\RR}$ integrierbar, und es gilt 
    \[
    \Int_{\RR^{k+m}} f(x,y) \dd (x,y) = 
    \Int_{\RR^m} F(y) \dd^m y,
    \]
    oder pr\"agnanter
    \[
    \boxed{ 
      \Int_{\RR^{k+m}} f(x,y) \dd (x,y) = 
      \Int_{\RR^m} \Big( \Int_{\RR^k} f(x,y) \dd^k x\Big) \dd^m y,
    }
    \]}%
  \noindent
  Der Satz von Fubini wurde im Spezialfall $f\in\calli{H}^\uparrow$ bereits 
  in Frage \ref{11_kleinerfubini} 
  bewiesen. Man kann an dieses Zwischenergebnis anknüpfen, 
  um den allgemeinen Fall zu beweisen, indem man den Satz Schritt für 
  Schritt gemäß der Konstruktion des Lebesgue-Integrals auf eine größere 
  Klasse von Funktionen und schließlich auf die Lebesgue-integrierbaren 
  Funktionen ausdehnt. \AntEnd
\end{antwort}

%% --- 79 --- %%
\begin{frage}\index{Satz!von Tonelli}
  Was besagt der \bold{Saz von Tonelli}?
\end{frage}

\begin{antwort}
  Der Satz von Tonelli liefert ein Kriterium für die Integrierbarkeit 
  einer Funktion über einen Produktraum $\RR^n\times \RR^m=\RR^{n+m}$ 
  durch Rückführung auf das Mehrfachintegral über die Faktoren. Der Satz 
  lautet

  \medskip
  \noindent\slanted{Eine lokal integrierbare 
    Funktion $f\fd \RR^{n+m}\to \RR$ ist genau dann über 
    $\RR^{n+m}$ integrierbar, wenn wenigstens eines der beiden 
    Integrale 
    \[
    \int_{\RR^n} \Big( \int_{\RR^m} \big| f(x,y) \big| \dd y \Big) \dd x 
    \qquad\text{oder}\qquad
    \int_{\RR^n} \Big( \int_{\RR^m} \big| f(x,y) \big| \dd x \Big) \dd y
    \]
    existiert. 
  }

  \medskip\noindent%
  Die Notwendigkeit der Bedingung folgt sofort mit dem Satz von Fubini. Für den 
  Beweis der anderen Richtung konstruiert man wieder eine monoton wachsende 
  Folge und wendet den Satz von Beppo-Levi an. Sei dazu 
  $f_k := \min \left( |f|, k \cdot \chi_{W_k} \right)$, wobei $W_k := [-k,k]^n$ 
  den abgeschlossenen Würfel mit Kantenlänge $k$ bezeichnet. Die $f_k$ sind dann 
  nach der Voraussetzung alle integrierbar und konvergieren monoton wachsend 
  gegen $f$. Ferner gilt mit dem Satz von Fubini
  \[
  \Int_{\RR^{n+m}} f_k(x,y) \dd (x,y) =
  \Int_{\RR^m} \Big( \Int_{\RR^n} f_k(x,y) \dd x \Big) \dd y \le  
  \Int_{\RR^m} \Big( \Int_{\RR^n} \left| f_k(x,y) \right| \dd x \Big) \dd y.
  \]
  Unter der Voraussetzung des Satzes, 
  dass das rechte Integral existiert, ist die Folge $(f_k)$ also beschränkt und damit 
  nach dem Saz von Beppo Levi $f$ über $\RR^{n+m}$ integrierbar. \AntEnd  
\end{antwort} 

%% --- 80 --- %%
\begin{frage}\index{Faltung}
  Was versteht man unter der \bold{Faltung} von zwei Funktionen 
  $f, g \in  \calli{L}^1 (\RR^n )$?
\end{frage}

\begin{antwort}
  Betrachtet man die Funktion 
  \[
  \RR^n \times \RR^n \to \RR, \qquad
  (x,y) \mapsto f(x) g(y-x),
  \]
  dann ist diese \"uber $\RR^{2n}$ integrierbar. Nach dem Satz von Fubini 
  existiert das Integral 
  \[
  (f \ast g)(y) = \int_{\RR^n} f(x)g(y-x) \difx
  \]
  f\"ur alle $y\in\RR^n \mengeminus N$, wobei $N\subset\RR^n$ 
  eine geeignete Nullmenge ist. Setzt man {\zB} $(f\ast g)(y)=0$ 
  f\"ur $y\in N$, dann erh\"alt man eine integrierbare Funktion 
  $f\ast g \fd \RR^n$, indem man 
  \[
  \boxed{
    (f \ast g)(y) := \int_{\RR^n} f(x)g(y-x) \difx} 
  \]
  setzt. 
  Die Funktion $f\ast g$ hei{\ss}t \slanted{Faltung} von $f$ und $g$. 
  Für das Integral über das Faltungsprodukt gilt
  \begin{align*}
    \int_{\RR^n} (f\ast g)(y) \dify &=
    \int_{\RR^{2n}}  f(x) g(y-x) \difx \dify \\
    &= \int_{\RR^n} f(x) \Big( \int_{\RR^n} g(y-x) \dd y \Big) \difx 
    =
    \int_{\RR^{n}} f(x) \difx \int_{\RR^n} g(y)\dify.
  \end{align*}
  Es ist leicht nachzuweisen, dass das Faltungsprodukt 
  kommutativ ist: 
  \[
  f\ast g = g\ast f, \qquad f,g\in\calli{L}^1(\RR^n).
  \EndTag
  \]
\end{antwort} 

%% --- 81 --- %%
\begin{frage}\index{Laapp1@$L^1$-Raum}
  \index{Banachraum!$L^1$}
  \nomenclature{$L^1(\RR^n)$}{Quotientenraum $\calli{L}^1/\calli{N}$}
  Wie ist der Raum $L^1 ( \RR^n ) $ definiert?
\end{frage}

\begin{antwort}
  Der Raum $\calli{L}^1(\RR^n)$ der Lebesgue-integrierbaren 
  Funktionen ist bezüglich der $\calli{L}^1$-Halbnorm kein normierter 
  Raum, da aus $\n{f}_{\calli{L}^1}=0$ nicht $f=0$ folgen muss, 
  sondern nur, dass $f$ eine Nullfunktion ist. 

  Aus diesem Grund bildet man den Quotientenraum 
  \[ 
  \boxed{
    L^1( \RR^n ) : = \calli{L}^1 (\RR^n )\Big/ \calli{N},
  }
  \] 
  wobei $\calli{N}$ die Menge der Nullfunktionen in $\calli{L}^1(\RR^n)$ 
  ist. Die Elemente des Raums $L^1(\RR^n)$ sind also 
  die \slanted{Äquivalenzklassen integrierbarer Funktionen}, 
  wobei zwei Funktionen $f, g \in \calli{L}^(\RR^n)$ genau dann äquivalent 
  sind, wenn $f-g$ eine Nullfunktion ist. 

  Die $\calli{L}^1$-Halbnorm induziert auf $L^1(\RR^n)$ 
  eine Norm, bezüglich der $L^1(\RR^n)$ ein \slanted{Banachraum} ist. 
  Die Vollständigkeit von $L^1(\RR^n)$ folgt dabei aus dem 
  \slanted{Satz von Riesz-Fischer} (vgl. Frage \ref{11_riesz} und 
  \ref{11_Lpraum}).
  \AntEnd  
\end{antwort}



\section{Der Banachraum $L^1$ und der Hilbertraum $L^2$}

Analog zur Definition des Raums $\calli{L}^1(\RR^n)$ 
der Lebesgue-integrierbaren Funktionen lassen sich für beliebiges 
$p\in \RR$ mit $p\ge 1$ die Funktionenräume $\calli{L}^p(\RR^n)$ definieren. 
Im Wesentlichen bestehen diese Räume aus den Funktionen $f\fd \RR^n\to\RR$, 
für die $|f|^p$ Lebesgue-integrierbar ist. Nach dem 
\slanted{Satz von Riesz-Fischer} sind diese Räume 
vollständig bezüglich der auf ihnen definierten $\calli{L}^p$-Halbnorm. 
Durch Übergang zum Quotientenraum $L^p(\RR^n)$ erhält man aus 
$\calli{L}^p(\RR^n)$ einen \slanted{Banachraum}. 

Wir konzentrieren uns hier im Wesentlichen auf die 
Fälle $p=1$ und $p=2$. 


%% --- 82 --- %%
\begin{frage}
  \index{LaaLP@$\calli{L}^p$!Halbnorm}
  \nomenclature{$\n{\,\;}_{\calli{L}^p}$}{$\calli{L}^p$-Halbnorm}
  \nomenclature{$\n{\,\;}_p$}{$\calli{L}^p$-Halbnorm bzw. $p$-Norm auf $L^p$}
  Für eine Funktion $f\fd\RR^n\to\overline{\RR}$ hatten wir durch 
  $\n{f}_{\calli{L}^1} = \n{f}_1 := I^*(|f|)$ eine Pseudo-Norm definiert. 
  Wie kann man diese Pseudo-Norm verallgemeinern?
\end{frage}

\begin{antwort}
  Man definiert für $p\in \RR$, $p\ge 1$:
  \[
  \boxed{
    \n{f}_{\calli{L}^p} = \n{f}_p := \big( I^*(|f|^p) \big)^{\frac{1}{p}} 
    \in \RR_+ \cup \{ \infty \}.
  } \EndTag
  \]
\end{antwort}



%% --- 83 --- %%
\begin{frage}\index{LaaLP@$\calli{L}^p$!Räume}
  \nomenclature{$\calli{L}^p(\RR^n)$}{$\{ f\fd \RR^n\to \RR \sets |f|^p 
    \text{Lebesgue-integrierbar} \} $} 
  Wie sind für $p\in\RR$, $p\ge 1$ die Räume $\calli{L}^p(\RR^n)$ definiert?
\end{frage}

\begin{antwort}
  Die Räume $\calli{L}^p(\RR^n)$ sind definiert durch
  \[
  \boxed{
    \calli{L}^p ( \RR^n ):= \big\{ f\fd \RR^n \to\RR \sets \,
    \text{$f$ lokal integrierbar},\; \n{f}_{\calli{L}^p} <\infty \big\}.
  }
  \]
  Von besonderem Interesse ist dabei neben dem Fall $p=1$ der 
  Raum $\calli{L}^2(\RR^n)$, der sich dadurch auszeichnet, dass man auf 
  ihm ein (Pseudo-)Skalarprodukt definieren kann (s. Frage \ref{11_l2}), 
  und der damit eine besonders reichhaltige Struktur besitzt. \AntEnd
\end{antwort}

%% --- 84 --- %%
\begin{frage}
  Wann konvergiert eine Folge $(f_k)$ von Funktionen $f_k \fd \RR^n\to \RR$ 
  f.\,ü. gegen eine Funktion $f\fd \RR^n \to \RR$? 
  Wann konvergiert eine Folge $(f_k)$ mit $f_k\in\calli{L}^p$ im Sinne der 
  $\calli{L}^p$-Halbnorm gegen $f\in\calli{L}^p(\RR^n)$. 
\end{frage} 

\begin{antwort}
  Eine Folge von Funktionen $f_k\fd \RR^n\to \RR$ konvergiert dann 
  f.\,ü. gegen $f$, wenn es eine Nullmenge $M$ gibt, so dass für alle 
  $x\in \RR^n\mengeminus M$ gilt: $\lim\limits_{k\to\infty} |f_k(x)-f(x)| =0$, 
  {\dasheisst} wenn $(f_k)$ f.\,ü. punktweise gegen $f$ konvergiert. 

  Die Funktionenfolge $(f_k)$ konvergiert im Sinne der $\calli{L}^p$-Halbnorm 
  gegen $f$, wenn $\lim\limits_{k\to\infty} \n{ f-f_k }_p=0$ gilt. 
  \AntEnd
\end{antwort}

%% --- 85 --- %%
\begin{frage}
  Welche Formel benutzt man um zu zeigen, dass die 
  $\calli{L}_p$-Halbnorm die Dreiecksungleichung erfüllt?
\end{frage}

\begin{antwort}
  Die Dreicksungleichung ergibt sich mit der Integralversion der 
  Hölder\sch en Ungleichung aus Frage \ref{01_hldi}. 
  \AntEnd
\end{antwort}

%% --- 86 --- %%
\begin{frage}\index{Konvergenz!im absoluten Mittel}
  \index{Konvergenz!im quadratischen Mittel}
  Wann spricht man von \bold{Konvergenz im absoluten Mittel}, 
  wann von \bold{Konvergenz im quadratischen Mittel}?
\end{frage}

\begin{antwort}
  \slanted{Konvergenz im absoluten Mittel} bedeutet "`Konvergenz bezüglich 
  der $\calli{L}^1$-Halbnorm"', 
  \slanted{Konvergenz im quadratischen Mittel} bedeutet 
  "`Konvergenz bezüglich der $\calli{L}^2$-Halbnorm"'. \AntEnd
\end{antwort} 

%% --- 87 --- %%
\begin{frage}\index{Cauchy-Folge!bzgl. der $\calli{L}^p$-Halbnorm}
  Was versteht man unter einer $\calli{L}^p$-Cauchy-Folge?
\end{frage}

\begin{antwort}
  Eine Folge $(f_k)$ von Funktionen $f_k \in \calli{L}^p$ ist eine 
  $\calli{L}^p$-Cauchy-Folge, wenn für alle $\eps>0$ ein $N\in \NN$ existiert, 
  so dass gilt 
  \[
  \n{ f_n-f_m }_{\calli{L}^p} < \eps \qquad\text{für alle $n,m>N$}.
  \EndTag
  \] 
\end{antwort}

%% --- 88 --- %%
\begin{frage}\label{11_riesz}
  Wenn eine Folge $(f_k)$ im Sinne der $\calli{L}^p$-Halbnorm 
  gegen eine Funktion $f$ konvergiert, dann ist $(f_k)$ eine 
  $\calli{L}^p$-Cauchy-Folge. Gilt hiervon auch die Umkehrung, 
  {\dasheisst}, hat jede Cauchy-Folge in $\calli{L}^p(\RR^n)$ einen Grenzwert 
  $f\in\calli{L}^p(\RR^n)$?
\end{frage}  

\begin{antwort}
  Die Umkehrung gilt auch. 
  Das ist die Aussage des \slanted{Satzes von Riesz-Fischer} über 
  die Vollständigkeit der Räume $\calli{L}^p$. 
  \index{Satz!von Riesz-Fischer}
  \index{Vollständigkeit!der Räume $\calli{L}^p$}

  \medskip\noindent
  \satz{Die Räume $\calli{L}^p(\RR^n)$ sind vollständig bezüglich der 
    $\calli{L}^p$-Halbnorm. Das heißt, 
    jede $\calli{L}^p$-Cauchy-Folge $(f_k)$ von Funktionen 
    $f_k\in\calli{L}^p(\RR^n)$ besitzt einen Grenzwert in 
    $\calli{L}^p(\RR^n)$.
    \AntEnd
  }
\end{antwort}

%% --- 89 --- %%
\begin{frage}\label{11_Lpraum}
  \index{Laapp-Raum@$L^p$-Räume}\index{Banachraum}
  \nomenclature{$L^p(\RR^n)$}{Quotientenraum $\calli{L}^p/\calli{N}_p$}
  Wie sind die Räume $L^p(\RR^n)$ definiert und warum handelt es sich dabei um 
  Banachräume?
\end{frage}

\begin{antwort}
  Die Elemente der Räume $L^p(\RR^n)$ sind die Äquivalenzklassen $[f]$ 
  von Funktionen $f\in\calli{L}^p(\RR^n)$, wobei $f$ und $g$ genau dann 
  äquivalent sind, wenn $f-g$ eine Nullfunktion bezüglich der 
  $\calli{L}^p$-Halbnorm ist. Bezeichnet $\calli{N}_p$ die Menge der 
  Nullfunktionen bezüglich $\n{\,\;}_{\calli{L}^p}$, dann ist also
  \nomenclature{$\calli{N}_p$}{Menge der Nullfunktionen 
    bzgl. der $\calli{L}^p$-Halbnorm} 
  \[
  \boxed{L^p(\RR^n) := \calli{L}^p(\RR^n) \big/ \calli{N}_p.}
  \] 
  Für $[f]\in L^p$ wähle 
  man einen Repräsentanten $f\in\calli{L}^p(\RR^n)$ und definiere die 
  $L^p$-Norm durch 
  \[
  \nb{ [f] }_p = \nb{ f+\calli{N}_p }_p := 
  \nb{ f }_{\calli{L}^p}.
  \]
  Die Definition ist offensichtlich unabhängig von der 
  Auswahl der Repräsentanten. Ferner übertragen sich die Eigenschaften 
  einer Halbnorm von $\n{\,\;}_{\calli{L}^p}$ auf $\n{\,\;}_p$. Zusätzlich 
  erfüllt $\n{\,\;}_p$ aber jetzt auch 
  \[
  \nb{ [f] }_p = 0 \LLa [f]=0.
  \]
  Damit ist $\n{\,\;}_p$ eine Norm auf $L^p(\RR^n)$. Mit dem Satz von 
  Riesz-Fischer folgt, dass $L^p(\RR^n)$ ein \slanted{vollständiger} 
  normierter Raum, also ein Banachraum ist.
  \AntEnd   
\end{antwort} 

%% --- 90 --- %%
\begin{frage}
  Wie definiert man für komplexwertige Funktionen $f\fd\RR^n\to\CC$ 
  die Räume $\calli{L}^p( \RR^n, \CC)$? 
\end{frage}

\begin{antwort}
  Für eine komplexwertige Funktion $f\fd\RR^n \to \CC$ gilt 
  \[
  \int f = \int \Re f + \i \int \Im f.
  \]
  Daraus folgt, dass $f$ genau dann (lokal) 
  integrierbar ist, wenn $\Re f$ und $\Im f$ integrierbare Funktionen sind. 
  In Analogie zum reellen Fall kann man also definieren:
  \[
  \calli{L}^p (\RR^n,\CC):= \big\{ f\fd \RR^n \to\CC \sets \quad 
  \text{$\Re f$ und $\Im f$ lokal integrierbar},\, 
  \n{f}_{\calli{L}^p} <\infty \big\}. \EndTag
  \]
\end{antwort} 

%% --- 91 --- %%
\begin{frage}\label{11_l2}
  \index{L2-Raum@$\calli{L}^2$-Raum}
  \index{L22-Raum@$L^2$-Raum}
  \index{Hilbertraum}
  Wie kann man auf $\calli{L}^2( \RR^n,\CC)$ eine positiv-semidefinite 
  Hermitesche Form definieren, die beim Übergang zu $L^2(\RR^n,\CC)$ 
  dort ein (positiv-definites) Skalarprodukt induziert, so dass 
  $\calli{L}^2(\RR^n,\CC)$ bezüglich der aus dem Skalarprodukt abgeleiteten 
  Norm vollständig, also ein Hilbertraum ist?
\end{frage}

\begin{antwort}
  Die Abbildung \index{Skalarprodukt!auf $L^2$}
  $\langle\;,\;\rangle \fd \calli{L}^1(\RR^n,\CC) \times 
  \calli{L}^1(\RR^n,\CC)  \to \CC$ mit 
  \[
  \langle f,g \rangle := \Int_{\RR^n} f\overline{g} \difx 
  \]
  erfüllt alle Rechenregeln einer positiv-semidefiniten Hermiteschen Form. 
  Offensichtlich gilt $\langle f,f \rangle = \nb{f}_{\calli{L}^2}$. Durch 
  Übergang zu $L^2(\RR^n)$ erhält man daraus ein positiv-definites 
  Skalarprodukt auf $L^2(\RR^n)$. Wegen seiner Vollständigkeit wird der 
  Raum $L^2(\RR^n)$ damit zu einem Hilbertraum.\AntEnd 
\end{antwort} 

\section{Parameterabhängige Integrale, Fouriertransformierte}

Mit Hilfe des Lebesgue\sch en Grenzwertsatzes erhält man starke 
Verallgemeinerungen der Sätze über Stetigkeit und Differenzierbarkeit 
parameterabhängiger Integrale aus Kapitel \ref{paramintegrale}. 
Wir legen folgende Notation zugrunde: Sei 
$X\subset \RR^n$ und $T \subset \RR^m$ und 
\[
f\fd X \times T \to\CC, \qquad 
(x,t) \mapsto f(x,t),
\asttag
\]
sodass für jeden fixierten Parameter $x$ die Funktion 
$t\mapsto f(x,t)$ über $T$ integrierbar ist. Die durch 
Integration über $T$ entstehende Funktion sei
\[
F(x) := \int_{T} f(x,t) \dift.
\]

%% --- 92 --- %%
\begin{frage}\label{11_paramstetigkeit}
  \index{Stetigkeitssatz für parameterabhängige Integrale}
  Was besagt der \bold{Stetigkeitssatz} für die Funktion $F$? 
\end{frage}

\begin{antwort}
  Der Satz besagt: 
  \satz{Besitzt die Funktion $f$ die Eigenschaften 
    {\setlength{\labelsep}{3mm}
      \begin{enumerate}
      \item[\desc{i}] Für jedes fixierte $t\in T$ ist $x\mapsto f(x,t)$ stetig,
        \\[-3.5mm]
      \item[\desc{ii}] Es gibt eine auf $T$ integrierbare Funktion $G\ge 0$ mit 
        \[
        | f(x,t)| \le G(t) \qquad\text{für alle $(x,t) \in \RR^m \times T$}.
        \]
      \end{enumerate}}%
    \noindent
    Dann ist die durch {\astref} definierte Funktion stetig.
  }

  \medskip\noindent
  Sei $(x_k)$ eine Folge in $\RR^m$ mit $x_k \to x$. Für die Stetigkeit 
  von $F$ in $x$ genügt es zu zeigen, dass daraus $F(x_k) \to F(x)$ folgt. 
  Dazu betrachte man die Folge der Funktionen $f_k \fd T \to \CC$ 
  mit $f_k(t) := f(x_k,t)$. 
  Die Folge $(f_k)$ konvergiert nach \desc{i} dann punktweise 
  gegen die Funktion $t\mapsto f(x,t)$, ferner gilt $|f_k| \le G$ für 
  alle $k\in\NN$. Damit sind die Voraussetzungen 
  des Lebesgue'schen Grenzwertsatzes für $(f_k)$ erfüllt, und aus diesem 
  folgt
  \[
  \lim_{k\to\infty} F(x_k)=
  \lim_{k\to\infty} \int_T f_k(t)\dift=
  \int_T f(x,t)\dift = F(x).\EndTag
  \]
\end{antwort}

%% --- 93 --- %%
\begin{frage}
  Können Sie zeigen, dass für eine integrierbare Funktion 
  $f\fd \RR\to\CC$ die durch 
  \[
  \widehat{f}(x) := \frac{1}{\sqrt{2\pi}} \int_\RR f(t)e^{-\i xt} \dift 
  \]
  definierte Funktion $\widehat{f}\fd \RR\to\CC$ stetig ist?
\end{frage}

\begin{antwort}
  Die Funktion $x\mapsto f(t)e^{\i xt}$ ist für alle $t\in\RR$ stetig. 
  Ferner ist $|f|$ eine integrierbare Majorante des Integranden. Aus dem 
  Stetigkeitssatz folgt damit die Stetigkeit von $\widehat{f}$. 
  \AntEnd
\end{antwort}

%% --- 94 --- %%
\begin{frage}\label{11_paramdifferenzierbarkeit}
  \index{Differentiationssatz!für parameterabhängige Integrale}
  Was besagt der \bold{Differenziationssatz} für $F$?
\end{frage}

\begin{antwort}
  Der Satz besagt:

  \medskip\noindent 
  \slanted{Sei $X\subset \RR^n$ offen und $f$ habe die folgenden 
    Eigenschaften 
    {\setlength{\labelsep}{4mm}
      \begin{enumerate}
      \item[\desc{i}] Für jedes fixierte $t\in T$ ist $x\mapsto f(x,t)$ stetig 
        partiell differenzierbar. \\[-3.5mm]
      \item[\desc{ii}] Es gibt eine auf $T$ integrierbare Funktion $G\ge 0$ mit 
        \[
        \left| \frac{\partial f}{\partial x_\nu} (x,t) \right| \le G(t) 
        \qquad\text{für alle $(x,t) \in X \times T$ und $\nu=1,\ldots,n$}.
        \]
      \end{enumerate}}\noindent
    Dann ist die durch \textrm{{\astref}} 
    definierte Funktion stetig differenzierbar. 
    Ferner ist für jedes $x\in X$ die Funktion 
    $t\mapsto \partial_{x_\nu} f(x,t)$ integrierbar, und es gilt 
    \[
    \frac{\partial F}{\partial x_\nu}(x)=\int_T 
    \frac{\partial f}{\partial x_\nu} (x,t) \dift. \aasttag
    \]
  }%
  \noindent
  Den Beweis erhält man wiederum durch eine Anwendung des Lebesgue'schen 
  Grenzwertsatzes. Sei $x_0 \in X$. Man wähle eine Nullfolge 
  $(h_k)$ reeller Zahlen derart, dass für alle $k\in\NN$ 
  die Punkte $x_k := x_0 + h_k e_\nu$ in $X$ liegen und betrachte die 
  Funktionen 
  \[
  \varphi_k (t) := \frac{f(x_k,t)-f(x_0,t)}{h_k}.
  \]
  Die $\varphi_k$ sind integrierbare Funktionen, und für jedes $t\in T$ gilt 
  $\lim\varphi_k (t) = \frac{\partial f}{\partial x_\nu} 
  (x_0,t)$. 
  Eine integrierbare Majorante für die $\varphi_k$ erhält man aus dem
  Schrankensatz der Differenzialrechnung einer Veränderlichen. Fasst man 
  $f$ als Funktion der $\nu$-ten Variablen auf, so gilt 
  $| f(x_k,t)-f(x_0,t) | \le \big| G(t) \big| \cdot h_k$,  
  also $| \varphi_k | \le G$. 

  Nach dem Lebesgue'schen Grenzwertsatz ist damit die 
  Grenzfunktion der $\varphi_k$ integrierbar, und es gilt 
  \[
  \lim_{k\to\infty} \int_T \varphi_k (t)\dift = 
  \int_T \frac{\partial f}{\partial x_\nu} (x_0,t) \dift.
  \]
  Wegen $
  \frac{1}{h_k}( F(x_k)-F(x_0) ) = \int_T \varphi_k (t)\dift$ 
  existieren die partiellen Ableitungen von $F$ und es gilt 
  {\astastref}.
  
  Schließlich ergibt sich aus {\astastref} zusammen mit dem Stetigkeitssatz 
  die Stetigkeit der partiellen Ableitungen $\partial_{x_\nu} F$. 
  \AntEnd
\end{antwort} 

%% --- 95 --- %%
\begin{frage}\label{11_gammafunktion}\index{Gammafunktion!Stetigkeit}
  \index{Gammafunktion!Differenzierbarkeit}
  Können Sie den Differenziationssatz anwenden, um zu zeigen, dass 
  die durch 
  \[
  \Gamma(x) := \int_0^\infty e^{-t}t^{x-1} \dift\qquad\text{für $x>0$}
  \]
  definierte $\Gamma$-Funktion auf $\RR_+$ stetig und unendlich 
  oft differenzierbar ist und dass für alle $k\in\NN$ gilt: 
  \[
  \Gamma^{(k)}(x)= \int_0^\infty (\log t)^k \cdot t^{x-1} e^{-t} \dift
  \fragezeichen 
  \]
\end{frage}

\begin{antwort}
  Der Integrand ist für jedes feste $t$ eine stetige Funktion von $x$. 
  Auf jedem fixierten kompakten 
  Intervall $[\alpha,\beta]\subset \RR_+$ ist ferner die Funktion 
  \[
  G(t) := \left\{ \begin{array}{ll} 
      t^{\alpha-1} & \text{für $0<t\le 1$}\\
      Me^{-t/2}    & \text{für $t>1$}
    \end{array} \right.
  \]
  für ein geeignetes (von $\beta$ abhängiges) $M\in\RR_+$ 
  eine Majorante des Integranden (vgl. Frage \ref{06_gamk}), 
  die uneigentlich Riemann-integrierbar und damit 
  nach Frage nach der Antwort zu Frage \ref{11_riemannlebesgue} 
  auch Lebesgue-integrierbar ist. Damit sind die Voraussetzungen 
  des Stetigkeitssatzes für die $\Gamma$-Funktion erfüllt. 

  Da ferner die stetige Funktion 
  $\frac{\partial^k}{\partial x}t^{x-1}e^{-t}=\log^k t \cdot t^{x-1}e^{-t}$ 
  für alle $x\in\open{\alpha,\beta}$ und alle $k\in \NN$ 
  die integrierbare Majorante $|\log^k t| G(t)$ besitzt, liefert der 
  Differenziationssatz auch den zweiten Zusammenhang. \AntEnd
\end{antwort}

%% --- 96 --- %%
\begin{frage}
  Was besagt der \bold{Holomorphiesatz} für die Funktion $F$?
\end{frage}

\begin{antwort}
  \slanted{Die Menge $X=U$ sei jetzt eine offene Menge in $\CC$ und 
    $f \fd U\times T \to \CC$ habe die Eigenschaften 
    {\setlength{\labelsep}{4mm}
      \begin{itemize}
      \item[\desc{i}] Für jedes fixierte $t\in T$ ist $z\mapsto f(z,t)$ 
        holomorph in $U$.\\[-3.5mm]
      \item[\desc{ii}] Es gibt eine über $T$ integrierbare Funktion 
        $G$ derart, dass 
        \[
        \big| f(z,t) \big| \le G(t) \qquad \text{für alle $(z,t) \in U\times T$}
        \]
      \end{itemize}}\noindent%
    Dann ist die durch $F(z):= \int_T f(z,t) \dift$ definierte Funktion 
    $F \fd U \to \CC$ holomorph, und es gilt
    \[
    F'(z) = \int_T \frac{\partial}{\partial z} f(z,t) \dift.
    \]
  }
  Man beweist den Satz, indem man zeigt, dass die Funktion $F$ 
  die Cauchy-Riemann'schen Differenzialgleichungen 
  $\partial_1 F = -\i \partial_2 F$ erfüllt, wobei man an einer 
  Stelle ausnutzt, dass für eine in $U$ holomorphe Funktion 
  in jeder abgeschlossenen Kreisscheibe $\overline{K}_{r}(a) \subset U$ 
  die Abschätzung 
  \[
  \big| f'(z) \big|_{\overline{K}_{r}(a)} 
  \le \frac{1}{r} \n{ f }_{\overline{K}_{r}(a)} 
  \]
  gilt, die sich aus der Cauchy'chen Integralformel 
  für die Ableitung ergibt (vgl. \citep{Freitag}). \AntEnd 
\end{antwort}

%% --- 97 --- %%
\begin{frage}\label{11_fouri}
  Können Sie mithilfe des Differenziationssatzes folgendes Integral berechnen: 
  \[
  \widehat{f}(x) = \frac{1}{\sqrt{2\pi}} \Int_\RR e^{-t^2/2} e^{-\i xt}\dift 
  \fragezeichen
  \]
\end{frage}

\begin{antwort}
  Der Integrand wird durch die integrierbare Funktion $e^{-t^2/2}$ majorisiert, 
  Differenziation unter dem Integralzeichen ist also erlaubt. Mit partieller 
  Integration erhält man
  \[
  \widehat{f}'(x) = \frac{-\i}{\sqrt{2\pi}} \Int_\RR te^{-t^2/2} e^{-\i xt}\dift
  = \frac{-x}{\sqrt{2\pi}} \Int_\RR e^{-t^2/2} e^{-\i xt} \dift
  =-x \widehat{f}(x).
  \]
  Daraus folgt $\frac{\dd}{\difx}(\widehat{f}(x) \cdot e^{x^2/2})=
  (-x\widehat{f}(x)+x\widehat{f}(x))e^{x^2/2}=0$,  
  also $\widehat{f}(x)=C\cdot e^{-x^2/2}$ mit einer Konstanten 
  $C\in\RR$. Das Ergebnis von Frage \ref{11_gaussintegral} liefert 
  $\widehat{F}(0)=1$ und damit $C=1$. Folglich gilt 
  $\widehat{f}(x)= e^{-x^2/2}$.
  \AntEnd
\end{antwort}

%% --- 98 --- %%
\begin{frage}\index{Fouriertransformierte}
  Wie ist allgemein die \bold{Fouriertransformierte} einer Funktion 
  $f\in\calli{L}^1(\RR^n)$ definiert?
\end{frage}

\begin{antwort}
  Für eine Funktion $f \in\calli{L}^1(\RR^n,\CC)$ ist die 
  \slanted{Fouriertransformierte} zu $f$ die Funktion 
  $\widehat{f} \fd \RR^n \to\CC$ mit 
  \[
  \boxed{ 
    \widehat{f}(x) := \frac{1}{(2\pi)^{n/2}} \Int_{\RR^n} 
    f(t)e^{-\i \langle x,t\rangle} \dift.
  }
  \]
  Dabei ist $\langle x,t \rangle$ das Standardskalarprodukt 
  von $x\in\RR^n$ und $t\in\RR^n$. 

  Man vergleiche diese Darstellung für den Fall $n=1$ auch mit der 
  Darstellung der \slanted{Fourierkoeffizienten} einer 
  $2\pi$-periodischen Regelfunktion aus Frage \ref{07_fourierkoeff}.   
  \AntEnd
\end{antwort}

%% --- 99 --- %%
\begin{frage}
  Wieso stimmt die durch 
  \[
  f(x):=e^{-\n{x}_2^2/2} = e^{-(x_1^2+\cdots + x_n^2)/2}
  \]
  definierte Funktion $f\fd \RR^n\to\RR$ mit ihrer Fouriertransformierten 
  überein?
\end{frage}

\begin{antwort}
  Mit dem Satz von Fubini und der Antwort zu Frage \ref{11_fouri} 
  erhält man

  \begin{align*}
    \widehat{f}(x) &= \frac{1}{(2\pi)^{n/2}} 
    \Int_{\RR^n} e^{-(x_1^2+\cdots +x_n^2)} e^{-\i \langle x,t \rangle} \dift  
    = \frac{1}{(2\pi)^{n/2}} \Int_{\RR^n} \prod_{\nu=1}^n e^{-x_\nu^2/2} 
    e^{-\i x_\nu t_\nu} \dift \\
    &= 
    \prod_{\nu=1}^n \frac{1}{\sqrt{2\pi}} \Int_\RR 
    e^{-x_\nu^2/2} e^{-\i x_\nu t_\nu} \dift_\nu = 
    \prod_{\nu=1}^n e^{-x_\nu^2/2} = f(x).
    \EndTag
  \end{align*}
\end{antwort}



%% --- 100 --- %%
\begin{frage}\index{Umkehrsatz der Fourier-Transformation}
  Was besagt der \bold{Fourier\sch e Umkehrsatz}?
\end{frage}

\begin{antwort}
  Die Aussage des Umkehrsatzes lässt sich als kontinuierliches Analagon 
  zu der Darstellung einer Funktion durch ihre Fourier\slanted{reihe} 
  verstehen. Er besagt:

  \medskip%
  \noindent%
  \slanted{Ist $f\in\calli{L}^1(\RR^n)$ 
    eine Funktion, deren Fourier-Transformierte 
    $\widehat{f}$ ebenfalls zu $\calli{L}^1(\RR^n)$ gehört. Dann gilt 
    für alle $t\in\RR^n$ mit eventueller Ausnahme einer Nullmenge
    \[
    \boxed{
      f(t) = \frac{1}{(2\pi)^{n/2}} \int_{\RR^n} \widehat{f}(x) 
      e^{\i\langle x,t \rangle } \difx.}
    \]
    Dabei gilt die Identität in jedem Stetigkeitspunkt $t\in\RR^n$ von $f$. 
    Insbesondere gilt dort $\widehat{\widehat{f}}(t)=f(-t)$. 
  }

  \medskip\noindent
  Ein Beweis dieses Satzes wird etwa in \citep{Koenig}, 
  \citep{Forster} oder \citep{Rudin} gegeben.

  Im Fall $n=1$ kann man den Umkehrsatz 
  durch ein Plausiblitätsargument 
  als Grenzfall aus der Darstellung einer periodischen 
  Funktion durch ihre Fourier\slanted{reihe} ableiten,  
  wobei der Grenzfall der Situation entspricht, dass die 
  Periode der Funktion \slanted{unendlich} wird, was 
  bedeutet, dass überhaupt keine Voraussetzungen mehr an 
  die Periodizität von $f$ gestellt werden.

  Wir betrachten für $N\in \NN$ eine Funktion $f$ mit der Periode 
  $N\pi$, die wir der Einfachheit wegen als \slanted{stetig} voraussetzen.  
  Die Funktion $g(t) := f(Nt)$ ist dann eine stetige $2\pi$-periodische 
  Funktion, und der Satz von Fej\'er liefert für diese die Darstellung 
  \[
  g(t) = \frac{1}{2\pi} \sum_{k=-\infty}^\infty \widehat{g}(k) e^{\i k t}
  \]
  bzw. nach der Substitution $Nt \mapsto t$
  \[
  f(t) = \frac{1}{2\pi} \sum_{k=-\infty}^\infty \widehat{g}(k) 
  e^{\i\frac{k}{N} t}.
  \]
  Die Funktion $f$ wird in dieser Darstellung in physikalischer Hinsicht 
  in ihre harmonischen Oberschwingungen zerlegt. 
  Die Periodenlängen dieser Oberschwingungen 
  durchlaufen die Zahlen $N\pi, N\pi/2, N\pi/3, \ldots$. 
  Mit wachsendem $N$ liegt die Menge dieser Zahlen beliebig dicht in 
  $[0,N\pi]$. Der Grenzügergang $N\to\infty$ führt damit 
  verständlicherweise auf ein Integral über $\RR$. 

  Wir wollen das auch noch rechnerisch nachvollziehen. 
  Für den Fourierkoeffizenten von $g$ erhält man nach Frage 
  \ref{fkoeffi}
  \[
  \widehat{g}( k ) = 
  \frac{1}{2\pi} \int_{-\pi}^\pi f(N u) e^{-\i k u}\difu=
  \frac{1}{2\pi N} \int_{-N \pi}^{N\pi} f(u) e^{\i \frac{k}{N} u} \difu.
  \]
  Die Fourierreihe von $g(t)=f(N t)$ lautet also 
  \[
  f(N t) = \frac{1}{2\pi N}\sum_{k=-\infty}^\infty 
  \Big( \int_{-N\pi}^{N\pi} 
  f(t) e^{-\i \frac{k}{N} u}\difu \Big)  e^{\i k t }.
  \]
  bzw. nach der Substitution $Nt \mapsto t$
  \[
  f(t) = \frac{1}{2\pi}\sum_{k=-\infty}^\infty 
  \Big( \int_{-N\pi}^{N\pi} 
  f(u) e^{- \i \frac{k}{N} u}\difu \Big)  e^{\i \frac{k}{N} t } \cdot 
  \frac{1}{N}.
  \]
  Für $N\to\infty$ liegen die Zahlen $\frac{k}{N}$ dicht in $\RR$. 
  Ferner wird jeder Summand der Reihe mit dem Faktor $\frac{1}{N}$ 
  gewichtet. Es liegt also nahe, für den 
  Grenzfall die Variable $\frac{k}{N}$ durch 
  die reelle Variable $x$ zu ersetzen, sowie $\frac{1}{N}$ durch $\difx$ 
  und die Summe durch ein Integral. Das führt auf 
  \[
  f(t) = \frac{1}{2\pi} \int_\RR \Big( \int_\RR f(u) e^{-i x u} \difu \Big) 
  e^{\i x t } \difx,
  \]
  und das ist gerade die Fourier'sche Umkehrformel im Fall $n=1$, 
  die auf diese Weise plausibel gemacht werden kann.
  \AntEnd  
\end{antwort}

\section{Die Transformationsformel f\"ur Lebesgue-integrierbare 
  Funktionen}

%% --- 101 --- %%
\begin{frage}\label{11_transformationsformel}
  \index{Transformationsformel!fur Lebesgue@für Lebesgue-integrierbare Funktionen}
  Was besagt die \bold{Transformationsformel} f\"ur das 
  $n$-dimensionale Lebesgue-Integral?
\end{frage}

\begin{antwort}
  Die Transformationsformel f\"ur das $n$-dimensionale 
  Lebesgue-Integral ist eine starke Verallgemeinerung 
  der Substitutionsregel f\"ur das eindimensionale Integral. Der 
  Beweis erfordert jedoch erheblich mehr Aufwand. 
  Die Transformationsformel besagt:

  \medskip
  \noindent\slanted{
    Sind $U,V \subset \RR^n$ nicht leere offene Mengen und 
    ist $\varphi\fd U\to V$ ein $\calli{C}^1$-Diffeomorphismus. 
    Dann ist eine Funktion $f\fd V\to \RR$ genau dann 
    integrierbar, wenn die Funktion 
    $(f\circ \varphi) \det \calli{J}(\varphi; \,\cdot\,)$ \"uber $U$ 
    integrierbar ist. Ferner gilt dann 
    \[
    \boxed{
      \int_U f \big( \varphi(x) \big) 
      \big| \det \calli{J}(\varphi; x ) \big| 
      \dd x = \int_V f(y)\dd y .
    }
    \] 
  }%
  \noindent
  Die Transformationsformel ist insofern geometrisch plausibel, 
  als der Faktor $\big|\det\calli{J}(\varphi;x)\big|$ ein Maß dafür ist, 
  wie das Volumen einer "`infinitesimalen Umgebung"' von $x$ unter der 
  Abbildung $\varphi$ verzerrt wird. \AntEnd  
\end{antwort}

%% --- 102 --- %%
\begin{frage}
  Haben Sie eine Idee, wie man die Transformationsformel beweisen 
  k\"onnte?
\end{frage}

\begin{antwort}
  Man kann die Transformationsformel gem\"a{\ss} der Einf\"uhrung des 
  Lebesgue-Integrals in mehreren Schritten beweisen. 

  Zun\"achst beweist man sie f\"ur stetige Funktionen mit kompaktem 
  Tr\"ager und f\"ur die speziellen Diffeomorphismen
  \[
  \varphi \fd  \RR^n \to \RR^n; \qquad
  x\mapsto Ax+b \quad\text{mit $A\in \mathrm{GL}(n,\RR)$}.
  \]
  Diesen Beweisschritt haben wir in der Antwort zu Frage 
  \ref{11_trans} ausgearbeitet.  

  F\"ur einen beliebigen Diffeomorphismus f\"uhrt man die 
  Transformationsformel 
  durch lineare Approximation dann auf den linearen Fall zur\"uck. Dabei 
  spielen spezielle W\"urfel\"uberdeckungen eine wichtige Rolle. 

  In einem n\"achsten Schritt zeigt man, dass sich die G\"ultigkeit 
  der Transformationsformel auf die nichtnegativen halbstetigen Funktionen 
  \"ubertr\"agt. Damit gilt sie dann auch f\"ur das Ober- und Unterintegral 
  einer beliebigen Funktion und damit schlie{\ss}lich allgemein 
  f\"ur die Lebesgue-integrierbaren Funktionen. 

  Entscheidend bei dem Beweis ist unter anderem die Tatsache, dass 
  ein Diffeomorphismus Nullmengen stets in Nullmengen \"uberf\"uhrt.
  \AntEnd
\end{antwort} 

%% --- 103 --- %%
\begin{frage}\label{11_polar}\index{Integration!mittels Polarkoordinaten}
  Wie lautet die Transformationsformel f\"ur ebene bzw. r\"aumliche 
  Polarkoordinaten?
\end{frage}

\begin{antwort}
  \desc{a} F\"ur ebene Polarkoordinaten ergibt sich die folgende Formulierung: 
  Ist $p\fd \RR_+^* \times \, ]-\pi, \pi [ \,$ 
  definiert durch 
  \[
  (r,\varphi) \mapsto (r \cos\varphi, r\sin\varphi ) = (x,y),
  \]
  dann ist eine Funktion $f\fd \RR^2 \to \RR$ genau dann 
  integrierbar, wenn die Funktion
  \[
  \RR_+^* \times \, ] -\pi, \pi [ \, \to \RR; 
  \qquad 
  (r,\varphi) \mapsto r f \big( p(r,\varphi) \big) 
  \]
  integrierbar ist, und es gilt dann 
  \[
  \Int_{\RR^2} f(x,y) \difx\dify = 
  \Int_{-\pi}^{\pi} \Int_0^\infty 
  f( r\cos\varphi, r\sin\varphi ) r \dd r \dd \varphi.
  \]

  \medskip\noindent
  \desc{b} Im dreidimensionalen Fall lautet 
  eine Polarkoordinatenabbildung (vgl. \Abb\ref{fig:12_polarkoord}) 
  \begin{align*}
    p\fd \RR_+^* \times \open{-\pi,\pi} 
    \times \open{-\tfrac{\pi}{2}, \tfrac{\pi}{2} } 
    &\to \RR^3 \\  
    (r,\varphi,\psi) &\mapsto (r \cos\varphi \cos \psi, r\sin\varphi \cos\psi, 
    r\sin\psi ) = (x,y,z).
  \end{align*}

  \begin{center}
    \includegraphics{mp/12_polarkoord}
    \captionof{figure}{Zum Verständnis der Polarkoordinatenabbildung.}
    \label{fig:12_polarkoord}
  \end{center}

  \noindent
  Mit der Transformationsformel ergibt sich,  
  dass eine Funktion $f\fd \RR^3 \to \RR$ genau dann 
  integrierbar ist, wenn die Funktion
  \begin{align*}
    \RR_+^* \times \open{-\pi,\pi}
    \times \open{-\tfrac{\pi}{2}, \tfrac{\pi}{2}} &\to \RR, \\ 
    (r,\varphi,\psi) &\mapsto  f \big( p(r,\varphi,\psi) \big) r^2  \cos \psi 
  \end{align*}
  \picskip{0}\vspace*{-3mm}%
  integrierbar ist, und in diesem Fall gilt: 
  \[
  \Int_{\RR^3} f(x,y,z) \difx\dify\dd z = 
  \Int_0^\infty
  \Int_{-\pi}^\pi 
  \Int_{-\frac{\pi}{2}}^{\frac{\pi}{2}} 
  f\big( p( r,\varphi,\psi) \big) r^2 \cos\psi  \dd \varphi \dd \psi \dd r.
  \EndTag
  \]
  
\end{antwort} 

%% --- 104 --- %%
\begin{frage}\index{Volumen!$n$-dimensionaler Kugeln}
  \nomenclature{$K_R(0)$}{$n$-dimensionale Kugel 
    mit Radius $R$ und Mittelpunkt $0$}
  K\"onnen Sie mit der letzten Formel das Volumen der 
  Kugel $K_R(0) \subset \RR^3$ berechnen?
\end{frage}

\begin{antwort}
  Man erh\"alt 
  \begin{align*}
    K_R(0) &=
    \Int_0^\infty
    \Int_{-\pi}^\pi 
    \Int_{-\frac{\pi}{2}}^{\frac{\pi}{2}} 
    \chi_{K_R(0)}
    \big( p( r,\varphi,\psi) \big) r^2 \cos\psi  \dd \varphi \dd \psi \dd r \\
    &= 
    \Int_0^R r^2 \dd r
    \Int_{-\pi}^\pi \dd \varphi 
    \Int_{-\frac{\pi}{2}}^{\frac{\pi}{2}}  \cos\psi \dd \psi = 
    \frac{R^3}{3} \cdot 2\pi \cdot 2 = \frac{4}{3} R^3 \pi. \EndTag
  \end{align*}
\end{antwort} 

%% --- 105 --- %%
\begin{frage}\label{11_gaussintegral}
  \index{Gauss-Integral@Gauß-Integral}
  Können Sie das \bold{Gauß-Integral} $I := \int_{\RR} e^{-t^2} \dift$
  berechnen, indem Sie es auf ein Doppelintegral zurückführen?
\end{frage}

\begin{antwort}
  Durch Quadrieren erhält man das Doppelintegral
  \[
  I^2 = \big( \int_{\RR} e^{-x^2} \difx \big)\cdot 
  \big( \int_{\RR} e^{-y^2} \dify \big) = 
  \int_{\RR} \int_{\RR} e^{-x^2-y^2} \difx\dify,  
  \]
  das man mittels Polarkoordinaten nun sehr leicht bestimmen kann:
  \[
  I^2 = \int_{-\pi}^\pi \int_{0}^\infty e^{-r^2} r \dd r \dd\varphi 
  = \int_{-\pi}^\pi \left[ -\frac{e^{-r^2}}{2} \right]_0^\infty \dd \varphi 
  = \pi.
  \]
  Daraus folgt 
  \[
  \boxed{\int_\RR e^{-t^2} \dift= \sqrt{\pi}.}\EndTag
  \]
\end{antwort} 

%% --- 106 --- %%
\begin{frage}
  Welche Variante der Transformationsformel ist in den Anwendungen 
  häufig nützlich?
\end{frage}

\begin{antwort}
  Die Variante lässt Nullmengen als Ausnahmemengen zu. Sie besagt 
  genauer: 

  \medskip%
  \noindent
  \slanted{Sei $U\subset\RR^n$ offen und $\varphi\fd U \to \RR^n$ 
    stetig differenzierbar sowie $A\subset U$ eine messbare 
    Teilmenge (eine Menge heißt messbar, wenn sie die abzählbare 
    Vereinigung endlich messbarer Mengen ist) mit den folgenden 
    Eigenschaften:
    {\setlength{\labelsep}{4mm}
      \begin{itemize}
      \item[\desc{i}] $A\setminus A^\circ$ ist eine Nullmenge \\[-3.5mm]
      \item[\desc{ii}] Die Einschränkung von $\varphi$ auf $A^\circ$ induziert 
        einen Diffeomorphismus zwischen $A^\circ$ und $\varphi(A^\circ)$
      \end{itemize}
    }
    Unter diesen Voraussetzungen ist eine Funktion $f\fd \varphi(A) \to\RR$ 
    genau dann über $\varphi(A)$ integrierbar, wenn 
    $(f\circ\varphi) \big| \det \calli{J}( \varphi; \cdot) \big|$ über $A$ 
    integrierbar ist, und es gilt dann
    \[
    \int_A  (f\circ\varphi) \big| \det \calli{J}( \varphi; \cdot) \big| \dd^n x =
    \int_{\varphi(A)} f(y) \dd^n y.
    \]
  }

  \medskip\noindent
  Denn $N:=A\mengeminus A^\circ$ ist nach Voraussetzung eine Nullmenge, 
  daher kann man bei der Integration $A$ durch $A^\circ$ ersetzen. 
  Es ist aber auch $\varphi(N)$ eine Nullmenge, und daher kann man auch 
  $\varphi(A)$ durch $\varphi(A^\circ)$ ersetzen. Die Behauptung folgt 
  nun aus dem allgemeinen Transformationssatz für den durch $\varphi$ 
  induzierten Diffeomorphismus zwischen $A^\circ$ und $\varphi(A^\circ)$. 
  \AntEnd
\end{antwort} 





\section{Integration über Untermannigfaltigkeiten im $\RR^n$}

Im Folgenden sollen speziell \slanted{Flächeninhalte} und 
\slanted{Flächenintegrale} über Flächen im $\RR^3$ definiert werden 
(klassischer Fall). Auch schon der Fall eindimensionaler Mannigfaltigkeiten 
(regulär parametrisierte Kurven) ist von Interesse. Allgemein betrachten 
wir $p$-dimensionale $\calli{C}^1$-Untermannigfaltigkeiten im $\RR^n$ 
und erinnern zur Vorbereitung an den 
\slanted{Äquivalenzsatz für Untermannigfaltigkeiten} 
(s. Frage \ref{10_manniaeq}). 

%% --- 107 --- %%
\begin{frage}\index{Mannigfaltigkeit!parametrisierbare}
  Wann heißt eine Teilmenge $M\subset\RR^n$ eine 
  \bold{zusammenhängende eindimensionale parametrisierbare 
    (Unter-)Mannigfaltigkeit}?
\end{frage}

\begin{antwort}
  $M\subset\RR^n$ heißt zusammenhängende eindimensionale parametrisierbare 
  (Unter-)Mannigfaltigkeit, wenn es eine topologische Abbildung 
  $\alpha\fd\open{0,1} \to M$ gibt, die regulär ist und 
  für die $\dot{\alpha}(t)\not=0$ für alle $t\in\open{0,1}$ gilt.

  Eine solche Abbildung nennt man auch \slanted{reguläre Parametrisierung} 
  von $M$. 
  \AntEnd
\end{antwort}

%% --- 108 --- %%
\begin{frage}\index{regulaere Parametrisierung@reguläre Parametrisierung}
  Wie unterscheiden sich zwei reguläre Parametrisierungen 
  $\alpha\fd\open{0,1}\to M$ und 
  $\beta\fd\open{0,1}\to M$?
\end{frage}

\begin{antwort}
  Sind $\alpha$ und $\beta$ reguläre Parametrisierungen, dann ist 
  \[
  \tau := \beta^{-1}\circ\alpha \fd \open{0,1}\to\open{0,1} 
  \]
  ein Diffeomorphismus.
  \AntEnd
\end{antwort}

%% --- 109 --- %%
\begin{frage}
  Wenn $M\subset\RR^n$ eine $1$-dimensionale 
  zusammenhängende parametrisierbare Mannigfaltigkeit und 
  $f\fd M\to\RR$ eine stetige Funktion ist, und wenn das Integral 
  \[
  \int_0^1  f\big(\alpha( t)\big) \nb{ \dot{\alpha}(t)}_2 \dift 
  \]
  für \slanted{eine} reguläre Parametrisierung $\alpha$ von $M$ 
  existiert, warum existiert es dann auch für jede andere reguläre 
  Parametrisierung von $M$ und warum haben die Integrale den gleichen Wert?
\end{frage}

\begin{antwort}
  Für eine zweite Parametrisierung $\beta$ gilt 
  nach der Antwort zur vorigen Frage $\beta =\alpha\circ\tau$ 
  mit einem Diffeomorphismus $\tau\fd\open{0,1}\to\open{0,1}$. 
  Die Integrierbarkeit folgt damit aus der Transformationsformel, 
  ebenso die Identität der Integrale:
  \begin{align*}
    \int_0^1  f\big(\alpha( t)\big) \nb{ \dot{\alpha}(t)}_2 \dift &= 
    \int_0^1  f\big(\alpha( \tau(u) )\big) \nb{ \dot{\alpha}(\tau(u))}_2 
    \big|\det \big( \tau'(u) \big)\big|\difu \\ &= 
    \int_0^1  f\big( \beta(u) \big) \nnb{ \frac{\dot{\beta}(u)}{\tau'(u)} }_2 
    |\tau'(u)|\difu = 
    \int_0^1  f\big( \beta(t) \big) \nb{ \dot{\beta}(t)}_2 \dift. \EndTag
  \end{align*}
\end{antwort} 

%% --- 110 --- %%
\begin{frage}\index{Kurvenintegral!skalares}
  \nomenclature{$\ell_f(M)$}{Kurvenintegral von $f$ über die 
    $1$-dimensionale Untermannigfaltigkeit $M$}
  Wie ist das \bold{skalare Kurvenintegral} 
  $\ell_f (M)$ einer Funktion $f\fd D \to \RR$ mit $D\subset\RR^n$ 
  und $M\subset D$ über eine $1$-dimensionale parametrisierbare 
  Untermannigfaltigkeit $M$ im $\RR^n$ definiert?
\end{frage}

\begin{antwort}
  Das skalare Kurvenintegral von $f$ über $M$ ist definiert durch
  \[
  \boxed{
    \ell_f (M) := \int_0^1 f \big(\alpha(t)\big) \nb{\dot{\alpha}(t) }_2 \dift, 
  }
  \]
  wobei $\alpha$ eine beliebige reguläre Parametrisierung von $M$ ist. 
  Nach der Antwort zur vorigen Frage hängt der Wert von $\ell_f(M)$ nicht 
  von der Parametrisierung ab. \AntEnd
\end{antwort} 

%% --- 111 --- %%
\begin{frage}\index{Kurvenlänge}
  Wie lässt sich die \bold{Länge der Kurve} $\alpha$ berechnen?
\end{frage}

\begin{antwort}
  Für $f \equiv 1$ erhält man 
  \[
  \ell_1 (M) = \int_0^1 \n{\dot{\alpha}(t)}_2\dift = 
  \int_0^1 \sqrt{ \langle \dot{\alpha}(t), \dot{\alpha}(t) \rangle }\dift
  = : \ell (\alpha ). 
  \]
  $\ell_1(M)$ kann man als eindimensionales Maß von $M$ bezeichnen, 
  es stimmt mit der Länge der Kurve $\alpha$ überein.
  \AntEnd
\end{antwort} 

%% --- 112 --- %%
\begin{frage}\index{Immersion}\index{Einbettung}
  Was versteht man unter einer \bold{Immersion} von einer offenen Menge 
  $V\subset\RR^p$ in den $\RR^n$ ($p<n$).
\end{frage}

\begin{antwort}
  Ist $V\subset\RR^p$ offen und $\alpha\fd V\to\RR^n$ stetig partiell 
  differenzierbar, dann heißt $\alpha$
  \slanted{Immersion}, falls das Differenzial 
  $\dd\alpha(v)$ für alle $v\in V$ injektiv abbildet, {\dasheisst} wenn 
  $\Rang \calli{J}(\alpha;v)=p$ für alle $v\in V$ gilt. 
  \AntEnd
\end{antwort} 

%% --- 113 --- %%
\begin{frage}\index{Einbettung}
  Was ist der Unterschied zwischen einer Immersion und einer 
  \bold{Einbettung}?
\end{frage}

\begin{antwort}
  Eine Immersion $\alpha\fd V\to\RR^n$ heißt \slanted{Einbettung}, 
  wenn $\alpha$ zusätzlich einen Homöomorphismus 
  zwischen $V$ und $\alpha(V)$ induziert. Dass eine Immersion diese 
  Eigenschaft nicht notwendigerweise besitzt, zeigt das Beispiel der 
  regulären Kurve in der Antwort zu Frage \ref{10_manniaeq}. 

  Für eine Einbettung $\alpha \fd V\to\RR^n$ mit $V\subset\RR^p$ 
  ist $\alpha(V)$ stets eine $p$-dimensionale Mannigfaltigkeit. Das folgt 
  aus dem Satz in Frage \ref{10_nomanni}. 
  \AntEnd
\end{antwort} 

%% --- 114 --- %%
\begin{frage}
  Ist $M$ eine $p$-dimensionale Untermannigfaltigkeit im $\RR^n$ und 
  $(\varphi,U)$ eine Karte auf $M$ ($U\subset M$ offen in $M$), warum gibt es 
  dann stets eine Einbettung $\alpha\fd V\to U$?
\end{frage}

\begin{antwort}
  Nach der Definition einer Karte in Frage \ref{10_mannidef} 
  gibt es offene Umgebungen $U', V'\subset \RR^n$ mit 
  $U=U'\cap M$, so dass $\varphi \fd U'\to V'$ ein Diffeomorphismus 
  ist, für den gilt
  \[
  \varphi(U)=V' \cap \RR_0^p.  
  \]
  Sei $V\subset \RR^p$ die Teilmenge mit $V'\cap\RR_0^p=V\times\{0\}$. 
  Die Abbildung 
  \[
  \alpha( u ) := \varphi^{-1}(u,0)
  \]
  bildet $V$ dann homöomorph auf $U$ ab. Ferner ist $\alpha$ injektiv, 
  denn die Jacobi-Matrix $\calli{J}(\alpha,u)$ besteht aus den 
  ersten $p$ Spalten der Jacobi-Matrix von $\varphi^{-1}$ in $u$. 
  Da $\varphi^{-1}$ ein Diffeomorphismus ist, sind diese Spalten 
  linear unabhängig. 
  \AntEnd
\end{antwort}

%% --- 115 --- %%
\begin{frage}\index{Masstensor@Maßtensor}
  \index{Gramsche Determinnate@Gram\sch e Determinante}
  Wie ist der \bold{Maßtensor}, 
  wie die \bold{Gram\sch e Determinante} 
  einer Einbettung $\alpha \fd V\to U$ definiert?  
  Wie lässt sich die Gram\sch e Determinente geometrisch interpretieren?
\end{frage}


\begin{antwort} 
  Für eine Einbettung $\alpha\fd V\to U$ mit $V\subset \RR^p$, 
  $V$ offen ist der Maßtensor von $\alpha$ im Punkt 
  $v\in V$ die positiv definite symmetrische $p\times p$-Matrix 
  \[
  \calli{J}(\alpha;v)^T \cdot \calli{J}(\alpha;v).
  \]
  Die Elemente der Matrix sind die Skalarprodukte 
  \[
  g_{ij}(v) := \langle \partial_i \alpha(v), \partial_j \alpha(v) \rangle
  \]
  der Spaltenvektoren von $\calli{J}(\alpha;v)$.

  Die \slanted{Gram'sche Determinante} $g^\alpha (v)$ 
  \nomenclature{$g^\alpha (v)$}{Gram\sch e Determinante der Einbettung 
    $\alpha$ im Punkt $v$}
  von $\alpha$ im Punkt $v\in V$ ist die Determinante des Maßtensors in $v$:
  \[
  \boxed{
    g^\alpha(v):= 
    \det\left( \calli{J}(\alpha;v)^T \cdot \calli{J}(\alpha;v) \right). 
  }
  \]
  Für eine lineare Abbildung $L\fd \RR^p\to\RR^n$, die durch die Matrix 
  $A$ beschrieben wird, ist $\sqrt{\det A^T A}$ 
  gerade das $p$-dimensionale Volumen des Bildes $L( W_p)$ des 
  Einheitswürfels $W_p = \open{0,1}^p$ unter $l$.

  \begin{center}
    \includegraphics{mp/12_gram}
    \captionof{figure}{%
      Die Gram'sche Determinante misst die 
      "`infinitesimale Volumenverzerrung"', die $\alpha$ in einem Punkt
      bewirkt. }
    \label{fig:12_gram}
  \end{center}

  Für eine allgemeine Einbettung $\alpha\fd\RR^n\to\RR^p$ 
  lässt sich die Gram\sch e Determinante (bzw. deren Quadratwurzel)
  somit als Maß der "`infinitesimalen Volumenverzerrung"', 
  die $\alpha$ in $x$ bewirkt, interpretieren. 
  \AntEnd
\end{antwort} 

%% --- 116 --- %%
\begin{frage}\index{Integral!ueber ein Kartengebiet@über ein Kartengebiet}
  Ist $(\varphi,U)$ eine Karte für $M$ und $\alpha\fd V\to U$ 
  eine lokale Parametrisierung von $U$ und $f\fd U\to\RR$ eine 
  stetige Funktion, wann heißt $f$ über $U$ integrierbar und wie 
  ist das Integral von $f$ über $U$ erklärt?
\end{frage}

\begin{antwort}
  $f$ heißt über $U$ integrierbar, wenn die Funktion
  \[
  V\to\RR, \qquad v\mapsto f\big( \alpha(v) \big) \sqrt{g^\alpha(v)}
  \]
  über $V$ integrierbar ist. In diesem Fall definiert man 
  \[\boxed{
    \int_U f \dd S := 
    \int_U f(x) \dd S(x) := 
    \int_V f\big(\alpha(v)\big) \sqrt{g^\alpha (v)} \dd v.}\asttag
  \]
  \nomenclature{$\int_U f\dd S$}{Integral über ein Kartengebiet $U$}  
  Die Funktion $f$ wird via Einbettung also auf $V$ 
  "`heruntergeholt"', was es ermöglicht, die Integration 
  über Teilmengen einer Mannigfaltigkeit 
  auf die Integration im $\RR^p$ zurückzuführen, wobei die dabei 
  auftretende Volumenverzerrung durch die 
  Gram\sch e Determinante in Rechnung gestellt wird.   
  
  \index{Flaechenelement@Flächenelement}
  \nomenclature{$\dd S$}{Flächenelement} 
  $\dd S= \sqrt{g^\alpha}\dd v$ nennt man \slanted{Flächenelement}. 
  In Frage \ref{tausend} interpretieren wir diese als Differenzialform. 
  \AntEnd
\end{antwort} 

%% --- 117 --- %%
\begin{frage}
  Warum ist die Definition des Integrals in {\astref} 
  unabhängig von der Parametrisierung $\alpha$?
\end{frage}

\begin{antwort}
  Ist $\beta\fd V'\to U$ eine weitere lokale Parametrisierung von $U$, 
  dann gilt $\alpha=\beta \circ T$ mit einem Diffeomorphismus 
  $T \fd V'\to V$. Für die Maßtensoren der beiden Einbettungen 
  liefert die Kettenregel in einem Punkt $u\in V$ und $v=T(u)\in V'$ 
  \[
  \calli{J}(\alpha; u)^T \cdot 
  \calli{J}(\alpha; u) = 
  \calli{J}(T; u)^T \cdot 
  \Big( \calli{J}(\beta; v)^T \cdot \calli{J}(\beta; v) \Big) \cdot 
  \calli{J}(T;u).
  \]
  Es folgt 
  \[
  \sqrt{ g^\alpha(u) } = \big| \det T(u) \big| \cdot 
  \sqrt{ g^\beta(v) }, 
  \]
  und daraus erhält man schließlich mit der Transformationsformel
  \[
  \int_V f \big( \alpha(u) \big) \cdot \sqrt{g^\alpha(u)} \difu =
  \int_{V'} f \big( \beta(v) \big) \cdot \sqrt{g^\beta(v)} \dd v.
  \EndTag
  \]
\end{antwort}

%% --- 118 --- %%
\begin{frage}\label{ZerlegungderEins}\index{Zerlegung der Eins}
  \index{Integral!einer Funktion über eine Untermannigfaltigkeit}
  Welche Technik verwendet man, um das Integral $\int_M f$ für eine 
  stetige Funktion auf einer $p$-dimensionalen Mannigfaltigkeit zu erklären?
\end{frage}

\begin{antwort}
  \index{Zerlegung der Eins}
  Man benutzt dazu eine \slanted{Zerlegung der Eins} auf $M$. 
  Darunter versteht man eine Familie stetiger 
  Funktionen $\eps_i\fd M\to [0,1]$, $i\in\NN$  mit den Eigenschaften
  \satz{\begin{itemize}[2mm]
    \item[\desc{i}] Zu jedem $x\in M$ gibt es eine Umgebung $U(x)$ derart, dass 
      alle bis auf endlich viele der Funktionen $\eps_i$ auf $U(x)$ verschwinden 
      (die Zerlegung ist \slanted{lokal endlich})\index{lokal endlich}.
    \item[\desc{ii}] Es ist $\sum_{i=1}^\infty \eps_i(x)=1$ für alle $x\in M$. 
    \end{itemize}} 
  Enscheidend ist nun, dass es zu jeder offenen Überdeckung 
  $\{U_\lambda\}_{\lambda\in\Lambda}$ von $x$ eine 
  \slanted{subordinierte Zerlegung der Eins} gibt, {\dasheisst}, dass der 
  Träger $\Tr \eps_i$ für jedes $i\in\NN$ in einer der Mengen 
  $U_\lambda$ enthalten ist. 

  Zu einem Atlas von $M$ kann man also eine diesem Atlas untergeordnete 
  Zerlegung der Eins $\{ \eps_i\}$ wählen. Die Funktion $\sum f\eps_i$ 
  stimmt dann auf $M$ mit $f$ überein und die Summanden $f\eps_i$ 
  verschwinden jeweils außerhalb eines Kartengebiets $U(i)$. 
  Ist $f\eps_i$ über $U(i)$ integrierbar, so kann man für diese Funktion 
  das Integral über $M$ einfach durch 
  $\int_M f\eps_i \dd S := \int_{U(i)} f\eps_i \dd S$ definieren. 

  Sind alle Funktionen $f\eps_i$ in diesem Sinne über $M$ integrierbar 
  und gilt zusätzlich noch 
  $\sum_{i=1}^\infty \int_M |f| \eps_i \dd S < \infty$, dann 
  definiert man 
  \[
  \boxed{ 
    \int_M f \dd S := \sum_{i=1}^\infty \int_M f\eps_i \dd S.
  }
  \]\nomenclature{$\int_M f \dd S$}{Integral über eine Untermannigfaltigkeit $M$}
  Dabei muss noch gezeigt werden, dass die Bedingungen und der Wert des 
  Integrals nicht von der 
  Wahl der Zerlegung der Eins abhängen. 
  \AntEnd 
\end{antwort}

%% --- 119 --- %%
\begin{frage}\label{11_pdimvol}
  \index{Volumen!einer Karte}
  Was versteht man unter dem $p$-dimensionalen Volumen einer 
  Karte $(\varphi,U)$ einer $p$-dimensionalen Mannigfaltigkeit?
\end{frage}

\begin{antwort}
  Falls die konstante Funktion $1$ über $U$ integrierbar ist, dann heißt 
  \[\boxed{
    \vol_p (U) := \int_U 1\cdot \dd S = \int_V \sqrt{g^\alpha (v) } \dd v}
  \]
  das $p$-dimensionale Volumen von $U$. 
  \AntEnd
\end{antwort} 

%% --- 120 --- %%
\begin{frage}
  Können Sie die in der Literatur häufig anzutreffende Schreibweise 
  \[
  \vol_2(M) = \int_V \sqrt{EG-F^2} \dd u \dd v \asttag
  \]
  erläutern?
\end{frage}

\begin{antwort}
  Für eine Einbettung $\alpha \fd V\to U$ 
  im Fall $V\subset\RR^2$ und $U\subset\RR^3$ bezeichnen die Buchstaben 
  $E$, $G$ und $F$ die Einträge in der Maßtensor-Matrix $g_{ij}(u)$, 
  und zwar ist $E=g_{11}$, $F=g_{12}=g_{21}$ und $G=g_{22}$. Mit diesen 
  Bezeichnungen schreibt sich der Maßtensor von $\alpha$ im Punkt $(u,v)$ 
  dann in der Form
  \[
  \begin{pmatrix} E(u,v) & F(u,v) 
    \\ F(u,v) & G(u,v) \end{pmatrix}.
  \]
  Für die Gramsche Determinante gilt damit
  $\sqrt{g^\alpha(u,v)}=\sqrt{ E(u,v)G(u,v) -F^2(u,v) }$. 
  Also ist {\astref} gleichbedeutend mit der Formel in Frage 
  \ref{11_pdimvol}\AntEnd 
\end{antwort}



%% --- 121 --- %%
\begin{frage}\index{Integration!zwiebelweise@"`zwiebelweise"'}
  \index{Satz!ueber die zwiebelweise@über die "`zwiebelweise Integration"'}
  Was besagt der Satz von der "`\bold{zwiebelweisen Integration}"'? 
\end{frage}

\begin{antwort}
  Der Satz besagt: 

  \medskip\noindent
  \satz{Ist $f\fd\RR^n\to\RR$ eine integrierbare Funktion, 
    dann ist für alle $r\in\RR_+$ außerhalb einer Nullmenge die Funktion 
    $f$ über die Sphäre $S_r := \{ x\in \RR^n \sets \n{x}_2 = r \}$ 
    integrierbar, und es gilt
    \[
    \boxed{
      \Int_{\RR^n} f(x) \dd^n x = 
      \Int_0^\infty 
      \Big( \Int_{S_r}  f(x) \dd S \Big) \dd r =
      \Int_0^\infty 
      \Big( \Int_{S^{n-1}}  f(r x) \dd S \Big) r^{n-1} \dd r. }
    \asttag
    \]
  }

  \medskip\noindent
  Wir zeigen den Satz hier im Spezialfall $n=3$. Der Beweis für die allgemeine 
  Version funktioniert nach demselben Prinzip, nur dass man dafür erst 
  Informationen über allgemeine Polarkoordinatenabbildungen 
  $\RR^n\to\RR^n$ und deren Funktionaldeterminanten sammeln muss. 

  Sei also $f\fd \RR^3\to\RR$ eine integrierbare Funktion. 
  Nach Frage \ref{11_polar} gilt mit der dort definierten 
  Polarkoordinatenabbildung 
  $p\fd \RR_+^* \times \open{-\pi,\pi} \times 
  \open{ -\frac{\pi}{2}, \frac{\pi}{2} }$ zunächst 
  \[
  \Int_{\RR^3} f(x) \difx = 
  \Int_0^\infty
  \Int_{-\pi}^\pi 
  \Int_{-\frac{\pi}{2}}^{\frac{\pi}{2}} 
  f\big( p( r,\varphi,\psi) \big) r^2 \cos\psi  \dd \varphi \dd \psi \dd r.
  \aasttag
  \]

  Nun berechnen wir für festes $r\in\RR$ das Oberflächenintegral $
  \int_{S_r} f \dd S$ und betrachten dazu die 
  Abbildung $\Phi \fd \open{-\pi,\pi} \times 
  \open{ -\frac{\pi}{2}, \frac{\pi}{2} } \to \RR^3$ mit 
  \[
  (\varphi,\psi) \mapsto ( 
  r  \cos \varphi \cos \psi,
  r   \sin \varphi \cos\psi, 
  r \sin \psi )^T
  \]

  \begin{center}
    \includegraphics{mp/12_polar}
    \captionof{figure}{Parametrisierung der Sphäre $S_r$ mittels 
      Polarkoordinaten.}
    \label{fig:12_polar}
  \end{center}

  Diese ist für jedes feste $r\in\RR$ eine zulässige Parametrisierung der 
  geschlitzten Sphäre 
  \[
  S'_r = S_r \mengeminus N\quad\text{mit}\quad 
  N := \{ (x, 0, z)\sets x \le 0 \}.
  \]
  Nach dem Satz von Fubini und dem Transformationssatz ist $f$ über $S_r$ 
  für jedes $r\in\RR_+$ außerhalb einer Nullmenge integrierbar. 
  Für den Maßtensor von $\Phi$ erhält man 
  \[
  \sqrt{g^\Phi(\varphi,\psi)} := \sqrt{\det \begin{pmatrix} 
      r^2 \cos^2 \psi & 0 \\ 0 & r^2 
    \end{pmatrix} } = r^2\cdot \cos \psi,
  \]
  und damit 
  \[
  \Int_{S_r} f(x) \dd S = 
  \Int_{S'_r} f(x) \dd S =\Int_{-\pi}^\pi 
  \Int_{-\frac{\pi}{2}}^{\frac{\pi}{2}} 
  f( \Phi(\varphi,\psi) ) r^2 \cos \psi 
  \dd\varphi  \dd \psi.
  \]
  Wegen $\Phi(\varphi,\psi)=p(r,\varphi,\psi)$ folgt daraus 
  durch Vergleich mit {\astastref} die erste Gleichung 
  in {\astref} für den Fall $n=3$, 
  die zweite ergibt sich mit der Transformationsformel. 
  \AntEnd 
\end{antwort}



%% --- 122 --- %%
\begin{frage}
  Wie kann man den Satz anwenden, um die Formel 
  \[
  \boxed{ 
    \omega_n = n \kappa_n
  }
  \]
  zu zeigen, wobei $\omega_n:=\vol_{n-1}(S^{n-1})$ die Oberfläche und 
  $\kappa_n := \vol_n ( K_1(0) )$ 
  wie in Frage das Volumen Einheitskugel im $\RR^n$ ist?
\end{frage}

\begin{antwort}
  Integration über die charakteristische Funktion $\chi_{K_0(1)}$ und 
  Anwendung des Satzes aus der letzten Frage liefert
  \begin{align}
    \kappa_n &= \Int_{\RR^n} \chi_{K_0(1)} (x) \difx = 
    \Int_0^1 \Big(\Int_{S^{n-1}}  \chi_{K_0(1)}(rx) r^{n-1} \dd S \Big) \dd r 
    \notag
    \\
    &= \Big( \Int_0^1 r^{n-1} \dd r\Big) \cdot \Big(\Int_{S^{n-1}} 1 \dd S \Big)=
    \frac{1}{n} \omega_n. \notag
  \end{align}
  Damit weiß man jetzt speziell auch über die Oberfläche der dreidimensionalen 
  Einheitskugel Bescheid. Deren Wert ist $\omega_3 = 4\pi$. \AntEnd
\end{antwort}

%% --- 123 --- %%
\begin{frage}\label{11_rot}\index{Funktion!rotationssymmetrische}
  \index{Integration!rotationssymmetrischer Funktionen}
  Was besagt der Satz über \bold{rotationssymmetrische Funktionen}? 
  Wie kann man ihn mit Hilfe des Satzes über zwiebelweise Integration beweisen?
\end{frage}

\begin{antwort}
  Der Satz lautet: \satz{Ist $f\fd \RR_+ \to\RR$ eine Funktion, für 
    welche die Funktion 
    \[
    g\fd \RR^n\to\RR,\qquad 
    x\mapsto g(x) := f( \n{x}_2 )
    \]
    über $\RR^n$ integrierbar ist, dann gilt
    \[
    \Int_\RR g(x) \dd^n x = 
    \Int_{\RR^n} f( \n{x}_2 ) \dd^n x =  
    \omega_n \Int_0^\infty f(r)r^{n-1} \dd r.
    \]
  } 
  
  \noindent
  Der Satz ergibt sich mit dem Satz über zwiebelweise Integration 
  folgendermaßen:  
  \begin{align}
    \Int_{\RR^n} f( \nnb{x}_2 ) \dd^n x &= 
    \Int_0^\infty 
    \Big( \Int_{S^{n-1}} f( r \n{x}_2 ) \dd S \Big) r^{n-1}\dd r \notag
    \\ 
    &=\Int_0^\infty 
    \Big( \Int_{S^{n-1}} f( r )  \dd S \Big) r^{n-1}\dd r =
    \Int_0^\infty r^{n-1} f( r )r^{n-1} \dd r \cdot 
    \Int_{S^{n-1}}  1 \dd S.
    \EndTag
  \end{align}
\end{antwort} 





