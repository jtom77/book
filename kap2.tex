\chapter{Folgen reeller und komplexer Zahlen}

Mit \slanted{Folgen} und deren Konvergenz beginnt die eigentliche 
\slanted{Welt der Analysis}. 

Der Konvergenzbegriff hat jedoch viele Facetten. 
Der einfachste Konvergenzbegriff 
ist sicherlich der für Folgen von reellen und komplexen Zahlen. 
Wichtige Prinzipien für das tiefere Verständnis des Konvergenzbegriffs 
({\zB} bei Funktionen) lassen sich am Beispiel konvergenter Zahlenfolgen 
exemplarisch verdeutlichen und üben. 
\slanted{Reihen} (reeller oder komplexer Zahlen) sind Folgen spezieller 
Bauart, deshalb gelten die für Folgen gültigen Konvergenzkriterien auch 
für Reihen. 
Daneben gibt es wegen der speziellen Bauart von Reihen aber auch zahlreiche 
spezifische Kriterien für deren Konvergenz. 

Folgen und Reihen sind \slanted{zentrale Konstruktionswerkzeuge} der 
Analysis. Mit ihrer Hilfe werden neue Objekte (etwa neue Zahlen oder 
Funktionen) begrifflich konzipiert und formelmäßig dargestellt, 
wie etwa die Euler'sche Zahl $e=2,718281828459\ldots$, die Kreiszahl 
$\pi=3,1415926535\ldots$, die (reelle oder komplexe) 
\slanted{Exponentialfunktion} oder die \slanted{Winkelfunktionen} 
$\sin$ und $\cos$.

Ziel ist es dabei ferner, die neuen Objekte mit \slanted{beliebiger 
  Genauigkeit} in endlich vielen Schritten zu berechnen, 
da sie exakt häufig nicht zu berechnen sind.

Wir beschränken uns in diesem Abschnitt auf Folgen und Reihen reeller 
oder komplexer Zahlen. Als Beispiele für Reihen werden jedoch auch einige 
Potenzreihen, speziell die Exponentialreihe, auftreten. 
Auch für Begriffe wie \slanted{Stetigkeit}, \slanted{Differenzierbarkeit}, 
\slanted{Integrierbarkeit} ist der einfache Konvergenzbegriff für 
Zahlenfolgen grundlegend. 

\section{Definitionen, Beispiele, grundlegende Feststellungen}

%% Question 1
\begin{frage}%
  \label{02_fdef}
  \index{Folge}
  Wenn $X$ eine beliebige nichtleere Menge ist, was versteht man dann 
  unter einer \bold{Folge von Elementen aus $X$}?
  Was versteht man speziell unter einer reellen oder komplexen 
  Zahlenfolge? 
\end{frage}

\begin{antwort}
  Um den Begriff der Folge präzise zu fassen, 
  benutzt man den Abbildungsbegriff. Die Definition lautet: 

  \medskip\noindent
  \slanted{Eine Folge in $X$  
    ist eine Abbildung $a\,:\, \NN\to X$ von den natürlichen Zahlen 
    in die Menge $X$.}

  \medskip\noindent
  Die Bilder $a(n)=:a_n \in X$ mit $n\in \NN$
  heißen $\slanted{Glieder der Folge}$.
  Um eine reelle bzw. komplexe Zahlenfolge handelt es sich dann, wenn 
  $X=\RR$ bzw 
  $X=\CC$ gilt. \AntEnd
\end{antwort}


%% Question 2
\begin{frage}%
  \label{02_fnot}\nomenclature{$(a_n)$}{Folge $a_n$}
  Welche Schreibweisen (Notationen) für Folgen sind Ihnen bekannt? 
  Geben Sie Beispiele an.
\end{frage}

\begin{antwort}
  Die Glieder einer Folge aus $X$ lassen sich als Bilder der 
  Abbildung $a\,:\, \NN \to X$ in der Form $a(n)$ oder $a_n$ schreiben, wobei 
  die zweite Darstellung die wesentlich üblichere ist. 

  Um die Folge selbst zu bezeichnen benutzt man den Ausdruck $(a_n)_{n\in\NN}$, 
  der, wo keine Missverständnisse zu befürchten sind, auch durch $(a_n)$ 
  abgekürzt werden kann. So bezeichnet beispielsweise $(1/n)_{n\in\NN}$ die 
  Folge der Kehrwerte der natürlichen Zahlen.

  Eine nicht ganz präzise, aber sehr suggestive Schreibweise 
  bietet sich dann an, wenn das Bildungsprinzip der Folge sich per Analogie 
  schon aus den ersten paar Gliedern erschließen lässt. In diesem Fall schreibt 
  man nur diese explizit auf und deutet die folgenden nach dem Motto 
  "`und so weiter"' durch Punkte an, wie in dem Beispiel $1,1/2,1/3,\ldots$. 
  \AntEnd
\end{antwort}

%% Question 3
\begin{frage}%
  \label{02_frek}
  \index{Folge!rekursive}
  Was versteht man unter einer \bold{rekursiven Folge} reeller oder 
  komplexer Zahlen?
\end{frage}

\begin{antwort}
  Das Adjektiv "`rekursiv"' bezieht sich darauf, wie die Folge definiert ist. 
  Bei einer rekursiven Folge wird der Wert eines 
  Folgenglieds $a_n$ durch die Werte des vorhergehenden oder 
  mehrerer vorhergehender Glieder der Folge zusammen mit der 
  expliziten Angabe eines oder mehrerer Startwerte $a_1, a_2, \ldots, a_{n_0}$ 
  bestimmt. 
  \AntEnd
\end{antwort}

%% Question 4
\begin{frage}\index{Folge!rekursive}
  Können Sie zwei Beispiele rekursiver Folgen nennen?
\end{frage}

\begin{antwort}
  Zum Beispiel lässt sich die Folge $2^1, 2^2, 2^3,\ldots$ 
  aller Zweierpotenzen rekursiv definieren durch
  \[
  a_1=2, \quad a_{n+1} = 2\cdot a_n.
  \]
  Die Folge $(e_n)_{n\in\NN}$ mit $E_n=\sum_{k=0}^n \frac{1}{k!}$ 
  (vgl. dazu auch die Frage \ref{02_edef})
  ließe sich ebenfalls rekursiv definieren durch
  \[
  E_1 =2, \qquad E_{n+1} = E_n + \frac{1}{(n+1)!}.\EndTag
  \] 
\end{antwort} 



%% Question 5
\begin{frage}%
  \label{02_fib}
  \index{Fibonacci-Folge}
  \index{Fibonacci@\textsc{Fibonacci}, Leonardo Pisano (ca. 1170-1250}
  Wie ist die (klassische) Fibonacci-Folge definiert?
\end{frage}

\begin{antwort}
  Die Fibonacci-Folge $(f_n)$ ist eine rekursiv gegebene Folge 
  natürlicher Zahlen, deren erste beide Glieder gleich $1$ und 
  deren übrige Glieder jeweils die Summe der beiden vorhergehenden sind. 
  Die Fibonacci-Folge ist somit gegeben durch  
  \begin{equation}
    f_1=f_2=1,\quad f_{n+1}=f_{n}+f_{n-1}.
  \end{equation}

Die Tabelle \ref{tab:fibo} listet die Werte der ersten 50 Fibonacci Zahlen auf:



Für die hundertste Fibonacci-Zahl gilt bereits $f_{100} \approx 3,54\cdot 10^{20}$.

Die Folge $\frac{F_{n+1}}{F_n}$ ist konvergent gegen die \slanted{goldene Zahl}
\index{goldene Zahl} $\phi = \frac{1+\sqrt{5}}{2}$. \AntEnd\end{antwort}

\begin{table}
\begin{tabular}{r|r||r|r||r|r||r|r||r|r}
$n$&$F_n$&$n$&$F_n$&$n$&$F_n$&$n$&$F_n$&$n$&$F_n$\\\hline
1&1&11&89&21&10.946&31&1.346.269&41&165.580.141\\
2&1&12&144&22&17.711&32&2.178.309&42&267.914.296\\
3&2&13&233&23&28.657&33&3.524.578&43&433.494.437\\
4&3&14&377&24&46.368&34&5.702.887&44&701.408.733\\
5&5&15&610&25&75.025&35&9.227.465&45&1.134.903.170\\
6&8&16&987&26&121.393&36&14.930.352&46&1.836.311.903\\
7&13&17&1.597&27&196.418&37&24.157.817&47&2.971.215.073\\
8&21&18&2.584&28&317.811&38&39.088.169&48&4.807.526.976\\
9&34&19&4.181&29&514.229&39&63.245.986&49&7.778.742.049\\
10&55&20&6.765&30&832.040&40&102.334.155&50&12.586.269.02\\
\end{tabular}
\caption{Die Fibonacci-Zahlen $F_1$ bis $F_{50}$.}
\label{tab:fibo}
\end{table}


%% Question 6
\begin{frage}%
  \label{02_fag}
  \index{Folge!arithmetische}
  \index{Folge!geometrische}
  Was ist eine \bold{arithmetische} bzw. \bold{geometrische} Folge?
\end{frage}

\begin{antwort}
  Bei einer \slanted{arithmetischen} Folge ist die \slanted{Differenz} 
  zweier aufeinanderfolgender Folgenglieder konstant, 
  es ist also stets $a_{n+1}-a_n=k$. Folglich sind die Folgenglieder gegeben 
  durch die Formel $a_{n+1}= a_1 + n k$ für alle $n\in\NN$. 
  Beispielsweise ist die Folge $2,5,8,11,\ldots$ eine arithmetische 
  Zahlenfolge.

  Eine \slanted{geometrische} Folge ist eine Folge, bei der der 
  \slanted{Quotient} zweier aufeinanderfolgender Glieder konstant ist, 
  bei der also stets $a_{n+1}/a_n=k$ gilt. Für jedes $n\in\NN$ gilt damit 
  $a_{n+1}=a_1 k^n$. Ein Beispiel dafür ist die  
  Folge $1,3,9,27,\ldots$. 
  \AntEnd
\end{antwort}

%% Question 7
\begin{frage}%
  \label{02_fmgl}
  Worin besteht der Unterschied zwischen einer Folge und der Menge ihrer 
  Folgenglieder?
\end{frage}

\begin{antwort}
  Bei einer Folge kommt es auf die \slanted{Reihenfolge} der Glieder an.
  Dagegen besitzt eine Menge keinerlei Informationen über irgendeine 
  "`Reihenfolge"' ihrer Elemente. 
  So sind die beiden Folgen $0,1,-1,2,-2,3,-3,\ldots$ und 
  $0,-1,1,-2,2,-3,3,\ldots$ verschieden, die Menge ihrer Folgenglieder 
  aber in beiden Fällen gleich $\ZZ$.

  Die Glieder einer Folge sind durch ihren Index voneinander unterschieden, 
  auch wenn sie dasselbe Element der Menge bezeichnen. Die Folge $1,1,1,\ldots$ 
  besitzt unendlich viele Glieder, wohingegen die Menge der Glieder nur 
  das Element $1$ enthält. 

  Natürlich sind Folgen und Mengen auch schon allein deswegen 
  verschieden, weil sie zu unterschiedlichen Klassen mathematischer 
  Objekte gehören. \AntEnd
\end{antwort}

%% Question 8
\begin{frage}%
  \label{02_fvis}
  Welche Möglichkeiten der \slanted{Visualisierung} einer reellen oder 
  komplexen Zahlenfolge sind Ihnen geläufig?
\end{frage}

\begin{antwort}
  Eine reelle Zahlenfolge lässt sich 
  als Graph der Abbildung $\NN\to \RR$ in einem 
  zweidimensionalen kartesischen Koordinatensystem eintragen. 
  Jeder Punkt im Koordinatensystem markiert ein Folgenglied, so wie 
  es in Abbildung (s. Abbildung \ref{fig:02_visual1}) veranschaulicht wird.

  \begin{center}
    \includegraphics{mp/02_visual1}
    \captionof{figure}{Graph einer Abbildung $\NN\to\RR$.}
    \label{fig:02_visual1}
  \end{center}
  
  
  Prinzipiell lässt sich die Darstellungsweise auch auf komplexe Folgen 
  übertragen, wenn man ein dreidimensionales Koordinatensystem zugrunde legt  
  und etwa die $xy$-Ebene als Gauß'sche Zahlenebene (den Bildbereich) 
  wählt und die natürlichen Zahlen (den Wertebereich) an der $z$-Achse 
  anträgt (s. \ref{fig:02_visual2}). 
  Die Folgenglieder sind dann durch Punkte im dreidimensionalen Raum markiert.

  \begin{center}
    \includegraphics{mp/02_visual2}
    \captionof{figure}{Graph einer Abbildung $\NN\to\CC$.}
    \label{fig:02_visual2}
  \end{center}
  
  Im Allgemeinen wird letzteres Verfahren jedoch eher ein verwirrendes 
  als aufschlussreiches Bild liefern. Es ist daher in den meisten 
  Fällen sinnvoll, die Folgenglieder nur im Bildbereich einzutragen 
  und die Reihenfolge -- wo sie 
  aus deren Anordnung nicht schon von selbst nahegelegt wird -- durch Indizes 
  oder Verbindungslinien aufeinanderfolgender Glieder zu markieren (s. Abbildung \ref{fig:02_visual3}).
  \AntEnd

  \begin{center}
    \includegraphics{mp/02_visual3}
    \captionof{figure}{Bild einer komplexen Zahlenfolge.}
    \label{fig:02_visual3}
  \end{center}
\end{antwort}

%% Question 9
\begin{frage}%
  \label{02_epsu}
  \index{eps@$\eps$-Umgebung}
  Was versteht man unter der $\eps$-Umgebung einer reellen oder 
  komplexen Zahl?
\end{frage}

\begin{antwort}
  Unter der $\eps$-Umgebung einer reellen oder komplexen Zahl $a$, 
  wobei $\eps$ eine positive reelle Zahl ist, versteht 
  man die Menge aller derjenigen reellen oder komplexen Zahlen, 
  die von $a$ einen kleineren Abstand als $\eps$ haben, also
  \[
  U_\eps( a) = \{ \xi \in \KK; \; |\xi-a|<\eps \}, \quad 
  \text{mit $\KK=\RR$ oder $\KK=\CC$}.
  \] 
  In $\RR$ entspricht das geometrisch dem offenen 
  Intervall $\open{a-\eps,a+\eps}$, in 
  $\CC$ der offenen Kreisscheibe mit Mittelpunkt $a$ und Radius $\eps$ 
  (vgl. dazu auch Frage \ref{01_ueps}).
  \AntEnd
\end{antwort}




%% Question 10
\begin{frage}%
  \label{02_fa}
  \index{fastalle@fast alle}
  Was bedeutet die Aussage: 
  "`Fast alle Glieder einer Folge liegen in $U_\eps(a)$"'.
\end{frage}

\begin{antwort}
  Das heißt, dass \slanted{alle, bis auf endlich viele} Glieder der 
  Folge in $U_\eps(a)$ liegen. M.\,a.\,W., es gibt ein $N\in\NN$, sodass 
  für alle $n>N$ gilt $a_n\in U_\eps(a)$.
  \AntEnd
\end{antwort}



%% Question 11
\begin{frage}%%\label{02_fkon}
  \index{Folge!Konvergenz}
  \index{Konvergenz!einer Folge}
  Wann heißt eine Folge reeller oder komplexer Zahlen \slanted{konvergent}?
\end{frage}

\begin{antwort}
  Eine Folge $(a_n)_{n\in\NN}$ heißt konvergent genau dann, 
  wenn es eine Zahl $a$ 
  ($a\in \RR$ oder $a\in\CC$) mit der folgenden Eigenschaft gibt: \satz{ 
    Zu jedem $\eps>0$ gibt es eine natürliche Zahl $N:=N(\eps)$, sodass gilt
    \begin{equation}
      \boxed{
        |a_n-a| < \eps \quad\text{für alle $n>N$}.
      } \tag{$\ast$} 
    \end{equation}}
  Gegebenenfalls heißt $a$ dann der \slanted{Grenzwert}\index{Grenzwert!einer Folge} der Folge und man 
  schreibt
  \[
  \dis a=\limm a_n 
  \]
  oder etwas flapsiger $a_n\to a$. Wir benutzen auch hin und wieder die 
  abkürzende Schreibweise $\lim a_n$ für $\lim\limits_{n\to\infty}$. 
  \AntEnd
\end{antwort}



%% Question 12
\begin{frage}%
  \label{02_fgeo}
  \index{Folge!Konvergenz}
  \index{Konvergenz!einer Folge}
  Welche geometrische Deutung besitzt die Konvergenz einer Folge?
\end{frage}

\begin{antwort}
  Die Eigenschaft $(\ast)$ aus der vorigen Frage ist gleichbedeutend 
  mit $a_n\in U_\eps(a)$ für alle $n>N$. Damit erhält man folgende 
  äquivalente Definition der Konvergenz mittels $\eps$-Umgebungen: 

  \medskip
  \slanted{Eine Folge $(a_n)_{n\in\NN}$ heißt konvergent genau dann, wenn eine 
    Zahl $a$ mit der Eigenschaft existiert, dass für jedes $\eps>0$ 
    fast alle Glieder der Folge in $U_\eps(a)$ liegen.}

  \medskip
  Abbildung \ref{fig:02_konvergenz_ueps} veranschaulicht den Sachverhalt. 
  \AntEnd
  
  \begin{center}
    \includegraphics{mp/02_konvergenz_ueps}
    \captionof{figure}{Fast alle Glieder einer konvergenten Folge liegen in der 
      $\eps$-Kreisscheibe um den Grenzwert.}
    \label{fig:02_konvergenz_ueps}
  \end{center}
\end{antwort}


%% Question 13
\begin{frage}%
  \label{02_fkon}
  \index{Folge!Konvergenz}
  \index{Folge!Grenzwert}
  \index{Grenzwert!Eindeutigkeit}
  Warum kann eine Folge in $\RR$ oder $\CC$ höchstens einen Grenzwert haben?
\end{frage}

\begin{antwort}
  Ist $a$ ein Grenzwert der Folge und $b$ eine Zahl $\not=a$, dann existieren 
  $\eps$-Umgebungen $U_\eps(a)$ und $U_\eps(b)$ mit 
  $U_\eps(a)\cap U_\eps(b)=\emptyset$. 
  In $U_\eps(a)$ liegen ab einem bestimmten 
  Index $N$ alle Folgenglieder, daher können in $U_\eps(b)$ höchstens endlich 
  viele Folgenglieder liegen, $b$ kann also kein Grenzwert der Folge sein. 

  \begin{center}
    \includegraphics{mp/02_zwei_grenzwerte}
    \captionof{figure}{Disjunkte Umgebungen 
      $U_\eps(a)$ und $U_\eps(b)$ können nicht beide fast 
      alle Folgenglieder enthalten.}
    \label{fig:02_zwei_grenzwerte}
  \end{center}

  Die Eindeutigkeit des Grenzwerts hängt also damit zusammen, 
  dass zwei disjunkte $\eps$-Umgebungen nicht beide fast alle 
  Glieder einer Folge enthalten können, 
  \sieheAbbildung\ref{fig:02_zwei_grenzwerte}.
  \AntEnd
\end{antwort}


%% Question 14
\begin{frage}\label{01_hausdorff}
  \index{Hausdorffsche@Hausdorff'sche Trennungseigenschaft}
  \index{Hausdorff@\textsc{Hausdorff}, Felix (1868-1942)}
  \hspace*{-2mm}\textbf{*} Auf welcher (topologischen) 
  Eigenschaft basiert die Eindeutigkeit 
  des Grenzwerts einer konvergenten Folge?
\end{frage}

\begin{antwort}
  Es handelt sich um die \slanted{Hausdorff'sche Trennungseigenschaft}. 
  Ein topologischer Raum $X$ besitzt diese Eigenschaft, wenn zu je zwei 
  verschiedenen Punkten disjunkte Umgebungen existieren, wie in 
  \Abb\ref{fig:02_hausdorff} veranschaulicht.  
  
  \begin{center}
    \includegraphics{mp/02_hausdorff}
    \captionof{figure}{In einem Hausdorff-Raum besitzen 
      je zwei verschiedene Punkte 
      disjunke Umgebungen.}
    \label{fig:02_hausdorff}
  \end{center}
  In der Antwort zu Frage~\ref{02_fkon} wurde diese Eigenschaft implizit 
  vorausgesetzt, was auch in Ordnung geht, da $\RR$ und $\CC$ als 
  metrische Räume automatisch hausdorffsch sind.  
  Allerdings gilt das nicht für beliebige topologischen Räume. 
  \AntEnd
\end{antwort}

%% Question 15
\begin{frage}%
  \label{02_fbsp}
  \index{Folge!Konvergenz}
  \index{Konvergenz!einer Folge}
  Welche in den Beispielen aufgeführten Folgen sind konvergent?
\end{frage}

\begin{antwort}
  Die meisten der bisher aufgeführten Folgen sind \slanted{nicht} 
  konvergent, so etwa die arithmetischen Folgen $(a_n)_{n\in\NN}$ mit 
  $a_n=a_0+k\cdot n$ und $k\not=0$. Die Differenz zweier Folgenglieder 
  $|a_{n+m}-a_n|=|km|$ wird mit zunehmendem $m\in\NN$ beliebig groß, 
  daher kann es zu einem angenommenen Grenzwert $a$ keinen Index $N$ mit
  $|a_n-a| < \eps$ für alle $n>N$ geben. 

  Eine triviale konvergente Folge unter den bisher genannten ist die 
  konstante Folge $1,1,1,\ldots$, die gemäß der Definition 
  gegen den Grenzwert $1$ konvergiert.  

  Ebenfalls konvergent ist die Folge $(1/n)_{n\in\NN}$. Diese hat den Grenzwert 
  $0$. Ist nämlich $\eps>0$ vorgegeben, dann ist $0<1/n<\eps$, sofern 
  nur $n\eps > 1$ ist, und da $\RR$ archimedisch angeordnet ist, 
  existiert ein $n$ mit dieser Eigenschaft. 
  (Die Voraussetzung, dass $\RR$ ein archimedisch 
  angeordneter Körper ist, ist tatsächlich eine wesentliche Bedingung. In einem 
  nicht-archimedischen Körper konvergiert die Folge $(1/n)_{n\in\NN}$ nicht.)   
  \AntEnd
\end{antwort}

%% Question 16
\begin{frage}%
  \label{02_fbes}
  \index{Beschränktheitskriterium!für Folgen}
  Wann heißt eine Folge beschränkt? Warum ist jede konvergente Folge 
  beschränkt? 
\end{frage}

\begin{antwort}
  Eine Folge $(a_n)_{n\in\NN}$ heißt 
  \slanted{beschränkt}, wenn es eine positive Zahl $S\in\RR$ gibt, 
  sodass für alle Folgenglieder $|a_n| \le S$ gilt.

  Ist eine Folge konvergent 
  gegen den Grenzwert $a$, so gilt $a_n\in U_1(a)$ für 
  alle $n>N$, wobei $n$ eine bestimmte natürliche Zahl ist. 
  Für diese Glieder gilt $|a_n| < |a|+1$. Die Menge der restlichen Folgenglieder 
  ist endlich und somit ebenfalls beschränkt. Insgesamt gilt also 
  $|a_n| \le \max\{|a_0|,\ldots,|a_{N}|, |a|+1 \}$ für alle $n\in\NN$, die 
  Folge ist somit beschränkt.
  \AntEnd
\end{antwort} 

%% Question 17
\begin{frage}
  Gilt auch die Umkehrung, ist also jede beschränkte Folge konvergent?
\end{frage}

\begin{antwort}
  Eine beschränkte Folge muss nicht konvergent sein. Als 
  typisches Gegenbeispiel dient die alternierende Folge 
  $1,-1,1,-1\ldots$, die beschränkt ist, aber offensichtlich  
  nicht konvergiert. 
  \AntEnd
\end{antwort}

%% Question 18
\begin{frage}%
  \label{02_tlf}
  \index{Teilfolge}
  Was versteht man unter einer \bold{Teilfolge} bzw. einer gegebenen Folge?
\end{frage}

\begin{antwort}
  Die Bedeutung des Begriffs "`Teilfolge"' entspricht genau dem, 
  was die Bezeichnung  nahelegt: 
  eben eine Folge $(b_k)_{k\in\NN}$, deren sämtliche Glieder 
  auch Glieder der Folge $(a_n)_{n\in\NN}$ sind, wobei die 
  Anordnung der Folgenglieder 
  erhalten bleibt. 
  Somit gilt für die Glieder einer Teilfolge $(b_k)_{k\in\NN}$ von 
  $(a_n)_{n\in\NN}$
  \[
  b_k = a_{n(k)} \quad\text{und}\quad n(k+1)>n(k).
  \] 
  Bei der Zuordnung $k\mapsto n(k)$ handelt es sich um eine 
  streng monoton wachsende Folge natürlicher Zahlen. Die präzise 
  Definition einer Teilfolge lautet damit: 
  \slanted{Ist $(n_k)$ eine streng monoton wachsende Folge 
    natürlicher Zahlen, dann ist die Folge $(a_{n_k})_{k\in\NN}$ 
    eine Teilfolge von $(a_n)_{n\in\NN}$.}
  \AntEnd
\end{antwort}



%% Question 19
\begin{frage}%
  \label{02_fibk}
  \index{Fibonacci-Folge}
  Ist die Fibonacci-Folge aus Frage \ref{02_fib} konvergent?
\end{frage}

\begin{antwort}
  Die Folge konvergiert nicht, weil sie nicht beschränkt ist. 
  \AntEnd 
\end{antwort}

%% Question 20
\begin{frage}\index{Umordnung einer Folge}
  Was versteht man unter einer \bold{Umordnung} einer gegebenen Folge?
\end{frage}

\begin{antwort}
  Eine \slanted{Umordnung} erhält man  
  durch eine \slanted{bijektive} Abbildung $\NN\to\NN$ 
  der Indizes. Die Definition lautet: 
  \slanted{Sei $\NN\to\NN$, $k\mapsto n(k)=:n_k$ bijektiv. 
    Dann ist die Folge $(a_{n_k})_{k\in\NN}$ eine 
    Umordnung der Folge $(a_n)_{n\in\NN}$.}
  \AntEnd
\end{antwort}

%% Question 21
\begin{frage}%
  Warum hat bei einer konvergenten Folge auch jede Teilfolge und 
  jede Umordnung denselben Grenzwert wie die Ausgangsfolge?
\end{frage}

\begin{antwort}
  Sei $a$ der Grenzwert der Ausgangsfolge. Dann liegen für jedes 
  $\eps >0$ höchs\-tens endlich viele ihrer Glieder außerhalb von 
  $U_\eps(a)$ und somit auch nur endlich viele Glieder ihrer 
  Teilfolgen bzw. ihrer Umordnungen. Das heißt, fast alle Folgenglieder 
  liegen in $U_\eps(a)$. 
  \AntEnd
\end{antwort}


\section{Einige wichtige Grenzwerte}

In den folgenden Fragen werden die Grenzwerte einiger wichtiger 
reeller und komplexer Folgen bestimmt. Neben den Ergebnissen, die an vielen 
Stellen weiterbenutzt werden, lohnt es sich auch, die Techniken genau zu studieren, 
mit denen die Abschätzungen hier ermittelt werden. 


\begin{frage}%
  \label{q:folge-geom-absch}
  Zeigen Sie
  \[
  \lim_{n\to\infty}q^n=0\quad\text{für $q\in\CC$,~$|q|<1$}.
  \]
\end{frage}

\begin{antwort}
  Wegen $|q|<1$ ist $|q|^{-1}>1$. 
  Man setze $|q|^{-1}=1+a$ mit $a>0$. Die 
  Bernoulli'sche Ungleichung (s.~Frage~\ref{q:bernoulli-ungleichung}) 
  liefert dann für $n\in \NN$ 
  \[
  \left( |q|^{-1} \right)^n =(1+a)^n>1+an>an \Ra |q|^n < \frac{1}{an}.
  \]
  Für jedes $\eps>0$ und $n>\frac{1}{\eps a}$ ist somit $|q|^n<\eps$ und 
  damit $q^n\in U_\eps(0)$.
  \AntEnd
\end{antwort}


\begin{frage}
  Zeigen Sie 
  \[
  \lim_{n\to\infty}\frac{z^n}{n!}=0\quad\text{für $z\in\CC$.}
  \]
\end{frage}

\begin{antwort}
  Sei $m:= [\, |z|\, ]$ 
  (Gauß-Klammer von $|z|$, vgl. Frage \ref{01_gkl}). Dann gilt 
  für $n>m$
  \[
  \left| \frac{z^n}{n!} \right| = \frac{ |z|^m }{ m! } \cdot 
  \frac{ |z|^{n-m} }{ (m+1)\cdots n } \le 
  |z|^m \cdot \left| \frac{z}{(m+1)} \right|^{n-m}.
  \]
  Der erste Faktor auf der rechten Seite ist konstant, der zweite konvergiert 
  wegen 
  $\left| \frac{z}{m+1} \right| < 1$ gegen $0$. 
  Insgesamt folgt daraus die Behauptung.
  \AntEnd
\end{antwort}

\begin{frage}
  Zeigen Sie
  \[
  \lim_{n\to\infty} \sqrt[n]{a} = 1\quad\text{für $a\in\RR,\, a>0$.}
  \]
\end{frage}

\begin{antwort}
  Man betrachte zunächst den Fall $a \ge 1$ und setze
  $x_n := \sqrt[n]{a}-1$. Mit der Bernoulli'schen Ungleichung gilt dann 
  \[
  a = (x_n+1)^n \ge 1+nx_n \Ra x_n < \frac{a}{n}, 
  \]
  und somit $\lim\limits_{n\to \infty} x_n = 0$ oder $\lim\limits_{n\to\infty} 
  \sqrt[n]{a}=1$. Den allgemeinen Fall kann man darauf nun 
  mithilfe der in Frage \ref{02_freg} behandelten Rechenregel 
  \desc{c} zurückführen: 
  \[
  \lim_{n\to\infty} \sqrt[n]{a} = \lim_{n\to\infty}\frac{1}{\sqrt[n]{a^{-1}}} = 
  \frac{1}{\lim\limits_{n\to\infty} \sqrt[n]{a^{-1}}} = 1. 
  \]
\end{antwort}

\begin{frage}
  Zeigen Sie
  \[
  \lim_{n\to\infty} \sqrt[n]{n}=1.
  \]
\end{frage}

\begin{antwort}
  Man setze $x_n:= \sqrt[n]{n}-1$. Dann erhält man durch binomische Entwicklung 
  \[
  n=(1+x_n)^n > 1+ \binom{n}{2}x_n^2 \Ra n-1 > \frac{n(n-1)}{2}x_n^2 
  \Ra x_n < \sqrt{\frac{2}{n}} 
  \Ra \lim_{n\to\infty} x_n =0, 
  \]
  und damit  $\lim\limits_{n\to\infty} \sqrt[n]{n}=1$.
\end{antwort}

\begin{frage}
  Zeigen Sie
  \[
  \lim_{n\to\infty} \frac{1}{\sqrt[n]{n!}} = 0.
  \]
\end{frage}


\begin{antwort}
  Für die Fakultät gilt die Abschätzung 
  $n! \ge \left(\frac{n}{2}\right)^{n/2}$, wie man mit vollständiger 
  leicht bestätigen kann. Mit dieser Abschätzung folgt  
  \[
  \sqrt[n]{n!} \ge   \left( \frac{n}{2} \right)^{
    \frac{n}{2}\cdot\frac{1}{n}} = 
  \sqrt{ \left( \frac{n}{2} \right ) }, \quad\text{und damit}
  \quad
  \lim\limits_{n\to\infty} \frac{1}{\sqrt[n]{n!}} =0.
  \]
\end{antwort}


\begin{frage}
  Zeigen Sie
  \[
  \lim_{n\to\infty} \frac{1}{n^s}=0\quad\text{für $s\in\QQ,\, s>0$}.
  \]
\end{frage}

\begin{antwort}
  Sei $s=p/q$ mit $p,q \in \NN$. Wegen $n^s=\left( \sqrt[q]{n} \right)^p$ 
  ist nur $\sqrt[q]{n} \to \infty $ zu zeigen. Der Wert 
  $\sqrt[q]{n}$ wird aber größer als jede beliebige Zahl $M$, sofern nur 
  $n>M^q$ gilt. 
\end{antwort}

\begin{frage}
Zeigen Sie
\[
  \lim_{n\to\infty} n^p z^n = 0\quad\text{für $p\in\NN,\,z\in\CC,\,|z|<1$.}
\]
\end{frage}

\begin{antwort}
  Um $\left| z^n \right|$ durch eine Potenz von $n$ abzuschätzen, betrachte man 
  die Binomialentwicklung von $\left| z^{-n} \right|$. Dazu 
  setze man $1<|z|^{-1}= (1+x)$ mit $x>0$. Damit erhält man für alle $n>2p$
  \[
  |z|^{-n}=(1+x)^n > \binom{n}{p+1} x^{p+1} = 
  \frac{ n(n-1)\cdots (n-p+1)}{ (p+1)! } x^{p+1} 
  > \left( \frac{n}{2} \right)^{p+1} \cdot \frac{x^{p+1}}{ (p+1)! }.
  \]
  Der erste Faktor rechts geht gegen unendlich für $n\to \infty$, während 
  der hintere konstant ist. Somit gilt
  \[
  \left| n^p z^n \right| = \left| \frac{n^p}{ z^{-n} } \right|  <  
  n^p \left(\frac{2}{n}\right)^{p+1} \cdot \frac{(p+1)!}{x^{p+1}} = 
  \frac{1}{n} \cdot \frac{2^{p+1} \cdot (p+1)!}{x^{p+1}}.
  \]
  Der zweite Faktor auf der rechten Seite konstant, während der erste 
  gegen 0 konvergiert. Daraus folgt insgesamt die Behauptung. 
  \AntEnd
\end{antwort}


\begin{frage}
  \label{q:folge_absch}
  Zeigen Sie
  \[
  \lim_{n\to\infty} \sqrt[n]{a^n+b^n}=\max\{a,b\} \quad\text{für $a,b\ge0$}
  \]
\end{frage}

\begin{antwort}
  Sei ohne Beschränkung der Allgemeinheit $a=\max\{ a,b \}$. 
  Das Konvergenzerhalten erkennt man, wenn man den Term in der 
  Form 
  \[
  \sqrt[n]{ a^n +b^n} = \sqrt[n]{ a^n ( 1+ (b/a)^n ) } = 
  a\cdot \sqrt[n]{ 1+(b/a)^n }
  \]
  schreibt. Wegen $b\le a$ konvergiert der Wurzelausdruck 
  für $n\to\infty$ gegen $1$ (dies gilt auch noch 
  im Fall der Gleichheit). Damit folgt die Behauptung.  
  \AntEnd
\end{antwort}




\section{Permanenzeigenschaften (Rechenregeln) 
  für konvergente  Folgen}


%% Question 23
\begin{frage}%
  \label{02_null}
  \index{Nullfolge}
  Wann heißt eine reelle oder komplexe Zahlenfolge eine \bold{Nullfolge}?
\end{frage}

\begin{antwort}
  Ein reelle oder komplexe Folge heißt \slanted{Nullfolge}, wenn sie gegen 
  den Grenzwert $0$ konvergiert. 
  \AntEnd
\end{antwort}

%% Question 24
\begin{frage}%
  \label{02_nuko}
  \index{Nullfolge}
  Wie kann man mithilfe des Begriffs "`Nullfolge"' die Konvergenz einer 
  Folge gegen eine Zahl $a$ beschreiben?
\end{frage}

\begin{antwort}
  
  Eine Folge $(a_n)$ konvergiert genau dann gegen den Grenzwert 
  $a$, wenn 
  die Folge $(a_n-a)$ eine Nullfolge ist. 
  \AntEnd
\end{antwort}

%% Question 25
\begin{frage}%
  \label{02_gmon}
  \index{Monotonie des Grenzwerts}
  Was besagt der Satz über die \bold{Monotonie des Grenzwerts} bei reellen 
  Zahlenfolgen?
\end{frage}

\begin{antwort}
  Der Satz lautet: 

  \medskip\noindent
  \satz{{\noindent}Seien $(a_n)$ und $(b_n)$ zwei konvergente 
    reelle Folgen mit $a_n \to a$ und $b_n \to b$. Gilt dann $a_n \le b_n$ für 
    fast alle $n\in\NN$, so folgt $a\le b$. }

  \medskip\noindent
  Wäre $b<a$, so gäbe es zwei 
  $\eps$-Umgebungen von $a$ und $b$ mit der Eigenschaft, dass 
  alle Elemente aus $U_\eps(b)$ kleiner  
  sind als diejenigen aus $U_\eps(a)$. Nun enthält aber 
  nach Voraussetzung $U_\eps(b)$ fast alle 
  $b_n$, $U_\eps(a)$ fast alle $a_n$. Also würde $b_n<a_n$ für fast alle 
  $n$ gelten, im Widerspruch zur Voraussetzung. 
  \AntEnd
\end{antwort}

%% Question 26
\begin{frage}%
  \label{02_sndw}
  \index{Sandwichtheorem}
  \index{Einschließungssatz}
  Was besagt das \bold{Sandwichtheorem} (der \bold{Einschließungssatz}) 
  für reelle Zahlenfolgen? 
\end{frage}

\begin{antwort}
  Das Sandwichtheorem lautet:

  \medskip
  \noindent\textit{Seien 
    $(A_n)$, $(a_n)$, $(B_n)$ drei reelle Folgen, und es gelte 
    $A_n\le a_n\le B_n$ für fast alle $n$. 
    Wenn $(a_n)$ und $(b_n)$ dann gegen denselben Grenzwert konvergieren, 
    so konvergiert auch $(a_n)$ gegen diesen Grenzwert. } 

  Um die Regel zu beweisen, geht man wie in der vorhergehenden Frage vor 
  und zeigt $\lim A_n \le \lim a_n \le \lim B_n$. Aus $\lim A_n=\lim B_n$ 
  folgt dann gleich die Behauptung. 

  Das Sandwichtheorem kann man zum Beispiel bei der Untersuchung   
  der Folge 
  \[
  (a_n):=\left( \sin (n)/ n \right)
  \]
  anwenden. 
  Die Folge wird durch die beiden Nullfolgen 
  $(a_n)= (-1/n)$ und $(b_n)=(1/n)$ eingeschlossen, nach dem 
  Sandwichtheorem kann man daraus $\limm a_n = 0$ folgern (\sieheAbbildung\ref{fig:02_sandwich}).    
  \AntEnd 

  \begin{center}
    \includegraphics{mp/02_sandwich}
    \captionof{figure}{Die schwarz dargestellte Folge wird von der oberen und 
      unteren Folge eingeschlossen.}
    \label{fig:02_sandwich}
  \end{center}
\end{antwort}

%% Question 27
\begin{frage}%
  \label{02_freg}
  \index{Folge!Rechenregeln}
  Wie verträgt sich der Grenzwertbegriff bei Folgen mit der Bildung von 
  \bold{Summen}, \bold{Produkten} und \bold{Quotienten} von Folgen?
\end{frage}

\begin{antwort}
  Für zwei Folgen $(a_n)$ und $(b_n)$ mit $\dis\limm a_n = a$ und 
  $\dis \limm b_n = b$ gelten die folgenden Rechenregeln:
  \satz{\setlength{\labelsep}{5mm}
    \begin{enumerate}
    \item[\desc{i}] $\lim\limits_{n\to\infty} (a_n+b_n)=a+b$,
    \item[\desc{ii}] $\lim\limits_{n\to\infty} (a_nb_n) = ab$,
    \item[\desc{iii}] Ist $b\not=0$, so sind fast alle $b_n\not=0$, 
      und es gilt $\lim\limits_{n\to\infty} (a_n/b_n) = a/b$. 
    \end{enumerate}}

  \medskip\noindent
  Die Regel \desc{i} ist eine unmittelbare Folge der Dreiecksungleichung: 
  \[
  \big|(a+b)-(a_n+b_n)\big| \le \big|a-a_n\big|+\big|b-b_n\big|.
  \]
  Regel \desc{ii} erkennt man nach der Umformung
  \begin{eqnarray*}
    \big|a_nb_n-ab\big| &=& \big|a_n(b_n-b)+b(a_n-a)\big| 
    \le \big|a_n\big|\big|b_n-b\big|+\big|b\big|\big|a_n-a\big|.
  \end{eqnarray*} 
  Die Faktoren $|a_n|$ und $|b|$ auf der rechten Seite
  sind nämlich beschränkt, 
  während $|a_n-a|$ und $|b_n-b|$ nach der Voraussetzung kleiner als $\eps$  
  werden, wenn man $n$ nur genügend groß wählt. 

  Um \desc{iii} zu zeigen, wähle man ein genügend kleines 
  $\eps$, sodass die Null nicht in $U_\eps(b)$ enthalten ist. Da fast 
  alle Glieder $b_n$ in $U_\eps(b)$ liegen, sind auch fast alle $b_n$ von 
  Null verschieden. Das beweist den ersten Teil von \desc{iii}, der zweite folgt 
  nun aus Kombination mit \desc{ii}. \AntEnd
\end{antwort}

%% Question 28
\begin{frage}%
  \label{02_flin}
  Warum ist die Abbildung 
  $\lim: \; V\to \KK$ 
  mit $(a_n)_{n\in\NN} \mapsto \limm a_n$  
  ($\KK=\RR$ oder $\KK=\CC$)  
  vom Vektorraum der konvergenten Folgen (aus $\KK$) in den 
  Grundkörper ein lineares Funktional?
\end{frage}

\begin{antwort}
  Es muss gezeigt werden, dass für je zwei Folgen 
  $(a_n)_{n\in\NN}$ und $(b_n)_{n\in\NN}$ aus $V$ und jedes $\lambda \in \KK$ die Beziehungen 
  \[
  \mathrm{(i)}\;\limm (a_n+b_n) = \limm a_n + \limm b_n, 
  \qquad
  \mathrm{(ii)}\;\limm \lambda a_n = \lambda \limm a_n 
  \]
  gelten. Die erste ist aber gerade die Regel \desc{i} aus der vorigen 
  Frage, die zweite folgt aus der Beziehung 
  $|a_n-a|<\eps/\lambda \Ra |\lambda a_n - \lambda a| < \eps$.   
  \AntEnd
\end{antwort}

%% Question 29
\begin{frage}%
  Können Sie begründen, warum für jedes Polynom 
  \[
  P\,:\, \KK\to \KK, \quad
  x \mapsto a_kx^k+a_{k-1}x^{k-1}+\cdots + a_0, \quad
  k\in\NN,\,a_j \in\KK,\, 0\le j\le k,
  \]
  und jede mit dem Grenzwert $\xi$ konvergente Folge $(x_n)$ gilt: 
  $\lim\limits_{n\to\infty} P(x_n) = P( \xi )$? (Das bedeutet, dass Polynome 
  \slanted{stetig} sind.)
\end{frage}

\begin{antwort}
  Der Zusammenhang ergibt sich durch wiederholte Anwendung der Regeln 
  \ref{02_freg}\,(i), 
  \ref{02_freg}\,(ii) und 
  \ref{02_flin}\,(ii). 
  \AntEnd
\end{antwort}

%% Question 30
\begin{frage}%
  \label{02_fkom}
  \index{Folge!komplexer Zahlen}
  Wenn eine \bold{komplexe Zahlenfolge} konvergiert, was kann man dann 
  über die Konvergenz der Folge der Beträge bzw. der Folge der Real- und 
  Imaginärteile aussagen?
\end{frage}

\begin{antwort}
  Unter diesen Bedingungen konvergieren sowohl 
  die Folge der Real- und Imaginärteile als auch die Beträge. 

  Abbildung~\ref{fig:02_komplex1} veranschaulicht den Zusammenhang. Liegen 
  fast alle Folgenglieder in der $\eps$-Umgebung des Grenzwertes 
  $a=x+iy$, dann 
  liegen die Imaginär- und Realteile der entsprechenden Glieder 
  in den \slanted{reellen} $\eps$-Umgebungen 
  $U_\eps(x)$ bzw. 
  $U_\eps(y)$. Aus ähnlichen Gründen folgt aus 
  $a_n \in U_\eps(a)$ auch $|a_n| \in U_\eps( |a| )$.\AntEnd

  \begin{center}
    \includegraphics{mp/02_komplex1}
    \captionof{figure}{Fast alle Realteile der Folge liegen in $U_\eps(x)$, 
      fast alle Imaginärteile in $U_\eps(y)$.} 
    \label{fig:02_komplex1}
  \end{center}
\end{antwort}

%% Question 31
\begin{frage}%
  \label{02_fbet}
  \index{Folge!Konvergenz}
  Kann man aus der Konvergenz der Folge der Beträge einer Folge auf die 
  Konvergenz der Folge schließen?
\end{frage}

\begin{antwort}
  Die Antwort lautet nein. Als typisches Gegenbeispiel betrachte man die 
  Folge $(a_n)_{n\in\NN}=((-1)^n)$ und die Folge ihrer Beträge $(1^n)$. Die 
  letzte konvergiert, die erste aber nicht.
  \AntEnd
\end{antwort}

%% Question 32
\begin{frage}%
  \label{02_fkoma}
  \index{Konvergenz!einer komplexen Folge}
  \index{Grenzwert!einer komplexen Folge}
  \index{Folge!komplexer Zahlen}
  Warum ist die Konvergenz einer komplexen Zahlenfolge gegen einen Grenzwert 
  $a\in\CC$ äquivalent mit der Konvergenz der Folge der Realteile gegen 
  $\Re a$ und der Konvergenz der Folge der Imaginärteile gegen $\Im a$?
\end{frage}

\begin{antwort}
  Ähnlich wie in der Antwort zu Frage \ref{02_fkom}
  lässt sich der Zusammenhang 
  durch ein geometrisches Argument veranschaulichen. 

  Liegen ab einem bestimmten Index $N$ alle Realteile der Folgenglieder 
  in $U_{\eps'}(x)$ und alle Imaginärteile in $U_{\eps'}(y)$, dann liegen 
  die komplexen Folgenglieder selbst in $U_{\eps}(a)$ mit 
  $\eps=\sqrt{2} \eps'$, \sieheAbbildung\ref{fig:02_komplex2}

  \begin{center}
    \includegraphics{mp/02_komplex2}
    \captionof{figure}{
      Die Menge $\{z ; \; \Re z \in U_{\eps'}(x), \Im z \in U_{\eps'}(y)\} 
      \subset \CC$ liegt innerhalb einer Kreisscheibe mit Radius $\sqrt2 \eps'$.}  
    \label{fig:02_komplex2}
  \end{center}
  Formal folgt das Ergebnis unter den gegebenen Voraussetzungen aus
  $|a_n-a| = \sqrt{ (x_n-a)^2 + (y_n-y)^2 } \le \sqrt{2{\eps'}^2} 
  = \sqrt{2}\eps'.$ \AntEnd
\end{antwort}

%% Question 33
\begin{frage}%
  \label{02_gkom}
  \index{Grenzwert!einer komplexen Folge}
  Kann der Grenzwert einer reellen Zahlenfolge eine komplexe Zahl sein?
\end{frage}

\begin{antwort}
  Nein. Der Trick beim Beweis besteht darin, 
  die komplexe Konjugation heranzuziehen  
  und die Tatsache auszunutzen, dass aus $z_n\to z$ stets auch 
  $\overline{z_n}\to \overline{z}$ folgt 
  (das ist eine direkte Folge von \ref{02_freg}\,(i)), und dass 
  für reelle Folgen natürlich $\overline{a_n}=a_n$ gilt. Daraus folgt 
  $\overline{\lim a_n}=\lim a_n$, und das geht nur 
  für $\lim a_n \in \RR$.
  \AntEnd  
\end{antwort} 



\section{Prinzipien der Konvergenztheorie}

%% Question 34
\begin{frage}%
  \label{02_fmo}
  \index{Folge!(streng) monoton wachsende}
  \index{Folge!(streng) monoton fallende}
  Was versteht man unter einer \bold{(streng) monoton wachsenden 
    bzw. (streng) monoton fallenden Folge} reeller Zahlen?
\end{frage}

\begin{antwort}
  Eine Folge $(a_n)$ reeller Zahlen heißt 
  \begin{itemize}[2mm]
  \item[\desc{i}] \slanted{monoton wachsend}, wenn $a_{n+1} \ge a_n $ 
    für alle $n\in\NN$ und 
  \item[\desc{ii}] \slanted{monoton fallend}, wenn $a_{n+1} \le a_n $ 
    für alle $n\in\NN$ gilt.
  \end{itemize} 

  \noindent 
  Analog dazu sind die Bedingungen 
  $a_{n+1}/a_n \ge 1$ bzw. $a_{n+1}/a_n \le 1$.  
  Bei \slanted{streng} monotonen Folgen handelt es sich dabei um echte 
  Ungleichungen, dann gilt also zudem $a_{n+1} \not= a_n$.
  \AntEnd
\end{antwort}

%% Question 35
\begin{frage}%
  \label{02_mokr}
  \index{Monotoniekriterium}
  Was besagt das \bold{Monotoniekriterium} für die Konvergenz einer 
  reellen Zahlenfolge?
\end{frage}

\begin{antwort}
  Das Monotoniekriterium lautet: 

  \medskip\noindent
  Jede 
  \emph{beschränkte monotone} 
  Folge reeller Zahlen konvergiert, und zwar 
  gilt 
  \[
  \lim_{n\to\infty} a_n = \left\{ \begin{array}{ll} 
      \sup\, \{a_n\sets n\in \NN\}, & \text{falls $(a_n)$ monoton 
        wächst,} \\
      \inf\, \{a_n;\; n\in \NN\}, &\text{falls $(a_n)$ monoton fällt.} 
    \end{array}\right. \EndTag
  \]
\end{antwort}

%% Question 36
\begin{frage}%
  \label{02_mokb}
  \index{Monotoniekriterium}
  Können Sie das Monotoniekriterium beweisen?
\end{frage}

\begin{antwort}
  Sei $(a_n)$ zunächst eine monoton \slanted{wachsende} Folge und sei 
  $a:=\sup\{ a_n;\; n\in\NN \}$. Zu jedem $\eps>0$ gibt es aufgrund der 
  Supremumseigenschaft von $a$ ein $N\in\NN$ mit $a-\eps < a_N<a$. Da die Folge 
  monoton wächst, gilt diese Ungleichung auch für alle $n$ mit $n > N$, 
  {\dasheisst}, fast alle Folgenglieder liegen in $U_\eps (a)$.

  Für monoton fallende Folgen verläuft der Beweis analog. An der Stelle des 
  Supremums betrachtet man hierbei das Infimum.
  \AntEnd
\end{antwort}

%% Question 37
\begin{frage}%
  \label{Euler}
  Können Sie zeigen, dass die Folge $(e_n)$ mit 
  \[
  e_n := \left( 1+\frac{1}{n} \right)^n 
  \]
  monoton wachsend und beschränkt sind?
\end{frage} 

\begin{antwort}
  Die Monotonie der Folge $(e_n)_{n\in\NN}$ folgt mit der Bernoulli'schen 
  Ungleichung aus
  \begin{eqnarray*}
    \frac{e_n}{e_{n-1}} &=& \left( \frac{n+1}{n} \right)^n 
    \left( \frac{n-1}{n} \right)^{n-1} = 
    \frac{n+1}{n} \left( \frac{n^2-1}{n^2} \right)^{n-1} =
    \frac{n+1}{n} \left( 1-\frac{1}{n^2} \right)^{n-1} \\
    & \ge & \left( 1+\frac{1}{n} \right) \left( 1-\frac{n-1}{n^2} \right) =
    1+\frac{1}{n^3} \ge 1.
  \end{eqnarray*}
  Um die Beschränktheit zu zeigen, betrachte man die Folge 
  $(\widetilde{e}_n)$ mit $\widetilde{e}_n= \left( 1+ \frac{1}{n} \right)^{n+1}$ 
  Offensichtlich gilt $e_n\le \widetilde{e}_n$ für alle 
  $n\in\NN$. Ferner ist die Folge $(\widetilde{e}_n)_{n\in\NN}$ monoton 
  fallend, das ergibt sich aus 
  \begin{eqnarray*}
    \frac{\widetilde{e}_{n-1}}{\widetilde{e}_{n}} &=& \left( \frac{n}{n-1} \right)^n 
    \left( \frac{n}{n+1} \right)^{n+1} = 
    \frac{n}{n+1} \left( \frac{n^2}{n^2-1} \right)^{n} =
    \frac{n}{n+1} \left( 1+\frac{1}{n^2-1} \right)^{n} \\
    & \ge & \frac{n}{n+1} \left( 1+\frac{1}{n^2-1} \right) 
    \ge \frac{n}{n+1} \left( 1+\frac{1}{n} \right) =1. 
  \end{eqnarray*}
  Hieraus erhält man insbesondere 
  $e_n \le \widetilde{e}_1=4$ für alle $n\in\NN$, die 
  Folge $(e_n)_{n\in\NN}$ ist somit beschränkt.\AntEnd
\end{antwort}

%% Question 38
\begin{frage}
  Die Folge $(E_n)$ mit  
  \[
  E_n := \sum_{k=0}^n \frac{1}{k!}
  \]
  ist ebenfalls monoton wachsend und beschränkt. Woraus folgt das?
\end{frage}

\begin{antwort}
  Die Folge ist offensichtlich monoton wachsend, da alle Summanden 
  positiv sind. Wegen $k! \ge 2^{k-1}$ für alle $k\in\NN$ ergibt sich die 
  Beschränktheit durch Vergleich mit der geometrischen Reihe. Es gilt
  \begin{equation}
    E_n \le 2 \sum_{k=0}^{n} 2^{-k} = 2\cdot 
    \frac{1-1/2^{n+1}}{1-1/2} \le 4.   \EndTag
  \end{equation}
\end{antwort} 

%% Question 39
\begin{frage}\label{02_edef}
  Können Sie $\lim\limits_{n\to\infty} e_n=\lim\limits_{n\to\infty} E_n$ 
  beweisen? 
\end{frage}

\begin{antwort}
  Um eine sinnvolle Abschätzung des Ausdrucks $|e_n-E_n|$ zu erhalten, 
  müssen wir den Term in Summanden aufspalten, die sich ihrerseits 
  gesondert abschätzen lassen. 

  Dazu benutzen wir die Tatsache, dass $(e_n)$ konvergiert. Man wähle 
  $K\in \NN$ so groß, dass $\sum_{k=K}^\infty \frac{1}{k!} < \frac{\eps}{3}$ 
  gilt. Mit der binomischen Entwicklung 
  von $(1+1/n)^n$ erhält man 
  \[
  \left| \left( 1+ \frac{1}{n} \right)^n - \sum_{k=0}^n \frac{1}{k!} \right|
  \le
  \sum_{k=0}^{K-1} \left| \binom{n}{k} \frac{1}{n^k} - \frac{1}{k!} \right|
  + \sum_{k=K}^n \binom{n}{k} \frac{1}{n^k} + 
  \sum_{k=K}^{n} \frac{1}{k!}.
  \]
  Der letzte Summand ist aufgrund der Wahl von $K$ kleiner als 
  $\frac{\eps}{3}$. Für die mittlere Summe gilt:
  \[
  \sum_{k=K}^n \binom{n}{k} \frac{1}{n^k} = 
  \sum_{k=K}^n \frac{n(n-1)\cdots (n-k+1)}{k!n^k} \le
  \sum_{k=K}^n \frac{1}{k!} < \frac{\eps}{3}.
  \]
  Für die erste Summe schließlich gilt:
  \[
  \limm \sum_{k=0}^{K-1} \left| \binom{n}{k} \frac{1}{n^k} - \frac{1}{k!} \right|
  =  \sum_{k=0}^{K-1} \limm
  \left| \frac{1}{k!} \cdot \left( 1-\frac{1}{n} \right )\cdots
    \left( 1-\frac{k+1}{n} \right ) -\frac{1}{k!} \right| = 0.
  \]
  Es gibt also ein $N \ge K$, sodass die erste Summe für alle $n>N$ kleiner 
  als $\eps/3$ ist. Für jedes $n>N$ gilt dann insgesamt 
  $|E_n-e_n| < \eps$, was zu zeigen war. 
\end{antwort}

%% Question 40
\begin{frage}
  Welche berühmte Zahl wird durch den gemeinsamen 
  Grenzwert von $(e_n)$ und 
  $(E_n)$ definiert?
  \nomenclature{$e$}{Euler'sche Zahl $e$}
  \index{Eulersche Zahl@Euler'sche Zahl $e$}
  \index{Euler@\textsc{Euler}, Leonhard (1707-1783)}
\end{frage}

\begin{antwort}
  Der Grenzwert der Folgen $(e_n)_{n\in\NN}$ und $(e_n)_{n\in\NN}$ ist 
  die \textit{Euler'sche Zahl}
  $e=2,718281828459\ldots$.
  \AntEnd
\end{antwort}

%% Question 41
\begin{frage}\label{02_quaf}
  \index{Quadratwurzel!Rekursionsfolge zur Berechnung}
  Sei $a$ eine positive reelle Zahl. Warum ist die rekursive Folge 
  $(x_n)$ mit 
  \[
  x_0 := a+1\qquad\text{und}\qquad x_{n+1}:= 
  \frac{1}{2} \left( x_n+\frac{a}{x_n} \right)
  \] 
  konvergent, und was ist ihr Grenzwert?
\end{frage}

\begin{antwort}
  Wegen $a>0$ sind alle $x_n$ positiv, die Folge ist also durch null 
  nach unten beschränkt. 
  Wenn man jetzt noch zeigen kann, dass die Folge monoton fällt, dann 
  ergibt sich die Konvergenz unmittelbar aus dem Monotoniekrierium. 

  Wegen $x_n-x_{n+1} = \frac{1}{2} \left( ( x_n^2-a ) / x_n \right)$ 
  ist die Folge jedenfalls dann monoton fallend, wenn 
  $x_n \ge \sqrt{a}$ für alle $n$ gilt. Das ergibt sich wegen der 
  Positivität der $x_n$ mit 
  \[
  x_{n}-\sqrt{a} = \frac{1}{2}\left( x_{n-1} + \frac{a}{x_{n-1}} \right) 
  - \sqrt{a} 
  = \frac{ x_{n-1}^2+a-2\sqrt{a}x_{n-1} }{ 2x_{n-1} } 
  = \frac{ (x_{n-1}-\sqrt{a})^2 }{2x_{n-1}} > 0.
  \] 
  Die Folge ist somit monoton fallend und besitzt einen Grenzwert in $\RR$.
  Der Grenzwert $x$ muss die Gleichung  $x=\frac{1}{2}\left( x+\frac{a}{x} \right)$, also  $x^2=a$  
  erfüllen. Daraus folgt $x=\sqrt{a}$. 

  \medskip
  \noindent
  \slanted{Bemerkung}: Bei der Konstruktion der Folge handelt es sich um einen 
  Spezialfall des \slanted{Newton-Verfahrens} (vgl. Frage~\ref{07_newtondef}).
  \AntEnd
\end{antwort}

%% Question 42
\begin{frage}%
  \label{02_mont}
  \index{Folge!Teilfolge}
  \index{Teilfolge}
  Warum hat \slanted{jede} reelle Zahlenfolge eine monotone Teilfolge?
\end{frage}

\begin{antwort}
  Angenommen, eine 
  Folge $(a_n)$ besitzt keine monoton \slanted{wachsende} Teilfolge. 
  Zu jedem Folgenglied $a_n$ betrachte man die längstmögliche 
  monoton wachsende Serie $a_n, a_{n_1}, \ldots, a_{n_\nu}$. 
  Diese bricht in jedem Fall nach höchstens endlich vielen Schritten ab 
  (andernfalls hätte man eine monoton wachsende Teilfolge),  
  und für die Endpunkte $a_{n_\nu}$ gilt $a_m < a_{n_\nu}$ für alle 
  $m>n_\nu$. Die Folge dieser "`Spitzen"' $(a_{n_u})$ ($n=1,2,3,\ldots$) ist 
  dann eine (sogar streng) monoton fallende Teilfolge von $(a_n)$.
  \AntEnd
\end{antwort}

%% Question 43
\begin{frage}%
  \label{02_bw}
  \index{Satz!von Bolzano-Weierstraß}
  \index{Weierstrass@\textsc{Weierstrass}, Karl Theodor Wilhelm (1815-1897)}
  \index{Bolzano@\textsc{Bolzano}, Bernard (1781-1848)}
  Was besagt der \bold{Satz von Bolzano-Weierstraß} für reelle Zahlenfolgen?
\end{frage}

\begin{antwort}
  Der Satz von Bolzano-Weierstraß lautet: 

  \medskip\noindent
  \slanted{Jede beschränkte Folge 
    reeller Zahlen  besitzt eine konvergente Teilfolge}. 
  \AntEnd
\end{antwort}

%% Question 44
\begin{frage}
  Gilt der Satz von Bolzano-Weierstraß auch für \slanted{komplexe} Folgen?
\end{frage}

\begin{antwort}
  Der Satz gilt auch für komplexe Folgen $(a_n)$ mit 
  $a_n := x_n +i y_n$ ($x,y\in\RR$) und folgt aus der reellen Version. 
  Die Folge der Realteile und die Folge der Imaginärteile 
  sind in diesem Fall nämlich ebenfalls beschränkt. 
  Man kann also nach dem Satz von Bolzano-Weierstraß für reelle Folgen 
  zunächst eine 
  konvergente Teilfolge $(x_{n_k})_{k\in\NN}$ von $(x_n)$ auswählen und 
  anschließend eine konvergente Teilfolge $(y_{n_{k_\ell}})_{\ell\in\NN}$ von 
  $(y_{n_k})_{k\in\NN}$. Die Folge $(a_{n_{k_\ell}})_{\ell\in\NN}$ 
  ist dann eine konvergente Teilfolge von $(a_n)_{n\in\NN}$.
  \AntEnd
\end{antwort}

%% Question 45
\begin{frage}%
  \label{02_hfw}
  \index{Haufungswert@Häufungswert}
  Was versteht man unter einem \bold{Häufungswert} einer Zahlenfolge? 
\end{frage}

\begin{antwort}
  Ein \slanted{Häufungswert} einer Zahlenfolge $(a_n)$ ist der Grenzwert einer 
  Teilfolge $(a_{n_k})$. Aus dieser Definition folgt sofort, dass 
  in jeder $\eps$-Umgebung eines Häufungswerts $h$ unendlich viele 
  Folgenglieder liegen. Hiervon gilt aber auch die Umkehrung. 
  Liegen nämlich in jeder $\eps$-Umgebung einer Zahl $h$ unendlich viele 
  Folgenglieder, dann lässt sich leicht eine Teilfolge auswählen, die 
  gegen den Wert $h$ konvergiert. Dazu bestimme man das erste Element der 
  Teilfolge aus $U_1(h)$, das zweite aus $U_{1/2}$ und allgemein 
  das $k$-te Glied aus $U_{1/k}(h)$. Diese Teilfolge konvergiert 
  dann gegen den Wert $h$.
  \AntEnd
\end{antwort}

%% Question 46
\begin{frage}%
  \label{02_lsup}
  \index{Limes Superior ($\lim\sup$)}
  \index{Limes Inferior ($\lim\inf$)}
  \nomenclature{$\lim\sup,\,\lim\inf$}{Limes Superior, Limes Inferior}
  Warum hat die Menge der \bold{Häufungswerte} einer 
  beschränkten reellen Zahlenfolge ein Maximum 
  (genannt $\lim\sup$) bzw. Minimum (genannt $\lim\inf$)? 
\end{frage}

\begin{antwort}
  Mit der Ausgangsfolge $(a_n)$ 
  ist auch die Menge $\calli{H}$ ihrer Häufungswerte nach 
  oben beschränkt und besitzt somit ein Supremum $s$, {\dasheisst},   
  zu jedem $\eps > 0$ gibt es ein Element $h\in \calli{H}$ mit 
  $s \ge h> s-\eps/2$. Da $h$ ein Häufungswert der Folge ist, 
  gibt es eine Teilfolge, die gegen $h$ konvergiert. Ab einem 
  bestimmten Index $N$ liegen alle Glieder dieser Teilfolge in 
  $U_{\eps/2}(h)$ und somit in $U_\eps(s)$. Es liegen also unendlich 
  viele Glieder der Folge $(a_n)$ in $U_\eps(s)$, also ist $s$ ein 
  Häufungspunkt. Aus $s\in H$ und $s=\sup \calli{H}$ folgt $s=\max \calli{H}$, 
  die Menge der Häufungswerte besitzt also ein Maximum. 

  Der Beweis für die Existenz des Minimums lässt sich darauf durch Übergang zur 
  Folge $(-a_n)$ zurückführen.
  \AntEnd
\end{antwort}

%% Question 47
\begin{frage}%
  \label{02_cf}
  \index{Cauchy@\textsc{Cauchy}, Augustin Louis (1789-1857)}
  \index{Cauchy-Folge}
  Wann heißt eine Folge reeller oder komplexer Zahlen eine 
  \bold{Cauchy-Folge}?
\end{frage}  

\begin{antwort}
  Eine Zahlenfolge heißt \slanted{Cauchy-Folge}, 
  wenn es eine natürliche Zahl 
  $N$ gibt, sodass für alle $n,m \ge N$ gilt: $|a_n - a_m|<\eps$.

  Eine äquivalente Formulierung lautet: zu jedem $\eps>0$ 
  existiert ein $N\in\NN$, sodass $a_n \in U_\eps(a_N)$ für 
  fast alle $n$ gilt.   

  \begin{center}
    \includegraphics{mp/02_cauchy}
    \captionof{figure}{
      Fast alle Folgenglieder liegen in der $\eps$-Kreisscheibe um $a_N$.
    }
    \label{fig:02_cauchy}
  \end{center}
  Bei einer Cauchy-Folge "`verdichten"' sich die Glieder $a_n$ 
  mit zunehmendem $n$ also beliebig stark, \sieheAbbildung\ref{fig:02_cauchy}.  
  \AntEnd
\end{antwort}

%% Question 48
\begin{frage}%
  \label{02_cfa}
  \index{Cauchy-Folge}
  Warum ist jede konvergente reelle oder komplexe Zahlenfolge eine 
  Cauchy-Folge?
\end{frage}

\begin{antwort}
  Sei $a$ der Grenzwert der Folge. Aus $|a_n-a|\le \eps/2$ für alle 
  $n>N$ folgt dann mit der Dreiecksungleichung für alle $k,m > N$:
  \begin{equation}
    |a_k-a_m| \le | a_k -a | + | a_m-a | < \eps.  \EndTag
  \end{equation}
\end{antwort}

%% Question 49
\begin{frage}%
  \label{02_cfb}
  \index{Vollständigkeit!von $\RR$ und $\CC$}
  Warum sind $\RR$ und $\CC$ \bold{vollständig} in dem Sinne, dass 
  jede Cauchy-Folge in $\RR$ oder $\CC$ einen reellen bzw. 
  komplexen Grenzwert besitzt?
\end{frage}

\begin{antwort}
  Sei $(a_n)$ eine Cauchy-Folge, sodass $|a_n-a_m| < \eps/2 $ für alle 
  $n,m \ge N$ gilt, \sieheAbbildung\ref{fig:02_cauchy2}. 
  Die Folge ist dann beschränkt, da nur endlich viele 
  Glieder außerhalb von $U_\eps( a_N)$ liegen und 
  besitzt somit nach Bolzano-Weierstraß eine konvergente 
  Teilfolge $(a_{n_k})_{k\in\NN}$. Sei $a$ deren 
  Grenzwert, dann gibt es ein $K\in\NN$, sodass für alle 
  $k>K$ gilt $|a-a_{n_k}| < \eps/2$. Es bleibt zu zeigen, 
  dass auch die Ausgangsfolge gegen $a$ konvergiert. 

  Man wähle $k>K$ so, dass auch $n_k>N$ gilt. 
  Dann folgt mit der Dreiecksungleichung 
  aufgrund der Konvergenz der Teilfolge und der 
  Cauchy-Eigenschaft der Ausgangsfolge für alle $n>N$
  \[
  |a-a_n| \le |a-a_{n_k}| + | a_{n_k} - a_n | < \eps.
  \]      
  Die Ausgangsfolge ist somit konvergent.
  \AntEnd

  \begin{center}
    \includegraphics{mp/02_cauchy2}
    \captionof{figure}{
      Jede Cauchy-Folge in $\CC$ (und $\RR$) besitzt einen Grenzwert.
    }
    \label{fig:02_cauchy2}
  \end{center}  
\end{antwort}

%% Question 50
\begin{frage}%
  \label{02_cfc}
  \index{Cauchy-Folge}
  \index{Vollständigkeit!von $\RR$ und $\CC$}
  Kennen Sie Beispiele für Cauchy-Folgen in $\QQ$, die in $\QQ$ keinen 
  Grenzwert besitzen? 
\end{frage}

\begin{antwort}
  Die Folge rationaler Zahlen $(x_n)$ mit $
  x_n := \frac{1}{2} \left( x_{n-1} + \frac{m}{x_{n-1}} \right)$ 
  konvergiert nach Frage \ref{02_quaf} für jedes 
  $m\in\NN$ gegen den Grenzwert $\sqrt{m}$. Ist $m$ keine Quadratzahl, so ist 
  dieser Grenzwert irrational.

  Die Euler'sche Zahl $e$ als 
  gemeinsamer Grenzwert der Folgen $(e_n)$ und $(E_n)$ aus 
  Frage \ref{02_edef} ist ebenfalls irrational 
  (siehe die Antwort zu Frage \ref{02_eirr}).
  \AntEnd
\end{antwort}

%% Question 51
\begin{frage}%
  \label{02_ints}
  \index{Intervallschachtelung}
  Was versteht man bezüglich $\RR$ unter einer 
  \bold{Intervallschachtelung}?
\end{frage}

\begin{antwort}[]
  \Ant Unter einer \slanted{Intervallschachtelung} versteht man eine Folge 
  $I_1,I_2,I_3,\ldots$ 
  kompakter Intervalle $I_n$, für die gilt
  \satz{\setlength{\labelsep}{4mm}
    \begin{enumerate}
    \item[\desc{a}] $I_{n+1} \subset I_n$ für alle $n\in\NN$.\\[-3.5mm]
    \item[\desc{b}] Die Folge $( |I_n| )$ der Intervallängen ist eine 
      Nullfolge. 
    \end{enumerate}}

  \noindent
  Beispielweise bildet die Folge $(I_n)$ mit $I_n := [ 0, 1/n ]$ eine 
  Intervallschachtelung.
  \AntEnd

  \begin{center}
    \includegraphics{mp/02_intschachtelung}
    \captionof{figure}{Bei einer Intervallschachtelung gilt $I_{n+1}\subset I_n$.}
    \label{fig:02_intschachtelung}
  \end{center}
\end{antwort}

%% Question 52
\begin{frage}%
  \label{02_intp}
  \index{Intervallschachtelung}
  Was besagt das \bold{Intervallschachtelungsprinzip}?
\end{frage}

\begin{antwort}
  Das Prinzip besagt: 

  \medskip%
  \noindent%
  \textit{Zu jeder Intervallschachtelung 
    $(I_n)_{n\in\NN} \subset \RR$ existiert eine 
    reelle Zahl, die in allen Intervallen $I_n$ enthalten ist.}

  \medskip\noindent
  Die entsprechende Zahl ist damit eindeutig bestimmt. 
  Wären nämlich $a,b\in\RR$ mit $a<b$ zwei verschiedene reelle Zahlen, die 
  in allen $I_n$ enthalten sind, so würde $[ a,b ] \subset I_n$ und damit 
  $|I_n| \ge b-a$ für alle $n\in\NN$ gelten. Die Folge der Intervallängen 
  wäre damit keine Nullfolge.
  \AntEnd
\end{antwort}

%% Question 53
\begin{frage}%
  \label{02_ein}
  \index{Intervallschachtelung}
  \index{Eulersche Zahl@Euler'sche Zahl $e$}
  Können Sie nachweisen, dass die Folgen $(e_n)$ und $(\widetilde{e}_n)$ mit 
  \[
  e_n := \left( 1+\frac{1}{n} \right)^n \quad\text{und}\quad
  \widetilde{e}_n :=  \left( 1+\frac{1}{n} \right)^{n+1}
  \]
  eine Intervallschachtelung bilden?
\end{frage}

\begin{antwort}
  Es wurde bereits in Frage \ref{02_edef} gezeigt, dass 
  die Folge $(e_n)$ monoton wächst  
  und die Folge $(\widetilde{e}_n)$ 
  monoton fällt. Wegen $\widetilde{e}_n \ge e_n$ 
  gilt daher $[e_n,\widetilde{e}_{n}]
  \subset [e_{n+1},\widetilde{e}_{n+1}]$ für alle $n\in \NN$.

  Es bleibt noch zu zeigen, dass die Folge $(\widetilde{e}_n-e_n)$ 
  der Intervallängen eine 
  Nullfolge ist. Dies ergibt sich sofort aus 
  \[
  \left( 1+\frac{1}{n} \right)^{n+1} -
  \left( 1+\frac{1}{n} \right)^n = 
  \left( 1+\frac{1}{n} \right)^n \cdot \frac{1}{n}.
  \]
  Der erste Faktor auf der rechten Seite ist beschränkt 
  (vgl. Frage \ref{02_edef}), der 
  zweite konvergiert gegen null. Also gilt $\limm (\widetilde{e}_n -e_n )=0$, und 
  die beiden Folgen bilden eine Intervallschachtelung.
  \AntEnd  
\end{antwort}

%% Question 54
\begin{frage}
  Welche Zahl wird durch die Intervallschachtelung aus der letzten Frage erfasst?
\end{frage}

\begin{antwort}
  Die dadurch erfasste Zahl ist wiederum die Euler'sche Zahl $e$.
  \AntEnd
\end{antwort} 

%% Question 55
\begin{frage}%
  \label{02_eina}
  \index{Intervallschachtelung}
  Warum bilden auch die Folgen $(E_n)$ und $(\widetilde{E}_n)$ mit 
  \[
  E_n := \sumkn \frac{1}{k!} \qquad\text{und}\qquad
  \widetilde{E}_n := E_n + \frac{1}{n\cdot n!}, \quad n\in\NN
  \]
  eine Intervallschachtelung?
\end{frage}

\begin{antwort}
  Die Folge $(E_n)$ ist offensichtlich monoton wachsend, die 
  Folge $(\widetilde{E}_n)$ wegen
  \[
  \widetilde{E}_{n+1} = E_n + \frac{1}{(n+1)!} + \frac{1}{(n+1)(n+1)!} 
  = E_n + \frac{n+1}{(n+1)^2\cdot n!} \le 
  E_n + \frac{1}{n\cdot n!} = \widetilde{E}_n
  \]
  monoton fallend. 
  Da die Folge der Differenzen 
  $(\widetilde{E}_n-E_n)=( \frac{1}{n\cdot n!} )$ 
  eine Nullfolge ist, bilden die Intervalle 
  $I_n := [ E_n,\widetilde{E}_n]$ somit eine Intervallschachtelung. 
\end{antwort}

%% Question 56
\begin{frage}
  Die durch $(E_n)$ und $(\widetilde{E}_n)$ gegebene Intervallschachtelung 
  erfasst wiederum die Euler\sch e Zahl 
  $e$. Können Sie das begründen?
\end{frage} 

\begin{antwort}
  Die Antwort ergibt sich aus dem Ergebnis zu Frage \ref{02_edef}, 
  demzufolge $(e_n)$ und $(E_n)$ 
  den gleichen Grenzwert besitzen. 
  \AntEnd
\end{antwort} 

%% Question 57
\begin{frage}
  Welche der beiden Intervallschachtelungen konvergiert schneller: die durch 
  $(E_n)$ und $(\widetilde{E}_n)$ oder die durch 
  $(e_n)$ und $(\widetilde{e}_n)$ gegebene?
\end{frage}

\begin{antwort}
  Die erste Intervallschachtelung konvergiert wesentlich schneller. Für die 
  Folge der Intervalllängen gilt hier $|(I_n)|=\frac{1}{n\cdot n!}$. 
  Schon die Länge des fünften Intervalls beträgt nur noch $1/600=0.001666\ldots$
  Dagegen sind die Intervalllängen $(|I'_n|)$ 
  der durch $(e_n)$ und $(\widetilde{e}_n)$ gegebenen Intervallschachtelung 
  größer als $\frac{1}{n}$.  
  \AntEnd
\end{antwort}

%% Question 58
\begin{frage}%
  \label{02_eirr}
  \index{Intervallschachtelung}
  \index{Eulersche Zahl@Euler'sche Zahl $e$}
  Können Sie beweisen, dass die Euler'sche Zahl 
  $\dis e=\lim e_n = \dis \lim E_n$ keine rationale Zahl ist?
\end{frage}

\begin{antwort}
  Angenommen, $e$ sei rational, etwa $e=m/n$ mit $m,n \in \NN$. Dann wäre
  $n!\, e$ eine ganze Zahl. Mit $\alpha := e-E_n$ wäre also auch  
  \[
  n!\,\alpha := n! \,\left( e- E_n \right) = 
  n!\, \left( e - 1-\frac{1}{2} - \cdots - \frac{1}{n!} \right).
  \]
  eine ganze Zahl. Auf der anderen Seite folgt aber aus $\widetilde{E}_n > E$ 
  \[
  \alpha = ( e-E_n ) < ( \widetilde{E}_n - E_n ) = \frac{1}{n\cdot n!} < \frac{2}{(n+1)!},
  \] 
  und damit $0<n! \, \alpha < 2/(n+1)$, im Widerspruch zur Ganzzahligkeit von 
  $n! \, \alpha$. 
  \AntEnd 
\end{antwort}

%% Question 59
\begin{frage}%
  \label{02_fqua}
  \index{Quadratwurzel!Rekursionsfolge zur Berechnung}
  Sei $(x_n)$ die in Frage \ref{02_quaf} definierte Folge zur Approximation 
  von Quadratwurzeln. 
  Versuchen Sie zu zeigen, dass  
  für den \slanted{Fehler} $R_n := x_n-\sqrt{a}$ gilt: $
  R_{n+1} \le \frac{1}{2\sqrt{a}} R_n^2$?
\end{frage}

\begin{antwort}
  Da die Folge für $n>1$ monoton fällt, erhält man
  \begin{equation}
    R_{n+1} = \frac{1}{2} \left( x_n + \frac{a}{x_n} \right) - \sqrt{a} = 
    \frac{1}{2}  \frac{ x_n^2 + a-2x_n \sqrt{a} }{ x_n } \le 
    \frac{ ( x_n-\sqrt{a} )^2 }{ 2\sqrt{a} } = \frac{R_n^2}{2\sqrt{a}}.
    \EndTag
  \end{equation}
\end{antwort}

%% Question 60
\begin{frage}%
  \label{02_quak}
  \index{Konvergenz!quadratische}
  Was versteht man unter \bold{quadratischer Konvergenz} einer Folge?
\end{frage}

\begin{antwort}
  Eine Folge $(a_n)$ konvergiert \slanted{quadratisch} gegen 
  den Grenzwert $a$, wenn es eine Konstante 
  $C$ gibt, sodass gilt:
  \[
  |a_{n+1}-a| \le C \cdot |a_n-a|^2\quad\text{für alle $n\in\NN$}.
  \]
  Für die Folge mit $x_n=\frac{1}{2}( x_n + a/x_n)$ liegt also aufgrund 
  der Ergebnisse aus der vorigen Frage quadratische Konvergenz vor. 
  In diesem Fall ist $C=1/2\sqrt{a}$. 
  \AntEnd
\end{antwort}


%%% Local Variables: 
%%% mode: latex
%%% TeX-master: "master"
%%% End: 
