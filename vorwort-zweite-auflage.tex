\chapter*{Vorwort zur zweiten Auflage}

In dieser zweiten Auflage des Prüfungstrainers Analysis haben wir zahlreiche 
Druckfehler korrigiert, auf die wir dankenswerterweise von Studierenden 
hingewiesen wurden. Inhaltlich wurde der Text an einigen Stellen überarbeitet und ergänzt in 
der Absicht, den Gebrauch des Textes zu erleichtern. 
Neu aufgenommen wurden Fragen zur Einführung der komplexen Zahlen:  
einmal über das Standardmodell $\CC=\RR^2$, über spezielle reelle $2\times 2$-Matrizen 
und über den Restklassenring $\RR[X]/(X^2+1)$. Völlig überarbeitet wurde der 
Abschnitt über das Newton-Verfahren. Insgesamt sind 24 neue Fragen dazugekommen.

Das Literaturverzeichnis wurde ergänzt und aktualisiert. Eingeflossen sind auch 
die Erfahrungen des erstgenannten Autors aus einer jüngst gehaltenen Vorlesung 
"`Höhere Mathematik für Physiker"'. Zwischen Analysis und Linearer Algebra gibt es mannigfache Querverbindungen, die auch in diesem Prüfungstrainer zum Ausdruck kommen (\zB\ in der 
Formulierung des Hauptsatzes der Differenzial- und Integralrechnung in der 
Sprache der Linearen Algebra). Dieses Wechselspiel aufzuzeigen, war ein 
besonderes Anliegen  in  "`Grundwissen Mathematikstudium"' 
(Arens, Busam  et al., vgl. die Angabe im Literaturverzeichnis \citep{Arens}).

Wie im Vorwort zur ersten Auflage ausgeführt, orientiert sich die Stoffauswahl 
an einer dreisemestrigen, einführenden Vorlesung zur Analysis. Dabei war der bewährte 
Analysiszyklus von Otto Forster (vgl.~\citep{Forster}) 
eine wichtige Orientierungshilfe. Ab der sechsten Auflage 
des dritten Bandes (2011) hat Otto Forster die Integrationstheorie in mehreren 
Variablen auf eine maßtheoretische Grundlage (Mengenalgebren) gestellt, während 
von der ersten bis zur fünften Auflage das Lebesgue-Integral mit Hilfe des 
Daniell-Lebesgue-Prozesses eingeführt wurde. Die von uns dargestellte Einführung des 
Lebesgue-Integrals entspricht der Einführung von Forster in den ersten fünf Auflagen. 
Dass beide Methoden letztlich auf denselben Integralbegriff führen, ergibt sich aus der 
Tatsache, dass das Lebesgue-Integral als Haar'sches Maß bis auf einen 
konstanten Faktor eindeutig bestimmt ist. 

Vielfältig wird die Umstellung der Diplom- und Staatsexamensstudiengänge auf 
gestufte Studiengänge als Erfolg gewertet, jedoch sind bei der Stoffauswahl 
etliche "`Perlen"' der Analysis auf der Strecke geblieben. Die oben genannte 
Vorlesung zur Physik aber hat z.\,B. gezeigt, wie wichtig etwa auch der Themen 
wie der Differenzialformenkalkül, 
die Integration über Untermannigfaltigkeiten oder die 
Integralsätze für die Anwendungen sind. 
 
Unser besonderer Dank gilt unseren beiden Lektoren, 
Herrn Dr. Andreas Rüdinger und Frau Bianca Alton für 
ihre fachkundige, aufmerksame und geduldige inhaltliche sowie 
ls organisatorische Unterstützung. 


\bigskip
\strut\hfill Berlin, Heidelberg im Mai 2014\newline
\strut\hfill Rolf Busam\newline
\strut\hfill Thomas Epp

%%% Local Variables: 
%%% mode: latex
%%% TeX-master: "master"
%%% End: 
