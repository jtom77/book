

\chapter{Die Systeme der reellen und komplexen Zahlen}  

Die zentralen Begriffe der Analysis wie \slanted{Konvergenz}, 
\slanted{Stetigkeit}, \slanted{Integrierbarkeit}, 
\slanted{Differenzierbarkeit} etc. basieren alle auf einem exakt 
definierten \slanted{Zahlbegriff}, dessen endgültige, befriedigende 
Präzisierung nach einer fast viertausendjährigen Entwicklung 
erst gegen Ende des 19. Jahrhunderts gelang. 

Bei den folgenden Fragen gehen wir
von einer \slanted{axiomatischen Beschreibung} der Menge der 
reellen Zahlen aus, {\dasheisst} wir betrachten die reellen Zahlen als 
gegeben durch\index{Axiome der reellen Zahlen}
\begin{itemize}[2mm]
\item[\desc{i}] die Körperaxiome 
\item[\desc{ii}] die Anordnungsaxiome
\item[\desc{iii}] ein Vollständigkeitsaxiom.
\end{itemize}
Als Vollständigkeitsaxiom
\index{R@$\RR$}
\index{Vollständigkeitsaxiom}
\index{Supremumsaxiom@Supremumsaxiom}
wählen wir das \slanted{Supremumsaxiom}. 
Durch diese drei Serien von Axiomen sind die reellen Zahlen bis auf 
Isomorphie eindeutig bestimmt (vgl. etwa \citep{Ebbinghaus}). 
Einen derartigen axiomatischen Zugang findet man in 
vielen Lehrbüchern der Analysis 
wieder (z.B. \citep{Barner}, \citep{Forster} oder \citep{Koenig}). 

Daneben gibt es die Möglichkeit, die reellen Zahlen ausgehend 
von den natürlichen Zahlen, wie sie in den \slanted{Peano-Axiomen}
\index{Peano-Axiome}
\index{Peano@\textsc{Peano}, Guiseppe (1858-1932)} 
fixiert sind, über die ganzen und rationalen Zahlen zu konstruieren. 
Dieser Weg ist etwas mühselig und zeitaufwendig, 
insbesondere die Konstruktion der reellen 
Zahlen aus den rationalen Zahlen erfordert Schlussweisen und Techniken, die 
für Studierende am Anfang ihres Studiums nur schwer verdaulich sind. 
Eine klassische Darstellung für diesen Weg findet man in dem Klassiker 
von E.~Landau (vgl. \citep{Landau}), eine moderne etwa bei J.~Kramer und 
A.-M.~Pippich (vgl. \citep{Kramer}).

Bei der axiomatischen Beschreibung der reellen Zahlen findet man die 
natürlichen, die ganzen und die rationalen Zahlen als spezielle 
Teilmengen der reellen Zahlen wieder. 
Hat man die reellen Zahlen zur Verfügung, dann ist das Standardmodell 
für die \textit{komplexen Zahlen} relativ einfach zu konstruieren. 
Auch wenn die komplexen Zahlen für den Aufbau der reellen Analysis im Prinzip 
entbehrlich sind, erweisen sie sich doch als ausgesprochen nützliche 
Hilfsmittel, etwa bei Behandlung der \textit{Schwingungsdifferenzialgleichung}, und in diesem Zusammenhang der Einführung der trigonometrischen Funktionen, 
ferner bei der \slanted{Partialbruchzerlegung} und 
\slanted{Integration der rationalen 
  Funktionen} oder der Behandlung von \slanted{Fourier-Reihen}. 
Ferner sei auch an den \slanted{Fundamentalsatz der Algebra} erinnert, der 
besagt, dass jedes nicht konstante reelle oder komplexe Polynom vom Grad $n$ 
($\ge 1$) auch $n$ (im Allgemeinen komplexe) Nullstellen hat.

\section{Axiomatische Einführung der reellen Zahlen}

%: Question 1
\begin{frage}
  \label{q:koerperaxiome}
  \nomenclature{$\RR$}{Menge der reellen Zahlen}
  \nomenclature{$\KK$}{allgemeiner Körper, in der Regel $\RR$ oder $\CC$}
  Was bedeutet die Aussage "`Die Menge der reellen Zahlen 
  bildet einen \textbf{Körper}\index{Koerper@Körper}"'?
\end{frage}

\begin{antwort}
  Die Eigenschaft der reellen Zahlen, einen 
  Körper\index{Koerper@Körper} zu bilden, bedeutet, 
  dass sich innerhalb von $\RR$ (fast) 
  ohne Einschränkung addieren, subtrahieren, multiplizieren 
  und dividieren lässt. Als Ergebnis dieser Operationen erhält man jedes Mal 
  wieder eine reelle Zahl. 
  (Die einzige Ausnahme hiervon ist die Division durch Null.)

  Genauer heißt das, dass die reellen Zahlen die folgende allgemeine Definition 
  eines Körpers\index{Koerper@Körper!Axiome} erfüllen.

  \medskip
  \index{Koerper@Körper}
  \index{Kommutativgesetze}
  \index{Assoziativgesetze}
  \index{Distributivgesetze}
  \index{neutrales Element}
  \index{inverses Element}

  \noindent
  Es sei $\KK$ eine Menge mit zwei inneren Verknüpfungen, d.\,h. Abbildungen
  \begin{alignat*}{4}
    + &: \KK\times\KK \to \KK,\qquad &(a,b) &\mapsto a+b \quad &&\text{(Addition)}, \\
    \cdot &:\KK\times\KK \to \KK,\qquad &(a,b) &\mapsto a\cdot b 
    \quad &&\text{(Multiplikation)}.
  \end{alignat*}
  Dann heißt $\KK$ (genauer: das Tripel $(\KK, + , \cdot)$ ein Körper, 
  wenn folgende Axiome erfüllt sind.

  \medskip
  \noindent
  {\renewcommand{\arraystretch}{1.4}
    \begin{tabular}{@{}l|p{60mm}|p{50mm}} \hline
      (K1)&
      Für alle $a,b \in \KK$ gilt $a+b=b+a$ & Kommutativgesetz bezüglich "`$+$"' \\\hline
      (K2)&
      Für alle $a,b,c \in \KK$ gilt $(a+b)+c=a+(b+c)$ & Assoziativgesetz bezüglich 
      "`$+$"' \\\hline
      (K3)&
      Es gibt eine Zahl $0\in\KK$, sodass $a+0=a$ für alle $a\in\KK$ gilt. & 
      Existenz eines neutralen Elements bezüglich "`$+$"' \\\hline 
      (K4)&
      Zu jedem $a\in\KK$ gibt es eine Zahl $(-a)\in\KK$ mit $a+(-a)=0$ & 
      Existenz eines inversen Elements bezüglich "`$+$"' \\\hline
      (K5)&
      Für alle $a,b \in K$ gilt $a\cdot b=b\cdot a$ & 
      Kommutativgesetz bezüglich "`$\cdot$"' \\\hline
      (K6)&
      Für alle $a,b,c \in K$ gilt $(a\cdot b)\cdot c=a\cdot (b\cdot c)$ & 
      Assoziativgesetz bezüglich "`$\cdot$"' \\\hline
      (K7)&
      Es gibt eine Zahl $1\in\KK$ ($1\not=0$), sodass $a\cdot 1=a$ für alle $a\in\KK$ gilt. & 
      Existenz eines neutralen Elements bezüglich "`$\cdot $"' \\\hline 
      (K8)&
      Zu jedem $a\in\KK\mengeminus\{0\}$ 
      gibt es eine Zahl $a^{-1}\in\KK$ mit $a\cdot a^{-1}=1$ & 
      Existenz eines inversen Elements bezüglich "`$\cdot$"' \\\hline
      (K9)&
      Für alle $a,b,c\in\KK$ gilt $a\cdot (b+c) = (a\cdot b)+ (a\cdot c$) & 
      Distributivgesetz \\\hline
    \end{tabular}
  }
  
  \medskip
  \noindent
  Die neutralen Elemente 0 und 1 sind dabei eindeutig bestimmt, ebenso 
  wie die inversen Elemente $-a$ und $a^{-1}$ (für $a\not=0$).

  \medskip
  Wir benutzen außerdem die "`Vorfahrtsregel"'\index{Vorfahrtsregel} 
  ("`Punktrechnung geht vor Strichrechnung"') sowie 
  die Abkürzung $ab$ für $a\cdot b$. Damit schreibt sich das Distributivgesetz 
  einfach in der Form $a(b+c) = ab+ac$.\AntEnd
\end{antwort}


%: Question 2
\begin{frage}\index{Gruppe}
  Wie lässt sich der Körperbegriff in der Sprache der Gruppentheorie ausdrücken?
\end{frage}

\begin{antwort}
  Ein Körper $\KK$ ist bezüglich der Verknüpfung "`Addition"' 
  eine additive Abel'sche Gruppe mit dem neutralen Element $0$,  
  die von $0$ verschiedenen Elemente aus $\KK$ bilden eine Abel'sche 
  Gruppe bezüglich der Multiplikation mit neutralem Element $1\not=0$, 
  und Addition und Multiplikation sind über das Distributivgesetz  
  miteinander verbunden.\AntEnd
\end{antwort}


%: Question 3
\begin{frage} 
  Kennen Sie außer dem Körper der reellen Zahlen noch andere Körper? 
\end{frage}

\begin{antwort}
  \index{Restklassenkörper}
  Weitere Körper sind beispielsweise die rationalen Zahlen $\QQ$ und 
  die komplexen Zahlen $\CC$. 
  Beispiele für Körper mit endlich vielen Elementen 
  sind für jede Primzahl $p$ die Restklassenkörper $\ZZ/p\ZZ$. 
  Zwischen $\QQ$ und $\RR$ bzw. $\CC$ liegen unendlich viele  
  \slanted{Zwischenkörper}, {\zB} die Körper
  \index{Zwischenkoerper@Zwischenkörper} 
  $\QQ(\sqrt{2}):=\{ a+b\sqrt{2};\; a,b\in\QQ \}$ oder 
  $\QQ(i) := \{ a+b\i;\; a,b\in \QQ \}$.
  
  Die natürlichen oder ganzen Zahlen sind dagegen keine Körper, 
weil sie beide \desc{K8} nicht erfüllen (die natürlichen Zahlen erfüllen 
zudem auch \desc{K4} nicht).

  Die Elemente eines Körpers müssen 
  nicht unbedingt Zahlen sein. So ist etwa die Menge der (gebrochen) 
  rationalen Funktionen\index{Koerper@Körper!der rationalen Funktionen} 
  ebenfalls ein Körper.  
  Die Elemente dieser Menge sind alle Funktionen des Typs $p(x)/q(x)$, 
  wobei $p$ und $q$ Polynome in einem Grundkörper $\KK$ sind und $q$ 
  nicht konstant null ist.
  \AntEnd
\end{antwort}

%: Question 4
\begin{frage} 
  \label{q:koerper-rechenregeln}
  Warum gelten in einem beliebigen Körper $\KK$ die folgenden 
  Rechenregeln? Dabei seien $a,b$ beliebige Elemente aus $\KK$. 
  \index{Koerper@Körper!Rechenregeln}
  \[
  \text{\desc{i}}\quad 0\cdot a=0, \hspace{0.8cm}
  \text{\desc{ii}}\quad  (-1)(-1)=1, \hspace{0.8cm}
  \text{\desc{iii}}\quad ab=0 \LLa (a=0 \text{ \rm{oder} } b=0)\,?
  \]
\end{frage}

\begin{antwort}
  \desc{i} Mit den Bezeichnungen für die Körperaxiome aus Frage~\ref{q:koerperaxiome} gilt
  \[
  0\cdot a = (0+0)\cdot a 
  \stackrel{K9}{=}0\cdot a+0\cdot a \stackrel{K4}{\Ra} 
  0\cdot a-0\cdot a = 0\cdot a 
  \stackrel{K4}{\Ra} 0=0\cdot a.
  \]
  Hieraus folgt auch der fundamentale Zusammenhang $(-1)a=-a$.

  \medskip
  \noindent
  \desc{ii} Zunächst gilt mit \desc{i}
  \[
  0 \stackrel{\text{\desc{i}}}{=}-1\cdot 0 \stackrel{K4}{=} -1(1-1)
  \stackrel{K9}{=} -1\cdot 1 +(-1)\cdot (-1) \stackrel{K7}{=} -1 +(-1)(-1).
  \]
  Die Addition von $1$ auf beiden Seiten der Gleichung liefert dann
  \[
  0+1 = -1+(-1)(-1)+1 \stackrel{K1,\,K4}{\Longrightarrow} 
  0+1 = 0+(-1)(-1)\stackrel{K3}{\Longrightarrow} 1=(-1)(-1).
  \]
 
  \medskip\noindent
  \desc{iii} Die Richtung "`$\La$"' folgt aus \desc{i}. 
  Für den Beweis der anderen Richtung nehmen wir {\oBdA} 
  $a\not=0$ an. Dann gibt es 
  ein Element $a^{-1}\in K$ mit $a^{-1}a=1$. Zusammen mit \desc{ii} folgt 
  dann
  \begin{equation}
    ab=0 \Ra a^{-1}a b = a^{-1} \cdot 0  
    \Ra 1\cdot b=0 \Ra b=0. \EndTag
  \end{equation}
\end{antwort}

%: Question 5
\begin{frage}
  Gilt die Eigenschaft \desc{iii} 
  ("`Nullteilerfreiheit"')\index{Nullteilerfreiheit} aus 
  Frage~\ref{q:koerper-rechenregeln} auch für reelle $2\times2$-Matrizen?
\end{frage}

\begin{antwort}
  Die Eigenschaft ist nicht erfüllt. Als Gegenbeispiel betrachte man etwa 
  \[
  \left(\begin{array}{rr} 0 & 0 \\ 0 & 1 \end{array}\right) \cdot 
  \left(\begin{array}{cc} 0 & 1 \\ 0 & 0 \end{array}\right) = 
  \left(\begin{array}{cc} 0 & 0 \\ 0 & 0  \end{array}\right),
  \]
wobei folgende Definition der 
Matrizenmultiplikation\index{Matrizenmultiplikation} 
verwendet wurde:
\[
\left(\begin{array}{rr} 
a_{11} & a_{12} \\ a_{21} & a_{22} 
\end{array}\right)
\cdot 
\left(\begin{array}{rr} 
b_{11} & b_{12} \\ b_{21} & b_{22} 
\end{array}\right)
=
\left(
\begin{array}{rr}
a_{11}b_{11} + a_{12}b_{21} \\
a_{21}b_{12} + a_{22}b_{22} 
\end{array}
\right).
\]
  Die reellen $2\times 2$-Matrizen bilden somit keinen Körper. 
  Das Beispiel zeigt die Anwendung eines sehr nützlichen Prinzips: 
  \slanted{Eine Menge, die nicht nullteilerfrei ist, kann kein Körper sein.} 
  \AntEnd
\end{antwort} 

%: Question 6
\begin{frage} 
  Was wissen Sie über die Elementanzahl eines 
  \textbf{endlichen Körpers}?
  \index{Koerper@Körper!endlicher}
\end{frage}

\begin{antwort}
  Die Anzahl der Elemente eines endlichen Körpers ist eine Primzahl 
  $p$ oder allgemeiner 
  eine Primzahlpotenz $q=p^n$ für eine natürliche Zahl $n\in\NN, n\ge 2$. 

  Im ersten Fall ist der Körper isomorph zum Restklassenkörper\index{Restklassenkörper}  
  $\FF_p:=\ZZ/p\ZZ$. 
  Die Körper mit $p^n$ Elementen sind dann allesamt  
  Erweiterungskörper\index{Erweiterungskörper} vom Grad 
  $n$ über $\FF_p$. 
  Das bedeutet, dass man sie aus $\FF_p$ 
  durch die "`Adjunktion"' der Nullstelle 
  $\alpha$ eines 
  irreduziblen Polynoms $f(X)\in\FF_p[X]$ vom Grad $n$ erhält. 
  Die Elemente des so erhaltenen Körpers besitzen dann alle 
  eine eindeutige Darstellung 
  \[
  a_{n-1} \alpha^{n-1} + a_{n-2}\alpha^{n-2}+\ldots + a_{1}\alpha + a_{0}, 
  \quad a_j \in \FF_p,
  \]
  wobei die algebraischen Beziehungen zwischen diesen durch die
  Gleichung $f(\alpha) = 0$ genau bestimmt sind. 
  Man beachte, dass für $n\ge2$ ein Körper mit $p^n$ Elementen grundverschieden ist vom  
  Restklassenring $\ZZ/p^n\ZZ$. Letzterer ist überhaupt kein Körper, da er 
  nicht nullteilerfrei ist. 
  \AntEnd
\end{antwort}

%: Question 7
\begin{frage} 
  Gibt es Körper mit $4$, $9$, $1\,024$, $65\,537$ bzw. $999\,997$ Elementen?
\end{frage}

\begin{antwort}
  Die Zahlen $4=2^2$, $9=3^2$ und $1\,024=2^{10}$ sind Primzahlpotenzen, 
  die Zahl $65\,537$ ist sogar selbst eine Primzahl 
  (nebenbei gesagt: Es ist die größte bekannte 
  \textit{Fermat'sche Primzahl} $F_4 := 2^{2^4}+1$).
  Also existiert nach 
  Frage 5 für diese Zahlen jeweils ein Körper mit der entsprechenden Anzahl 
  an Elementen. 
  Wegen $999\,997=757\cdot 1\,321$ gibt es 
  aber keinen Körper mit $999\,997$ Elementen.
  \AntEnd
\end{antwort}


%: Question 8
\begin{frage}
  \index{Koerper@Körper!mit zwei Elementen}
  Der "`kleinste"' Körper hat zwei Elemente, 
  nennen wir sie 
  $\overline{0}$ und $\overline{1}$. Wie addiert und multipliziert man 
  in diesem Körper?
\end{frage}

\begin{antwort}
  Da die Körperaxiome $\ov{1}+\ov{0}=\ov{1}$ festschreiben, bleibt als 
  additiv inverses Element der $\ov{1}$ nur $\ov{1}$ selbst übrig, es 
  muss also $\ov{1}+\ov{1}=\ov{0}$ gelten. 

  Die Wirkung der Multiplikation mit $\ov{0}$ und diejenige 
  der Multiplikation mit $\ov{1}$ ist durch die Körperaxiome 
  \textit{a priori} festgelegt. 
  Damit sind die Ergebnisse aller möglichen Operationen 
  bestimmt. Diese führen auf die in Abbildung~\ref{fig:01_koerper} 
  stehenden Verknüpfungstafeln.
  \AntEnd

  \begin{center}
    \includegraphics{mp/01_koerper}
    \captionof{figure}{Verknüpfungstafeln für den Körper mit zwei Elementen}
    \label{fig:01_koerper}
  \end{center}

  Dieser Körper wird in der Literatur mehrheitlich mit $\ZZ/2\ZZ$ oder $\FF_2$ bezeichnet.
\end{antwort}


%: Question 9
\begin{frage} 
  Wie lassen sich in einem beliebigen Körper $\KK$ die folgenden 
  \bold{Regeln der Bruchrechnung}\index{Bruchrechnung, Regeln}
  möglichst einfach beweisen? 
  ($a,b,c,d \in \KK, \, b\not=0, \, d\not=0$)
  \[
  \begin{array}{lcl}
    \displaystyle \text{\desc{i}}\quad
    \frac{a}{b} = \frac{c}{d} \LLa ad = bc & \rule{20mm}{0mm} &
    \displaystyle \text{\desc{ii}}\quad
    \frac{a}{b} \pm \frac{c}{d} = \frac{ad \pm bc}{bd} \\[3mm]
    \displaystyle \text{\desc{iii}}\quad
    \frac{a}{b}\cdot \frac{c}{d} = \frac{ac}{bd}  & \rule{20mm}{0mm} &
    \displaystyle \text{\desc{iv}}\quad
    \frac{  \frac{a}{b} }{ \frac{c}{d} } = \frac{ad}{bc} 
    \quad (\text{falls auch $c\not=0$})
  \end{array}
  \] 
\end{frage}


\begin{antwort}
  \desc{i} "`$\Ra$"': Multiplikation der ersten 
  Gleichung mit $bd \not=0$ ergibt 
  \[
  \frac{a}{b} = \frac{c}{d} \Ra 
  (ab^{-1})bd=(cd^{-1})bd \Ra 
  a(b^{-1}b)d=bc(dd^{-1}) \Ra ad = bc.
  \]
  "`$\La$"': Multiplikation der zweiten Gleichung mit $(bd)^{-1}$ ($bd\not=0$) 
  liefert
  \[
  ad=bc \Ra 
  ad(bd)^{-1} = bc (bd)^{-1} \Ra
  a(dd^{-1})b^{-1} = (bb^{-1}) c d^{-1} \Ra 
  \frac{a}{b} = \frac{c}{d}. 
  \]
  Folgerung: Mit $c=ax$ und $d=bx$ ($x\not=0$) folgt hieraus 
  die "`Kürzungsregel"': 
  $\frac{a}{b}=\frac{ax}{bx}$.

  \medskip\noindent
  \desc{ii}\quad Aus der Kürzungsregel und dem Distributivgesetz folgt
  \[
  \frac{a}{b}\pm \frac{c}{d} = 
  \frac{ad}{bd} \pm \frac{cb}{db} = 
  \frac{ad}{bd} \pm \frac{cb}{bd} = 
  \frac{ad+cb}{bd}.
  \]

  \medskip\noindent
  \desc{iii}\quad
  $\displaystyle 
  \frac{a}{b}\cdot\frac{c}{d} = 
  d^{-1}b^{-1}( ac ) =  
  (bd)^{-1} ac = \frac{ac}{bd}$.

  \medskip\noindent
  \desc{iv}\quad
  $\displaystyle 
  \frac{a/b}{c/d} = (d^{-1}c)^{-1} b^{-1} a = 
  c^{-1} db^{-1} a = (bc)^{-1} ad = \frac{ad}{bc}$ \quad mit $c\not=0$.
  \AntEnd
\end{antwort}


%: Question 10
\begin{frage} 
  Wie kann man für zwei Elemente $a,b$ eines Körpers $\KK$
  mithilfe der Axiome die 
  \textbf{Binomische Formel}\index{Binomische Formel}
  \[
  (a+b)^2 =a^2+2ab +b^2 
  \]
  beweisen? Dabei sei $2:=1+1$ und $x^2 := x\cdot x$ f\"ur jedes 
  $x\in\KK$.  
\end{frage} 

\begin{antwort}
  Die Formel ergibt sich im Wesentlichen 
  durch zweimaliges Anwenden des Distributivgesetzes 
  und einer Anwendung des Kommutativgesetzes:
  \nobelowskip
  \begin{align}
    (a+b)\cdot (a+b) &= a(a+b)+b(a+b)= (aa +ab) +(ba+bb) = 
    aa +( ab+ab ) +bb \notag \\
    &= a^2 + ( (1+1)ab )+b^2 = a^2+2ab +b^2. \EndTag
  \end{align}
\end{antwort}






\begin{frage}
  \label{01_anax} 
  Was bedeutet die Aussage "`Der Körper $\RR$ der reellen Zahlen ist ein 
  \textbf{angeordneter Körper}\index{Koerper@Körper!angeordneter}"'?
  
  \medskip
  Was versteht man allgemein unter einem angeordneten Körper?
\end{frage}

\begin{antwort}
  Das bedeutet, dass in $\RR$ gewisse Zahlen als \textit{positiv} ausgezeichnet 
  sind. Genauer: Auf $\RR$ ist eine Relation "`$>$"' gegeben, die die 
  folgenden beiden Eigenschaften besitzt\index{Anordnungsaxiome}
  \begin{itemize}[3mm]
  \item[(A1)] \slanted{Für jede reelle Zahl $a$ gilt entweder 
      $a>0$, $-a>0$ oder $a=0$.}
  \item[(A2)] \slanted{Aus $a>0$ und $b>0$ folgt $a+b>0$ und $ab>0$.}
  \end{itemize}
  Diese Eigenschaft von $\RR$ ist die Grundlage dafür, 
  dass Größenverhältnisse durch die Angabe einer reellen 
  Zahl angegeben werden können. 

  \medskip
  Allgemein heißt ein Körper $\KK$ angeordnet, 
  wenn eine Menge $P\subset \KK$ (ein 
  \textbf{Positivitätsbereich}\index{Positivitätsbereich}) mit den folgenden 
  Eigenschaften ausgezeichnet ist:
  \begin{itemize}
  \item[\desc{1}] Für jedes $x\in\KK$ gilt genau eine der beiden Beziehungen $x\in P$, $x\not=0$ oder $-x\in P$ (Trichotomie)
  \item[\desc{2}] $x,y \in P \Ra x+y \in P$ (Abgeschlossenheit gegenüber Addition)
  \item[\desc{3}] $x,y \in P \Ra x\cdot y \in P$ (Abgeschlossenheit gegenüber Multiplikation)
  \end{itemize}
  Im Fall $K=\RR$ ist $P=\{x\in\RR;\; x>0\}$.\AntEnd

\end{antwort}





%: Question 12
\begin{frage} 
  \label{rationalefunktionen}
  Gibt es außer dem Körper der reellen Zahlen noch 
  andere angeordnete Körper? Zählen Sie einige auf.
  \index{Koerper@Körper!angeordneter}
\end{frage}

\begin{antwort}
  Beispielsweise ist $\QQ$ als Teilkörper der reellen Zahlen mit der 
  Ordnungsstruktur von $\RR$ automatisch ebenfalls 
  ein angeordneter Körper. 

  Der Körper der rationalen reellen 
  Funktionen\index{Koerper@Körper!der rationalen Funktionen}
  lässt sich ebenfalls anordnen. 
  Dazu schreibe man $p(x)/q(x)$ in der (eindeutig bestimmten) Form 
  \[
  \frac{p(x)}{q(x)} = g(x)+ \frac{r(x)}{q(x)} , \qquad \deg r < \deg q.
  \]
  Eine Anordnung auf dem Körper der rationalen Funktionen erhält man 
  damit folgendermaßen: Die Funktion von $\frac{p(x)}{q(x)}$ 
  ist positiv genau dann, wenn gilt
  {\setlength{\labelsep}{4mm}
    \begin{itemize}
    \item[(i)] \slanted{$g(x)\not=0$ und der Koeffizient der höchsten Potenz 
        von $g(x)$ ist $>0$ oder} \\[-3.5mm]
    \item[(ii)]\slanted{$g(x)=0$ und der Koeffizient der höchsten 
        Potenz von $r(x)$ ist $>0$.}
    \end{itemize}}
  Man zeigt leicht, dass die Definition die beiden Axiome 
  \desc{A1} und 
  \desc{A2} erfüllt und damit tatsächlich eine Ordung auf dem Körper der 
  rationalen Funktionen gegeben ist.

  Diejenigen Körper, für die \textit{keine} Anordung existiert,  
  sind weit häufiger. Zum Beispiel lässt sich $\CC$ wegen 
  $\i^2=-1$ nicht anordnen, ebensowenig die endlichen Körper. 
  \AntEnd
\end{antwort}

%: Question 13
\begin{frage} 
  Warum lässt sich ein endlicher Körper\index{Koerper@Körper!endlicher} 
  nicht anordnen? 
\end{frage}

\begin{antwort}
  In einem angeordneten Körper ist stets $1>0$. Damit 
  gilt aufgrund von \desc{A2} auch für alle endlichen Summen $
  1+1+\cdots+1 > 0$.
  In einem endlichen Körper mit $q=p^n$, $n\in\NN$ 
  Elementen ist aber stets 
  $\underbrace{1+1+\cdots + 1}_{p\text{-mal}} =0.$\AntEnd
\end{antwort}

%: Question 14
\begin{frage} 
  Wie ist der \textbf{(Absolut-) Betrag}\index{Betrag!einer reellen Zahl}
  einer reellen Zahl definiert und welche Haupteigenschaften hat er?
\end{frage}

\begin{antwort}
  Der Betrag $|a|$ einer reellen Zahl ist definiert durch
  \[
  |a| := \left\{ \begin{array}{rl} a \quad & \text{für $a\ge 0$}, \\
      -a \quad & \text{für $a<0$}. 
    \end{array} \right. \EndTag
  \]
\end{antwort} 


%: Question 15
\begin{frage}\label{01_betrag_eigensch}\index{Betrag!einer reellen Zahl}
  Welche Haupteigenschaften besitzt der Betrag?
\end{frage}

\begin{antwort}
  Die wesentlichen Eigenschaften sind
  \[
  \begin{array}{ll@{\qquad}ll}
    \text{\desc{1}} & |a| \ge 0  \text{ und } |a|=0\Leftrightarrow a=0& 
    \text{\desc{2}} & -|a| \le a \le |a| \\
    \text{\desc{3}} & |ab| = |a|\cdot |b| & 
\text{\desc{4}} & \left|\frac{a}{b}\right| = \frac{|a|}{|b|} \quad\text{für $b\not=0$}\\
    \text{\desc{5}} & |a+b| \le |a|+|b|  &  
    \text{\desc{6}} & \big||a| - |b|\big| \le |a-b|. \\
  \end{array}
  \]
  Die Eigenschaften \desc{1} und \desc{2} folgen 
  unmittelbar aus der Definition des 
  Absolutbetrages, 
  \desc{3} ergibt sich mithilfe einer Fallunterscheidung, 
  \desc{4} und \desc{5} werden 
  in den nächsten beiden Fragen beantwortet. \AntEnd
\end{antwort}


%: Dreiecksungleichung
\begin{frage}
  \label{dreiecksungleichung}
  Was besagt die \textbf{Dreiecksungleichung}
  \index{Dreiecksungleichung!fuer reelle Zahlen@für reelle Zahlen}
  für reelle Zahlen?
\end{frage}

\begin{antwort}
  Die Dreiecksungleichung lautet: \satz{Für zwei reelle Zahlen $a,b$ 
    gilt stets}
    \[
    \boxed{
      |a+b| \le |a|+|b| .
    }
    \]
\noindent%
Die Dreiecksungleichung folgt mit Eigenschaft \desc{2} 
  des Absolutbetrages aus den beiden  
  \[
  a+b\le |a|+|b|, \qquad -(a+b)\le |a|+|b|. 
  \] 
  Wendet man darauf die Definition des Absolutbetrages an, 
  so erhält man die Dreiecksungleichung. \AntEnd
\end{antwort}

%: Question 17
\begin{frage}
  \label{01_dru}
  Wie lautet die Dreiecksungleichung für Abschätzungen
  \index{Dreiecksungleichung!fuer Abschaetzungen@für Abschätzungen nach unten}
  nach unten?
\end{frage}

\begin{antwort}
  Das ist die Ungleichung \desc{5} aus Antwort \ref{01_betrag_eigensch}:
  \[
  \boxed{
    \big| |a|-|b| \big| \le |a-b|. 
  }
  \]
  Man erhält sie aus der Dreiecksungleichung. Mit dieser gilt zunächst
  \[
  |a| = |a-b+b| \le |a-b|+|b| \quad\text{ und } \quad
  |b| = |b-a+a| \le |a-b|+|a|.
  \]
  Daraus folgt $(|a|-|b|)\le |a-b|$ und $-(|a|-|b|)\le |a-b|$ und die 
  Dreiecksungleichung für Abschätzungen nach unten.  Den Beweistrick einer 
  Addition mit $0$ sollte man sich übrigens merken. 
  Er ermöglicht es einem, die Dreiecksungleichung anzuwenden und damit 
  die Struktur metrischer Räume in Beweisen auch wirklich auszunutzen.
  \AntEnd
\end{antwort}


%: Question 18
\begin{frage}\label{q:metrischer-raum}
  Wieso hat die Zuordnung (Abbildung) 
  \[
  d\,:\, \RR \times \RR \to \RR, \quad (x,y) \mapsto d(x,y)=|x-y| 
  \]
  für $x,y,z\in\RR$ die folgenden Eigenschaften:
  \begin{itemize}[14mm]
  \item[(M1)] $d(x,y)=0 \Leftrightarrow x=y$,\\[-3.5mm]
  \item[(M2)] $d(x,y)=d(y,x)$,\\[-3.5mm]
  \item[(M3)] $d(x,z) \le d(x,y)+d(y,z)$.
  \end{itemize}
  \noindent
  $d(x,y) := |x-y|$ heißt der Abstand von $x$ und $y$.\index{Abstand}
  \index{Metrik!der reellen Zahlen}
\end{frage}

\begin{antwort}
  Eigenschaft \desc{M1} folgt aus $|x-y|=0 \LLa x-y=0$, 
  \desc{M2} ergibt sich wegen 
  $(x-y)=-(y-x)$ aus der Definition des Absolutbetrages, \desc{M3} 
  erhält man mit der Dreiecksungleichung: 
  $|x-z| = |x-y+y-z| \le |x-y|+|y-z|$. 

  Die Eigenschaften \desc{M1}, \desc{M2} und \desc{M3} besagen zusammen,
  dass durch $d$ eine \textit{Metrik} auf $\RR$ gegeben ist.   
  \AntEnd
  
\end{antwort}

%: Question 19
\begin{frage}
  Was versteht man unter einem 
  \textbf{metrischen Raum}?\index{Metrischer Raum}
\end{frage}

\begin{antwort}
  Ein metrischer Raum ist eine nichtleere Menge $X$ zusammen mit einer 
  Abbildung $d\fd X \times X\to \RR$ (einer \textit{Metrik}), 
  die die Eigenschaften \desc{M1}, \desc{M2}
  und \desc{M3} aus Frage~\ref{q:metrischer-raum} erfüllt. 

  Anschaulich ist ein metrischer Raum eine Menge, 
  für die ein Begriff des \slanted{Abstands} je zweier Elemente existiert. 
  Die Axiome \desc{M1}, \desc{M2} und \desc{M3} fordern, 
  dass dieser Abstandsbegriff geometrisch sinnvoll ist. 
  \AntEnd
\end{antwort} 

%: Question 20
\begin{frage}
  Warum folgt aus \desc{M1}, \desc{M2} und 
  \desc{M3} stets $d(x,y)\ge 0$ für alle 
  $x,y \in\RR$?
\end{frage}

\begin{antwort}
  Für alle $x,y \in \RR$ gilt 
  $0 = d(x,x) \le d(x,y)+d(y,x) = 2d(x,y)$,
  also $d(x,y)\ge 0$. 
  \AntEnd
\end{antwort}

%: Question 21
\begin{frage} 
  Kennen Sie außer $\RR$ weitere metrische Räume?
  \index{metrischer Raum}
\end{frage}

\begin{antwort}
  Die metrischen Räume sind breit gefächert und tauchen speziell in der 
  Analysis in den verschiedensten Formen auf. 
  $\NN$, $\ZZ$ und $\QQ$, versehen mit durch den Absolutbetrag 
  induzierten Metrik, sind Beispiele metrischer Räume.
  Die komplexen Zahlen $\CC$ mit $d(z,w):=|z-w|$  
  sowie die endlichdimensionalen euklidischen 
  Vektorräume $\RR^n$ sind mit 
  $d(x,y):=\sqrt{(x_1-y_1)^2+\ldots+(x_n-y_n)^2}$ ebenfalls metrische Räume. 

  Wichtige Untersuchungsgegenstände der Analysis sind 
  \slanted{Funktionenräume}. Diese lassen sich ebenfalls mit einer 
  (dem jeweiligen Problemkreis angemessenen) Metrik versehen. Im Raum der 
  stetigen Funktionen $\calli{C}([a,b])$ auf einem kompakten 
  Intervall $[a,b]$ ist etwa durch 
  \[
  d(f,g) := \sup\,\big\{ |f(x)-g(x) | \sets x\in [a,b] \big\}
  \]
  eine Metrik gegeben. Eine andere Metrik erhielte man auf demselben Raum 
  {\zB} durch 
  \[
  d(f,g) := \int_a^b \big|f(x)-g(x) \big| \difx. \EndTag
  \]
\end{antwort}

%: Question 22
\begin{frage}
  Wie lässt sich eine beliebige Menge $X$ mit mindestens zwei Elementen 
  zu einem metrischen Raum machen?
\end{frage}

\begin{antwort}
  Durch die Abbildung 
  \[
  d(x,y) := \left\{ \begin{array}{ll} 
      1\quad & \text{für $x\not=y$,} \\
      0\quad & \text{für $x=y$} \end{array} \right.
  \]
  ist auf $X$ eine Metrik (die sogenannte \slanted{diskrete Metrik})
  \index{diskrete Metrik} definiert.

  Wir beschränken uns auf den Nachweise der Dreiecksungleichung. Dabei können 
  wir annehmen, dass $x$, $y$ und $z$ paarweise verschieden sind. Dann ist
  \[
  d(x,z) = 1 \le d(x,y) + d(y,z) = 2. 
  \]
  \AntEnd
\end{antwort}

%: Question 23
\begin{frage}\index{Schranke}
  Was versteht man unter einer oberen bzw. unteren \bold{Schranke} 
  einer nichtleeren Teilmenge $M\subset\RR$?
\end{frage}

\begin{antwort}  
  Eine Zahl $C\in \RR$ heißt 
  \textit{obere bzw. untere Schranke}\index{Schranke}
  von $M\subset \RR$, 
  wenn für jedes $a\in M$ gilt: $a\le C$ bzw. $a\ge C$.\AntEnd
\end{antwort} 

%: Question 24
\begin{frage}
  Wie sind $\sup M$ \index{Supremum} und 
  $\inf M$\index{Infimum} \nomenclature{$\inf M$}{Infimum von der Menge $M$}
  \nomenclature{$\sup M$}{Supremum der Menge $M$}
  für eine nichtleere, nach oben 
  bzw. nach unten beschränkte Teilmenge $M$ von $\RR$ definiert?
\end{frage}

\begin{antwort}
  $\sup M$ bzw. $\inf M$ sind definiert als \textit{kleinste 
    obere bzw. größte untere Schranke} von $M$. Das bedeutet
  \begin{eqnarray*}
    s = \sup M &\LLa & \left\{ \begin{array}{ll}
        \text{\desc{i}} &  \text{$s$ ist obere Schranke von $M$ und} \\[-1mm]
        \text{\desc{ii}} & \text{für alle oberen Schranken $a$ von $M$ gilt $s\le a$},
      \end{array} \right. 
    \\
    t = \inf M &\LLa& \left\{ \begin{array}{ll}
        \text{\desc{i}} & \text{$t$ ist untere Schranke von $M$ und} \\[-1mm]
        \text{\desc{ii}} & \text{für alle unteren Schranken $b$ von $M$ gilt $t\ge b$}.
      \end{array} \right. 
  \end{eqnarray*}
  Man nennt $\sup M$ das \textit{Supremum} und 
  $\inf M$ das \textit{Infimum} von M.
  \AntEnd
\end{antwort}

%: Question 25
\begin{frage}
  Was besagt das 
  \bold{Vollständigkeitsaxiom}\index{Vollständigkeitsaxiom} 
  in Form des Supremumsaxioms
  für die reellen Zahlen?
\end{frage}

\begin{antwort}
  Das Vollständigkeitsaxiom lautet in dieser Form: 

  \medskip\noindent\satz{\noindent
    Jede nichtleere, nach oben beschränkte Teilmenge reeller Zahlen 
    besitzt ein Supremum}.\AntEnd
\end{antwort}


%: Question 26
\begin{frage}
  Können Sie zeigen, dass 
  $\inf M=-\sup(-M)$ für jede nichtleere nach unten beschränkte 
  Teilmenge $M$ von $\RR$ gilt? 
\end{frage}

\begin{antwort}
  Die Identität ergibt sich jeweils aus einer 
  der beiden "`Spiegelungsformeln"'   
  \[
  -\sup M =\inf (-M), \qquad
  -\inf M = \sup (-M),  
  \]
  die in der Abbildung~\ref{fig:01_supinf} veranschaulicht sind. 

  \begin{center}
    \includegraphics{mp/01_supinf}
    \captionof{figure}{Spiegelung von $M\subset\RR$ am Nullpunkt}
    \label{fig:01_supinf}
  \end{center}

  \noindent
  Wir zeigen die erste der beiden Formeln, die 
  zweite erhält man auf analoge Weise. 

  Aus $\sup M \ge a$ für alle $a\in M$ folgt zunächst 
  $-\sup M \le -a$ für alle $a \in M$ oder, was dasselbe ist,  
  $-\sup M \le b$ für alle $b\in -M$. 
  Folglich ist $-\sup M$ eine untere Schranke von $-M$, und wir 
  brauchen nur noch zu zeigen, dass es die größte untere Schranke ist. 
  
  Sei dazu $S$ eine weitere untere Schranke von $-M$.  
  Dann gilt $S\le b$ für alle $b\in -M$ und somit
  $-S \ge a$ für alle $a\in M$, also ist $-S$ eine obere Schranke von $M$ und 
  damit $-S \ge \sup M$. Es folgt $S \le -\sup M$. Somit ist  
  $-\sup M$ tatsächlich die größte untere Schranke von $-M$, {\dasheisst}    
  $-\sup(M)=\inf (-M)$, was zu zeigen war.
  \AntEnd
\end{antwort}

%: Question 27
\begin{frage}
  \index{Supremum}
  \index{Infimum}
  \index{Maximum}
  Was ist der Unterschied zwischen $\sup M$ und $\max M$? 
\end{frage}

\begin{antwort}
  Im Allgemeinen ist $\sup M$ kein Element von $M$, 
  beispielsweise ist bei einem offenen Intervalle $]a,b[$ das Supremum 
  $b$ zwar die kleinste obere Schranke des 
  Intervalls, jedoch nicht darin enthalten. 
  Im Gegensatz dazu ist das Maximum einer Menge stets auch ein Element dieser 
  Menge. \AntEnd
\end{antwort} 

%: Question 28
\begin{frage}
  Wann gilt $\sup M=\max M$ für eine nach oben 
  beschränkte Menge $M\subset\RR$?
\end{frage}

\begin{antwort}
  Es ist $\sup M =\max M$ genau dann, wenn 
  $\sup M \in M$ gilt. In diesem \textit{und nur in diesem} Fall existiert 
  $\max M$.
  \AntEnd
\end{antwort}


%: Question 29
\begin{frage}
  \label{q:sup-eps-charak}
  \index{Supremum!epsilon@$\eps$-Charakterisierung}
  \index{Infimum!epsilon@$\eps$-Charakterisierung}
  Wie lauten die $\eps$-Charakterisierungen von $\sup M$ bzw. $\inf M$?
\end{frage}

\begin{antwort}
  Die Eigenschaft 
  von $\sup M$, unter allen oberen Schranken die 
  \textit{kleinste} zu sein, ist offenbar gleichbedeutend 
  damit, dass für jedes $\eps>0$ 
  die Zahl $\sup M-\eps$ keine obere 
  Schranke von $M$ mehr sein kann, denn 
  andernfalls wäre $\sup M$ nicht die kleinste obere Schranke. 
  Genauso ist $\inf M$ unter allen unteren Schranken von $M$ dadurch 
  ausgezeichnet, dass $\inf M+\eps$ für jedes $\eps >0$ 
  keine untere Schranke von $M$ mehr sein kann. Infimum und Supremum lassen 
  sich somit folgendermaßen charakterisieren 
  \begin{align}
    s = \sup M &\LLa \left\{ \begin{array}{ll}
        \text{\desc{i}} &  \text{$s$ ist obere Schranke von $M$ und} \\
        \text{\desc{ii}} & \text{für alle $\eps>0$ gibt es ein $a\in M$ mit $s-\eps<a$},
      \end{array} \right. \notag 
    \\
    t = \inf M &\LLa \left\{ \begin{array}{ll}
        \text{\desc{i}} & \text{$t$ ist untere Schranke von $M$ und} \\
        \text{\desc{ii}} & \text{für alle $\eps>0$ gibt es ein $b\in M$ mit $b<t+\eps$}.
      \end{array} \right. \EndTag
  \end{align}
\end{antwort}

%: Archimedische Eigenschaft
%: Question 30
\begin{frage}
  Können Sie zeigen, dass das Infimum der Menge 
  $M:=\left\{ 1, \frac{1}{2}, \frac{1}{3}, \frac{1}{4}, \ldots \right\}$ 
  gleich null ist?  Wieso gilt $\max M=\sup M=1$?
\end{frage}

\begin{antwort}
  Für alle $n \in \NN$ ist $\frac{1}{n}>0$, 
  und somit ist $0$ eine untere Schranke von $M$. 
  Sei nun $\varepsilon > 0$ gegeben. Gemäß der  
  \textit{Archimedischen Eigenschaft} (vgl. Frage \ref{01_archimedes}) 
  gibt es eine natürliche Zahl $m$ mit $m>\frac{1}{\varepsilon}$, woraus 
  $\frac{1}{m} < \varepsilon = 0+\varepsilon $ folgt. Wegen $\frac{1}{m}\in M$ 
  erfüllt die Null also auch die Bedingung (ii) aus der Antwort zur vorigen 
  Frage, woraus $\inf M=0$ folgt.

  Zum Nachweis von $\sup M=1$ genügt es zu bemerken, dass wegen $n \ge 1$ 
  für alle natürlichen Zahlen $\frac{1}{n}\le 1$ gilt. 
  Damit ist $1$ eine obere Schranke von $M$,  
  wegen $1\in M$ gilt dann aber schon $1=\max M=\sup M$.
  \AntEnd
\end{antwort}

%: Question 31
\begin{frage}
  Seien $A$ und $B$ nichtleere Teilmengen von $\RR$, und es gelte 
  $a\le b$ für alle $a\in A$ und alle $b\in B$. 
  Zeigen Sie das sogenannte \textbf{Riemann-Kriterium}: 
  \index{Riemann-Kriterium}
  Es gilt 
  $\sup A=\inf B$ genau dann, wenn es zu jedem $\eps > 0$ ein $a\in A$ 
  und ein $b\in B$ mit $b-a < \eps$ gibt. 
\end{frage}

\begin{antwort}
  "`$\Ra$"':\quad Sei $\sup A = \inf B$ und $\eps>0$ gegeben. 
  Dann gibt es Zahlen $a\in A$ und $b\in B$, {\sd} die folgenden 
  beiden Ungleichungen erfüllt sind: 
  \[
  \inf B+\frac{\eps}{2} > b, \quad{\text{ und }}\quad 
  \sup A-\frac{\eps}{2} = \inf B - \frac{\eps}{2} < a.
  \]
  Beide Ungleichungen zusammen ergeben $b-a<\eps$.

  "`$\La$"':\quad Sei $\eps>0$ gegeben und es gelte $b-a<\eps$ 
  für ein $a\in A$ und ein $b\in B$. Damit erhält man zunächst 
  \begin{equation}
    b-a<\eps \Ra
    a-b > -\eps \Ra
    \sup A-b > - \eps \Ra
    \sup A+\eps > b. \tag{$\ast$}
  \end{equation}
  Andererseits ist $\sup A$ auch eine untere Schranke von $B$.
  Denn andernfalls 
  gäbe es ein $b'\in B$, das -- wie jedes Element aus $B$ -- eine 
  obere Schranke von $A$ ist, für das aber $b'<\sup A$ gelten würde. 
  Zusammen mit ($\ast$) folgt dann, dass $\sup A$ sogar die 
  \textit{größte} untere Schranke 
  von $B$ ist, also gleich dem Infimum von $B$.     
  \AntEnd
\end{antwort}


%: Question 32
\begin{frage}
  \index{Quadratwurzel!Existenz}
  Wie kann man mithilfe des Supremumsaxioms die Existenz von Quadratwurzeln 
  nichtnegativer reeller Zahlen zeigen?
\end{frage}

\begin{antwort}
  Es genügt, die Behauptung für reelle Zahlen $a \ge 1$ zu zeigen, 
  der Fall $a<1$ folgt daraus durch Übergang zu $\frac{1}{a}$. Der  
  Fall $a=0$ ist einzeln zu behandeln, aber offensichtlich 
  trivial. 

  Sei also $a\ge 1$ gegeben. Die Beweisstrategie 
  besteht darin zu zeigen, dass das Supremum der Menge  
  \[
  M:=\{ x \in \RR\sets x^2<a \}
  \] 
  gerade die gesuchte Quadratwurzel von $a$ ist. 
  Die Existenz von $s:=\sup M$ ist dabei 
  aufgrund des Supremumsaxioms gewährleistet, 
  denn $M$ ist nicht leer (wegen $0\in M$) und durch $a$ 
  nach oben beschränkt. Um $s^2=a$ zu zeigen, führt man die Annahmen 
  $s^2<a$ und $s^2>a$ nun jeweils gesondert zum Widerspruch. 

  Im ersten Fall ist $a-s^2$ positiv, und man kann daher ein 
  $n\in\NN$ so wählen, dass $n(a-s^2) > 2s+1$ gilt. Für dieses 
  $n$ erhält man mit der Binomischen Formel die Ungleichung 
  \[
  \left( s+\frac{1}{n} \right)^2 = s^2 + \frac{2s}{n} + \frac{1}{n^2} 
  < s^2 + \frac{2s+1}{n} < a.
  \] 
  Aus dieser folgt $(s+1/n) \in M$, was aber der Supremumseigenschaft von 
  $s$ widerspricht.   

  Wäre auf der anderen Seite 
  $s^2>a$, dann könnte man ein $n\in\NN$ so wählen, dass 
  $n(s^2-a)>2s$ ist, und damit erhielte man 
  \[
  \left( s-\frac{1}{n}\right)^2  > s^2 -\frac{2s}{n} >a,
  \] 
  womit $s-1/n$ eine obere Schranke von $M$ wäre, ebenfalls im 
  Widerspruch zu $s=\sup M$. 

  Daraus folgt insgesamt $s^2=(\sup M)^2=a$, die 
  nichtnegative Zahl $a$ besitzt also 
  eine Quadratwurzel in $\RR$. 
  \AntEnd
\end{antwort}


%: Question 33
\begin{frage} 
  \index{irrationale Zahl}
  Warum gibt es keine rationale Zahl $r$ mit $r^2=3$?
\end{frage}

\begin{antwort}
  Angenommen, es existieren Zahlen $p,q \in \ZZ$, $q\not=0$  mit 
  $\left(\frac{p}{q}\right)^2 =3$. 
  Man kann davon ausgehen, dass die Zahlen $p$ und $q$ keinen gemeinsamen 
  Teiler besitzen, andernfalls betrachte man den gekürzten Bruch, der ja 
  dieselbe rationale Zahl darstellt. 

  Die Identität $p^2=3q^2$ impliziert, 
  dass $3$ ein Teiler von $p$ ist. Denn $3$ ist eine Primzahl und 
  in der Primfaktorzerlegung von $p^2$ enthalten, folglich also auch in 
  derjenigen 
  von $p$. Es gibt daher eine natürliche Zahl $m$ mit $p=3m$. 

  Dieses Ergebnis lässt sich nun wieder in die Gleichung 
  $p^2=3q^2$ einsetzen und liefert $9m^2=3q^2$, also $3m^2=q^2$. 
  Mit demselben Argument wie oben schließt man daraus, dass $q$ 
  ebenfalls durch $3$ teilbar sein muss.
  Die Zahlen $p$ und $q$ besitzen damit den gemeinsamen Teiler 
  $3$, im Widerspruch zur Voraussetzung ihrer Teilerfremdheit. Die 
  Quadratwurzel aus $3$ kann also keine rationale Zahl sein.
  \AntEnd
\end{antwort}


%: Question 34
\begin{frage}\index{Intervall}
  \nomenclature{$\open{a,b},[a,b],\ropen{a,b},\lopen{a,b}$}{Intervalle}
  Welche Typen von Intervallen sind Ihnen bekannt?
\end{frage}

\begin{antwort}
  Sei $a,b\in\RR$ mit $a<b$. Es gibt die folgenden vier Typen 
  \slanted{beschränkter Intervalle}
  \begin{align*}
    [a,b] &:= \{ x\in\RR \sets a\le x\le b \} 
    & &\text{abgeschlossenes Intervall}, \\
    ] a,b [ &:= \{ x\in\RR \sets a < x < b \} 
    & &\text{offenes Intervall}, \\
    \ropen{ a,b } &:= \{ x\in\RR \sets a \le x < b \} 
    & &\text{(nach rechts) halboffenes Intervall}, \\
    \lopen{ a,b } &:= \{ x\in\RR \sets a < x \le b \} 
    & &\text{(nach links) halboffenes Intervall} 
  \end{align*}
  und die folgenden fünf Typen 
  \slanted{(einseitig) unbeschränkter Intervalle} 
  \begin{align*}
    \ropen{a,\infty} &:= \{ x\in\RR\sets x\ge a \}, 
    &\open{a,\infty} &:= \{ x\in\RR\sets x > a \}, 
    &                & 
    \\
    \lopen{\infty, a} &:= \{ x\in\RR\sets x\le a \}, 
    &\open{\infty, a} &:= \{ x\in\RR\sets x < a \}, 
    &\open{\infty, \infty} &:= \RR. 
  \end{align*}
  Als Spezialfälle \slanted{entarteter Intervalle} gibt es noch 
  \[
  [a,a] := \{ a \}, \qquad
  \open{a,a} := \emptyset \EndTag
  \]
\end{antwort} 


%: Question 35
\begin{frage}
  Welche Intervalltypen haben eine endliche Länge, und wie ist diese 
  definiert?
\end{frage}

\begin{antwort}
  Eine endliche Länge besitzen die beidseitig beschränkten  
  Intervalle und nur diese. Die \textit{Länge} 
  eines beschränkten Intervalls ist mit den Bezeichnungen aus der 
  vorigen Antwort die Zahl $b-a$. 
  \AntEnd
\end{antwort}

\section{Natürliche Zahlen und vollständige Induktion}

Durch unsere Herangehensweise, die reellen Zahlen axiomatisch 
einzuführen, sind wir nun in der Lage, die natürlichen Zahlen 
als spezielle Teilmenge der reellen Zahlen zu definieren 
(s.~Frage~\ref{q:nat-numbers-def}).

%: Question 36
\begin{frage}
  Was versteht man unter einer \textbf{induktiven Teilmenge}
  \index{induktive Teilmenge} 
  (auch Zählmenge oder Nachfolgermenge genannt) von $\RR$?
\end{frage}

\begin{antwort}
  Eine Menge $M \subset \RR$ heißt \slanted{induktiv}, wenn sie die folgenden 
  beiden Eigenschaften besitzt
  \begin{itemize}[2mm]
  \item[\desc{i}] $1\in M, $
  \item[\desc{ii}] $m \in M \Ra m+1\in M.$
  \end{itemize}
  Man beachte, dass nach dieser Definition die reellen Zahlen selbst 
  eine induktive Teilmenge von $\RR$ sind, 
  die Klasse der induktiven Teilmengen 
  also nicht leer ist.
  \AntEnd
\end{antwort}



%: Question 37
\begin{frage}
  \label{q:nat-numbers-def}
  \index{naturliche Zahl@natürliche Zahlen}
  \nomenclature{$\NN$}{Menge der natürlichen Zahlen: $\{1,2,3,\ldots\}$}
  \nomenclature{$\NN_0$}{$=\NN \cup \{0\}$}
  Wie ist die Menge $\NN$ der natürlichen Zahlen als Teilmenge von $\RR$ 
  definiert?
\end{frage}

\begin{antwort}
  Bei dieser Herangehensweise ({\dasheisst} im Unterschied zur axiomatischen 
  Festlegung der natürlichen Zahlen etwa durch die Peano-Axiome)
  definiert man die natürlichen Zahlen als 
  den \textit{Durchschnitt aller induktiven Teilmengen von $\RR$}, man setzt 
  also
  \begin{equation}
    \boxed{
      \dis \NN := \bigcap_{\substack{M \subset \RR \\ \text{$M$ induktiv}}} M.
    }
    \EndTag
  \end{equation}
\end{antwort}


%: Question 38
\begin{frage} 
  Was besagt die Aussage 
  "`$\NN$ ist die kleinste induktive Teilmenge von $\RR$"'?
\end{frage}

\begin{antwort}
  Die Definition der natürlichen Zahlen als Durchschnitt aller 
  induktiven Teilmengen beinhaltet zunächst, dass $\NN$ selbst eine induktive 
  Teilmenge von $\RR$ ist. Denn aus dieser Definition folgt erstens
  $1\in\NN$. Ferner gilt: Ist $a\in \NN$, dann ist $a$ nach Definition in 
  jeder induktiven Teilmenge von $\RR$ enthalten. 
  Damit ist aber auch $a+1$ in jeder dieser Mengen enthalten und 
  folglich auch in $\NN$. 
  Also ist $\NN$ eine induktive Teilmenge von $\RR$.

  Aus der Definition folgt ferner, 
  dass jede induktive Teilmenge von $\RR$ 
  die natürlichen Zahlen als Teilmenge enthält. 
  In diesem Sinne ("`$\NN$ ist in jeder induktiven Teilmenge der reellen 
  Zahlen enthalten"') ist $\NN$ die \textit{kleinste} induktive Teilmenge von $\RR$ (bezüglich des Enthaltenseins).
  \AntEnd
\end{antwort}


\begin{frage}%
  \label{q:archimedische-eigenschaft}
  Warum ist die Menge $\NN$ der natürlichen Zahlen nicht nach oben beschränkt?
\end{frage}

\begin{antwort}
  Wäre $\NN$ nach oben beschränkt, so besäße $\NN$ nach 
  dem Vollständigkeitsaxiom eine kleinste obere Schranke $s$, so dass 
  $n\le s$ für jedes $n\in\NN$ gelten würde.
  
  Nach der $\eps$-Charakterisierung des Supremums (vgl. Frage~\ref{q:sup-eps-charak}) 
  ist $s-1$ nicht mehr obere Schranke von $\NN$, es gibt also eine natürliche Zahl 
  $n_0$ mit $n_0>s-1$. Hieraus folgt aber $n_0+1 > s$, im Widerspruch zur Annahme, dass 
  $s$ obere Schranke von $\NN$ ist. Es kann also nicht $n\le s$ für alle $n\in\NN$ gelten.

  \medskip
  \noindent
  Bemerkung: Diese Eigenschaft, die schon Eudoxos und Archimedes 
  geläufig war, nennt man die 
  \slanted{Archimedische Eigenschaft}\index{Archimedische Eigenschaft} 
  der reellen Zahlen, vgl. dazu auch die Frage~\ref{01_archimedes}. \AntEnd
  \end{antwort}

%: Question 39
\begin{frage}
  Warum ist eine Teilmenge $W\subset \NN$, die die 
  $1$ und mit jedem $w$ auch $w+1$ enthält,  
  mit $\NN$ identisch?
\end{frage}

\begin{antwort}
  $W$ ist nach Definition eine induktive Teilmenge von $\RR$. 
  Damit gilt $
  \NN\subset W$, 
  denn $\NN$ ist die kleinste induktive Teilmenge. 
  Zusammen mit $
  W\subset\NN$ 
  folgt daraus $W=\NN$.
  \AntEnd
\end{antwort}

%: Question 40
\begin{frage} 
  \label{dazwischen}
  Warum gibt es keine natürliche Zahl $n$ mit $1 < n< 2 $ ?
\end{frage}

\begin{antwort}
  Man betrachte die Menge 
  \[
  M : = \{1\} \cup \{ x\in \RR; \, x \ge 2 \}.
  \]
  $M$ ist offensichtlich eine induktive Teilmenge von $\RR$. 
  Folglich ist $\NN \subset M$. 
  Da $M$ aber keine Zahl mit den gesuchten Eigenschaften 
  enthält, kann es eine solche auch in $\NN$ nicht geben.
  \AntEnd
\end{antwort}


%: Question 41
\begin{frage}\index{vollstaendige Induktion@vollst\"andige Induktion} 
  Was besagt das \bold{Beweisprinzip der vollständigen Induktion}?
\end{frage}

\begin{antwort}
  Zu jeder natürlichen Zahl $n$ sei eine Aussage $A(n)$ gegeben. 
  Das Beweisprinzip der vollständigen Induktion besagt: 

  \medskip\noindent
  \satz{\noindent
    Die Aussage $A(n)$ gilt für alle $n\in\NN$, sofern folgende 
    beide Bedingungen erfüllt sind:
    {\setlength{\labelsep}{4mm}
      \begin{enumerate}
      \item[\desc{i}] $A(1)$ ist wahr.\hspace*{55mm}(Induktionsanfang) 
        \\[-3.5mm]
      \item[\desc{ii}] Für jedes $n\in\NN$ gilt: aus $A(n)$ folgt $A(n+1)$.
        \qquad(Induktionsschritt)
      \end{enumerate}}}
  \noindent
  Die Bedingungen \desc{i} und \desc{ii} implizieren zusammen nämlich, 
  dass $A(2)$ richtig ist. Daraus folgt mit \desc{ii}, dass 
  auch $A(3)$ gilt, und hieraus folgt $A(4)$ usw. \slanted{ad infinitum}. 
  \AntEnd
\end{antwort}



%: Question 42
\begin{frage} 
  Welche Varianten des Beweisprinzips der vollständigen Induktion 
  sind Ihnen bekannt? 
  Wie lassen sich diese auf das ursprüngliche Prinzip zurückführen?
\end{frage}

\begin{antwort}
  Hier sind nur drei Beispiele für Varianten induktiver Beweise aufgeführt, 
  die recht häufig anzutreffen sind. Die Liste ist jedoch 
  keineswegs vollständig.   

  \medskip\noindent
  (i) Der Induktionsanfang liegt nicht bei $1$, sondern bei irgendeiner 
  anderen konkreten natürlichen Zahl $N$. In diesem Fall wird die Aussage 
  $A(n)$ nur für alle natürlichen Zahlen $n$ mit $n\ge N$ bewiesen. 
  Offensichtlich ist das gleichbedeutend damit, die Aussagen $A(N-1+n)$ 
  für \textit{alle} natürlichen Zahlen $n$ zu beweisen. 

  \medskip\noindent
  (ii) Im Induktionsschritt wird die Richtigkeit mehrerer Vorgängeraussagen 
  vorausgesetzt, {\zB} "`aus $A(n)$ und $A(n+1)$ folgt $A(n+2)$"'. 
  In diesen Fällen muss die Aussage am Anfang auch für die zwei 
  Zahlen $1$ und $2$ konkret bewiesen werden. Die Übereinstimmung mit dem
  Induktionsprinzip ergibt sich aus der Tatsache, 
  dass das Vorgehen in diesen Fällen äquivalent dazu ist, die Aussage 
  $A'(n)=A(n) \wedge A(n+1)$ mit einem "`normalen"' Induktionsargument 
  zu beweisen.
  
  \medskip\noindent
  (iii) Im Induktionsschritt wird die Richtigkeit der Aussage $A(n+1)$ 
  aus der Richtigkeit aller Aussagen $A(m)$ mit $m\le n$ hergeleitet. 
  Zusammen mit der Richtigkeit von $A(1)$ beinhaltet das dann 
  aber auch die Beziehung $A(n) \Ra A(n+1)$. 
  Denn andernfalls gäbe es (nach dem Wohlordnungsprinzip, vgl. Frage 
  \ref{wohlordnungssatz}) 
  eine \textit{kleinste} natürliche Zahl $n'>1$, {\sd} $A(n') \Ra A(n'+1)$ 
  \textit{nicht} gilt. Da die 
  Folgerung für alle kleineren Zahlen aber zutrifft, erlaubt das 
  Induktionsprinzip die Herleitung der Aussagen 
  $A(1),A(2),\ldots,A(n')$, und 
  aus diesen folgt aufgrund der Voraussetzung $A(n'+1)$. Also trifft die 
  Implikation $A(n') \Ra A(n'+1)$ doch zu, und die beiden Induktionsvarianten 
  sind äquivalent.
  \AntEnd
\end{antwort}


%: Question 43
\begin{frage}
  Für welche $n\in\NN$ gilt
  \[
  2^n > n^2\,? 
  \]
\end{frage}

\begin{antwort}
  Die Ungleichung gilt nicht für $n=2, 3, 4$; für $n=5$ ist sie wiederum richtig. Wir zeigen, dass sie für alle natürlichen Zahlen $n\ge 5$ gilt. 
  {\setlength{\labelsep}{4mm}
    \begin{enumerate}
    \item[\desc{i}] $32=2^5 > 25 = 5^2$ (Induktionsanfang) \\[-3.5mm]
    \item[\desc{ii}] Sei $n\ge 5$ und es gelte $2^n > n^2$. 
      Wegen $n^2\ge n\cdot 3> n\left(2+\frac 1n\right) = 2n+1$ folgt
      \[
      2^{n+1} = 2\cdot 2^n > 2\cdot n^2 = n^2+n^2 > n^2+2n+1 = (n+1)^2.
      \] 
    \end{enumerate}}
  Damit ist $2^n > n^2$ für alle $n \ge 5$ gezeigt.
  \AntEnd
\end{antwort}

%: Question 44
\begin{frage}\index{Pascalzahlen}\index{Pascalsches@Pascal'sches Dreieck}
  \index{Pascal@\textsc{Pascal}, Blaise (1623-1662)}
  Wie sind die Zahlen $C_k^n$ des \bold{Pascal\sch en Dreiecks} definiert? 
\end{frage}

\begin{antwort}[]
  \Ant 
  Die Pascalzahlen sind rekursiv definiert durch  
  $C_0^0=C_0^n=C_n^n=1$ für alle $n\in\NN$ und 
  \[
  C_k^n = C_{k-1}^{n-1} + C_k^{n-1}, \qquad{k\le n}.
  \]
  \picskip{5}\noindent
  Die Zahlen sind die Einträge des  
  \textit{Pascal\sch en Dreiecks}, 
  dessen Ränder alle aus Einsen bestehen 
  und dessen übrige Zahlen in der 
  $(n+1)$-ten Zeile die Summe der beiden schräg darüberstehenden 
  Zahlen der $n$-ten Zeile sind, \sieheAbbildung\ref{fig:01_pascal}. 
  \AntEnd

  \begin{center}
    \includegraphics{mp/01_pascal}
    \captionof{figure}{Ein Ausschnitt des Pascal'schen Dreiecks 
      für $n\le 6$. In den Reihen stehen die Zahlen $C_k^n$.}
    \label{fig:01_pascal}
  \end{center}
\end{antwort}


%: Question 45
\begin{frage}
  Warum gilt die Beziehung $C_k^n=C_{n-k}^n$?
\end{frage}

\begin{antwort}
  Die Richtigkeit der Formel erkennt man anschaulich 
  an der Symmetrie des Pascal'schen Dreiecks, formal beweisen  
  lässt sie sich mittels vollständiger Induktion über $n$. 
  Für $n=1$ ist die Formel wegen $C_1^1=C_0^1=1$ richtig. Ist sie 
  für $n\in\NN$ bereits gezeigt, so folgt im Induktionsschritt
  \[
  C_k^{n+1} =
  C_{k-1}^n + C_k^n = 
  C_{n-(k-1)}^n+C_{n-k}^n =
  C^{n+1}_{n-k+1} =
  C^{n+1}_{(n+1)-k},
  \]
  und das ist die gesuchte Identität für $n+1$.
  \AntEnd
\end{antwort}






%: Question 46
\begin{frage}\index{Binomialkoeffizient}
  \nomenclature{$\binom{n}{k}$}{Binomialkoeffizient}
  Wie sind die \bold{Binomialkoeffizienten} $\binom{n}{k}$ definiert? 
\end{frage}

\begin{antwort}
  Man setzt $\binom{n}{0}:=1$ und für $k,n\in\NN$, $k\le n$ 
  durch 
  \[
  \boxed{
    \binom{n}{k} := \frac{n (n-1)(n-2)\cdots(n-k+1)}
    { 1\cdot 2\cdots k }.
  }
  \EndTag
  \]
\end{antwort}






%: Question 47
\begin{frage}
  Wieso sind die Binomialkoeffizienten $\binom{n}{k}$ 
  gleich den Pascalzahlen $C_k^n$?
\end{frage}

\begin{antwort}
  Die Übereinstimmung ergibt sich durch  vollständige 
  Induktion über $k$. 
  Für $k=0$ ist die Formel $\binom{n}{k}=C^n_k$ aufgrund der 
  Definition richtig, und für $k>0$ erhält man 
  \begin{align*}
    \binom{n}{k}+\binom{n}{k+1} &=
    \frac{n(n-1)\cdots(n-k+1)}{k!}+
    \frac{n(n-1)\cdots(n-k)}{k!\cdot (k+1)} \\
    &=
    \frac{n(n-1)\cdots(n-k+1) \cdot \big( (k+1) + (n-k) \big)}{(k+1)!} \\
    &= 
    \frac{(n+1)n\cdots (n+1-k)}{(k+1)!} = 
    \binom{n+1}{k+1}.
  \end{align*}
  Die Binomialkoeffizienten erfüllen also dieselbe Rekursionsgleichung wie die 
  Pascalzahlen. Daraus folgt zusammen mit $\binom{n}{0}=C^n_0$ die Behauptung.
  \AntEnd
\end{antwort}







%: Question 48
\begin{frage}\label{01_permutationen}\index{Permutation}\index{Fakultät}
  Warum gibt es für eine endliche Menge mit $n$ Elementen genau $n!$ 
  bijektive Selbstabbildungen (Permutationen)?
\end{frage}

\begin{antwort}
  Der Zusammenhang lässt sich mit vollständiger Induktion 
  über die Mächtigkeit der Menge begründen.  
  Für Mengen mit nur einem Element gibt es offensichtlich nur eine ($=1!$) 
  bijektive Selbstabbildung, die Behauptung ist in diesem Fall also richtig. 

  Sei die Behauptung für $n\in \NN$ bereits gezeigt und sei 
  \[
  M := \{ m_1, \ldots, m_n, m_{n+1} \}
  \] 
  eine Menge mit $n+1$ Elementen. 
  Diejenigen Selbstabbildungen von $M$, die das Element $m_1$ 
  auf ein bestimmtes $m_k$ mit $k=1,\ldots,n+1$ abbilden, bilden 
  zusammen eine Menge $\calli{B}_k$ von Bijektionen $M\to M$. 

  Jede der $(k+1)$ Mengen $\calli{B}_k$ enthält genauso viele Elemente, 
  wie es Bijektionen 
  $M \mengeminus \{m_1\} \leftrightarrow M \mengeminus \{ m_k \}$ gibt,  
  und nach der Induktionsvoraussetzung sind das genau $n!$ Stück. 
  Da die Mengen $\calli{B}_k$ paarweise disjunkt sind, folgt aus
  \[
  \calli{B}:=\{ \text{ Bijektionen $ M\to M$ } \} = 
  \calli{B}_1 \cup \ldots \cup \calli{B}_{n+1}
  \]  
  die behauptete Formel $|\calli{B}|=(n+1)\cdot n!=(n+1)!$.
  \nomenclature{$|\mathfrak{M}|$}{M\"achtigkeit der Menge $\mathfrak{M}$}
  \AntEnd
\end{antwort}

%: Question 49
\begin{frage}\index{Binomialkoeffizient}
  Welche kombinatorische Bedeutung haben die Zahlen $\binom{n}{k}$?
\end{frage}

\begin{antwort}
  Die Zahl $\binom{n}{k}$ entspricht der Anzahl der Möglichkeiten, aus einer 
  Menge mit $n$ Elementen eine Teilmenge mit $k$ Elementen auszuwählen, oder 
  anders gesagt:    
  \textit{$\binom{n}{k}$ ist die Anzahl der $k$-elementigen 
    Teilmengen einer $n$-elementigen Menge.}

  Das lässt sich folgendermaßen begründen. 
  Bei Auswahl einer $k$-elementigen Teilmenge einer 
  $n$-elementigen Menge lässt sich das erste Element auf $n$ Arten bestimmen, 
  das zweite auf $(n-1)$ Arten usw. bis zum $k$-ten Element, für 
  das dann noch $(n-k+1)$ Möglichkeiten übrig bleiben. Das führt auf  
  $n(n-1) \cdots (n-k+1)$ verschiedene Serien aus $k$ Elementen, 
  von denen jedoch jeweils $k!$ (vgl. die 
  Antwort zur nächsten Aufgabe) dieselbe Menge bestimmen. Somit ergibt sich als 
  Anzahl der $k$-elementigen Teilmengen 
  \begin{equation}
    \frac{ n(n-1)\cdots (n-k+1)}{k!} = \binom{n}{k}. \EndTag
  \end{equation}
\end{antwort}


%: Question 50
\begin{frage}
  Warum ist die Anzahl der injektiven Abbildungen der Menge 
  $A:=\{ 1,2,\ldots,k\}$ 
  in die Menge $B:=\{ 1,2,\ldots,n \}$ 
  gegeben durch 
  \[
  P(n,k):=n(n-1)\cdots (n-k+1) = k! \binom{n}{k}\,\text{?}
  \]
\end{frage}

\begin{antwort}
  Es gibt genau $\binom{n}{k}$ Teilmengen $B_k$ von $B$ mit 
  $k$ Elementen. 
  Jede injektive Abbildung $A \to B$ ist eine Bijektion 
  zwischen $A$ und einer dieser Teilmengen $B_k$. Davon gibt es nach der 
  Antwort zur vorigen Frage für jedes gegebene $B_k$ genau $k!$ Stück.  
  Insgesamt gibt es also $P(n,k)=k! \cdot \binom{n}{k}$ injektive 
  Abbildungen $A\to B$.
  \AntEnd
\end{antwort}


%: Question 51
\begin{frage}
  Warum ist $P(n,k)$ die Anzahl der geordneten Stichproben aus der  
  Menge $\{ 1, \ldots, n \}$ vom Umfang $k$ ohne Wiederholung?
\end{frage}

\begin{antwort}
  Die $k$ Elemente jeder Stichprobe bilden zusammen eine 
  $k$-elementige Teilmenge von $\{ 1,\ldots, n\}$. Es gibt 
  $\binom{n}{k}$ solcher Teilmengen, in der allerdings die Reihenfolge der 
  Elemente nicht mitberücksichtigt ist. 
  Die $k$-Elemente jeder Teilmenge lassen sich auf $k!$ Arten anordnen. 
  Daraus folgt die Behauptung.
  \AntEnd
\end{antwort}






%: Question 52
\begin{frage}\index{Stichprobe!geordnete}
  Warum ist die Anzahl der geordneten Stichproben aus der Menge 
  $\{ 1,\ldots, n \} $ vom Umfang $k$ mit Wiederholungen durch 
  $W(n,k)=n^k$ gegeben?
\end{frage}

\begin{antwort}
  Für das erste Element der Stichprobe gibt es $n$ Kandidaten,  
  ebenso für das zweite, das dritte usw. bis zum $k$-ten Element. 
  Die Anzahl der möglichen Stichproben beträgt daher
  \[
  \underbrace{ n\cdot n \cdots n }_{\text{$k$ Faktoren}} = n^k.\EndTag
  \]
\end{antwort}






%: Question 53
\begin{frage}\index{Stichprobe!ungeordnete}
  Warum ist die Anzahl der ungeordneten Stichproben aus der Menge 
  $\{1,\ldots,n\}$ vom Umfang $k$ gegeben durch den Binomialkoeffizienten 
  $\binom{n}{k}$?
\end{frage}

\begin{antwort}
  Ohne Wiederholung und ohne Berücksichtigung der Reihenfolge ist die Bestimmung 
  einer Stichprobe gleichwertig damit, eine $k$-elementige Teilmenge aus 
  $\{ 1,\ldots, n \}$ auszuwählen. 
  Dafür gibt es nach Frage 40 genau $\binom{n}{k}$ Möglichkeiten.
  \AntEnd
\end{antwort}






%: Question 54
\begin{frage}
  Wie kann man die Formel $\sum_{k=0}^n \binom{n}{k} = 2^n$
  als Anzahlformel interpretieren?
\end{frage}

\begin{antwort}
  Die Summe auf der linken Seite ist die Anzahl aller Teilmengen einer 
  Menge mit $n$ Elementen (also die Anzahl aller Teilmengen mit null Elementen 
  \textit{plus} die Anzahl aller 
  Teilmengen mit einem Element \textit{plus} ... usw.). 

  Für jedes Element einer $n$-elementigen Menge $M$ und jede Teilmenge 
  von $M'\subset M$ gibt es genau zwei Möglichkeiten: entweder $m\in M'$ oder 
  $m\not\in M'$. Daraus erklärt sich der Wert $2^n$ als Anzahl der 
  Teilmengen von $M$.
  \AntEnd  
\end{antwort}






%: Question 55
\begin{frage}
  \label{wohlordnungssatz}\index{Wohlordnungssatz für $\NN$}
  Was besagt der \textbf{Wohlordnungssatz} für die natürlichen Zahlen? 
\end{frage}

\begin{antwort}
  Der Wohlordnungssatz besagt: \textit{Jede nichtleere Teilmenge 
    der natürlichen 
    Zahlen besitzt ein kleinstes Element.}

  Der Wohlordnungssatz kann wie das Beweisprinzip der vollständigen Induktion 
  zum Nachweis dafür benutzt werden, dass eine Aussage $A$ auf 
  alle natürlichen Zahlen zutrifft. Dazu genügt es, Folgendes 
  zu zeigen
  \slanted{\setlength{\labelsep}{4mm}
    \begin{enumerate}
    \item[\desc{i}] $A(1)$ ist richtig.\\[-3.5mm]  
    \item[\desc{ii}] Für alle $n\in \NN$ gilt: 
      aus $\neg A(n)$ folgt $\neg A(m)$ für eine 
      natürliche Zahl $m<n$.
    \end{enumerate}}

  Gilt nämlich \desc{i}, dann folgt mit dem Wohlordnungsprinzip 
  aus der Annahme, dass eine Aussage $A$ \textit{nicht} auf 
  alle natürlichen Zahlen zutrifft, dass es eine 
  kleinste natürliche Zahl $n'$ 
  geben muss, {\sd} $A(n')$ \textit{falsch} ist. Mit 
  \desc{ii} folgt daraus aber der Widerspruch, dass 
  es noch eine kleinere Zahl als $n$ gibt, für die die Aussage falsch ist. 
  Folglich gilt $A(n)$ für alle $n\in\NN$.
\end{antwort} 






%: Question 56
\begin{frage}
  Können Sie zeigen, dass aus dem Wohlordnungssatz die Gültigkeit des 
  Beweisprinzips vollständiger Induktion folgt?
\end{frage}

\begin{antwort}
  Sei die Eigenschaft 
  \desc{i} aus der vorigen Frage gegeben und gelte außerdem 
  {\setlength{\labelsep}{5mm}
    \begin{enumerate}
    \item[\desc{ii'}] $A(n)\Ra A(n+1)$.   
    \end{enumerate}}
  Es gilt zu zeigen, dass zusammen 
  mit dem Wohlordnungsprinzip daraus die Richtigkeit von $A$ für alle 
  natürlichen Zahlen folgt. 
  Angenommen, das trifft nicht zu. Dann gibt es nach dem 
  Wohlordnungsprinzip eine 
  kleinste natürliche Zahl $n'$, 
  {\sd} $A(n')$ falsch ist. Dann ist aber $A(n'-1)$ 
  richtig und wegen \desc{ii'} auch $A(n')$, im Widerspruch 
  zur Annahme. \AntEnd
\end{antwort} 






%: Question 57
\begin{frage}
  Können Sie umgekehrt den Wohlordnungssatz mittels  
  vollständiger Induktion herleiten?
\end{frage}

\begin{antwort}
  Angenommen, die nichtleere Menge $M\subset \NN$ 
  besitzt kein kleinstes Element. Dann 
  ist $1\not\in M$, denn andernfalls wäre $1$ das kleinste Element. 
  Mit $n$ kann aber auch $n+1$ nicht in der Menge enthalten sein, denn 
  andernfalls gäbe es ein Element $m\in M$ mit der Eigenschaft $n<m<n+1$, 
  die aber keine natürliche Zahl besitzt (vgl. Frage \ref{dazwischen}). 
  Aus dem Induktionsprinzip folgt daraus $n \not\in M$ für alle $n\in \NN$, 
  also $M=\emptyset$, im Widerspruch zur Voraussetzung, dass $M$ nicht leer ist.
  \AntEnd
\end{antwort}






%: Question 58
\begin{frage}
  Kennen Sie eine Anwendung des Wohlordnungssatzes?
\end{frage}

\begin{antwort}
  Ein typisches Beispiel 
  wäre ein Beweis der Existenzbehauptung im 
  Fundamentalsatz der Arithmetik:  
  \textit{Jede natürliche Zahl $>1$ lässt sich 
    in Primfaktoren zerlegen.} (Die Eindeutigkeit bekommt man allerdings 
  nicht so leicht.) 

  Die Aussage ist sicherlich richtig für $n=2$. 
  Angenommen, $n'\in \NN$ sei die kleinste Zahl, für die das nicht gilt. 
  Dann kann $n'$ keine Primzahl sein, also gibt es Zahlen $m,n \in \NN$ mit 
  $m,n < n'$ und $mn=n'$. Zumindest eine der 
  Zahlen $n$ oder $m$ kann sich dann ebenfalls nicht in 
  Primfaktoren zerlegen lassen -- Widerspruch. 
  \AntEnd
\end{antwort}

%: Question 59
\begin{frage}\label{01_archimedes}
  \index{archimedisch angeordnet}\index{Archimedische Eigenschaft}
  \index{Archimedes@\textsc{Archimedes} von Syrakus (ca. 287-212 v.Chr.)}
  Was bedeutet die Aussage: "`Der Körper der reellen Zahlen ist 
  \bold{archimedisch angeordnet}"'?
\end{frage}

\begin{antwort}
  Das heißt, dass in $\RR$ die  
  \slanted{Archimedische Eigenschaft} gilt:

  \medskip\noindent 
  \slanted{Zu je zwei Zahlen $x,y$ mit $x>0$ existiert eine 
    natürliche Zahl $n$, {\sd} $nx>y$}. 

  \medskip
  \noindent
  Denn wäre $nx\le y$ für alle $n\in\NN$, dann wäre auch $n\le\frac yx$ für alle 
  $n\in\NN$, d.\,h. $\frac yx$ wäre obere Schranke für $\NN$, im Widerspruch zur 
  Antwort auf Frage~\ref{q:archimedische-eigenschaft}.

  \medskip
  \noindent
  Die Unbeschränktheit von $\NN$ nach oben und die in der Antwort zu Frage \ref{01_archimedes} 
  genannte Eigenschaft sind äquivalent.
  \AntEnd
\end{antwort}






%: Question 60
\begin{frage}
  Warum gibt es zu jeder reellen Zahl $y$ eine natürliche Zahl $n$ mit $n>y$?
\end{frage}

\begin{antwort}
  Die Aussage ist ein Spezialfall der Archimedischen Eigenschaft  
  (Frage~\ref{q:archimedische-eigenschaft}) für $x=1$.
  \AntEnd
\end{antwort}






%: Question 61
\begin{frage}
  Kennen Sie ein Beispiel für einen Körper, 
  der zwar angeordnet, aber nicht 
  archimedisch angeordnet ist?
\end{frage}

\begin{antwort}
  Der Körper der rationalen Funktionen ist mit der in Frage 
  \ref{rationalefunktionen} definierten 
  Ordnung kein archimedisch angeordneter Körper. Zum Beispiel sind in diesem 
  Körper alle natürlichen Zahlen (aufgefasst als konstante Funktionen) durch $x$ 
  nach oben beschränkt: Es gilt $n<x$ für alle $n\in\NN$.
  \AntEnd
\end{antwort}


\section{Die ganzen und rationalen Zahlen}






%: Question 62
\begin{frage}\index{Z@$\ZZ$}\index{ganze Zahlen}
  \nomenclature{$\ZZ$}{Menge der ganzen Zahlen}
  Wie ist die Menge $\ZZ$ der \bold{ganzen Zahlen} 
  als Teilmenge von $\RR$ definiert?
\end{frage}

\begin{antwort}
  Ausgehend von $\RR$ definiert man die Menge der ganzen Zahlen einfach 
  durch $\ZZ := \NN \cup \{ 0 \} \cup \{ -n; \; n\in\NN \}$. Damit 
  ist $\ZZ$ der kleinste \textit{Unterring} von $\RR$.  

  Man beachte, dass wir die Menge $\RR$ als gegeben voraussetzen und damit auch 
  die ganzen Zahlen mit ihrer Ringstruktur automatisch "`schon haben"'. 
  Baut man im Unterschied dazu das Zahlsystem ausgehend von den natürlichen 
  Zahlen auf, dann muss man die ganzen Zahlen aus den natürlichen 
  durch algebraische Erweiterungsprozesse erst konstruieren, siehe dazu \zB\ 
  \citep{Kramer}.
  \AntEnd
\end{antwort} 






%: Question 63
\begin{frage}\index{Q@$\QQ$}
  \nomenclature{$\QQ$}{Menge der rationalen Zahlen}
  Wie ist die Menge $\QQ$ der \bold{rationalen Zahlen} 
  als Teilmenge von $\RR$ definiert?
\end{frage}

\begin{antwort}
  Der Körper der rationalen Zahlen 
  ist der \textit{Quotientenkörper}\index{Quotientenkörper} der 
  ganzen Zahlen, {\dasheisst}  
  $\QQ := \{ ab^{-1}; \; a,b \in \ZZ, \, b\not=0 \}$. 
  Die Regeln der Bruchrechnung in $\RR$ implizieren dann bereits, dass zwei 
  "`Brüche"' $ab^{-1}$ und $cd^{-1}$ ($a,b,c,d \in \ZZ,\, b,d\not=0$)
  dieselbe rationale Zahl darstellen, 
  wenn $ad=bc$ gilt. 
  Mit dieser Definition ist $\QQ$ der kleinste 
  \textit{Unterkörper} der reellen Zahlen.   
  \AntEnd
\end{antwort}






%: Question 64
\begin{frage}
  Welche \bold{algebraische Struktur} besitzen $\ZZ$ bzw. $\QQ$?
\end{frage}

\begin{antwort}
  Die ganzen Zahlen bilden einen \textit{kommutativen Ring mit Eins}, 
  {\dasheisst} eine additive Abel'sche 
  Gruppe, auf der eine Multiplikation definiert ist, {\sd} die Kommutativ-, 
  Assoziativ- und Distributivgesetze gelten, 
  und die das Einselement bezüglich der Multiplikation enthält. 
  Die rationalen Zahlen $\QQ$ bilden einen Körper.
  \AntEnd
\end{antwort}






%: Question 65
\begin{frage}
  Was besagt die Aussage "`$\QQ$ ist dicht in $\RR$"'?
\end{frage}

\begin{antwort}
  Die Aussage bedeutet, dass in jedem Intervall 
  $\open{x-\eps,x+\eps}=U_\eps(x)$ 
  mit $x,\eps \in \RR$ und $\eps >0$ 
  der reellen Zahlen eine rationale Zahl liegt: $U_\eps(x) \cap \QQ \not= 
  \emptyset$.  

  Man kann das beweisen, indem man für reelle Zahlen $y>1$ 
  zunächst zeigt, dass das Intervall $\open{y-1,y+1}$ 
  mindestens eine natürliche Zahl enthält, 
  etwa die Zahl $\min \{ n\in\NN;\; n>y-1 \}$. 

  Ausgehend von einem gegebenen Intervall $\open{x-\eps,x+\eps}$ mit $x>1$ 
  wähle man dann $n$ so groß, dass $1/n < \eps$ ist, und bestimme 
  die natürliche Zahl $m$, für die $nx-1 < m \le nx+1$ gilt. Dann liegt 
  $\frac{m}{n}$ im Intervall $\open{x-\eps,x+\eps}$.
  
  Aus diesem Ergebnis für reelle Zahlen $x>1$ folgt nun leicht 
  die allgemeine Behauptung.   
  \AntEnd
\end{antwort}






%: Question 66
\begin{frage}\index{irrationale Zahl}
  Wieso ist auch die Menge $\RR\mengeminus\QQ$ der irrationalen Zahlen 
  dicht in $\RR$?
\end{frage}

\begin{antwort}
  Sei $\open{x-\eps,x+\eps}\subset \RR$ ein reelles Intervall. Um zu zeigen, dass 
  darin eine irrationale Zahl liegt, unterscheiden wir folgende 
  Fälle:

  \desc{1} 
  $x\in\RR \setminus \QQ$. In diesem Fall wähle man ein $n\in \NN$ mit 
  $n>\frac{1}{\eps}$. Die Zahl $x+\frac{1}{n}$ ist dann irrational 
  (denn aus $x+\frac{1}{n} = \frac{p}{q}$ 
  mit ganzen Zahlen $p$ und $q$ würde $x=\frac{p}{q}-\frac{1}{n}\in\QQ$ folgen) 
  und im Intervall $\open{x,x+\eps}$ enthalten. 

  \desc{2} $x\in\QQ$. In diesem Fall wähle man eine bestimmte 
  irrationale Zahl $\varrho$ (etwa $\varrho=\sqrt{3}$, die wir schon kennen) 
  und $n$ so groß, dass $\varrho < n\eps$ gilt. 
  Dann ist $x + \frac{\varrho}{n} \in \open{x-\eps,x+\eps}$, und mit demselben 
  Argument wie in (1) zeigt man, dass $x+\frac{\varrho}{n}$ irrational ist.
  \AntEnd
\end{antwort}






%: Question 67
\begin{frage}\label{01_gkl}
  \index{Gauss-Klammer@Gauß-Klammer}
  \nomenclature{$[ \,\; ]$}{Gauß-Klammer}
  Wie ist $[ x ]$ 
  (Gauß-Klammer von $x$) für $x\in \RR$ definiert? (Speziell in der 
  Computerliteratur verwendet man hierfür oft die Bezeichnung 
  "`$\mathrm{floor}(x)$"' oder "`$\lfloor x \rfloor$"'.)
\end{frage}

\begin{antwort}
  Der Wert von $[x]$ 
  ist definiert als die eindeutig bestimmte ganze Zahl $k$ 
  mit $k\le x <k+1$, also
  \begin{equation}
    [ x ] := \max \big\{ k\in\ZZ;\; k\le x \big\}. \notag
  \end{equation}
  \noindent
  Beispiel: Für die Kreiszahl $\pi=3,14159265\ldots$ ist 
  $[ \pi ]=3$ und $[ -\pi ]=-4$. \AntEnd
\end{antwort}






%: Question 68
\begin{frage}\index{Division mit Rest}
  Was besagt der Satz über die \bold{Division mit Rest} in $\ZZ$?
\end{frage}

\begin{antwort}
  Der Satz besagt, dass zu je zwei ganzen Zahlen $a,b \in \ZZ$ 
  mit $b>0$ 
  eine Zahl $q\in \ZZ$ sowie ein Rest 
  $r \in \{ 0, 1, \ldots ,b-1 \}$ 
  existieren, {\sd} gilt 
  \[
  a=bq+r.  
  \]
  Dabei sind $q$ und $r$ eindeutig bestimmt. 
  Man kann das beweisen, indem man für $q$ den ganzzahligen Anteil 
  $\lfloor a/b \rfloor$ von $a/b$ nimmt und dann zeigt, 
  dass mit $r:=a-bq$ notwendig $0 \le r < b$ gilt.

  Beispiel: $16=7\cdot 2 + 2$.
  \AntEnd
\end{antwort}






%: Question 69
\begin{frage}\index{Euklidischer Algorithmus}
  \index{Euklid@\textsc{Euklid} (ca. 235-265 v.\,Chr.)}
  Was besagt der Satz über den 
  \bold{Euklidischen Algorithmus} in $\ZZ$?
\end{frage}  

\begin{antwort}
  Zu zwei Zahlen $a,b\in\ZZ$ mit $b>0$ 
  lässt sich durch sukzessive 
  Division mit Rest  
  auf eindeutige Weise folgendes Schema konstruieren
  \[
  \begin{array}{rclp{12mm}rcl}
    a &=& bq_1+r_1,       & &    0 &\le& r_1 < b\\[-1mm]
    b &=& r_1 q_2 + r_2,  & &    0 &\le& r_2 < r_1 \\[-1mm]
    &\vdots &           & &      &\vdots&        \\[-1mm]
    r_{n-2} &=& r_{n-1}q_n+ r_n, & &  0&\le& r_n < r_{n-1} \\[-1mm]
    r_{n-1} &=& r_n q_{n+1} + 0. & &   &   &             
  \end{array}
  \]
  Die Folge $r_1, r_2, \ldots$ der Reste ist streng monoton 
  fallend, und jeder Term ist nichtnegativ, folglich muss sie 
  an einer Stelle abbrechen. 

  Die Methode der Konstruktion dieser Folge heißt 
  \textit{Euklidischer Algorithmus}, die
  Aussage des zugehörigen Satzes lautet, dass der  
  letzte nicht verschwindende Term $r_n$ der Folge 
  der \textit{größte gemeinsame Teiler} der Zahlen 
  $a$ und $b$ ist. Dabei heißt $d\in\ZZ$ größter gemeinsamer 
  Teiler von $a$ und $b$, wenn
  {\setlength{\labelsep}{4mm}
    \begin{enumerate}\index{groster gemeinsamer@größter gemeinsamer Teiler}
    \item[\desc{i}] $d$ ein Teiler von $a$ und ein Teiler von $b$ ist und\\[-3.5mm]
    \item[\desc{ii}] für alle $e\in\ZZ$ gilt: aus $e \,|\, a$ und $e\,|\,b$ 
      folgt $e\,|\,d$.
    \end{enumerate}}
  Der in dem Satz ausgesprochene 
  Zusammenhang ergibt sich im Wesentlichen aus der einfachen 
  Beobachtung, dass aus einer Beziehung der Form 
  \[
  a = bq+r
  \]
  stets folgt, dass ein gemeinsamer Teiler von $r$ und $b$ auch 
  ein Teiler von $a$ sein muss. 
  Liest man das Schema von unten nach oben, so erkennt man damit,  
  dass tatsächlich $r_n=\ggT(a,b)$ gilt.
  \AntEnd
\end{antwort}






%: Question 70
\begin{frage}\index{Potenz!natürliche}
  Wie definiert man rekursiv für $n\in\NN_0 := N\cup \{ 0 \}$ 
  die Potenzen $x^n$ einer reellen Zahl $x$? 
  Wie definiert man rekursiv die Summe bzw. das Produkt von $n$ reellen 
  Zahlen $x_1,\ldots, x_n$?
\end{frage}

\begin{antwort}
  Die Potenzen sind durch 
  $x^0 := 1$ und die Rekursionsvorschrift
  $x^{n} := x^{n-1} \cdot x$ definiert.

  Ausgehend von Summe und Produkt zweier reeller Zahlen 
  definiert man rekursiv die Summe und das Produkt endlich vieler 
  reeller Zahlen durch 
  \[
  x_1 \ast \ldots \ast x_{n-1} \ast x_n = (x_1 \ast \ldots \ast x_{n-1}) 
  \ast x_n. 
  \]
  Dabei ist "`$\ast$"' entweder als Additions- oder Multiplikationszeichen 
  zu lesen.
  \AntEnd
\end{antwort}






%: Question 71
\begin{frage}\index{Potenz!ganzzahlige}
  Wie verallgemeinert man die Definition von $x^n$ für negative 
  Potenzen, falls $x\not=0$ gilt?
\end{frage}

\begin{antwort}
  Man setzt $x^{-n} := (x^{-1})^n.$ 
  Diese Definition ist die einzige Möglichkeit, die 
  für natürliche Zahlen $n,m$ gültige Gleichung  
  $x^{n+m}=x^n\cdot x^m$ zusammen mit $x^0=1$ auf die ganzen 
  Zahlen fortzusetzen.
  \AntEnd
\end{antwort}






%: Question 72
\begin{frage}\index{Potenz!Rechenregeln}
  Welche Rechenregeln für Potenzen sind Ihnen geläufig?
\end{frage}

\begin{antwort}
  Wesentlich sind die folgenden drei Regeln. 
  \[
  x^n \cdot y^n = (xy)^n, \quad x^{n+m} =x^n\cdot x^m, 
  \quad (x^n)^m= x^{nm} = (x^m)^n, 
  \qquad (x,y\in\RR,\, m,n \in \ZZ).
  \]
  Falls negative Potenzen auftreten, ist $x\not=0$ bzw. $y\not=0$ 
  vorauszusetzen.
  \AntEnd
\end{antwort}






%: Question 73
\begin{frage}
  \label{01_reks}\index{Rekursionssatz, allgemeiner}
  Was besagt der \bold{allgemeine Rekursionssatz}? 
  Kennen Sie eine Anwendung?
\end{frage} 

\begin{antwort} Der allgemeine Rekursionssatz besagt:

  \medskip\noindent
  \slanted{Sei $\mathfrak{M}$ eine Menge, und für jedes $n\in \NN$ sei eine  
    Abbildung $g_n\,:\, \mathfrak{M} \to \mathfrak{M}$ gegeben. Sei ferner $m\in \mathfrak{M}$ gegeben. 
    Dann gibt es genau eine Abbildung $f\,:\,\NN\to \mathfrak{M}$ mit }
  \begin{align*}
    \text{\desc{i}} &\qquad  f(1)=m \\
    \text{\desc{ii}} &\qquad  f(n+1)= g_n(f(n)).
  \end{align*} 
  Der Rekursionssatz ist ein sehr fundamentaler 
  "`metatheoretischer"' Satz, der die theoretische 
  Begründung dafür liefert, dass Abbildungen  
  $\NN \to \mathfrak{M}$ (d.h. Folgen) durch die Angabe eines Startwertes und 
  einer Rekursionsvorschrift (mit der ja, streng genommen, zunächst nur ein 
  unendlicher Prozess gegeben ist und kein mathematisches Objekt)
  eindeutig definiert werden können. 

  Mithilfe des Rekursionssatzes lässt sich beispielsweise \textit{beweisen},
  dass durch die Vorschrift $
  f(1) = 1$ und $f(n+1)=(n+1) f(n)$ 
  tatsächlich eine Abbildung $\NN \to \NN, \, n\mapsto f(n)=n!$ definiert ist.
  \AntEnd
\end{antwort}




\section{Der Körper der komplexen Zahlen}



Für die Konstruktion der komplexen Zahlen existieren verschiedene Modelle. 
Wir konzentrieren uns hier auf die drei gängigsten. 

1. Beim \slanted{Standardmodell} geht man aus von der Menge 
\[
\calli{C} := \RR^2 = \RR \times \RR = \{(a,b) ; a,b \in \RR\}.
\]
Man weiß aus der linearen Algebra, dass $\calli{C}$ bezüglich der 
komponentenweisen Addition 
\[
(A) : (a,b) \oplus (a',b') = (a+a', b+ b') 
\]
eine abelsche Gruppe ist mit dem neutralen Element $0 = (0,0)$. 

Ferner ist $\calli{C} = \RR^2$ ein 
2-dimensionaler Vektorraum mit der skalaren Multiplikation 
\[
r(a,b) = (ra,rb),\qquad r\in\RR.
\]
Eine Multiplikation von Zahlenpaaren liegt zunächst allerdings nicht auf 
der Hand. Die naheliegende Definition $(a,b) \cdot (a',b') = (aa',bb')$ 
führt zu keiner Körperstruktur auf $\calli{C}$, denn wegen 
$(1,0)\cdot(0,1)=(0,0)$ ist $\calli{C}$ mit 
\slanted{dieser} Multiplikation nicht nullteilerfrei und 
nach Eigenschaft $\desc{iii}$ aus Frage \ref{q:koerper-rechenregeln}  
$\calli{C}$ damit kein Körper sein. Die Multiplikation auf $\calli{C}$ muss 
also anders definiert sein.

2. Das zweite Modell ist an der Geometrie des $\RR^2$ orientiert. 
Man betrachtet die Menge 
\[
\calli{C} := \left\{\begin{pmatrix} a &-b \\ b & a \end{pmatrix} ; a,b \in \RR\right\}
\]
von speziellen $2\times 2$-Matrizen mit der Addition und 
Multiplikation von $2\times 2$-Matrizen und weist nach, dass $\calli{C}$ 
bezüglich dieser Verknüpfungen ein Körper ist. 

3. Das dritte Modell ist algebraisch orientiert. Man betrachtet den Polynomring 
$\RR[X]$ und das von dem irreduziblen Polynom $P=X^2+1$ erzeugte Ideal 
und den Restklassenring 
\[
\RR[X] / (X^2+1).
\]
Diese Methode geht auf Cauchy (1847) zurück und ist vor allem wegen ihrer 
Verallgemeinerungsfähigkeit (Satz von Kronecker, vgl.~\citep{Karpfinger}) von Bedeutung.

Es stellt sich heraus, dass die verschiedenen Modelle isomorph sind. 

\begin{frage}
Können sie folgende Aussagen beweisen?

\begin{enumerate}
\item Definiert man auf der Menge 
\[
\CC := \RR^2 = \RR\times \RR = \left\{ (a,b) ; a,b \in \RR \right\}
\]
aller reellen Zahlenpaare eine Addition 
\[
(A)\quad  (a,b) \oplus (a',b') = (a+a',b+b')
\]
sowie eine Multiplikation
\[
(M)\quad (a,b) \otimes (a',b') = (aa'-bb', ab'+a'b),
\]
dann ist $(\CC,\oplus, \otimes)$ ein Körper.   

\item Die Menge 
\[
\CC_{\RR} = \left\{ (a,0) | a\in\RR \right\}
\]
ist ein zu $\RR$ isomorpher Unterkörper von $\CC$. 
Durch $a\mapsto (a,0)$ wird $\RR$ mit $\CC_{\RR}$ identifiziert.

Schreibt man eine reelle Zahl $a\in\RR$ als $(a,0)$, dann erkennt man, dass 
$\RR$ ein Unterkörper von $\CC$ ist. 

Speziell schreibt man $0=(0,0)$ und $(1,0)=1$. Für das Element 
$\i:=(0,1)\in\CC$ gilt $\i^2 = (-1,0) = -1$.

In $\CC$ hat die Gleichung $z^2 +1=0$ die beiden einzigen Lösungen $\i$ und $-\i$. 

Jedes $z\in\CC$ besitzt eine eindeutige Normaldarstellung 
\[
z = a+b\i \quad\text{mit}\quad a,b\in\RR.
\]
Dabei heißt $a$ der Realteil und $b$ der Imaginärteil von $z$.
\end{enumerate}
\end{frage}

\begin{antwort}
Um nachzuweisen, dass $\CC$ ein Körper ist, muss man die 9 Körperaxiome einzeln nachprüfen. 

$(\CC,\oplus)$ ist eine abelsche Gruppe mit dem neutralen Element $0=(0,0)$ und das 
additive Inverse zu $z=(a,b)$ ist $z':=(-a,-b)$. 

$(\CC\mengeminus\{0\},\otimes)$ ist eine abelsche Gruppe mit dem neutralen Element $(1,0)$. Das multiplikativ Inverse $(x,y)$ zu $(a,b) \not= (0,0)$ ergibt sich 
aus dem eindeutig lösbaren linearen Gleichungssystem 
\[
ax-by = 1, \qquad bx+ay = 0.
\]
Es ist 
\[
x=\frac{a}{a^2+b^2}\quad\text{und}\quad
y=\frac{-b}{a^2+b^2}.
\]
Die Gültigkeit des Distributivegesetzes ist unmittelbar zu sehen, 
und das Assoziativgesetz für die Multiplikation ergibt sich einfach 
durch Nachrechnen aus dem Assoziativgesetz in $\RR$.

Ebenfalls durch elementares Nachrechnen bestätigt man, dass die Menge 
\[
\CC_{\RR} := \left\{ (a,0) | a\in\RR \right\}
\]
mit der auf $\CC_{\RR}$ eingeschränkten Addition und Multiplikation ein Körper ist 
und die Abbildung 
\[
j: \RR\to\RR, a \mapsto (a,0)
\]
ein Körperisomorphismus ist.

Identifiziert man die komplexe Zahl $(a,0)$ mit der reellen Zahl $a$, 
so wird $\RR$ zu einem Unterkörper von $\CC$.

Für das spezielle Element $\i=(0,1) \in \CC$ gilt 
\[
\i^2 = (0,1) \cdot (0,1) = (-1,0) = -1.
\]
Da ein Polynom vom Grad 2 höchstens 2 Nullstellen hat, sind also $\i$ und 
$-\i$ die einzigen Nullstellen der Gleichung $z^2 +1=0$. 

Die Normaldarstellung einer komplexen Zahl $z=(a,b)$ erhält man aus 
\[
z=(a,b) = (a,0) + (0,b) = a + (b,0) (0,1) = a+b\i.
\]

Die Eindeutigkeit ist evident, denn aus 
\[
a+b\i = a'+b'\i, \quad a\not=a', a,a',b,b' \in\RR
\]
folgt 
\[
a-a' = (b'-b) \i.
\]
Wäre $b'\not b$, dann folgte der Widerspruch $\frac{a-a'}{b-b'} = \i \in\RR$.

Also ist $b=b'$ und damit auch $a=a'$.   
\end{antwort}

\bigskip

Die zunächst als ziemlich willkürlich erscheinende Multiplikation 
(M) komplexer Zahlen wird also nachträglich dadurch gerechtfertigt, dass 
$(\CC, \oplus, \otimes)$ ein Körper ist. 

Da die Einschränkung der Addition und Multiplikation komplexer Zahlen 
auf reelle Zahlen gerade die dort schon existierende Addition und Multiplikation 
ist, verwendet man für die Addition und Multiplikation komplexer Zahlen 
auch einfach die Bezeichnungen
\begin{align*}
(a,b) + (a',b') &= (a+a', b+b')\quad \text{bzw.}\\
(a,b) \cdot (a',b') &= (a,b) (a',b').
\end{align*}
Ferner stimmt das Produkt 
\[
(r,0) (a,b) = (ra,rb)
\]
mit dem Ergebnis der skalaren Multiplikation $r(a,b) = (ra,rb)$ in $\RR^2$ überein.


\begin{frage}
Können Sie die Multiplikation (M) für die komplexen Zahlen motivieren?

Tipp: Nehmen Sie an, dass es einen Körper $\KK$ gibt, der die reellen Zahlen 
als Unterkörper enthält und in welchem es ein Element $\i$ mit $\i^2=-1$ 
gibt. Wie rechnet man dann in diesem Körper?
\end{frage}

\begin{antwort}
Bereits die Teilmenge $\CC:=\{a+b\i; a,b\in\RR\}$ ist ein Körper, denn für 
$z=a+b\i$ und $z'=a'+b'\i$ ($z,z'\in\CC$) gilt
\begin{align*}
z+z' &= (a+a') + (b+b') \i \in \CC\quad\text{und}\\
zz' &= (aa' + bb' i^2) + (ab'+ a'b) \i \\
&= (aa'-bb') + (ab'+a'b)\in \in \CC
\end{align*}
und, falls $a+b\i\not=0$ ist
\[
(a+b\i)^{-1} = \frac{a-b\i}{(a+b)(a-b\i)} = 
\frac{a}{a^2+b^2} + \frac{-b}{a^2+b^2}\in\CC.
\]
Die Definition der Multiplikation in $\CC$ ist also zwangsläufig. 

Anmerkung: Hat man die Formel für die Multiplikation komplexer Zahlen 
vergessen, so schreibe man $z=a+b\i$ und $z'=a'+b'\i$ und rechne 
distributiv unter Verwendung von $\i^2 = -1$.

Beispiel: $z=2-6\i, z'=1+\i$.
\[
zz' = aa' - bb' + (ab'+a'b) \i = 2\cdot 1 - 6\i^2 + (2-6\i) = 8-4\i.
\]
\end{antwort}



\begin{frage}
Durch welche Eigenschaften ist der Körper der komplexen Zahlen bis auf 
Isomorphie eindeutig bestimmt?
\end{frage}

\begin{antwort}
Der Körper $\CC$ der komplexen Zahlen ist bis auf Isomorphie durch folgende 
Eigenschaften eindeutig bestimmt. 
\begin{enumerate}
\item $\CC$ enthält $\RR$ als Unterkörper.
\item In $\CC$ gibt es ein Element $\i$ mit $\i^2=-1$.
\item Jedes $z\in\CC$ lässt sich eindeutig in der Gestalt 
$z=a+b\i$ mit $a,b\in\RR$ darstellen. 
\end{enumerate}

Dass $\CC$ die drei Eigenschaften hat, ist aufgrund der Konstruktion 
klar. Um $\RR$ als Unterkörper von $\CC$ aufzufassen, muss man allerdings 
$\CC_{\RR}$ und $\RR$ identifizieren, was aufgrund des Isomorphismus
\[
j : \RR\to \CC_{\RR}, \quad a\mapsto (a,0)
\]
gerechtfertigt ist.

Durch Betrachung von $\tilde{\CC}:=(\CC \mengeminus \CC_{\RR} ) \cup \RR$ 
kann man jedoch auch einen Körper konstruieren, der $\RR$ von vorne herein 
als Unterkörper enthält. Wir haben oben auf diese algebraische Konstruktion 
verzichtet und haben $\CC_{\RR}$ mit $\RR$ identifiziert. 
Ist nun $\CC'$ ein weiterer Körper, der $\RR$ als Unterkörper enthält und 
in dem es ein Element $\i'$ mit $\i'^2 = -1$ gibt und für den 
$\CC' = \RR + \RR\i'$ gilt, so betrachte man die Abbildung 
\[
\Psi : \CC' \to \CC, a'+b'\i \mapsto a+b\i\quad(a,b\in\RR).
\]
Diese ist offensichtlich ein Isomorphismus.
\end{antwort}

\begin{frage}
Können Sie nachweisen, dass die Menge 
\[
\calli{C} := \left\{ \begin{pmatrix} a & -b \\ b & a \end{pmatrix} ; 
a,b \in \RR \right\}
\]
der speziellen reellen $2\times 2$-Matrizen zusammen mit der Matrizenaddition 
und -multiplikation ein Körper ist, der zum Körper $\CC$ isomorph ist?
\end{frage}

\begin{antwort}
Die Menge $\calli C$ ist abgeschlossen gegenüber der Addition von Matrizen, 
das Nullelement ist die Nullmatrix $\begin{pmatrix} 0 &0 \\ 0 &0\end{pmatrix}$. 

$\calli C$ ist auch gegenüber der Multiplikation abgeschlossen und die Multiplikation 
ist für diese speziellen Matrizen kommutativ, wie man sofort nachprüft.

Die inverse Matrix zu 
$A=\begin{pmatrix} a & b \\ -b & a \end{pmatrix}$ ist 
\[
A^{-1} = \frac{1}{\det A} A^T = \frac{1}{a^2+b^2} 
\begin{pmatrix} a & b \\ -b & a \end{pmatrix}.
\]
Für die spezielle Matrix $I:=\begin{pmatrix} 0 & -1 \\ 1 & 0 \end{pmatrix}$ 
gilt $I^2 = - \begin{pmatrix} 1 & 0 \\ 0 & 1 \end{pmatrix} = -E_2$.

Durch die Abbildung 
\[
\Psi: \calli{C} \to \CC, 
\quad
\begin{pmatrix} 
a &-b \\ b & a
\end{pmatrix}
= 
a \begin{pmatrix} 
1&0 \\ 0&1
\end{pmatrix}
+ b \begin{pmatrix} 
0 & -1 \\ 1 & 0 
\end{pmatrix}
\mapsto a+b\i
\]
erhält man einen Isomorphismus.\AntEnd
\end{antwort}


\begin{frage}
Können Sie erläutern, wie man $\CC$ als \slanted{algebraische Körpererweiterung} 
von $\RR$ erhält?
\end{frage}

\begin{antwort}
Im Polynomring $\RR[X]$ bildet man die Restklassen $\RR[X]/(X^2 +1)$.

Zwei Polynome $f$ und $g$ aus $\RR[X]$ sind also genau dann äquivalent, 
wenn sie bei Division durch $X^2+1$ denselben Rest lassen. 
Die Reste haben hier die Gestalt $a+bX$ mit $a,b\in\RR$ und bilden 
ein Repräsentantensystem für $R[X]/(X^2+1)$. 

Für die Repräsentanten rechnet man wie folgt
\begin{align*}
[a+bX] \oplus [c+dX] &= \left[(a+c)+(b+d) X \right] \\
[a+bX] \otimes [c+dX] &= \left[ac+bdX^2 + (ad+bc)X \right] \\
&= \left[ (ac-bd) + (ad+bc) X \right].
\end{align*}

Die zu $[a+bX]$ inverse Äquivalenzklasse ist .
\[
\left[\frac{a}{a^2+b^2}-\frac{b}{a^2+b^2} X\right]\quad
(a,b) \not= (0,0).
\]
Durch 
\[
\RR \to \RR[X] / (X^2+1), \quad 
a \mapsto [a]
\]
wird $\RR$ in $\RR[X]/(X^2+1)$ eingebettet. 

Setzt man $j:= [X]$, dann ist $j$ Nullstelle des Polynoms 
$X^2+1$, \dasheisst\ $j^2=-1$, und es ergibt sich
\[
[a+bX] = a+b[X] = a+bj\quad \text{mit} j^2=-1.
\]
Die Abbildung 
\[
\Psi: \RR[X] / (X^2+1) \to \CC, \quad 
[a+bX] \mapsto a+b\i
\]
ist ein Isomorphismus.

\medskip
\noindent
Hintergrund: Die Konstruktion der komplexen Zahlen als 
Restklassenring nach dem von $X^2+1$ erzeugten Ideal geht auf Cauchy 
(1847) zurück. Der von Cauchy beobachtete Fall ist ein Spezialfall 
eines allgemeinen Satzes von C.~Kronecker~(1882) (vgl. \citep{Karpfinger}).

\satz{%
Für jedes irreduzible Polynom $P\in K[X]$ existiert ein 
Erweiterungskörper $L$ von $K$ mit $\deg (L|K) = \deg P$ und ein 
Element $a\in L$ mit $P(a)=0$. }

In unserem Sepzialfall ist $K=\RR$, $P=X^2+1\in \RR[X]$ und 
$a=j=[X]$.
\AntEnd
\end{antwort}


\begin{frage}
Warum lässt sich der Körper der komplexen Zahlen nicht 
(wie die reellen Zahlen) anordnen?
\end{frage}

\begin{antwort}
In einem angeordneten Körper gilt für ein Element $a\not=0$ stets 
$a^2 >0$. Wegen $\i^2 = -1< 0$ und $\i\not=0$ gilt dies in $\CC$ nicht, 
und daher kann $\CC$ nicht angeordnet sein. 

Beachte: Komplexe Zahlen lassen sich nicht der Größe nach vergleichen! 
Jedoch ist der Betrag einer komplexen Zahl eine nicht negative reelle 
Zahl und die Beträge komplexer Zahlen lassen sich der Größe 
nach vergleichen. 
\end{antwort}








%%% Local Variables: 
%%% mode: latex
%%% TeX-master: "master"
%%% End: 




%: Question 74
\begin{frage}
  Warum besitzt die Gleichung $x^2+1=0$ keine Lösung in $\RR$?
\end{frage}

\begin{antwort}
  Die Gleichung impliziert $x^2 =-1 <0$, das Quadrat einer reellen Zahl 
  ist aber stets positiv oder Null.  
  \AntEnd
\end{antwort}






%: Question 75
\begin{frage}\index{C@$\CC$}
  \nomenclature{$\CC$}{Körper der komplexen Zahlen}
  Welche Gründe gibt es, den Körper $\RR$ der reellen Zahlen zum  
  Körper $\CC$ der komplexen Zahlen zu erweitern?
\end{frage}

\begin{antwort}
  Ein Grund dafür liegt in der negativen Antwort zur vorigen Frage. 
  Nicht jede Gleichung, die sich im Körper der reellen Zahlen 
  formulieren lässt, besitzt innerhalb von $\RR$ auch eine 
  Lösung. 

  Um zu erreichen, dass jede polynomiale Gleichung eine Lösung besitzt, 
  gibt es nur die Möglichkeit, zu einem Erweiterungskörper von $\RR$ 
  überzugehen, der diese Eigenschaft hat. 
  Für jeden Körper gibt es einen solchen  
  \satz{algebraisch abgeschlossenen} Oberkörper. 
  Dass $\CC$ als Oberkörper der reellen Zahlen algebraisch abgeschlossen 
  ist, ist die Aussage des \satz{Fundamentalsatzes der Algebra: Jedes 
    nichtkonstante komplexe Polynom besitzt mindestens eine Nullstelle in $\CC$}.

  Das ganze Ausmaß der Bedeutung und Nützlichkeit der komplexen Zahlen 
  ist durch diese theoretische Motivation allerdings noch nicht annäherungsweise 
  erschöpft. Selbst in der reellen Analysis spielen sie an vielen Stellen  
  eine große Rolle. Stichworte sind \slanted{Fourierreihen}, 
  \slanted{Partialbruchzerlegung rationaler Funktionen}, 
  \slanted{Schwingungsdifferenzialgleichungen} u.v.m. 

  Für die komplexe Analysis (Funktionentheorie) sind die komplexen Zahlen 
  die unverzichtbare Grundlage.
  \AntEnd
\end{antwort}


%: Question 76
\begin{frage}\index{Addition komplexer Zahlen}
  \index{Multiplikation komplexer Zahlen}
  \index{i (imaginäre Einheit)}
  \nomenclature{$\i$}{imaginäre Einheit}
  Wie addiert bzw. multipliziert man zwei komplexe Zahlen $z$, 
  $z'$, wenn diese gegeben sind durch 
  $z=a+\i b$ bzw. $z'=a'+\i b'$, ($a,a',b,b' \in \RR,\; \i ^2=-1$)?
\end{frage}



\begin{antwort}[]
  \Ant Als Summe zweier komplexer Zahlen erhält man
  \begin{equation*}
    z+z' = (a+\i b)+(a+\i b')=(a+a')+\i (b+b').
  \end{equation*}
  Geometrisch entspricht das der Addition zweier Vektoren im $\RR^2$, 
  \sieheAbbildung\ref{fig:01_add_komplex}.

  \noindent
  Für das Produkt $zz'$ gilt
  \begin{align*}
    z z' &= (a+\i b)(a'+ \i b')=aa'+iab'+\i a'b +\i ^2 bb' \\ 
    &= (aa'-bb')+\i (ab'+a'b).
  \end{align*}
  Eine geometrische Interpretation der Multiplikation 
  erhält man leicht mittels Polarkoordinaten, vergleiche dazu die Fragen 
  \ref{polar} und \ref{polarmult}. \AntEnd

  \begin{center}
    \includegraphics{mp/01_add_komplex}
    \captionof{figure}{Die Addition zweier Zahlen in $\CC$ entspricht der 
      Vektoraddition in $\RR^2$.}
    \label{fig:01_add_komplex}
  \end{center}

\end{antwort}






%: Question 77
\begin{frage}
  Wie berechnet man zu einer komplexen Zahl $z=a+\i b$ 
  mit $(a,b)\not=(0,0)$ das \bold{multiplikativ inverse} 
  Element?
\end{frage}

\begin{antwort}
  Sei $z^{-1}:=x+\i y$ das gesuchte Inverse. Der Ansatz 
  $z\cdot z^{-1} = (a+\i b)(x+\i y) = 1 = 1+ \i\cdot 0$ 
  führt auf die beiden Gleichungen $ax-by=1$ und $ay+bx=0$, und mit 
  einer direkten Rechnung erhält man daraus die Lösungen 
  \begin{equation}
    x=\frac{a}{a^2+b^2}, \qquad\qquad y=-\frac{b}{a^2+b^2}. 
    \EndTag
  \end{equation}
\end{antwort}






%: Question 78
\begin{frage}\index{konjugiert komplexe Zahl}
  \nomenclature{$\overline{z}$}{konjugiert komplexe Zahl}
  Was versteht man unter der zu einer komplexen Zahl $z=a+b\i$ 
  \bold{konjugiert komplexen Zahl}?
\end{frage}

\begin{antwort}
  Die komplexe Zahl $\ov{z}:=a-\i b$. 
  \AntEnd
\end{antwort}






%: Question 79
\begin{frage}
  Welche Haupteigenschaften hat die Abbildung 
  $\ov{\rule{3mm}{0mm} \rule{0mm}{1.5mm}}:\,\CC \to \CC, \; 
  z=a+\i b \mapsto \ov{z}=a-\i b$.
\end{frage}


\begin{antwort}
  Es gelten die folgenden Eigenschaften:
  \[
  \ov{\ov{z}}=z, \qquad \ov{z\pm z'} = \ov{z} \pm \ov{z'}, \qquad  
  \ov{zz'}=\ov{z}\cdot \ov{z'}, \qquad 
  a=\frac{z+\ov{z}}{2},\;b=\frac{z-\ov{z}}{2\i},
  \]
  sowie ferner
  \[
  z \in \RR \LLa z=\ov{z}, z\in \i\RR \LLa \ov{z}=-z,
  \qquad\qquad z\ov{z}=a^2+b^2 \in \RR_{\ge 0},
  \]
  \sieheAbbildung\ref{fig:01_kompl_konj}.
  \begin{center}
    \includegraphics{mp/01_kompl_konj}
    \captionof{figure}{Zum Verhältnis von Negativem und konjugiert Komplexem einer komplexen Zahl.} 
    \label{fig:01_kompl_konj}
  \end{center}
  Alle Eigenschaften lassen sich mit direkten Rechnungen verifizieren. 

  Anmerkung: Die Abbildung $\CC\to\overline{\CC}$ ist ein 
  \textbf{involutorischer Automorphismus}\index{Automorphismus} mit dem 
  Fixkörper $\RR$. 
  \AntEnd
\end{antwort}






%: Question 80
\begin{frage}
  \index{Betrag!einer komplexen Zahl}
  Wie ist der \bold{Betrag einer komplexen Zahl} 
  $z=a+\i b$ ($a,b\in\RR$)? 
\end{frage}

\begin{antwort}
  Der Absolutbetrag ist für $z,w\in\CC$ definiert durch 
  \[
  \boxed{
    |z|:= \sqrt{z\ov{z}} = \sqrt{a^2+b^2}.}
  \]
  Interpretiert man die komplexe Zahlenebene als den $\RR^2$ und die 
  komplexe Zahl $z=a+\i b$ als den Vektor $(a,b)$, 
  so entspricht der komplexe Betrag also gerade der Euklidischen 
  des assoziierten Vektors. \AntEnd 
\end{antwort} 



%: Question 81
\begin{frage}\label{01_komplexbetrag}
  Welche Haupteigenschaften gelten für den Betrag einer komplexen Zahl?
\end{frage}

\begin{antwort}
  Für $z,w\in\CC$ gilt

  \medskip
  \begin{tabular}{llp{3mm}ll}
    \desc{1} & $|z|\ge0$ und $|z|=0\Leftrightarrow z=0$, & &  
    \desc{4} & $|z\cdot w| = |z|\cdot |w|,$ \\ 
    \desc{2} & $|z+w| \le |z|+|w|$, & &
    \desc{5} & $|\Re z | \le |z|, \quad |\Im z | \le |z|,$ \\
    \desc{3} & $|z-w| \ge \big| |z|-|w| \big|.$
  \end{tabular}

  \medskip
  \noindent
  Diese Eigenschaften werden in den folgenden Fragen nachgewiesen.
  \AntEnd
  
\end{antwort}






%: Question 82
\begin{frage}\index{Realteil}\index{Imaginärteil}
  \nomenclature{$\Re z$}{Realteil von $z$}
  \nomenclature{$\Im z$}{Imaginärteil von $z$}
  Warum gilt für den Realteil bzw. Imaginärteil einer komplexen Zahl $z$ 
  stets $|\Re(z)| \le |z|$ bzw. $|\Im(z)| \le |z|$?
\end{frage}

\begin{antwort}
  Es gilt 
  $\dis |z|=\sqrt{ (\Re z)^2 + (\Im z)^2 } \ge \sqrt{(\Re z)^2} = |\Re z|$, 
  und auf demselben Weg erhält man die 
  Abschätzung für den Imaginärteil.
  \AntEnd
\end{antwort}






%: Question 83
\begin{frage}\index{Dreiecksungleichung!in $\CC$}
  Wie lauten die Dreiecksungleichungen im Komplexen 
  für Abschätzungen nach oben bzw. nach 
  unten?
\end{frage}

\begin{antwort}
  Es handelt sich um die Eigenschaften \desc{3} bzw. \desc{4} aus 
  Frage \ref{01_komplexbetrag}. Für $z:=x+\i y$ und $w := u+\i v$ 
  folgt \desc{3} aus 
  \begin{gather}
    \left(\sqrt{x^2+y^2} + \sqrt{u^2+v^2} \right)^2 =
    x^2+y^2 + 2\sqrt{x^2+y^2}\sqrt{u^2+v^2} + u^2 + v^2 \notag \\
    \le  x^2+2xu+u^2 + y^2+2yv +v^2 
    = \sqrt{ (x+u)^2+(y+v)^2 }^2. \notag
  \end{gather}
  Mit der Dreiecksungleichung erhält man \desc{4} auf dieselbe Weise, wie 
  die entsprechende reelle Ungleichung in \ref{01_dru} hergeleitet 
  wurde.  
  \AntEnd
\end{antwort} 






%: Question 84
\begin{frage}
  Warum wird durch $d(z,w):=|z-w|$ für $z,w\in\CC$ eine Metrik auf 
  $\CC$ definiert?
\end{frage} 

\begin{antwort}
  Die Abbildung $d\fd\CC\times \CC\to \RR$ erfüllt alle Eigenschaften einer 
  Metrik, das heißt
  {\setlength{\labelsep}{8mm}
    \begin{enumerate}
    \item[\desc{M1}] $|z-w|=0 \Longleftrightarrow z=w,$ \\[-3,5mm]
    \item[\desc{M2}] $|z-w|=|w-z|,$\\[-3.5mm]
    \item[\desc{M3}] \slanted{$|z-w| \le |z-\zeta| + |\zeta -w|$ für alle $\zeta\in\CC$ 
        (Dreiecksungleichung).}
    \end{enumerate}
  }
  \noindent
  Eigenschaft \desc{M1} folgt aus $|z|=0 \LLa z=0$, 
  \desc{M2} ist eine Folge von $|z|=|-z|$, und \desc{M3} ergibt sich 
  mit der Dreiecksungleichung für den komplexen Betrag:
  \begin{equation}
    |z-w| = |z - \zeta +\zeta - w| \le |z-\zeta|+|\zeta-w|. \EndTag
  \end{equation}
\end{antwort}






%: Question 85
\begin{frage}\index{Standardskalarprodukt}
  Wieso ist für $z=x+\i y$ ($x,y\in\RR$) und $w=u+\i v$ ($u,v\in\RR$)
  $\Re(z\ov{w})=\Re(\ov{z}w)$ gleich dem Standardskalarprodukt 
  von $z=(x,y)\in\RR^2$ und $w=(u,v)\in\RR^2$?
\end{frage}

\begin{antwort}
  Es gilt 
  \[
  \Re (z\ov{w})= \Re(\ov{z}w)= 
  \Re ( (x-iy)(u+iv)) = xu +yv=\langle z, w \rangle.
  \]
  Dabei sind die komplexen Zahlen $z$ und $w$ als Vektoren 
  des $\RR^2$ zu interpretieren. 
  \AntEnd
\end{antwort}






%: Question 86
\begin{frage}\index{Cauchy-Schwarz'sche Ungleichung}
  Warum ist die Ungleichung 
  $|\Re(z\ov{w})|=|\Re(\ov{z}w)|\le |z| \cdot |w|$ mit der 
  Cauchy-Schwarz'schen 
  Ungleichung im $\RR^2$ identisch?
\end{frage}

\begin{antwort}
  Im Sinne der Antwort zur vorigen Frage \slanted{ist} die 
  Ungleichung nichts anderes als die 
  Cauchy-Schwarz'sche Ungleichung                                                
  \[
  | \langle z,w \rangle | \le |z|\cdot|w|, \quad z,w \in\CC \simeq \RR^2.
  \]
  Dabei bezeichnet $|(x,y)|:=\sqrt{x^2+y^2}$ die euklidische Norm 
  im $\RR^2$.
  \AntEnd
  
\end{antwort}






%: Question 87
\begin{frage}\index{offene Kreisscheibe}
  \index{eps@$\eps$-Umgebung}
  \nomenclature{$U_\eps(a)$}{$\eps$-Umgebung von $a$}
  \label{01_ueps}
  Wie ist die \bold{offene Kreisscheibe} 
  $U_\eps(a)$ mit Mittelpunkt $a\in\CC$ und Radius 
  $\eps > 0$ in $\CC$ definiert?
\end{frage}


\begin{antwort}[]
  \Ant $U_\eps (a)$ ist die Menge aller $z\in \CC$, die von $a$ einen 
  kleineren Abstand als $\eps$ haben,
  \begin{equation}
    U_\eps (a) := \{ z\in\CC; \; |z-a| < \eps \}. 
  \end{equation}

  Geometrisch entspricht $U_\eps(a)$ einer offenen Kreisscheibe um $a$ mit Radius $\eps$, 
  \sieheAbbildung\ref{fig:01_ueps}.

  \begin{center}
    \includegraphics{mp/01_ueps}
    \captionof{figure}{Die offene Kreisscheibe $U_\eps(a)$}
    \label{fig:01_ueps}
  \end{center}
\end{antwort}



%: Question 88
\begin{frage}\index{Einheitskreislinie}
  \nomenclature{$S^1$}{Einheitskreislinie}
  Wie ist die \slanted{Einheitskreislinie}\index{Einheitskreislinie} 
  $S^1$ in $\CC$ definiert?

  Warum ist $S^1$ eine Gruppe bezüglich der Multiplikation in $\CC$ und warum 
  gilt $z^{-1}=\overline{z}$?
\end{frage}

\begin{antwort}
  $S^1$ ist die Menge aller komplexen Zahlen, die den Betrag $1$ haben,
  \[
  S^1 := \{ z\in\CC; \; |z|=1 \}.
  \]
  Geometrisch handelt es sich dabei um einen Kreis 
  mit Radius $1$ um den Nullpunkt.

  \medskip
  Mit $z,w\in S^1$ gilt wegen $|z|=|w|=1$ auch $|zw|=|z||w|=1$. 
  Also ist mit $z$ und $w$ auch das Produkt $zw$ ein Element von $S^1$. 
  Ferner ist $1\in S^1$. Wegen $|z|=|\overline z|$ für alle 
  $z\in \CC$ folgt aus $z\in S^1$ auch $\overline z \in S^1$, und 
  es gilt $z\overline z = |z|^2=1$. Also hat jedes $z\in S^1$ mit 
  $\overline z$ ein multiplikativ Inverses in $S^1$. Insgesamt folgt, 
  dass $S^1$ eine Gruppe ist.
  \AntEnd
\end{antwort}






%: Question 89
\begin{frage}\label{01_spiegelpunkt}
  \index{Spiegelung an der Einheitskreislinie}
  Was versteht man unter der \bold{Spiegelung an der Einheitskreislinie} 
  (häufig auch Spiegelung am Einheitskreis genannt)?
\end{frage}

\begin{antwort}[]
  \Ant Die Spiegelung an der Einheitskreislinie ist die Lösung 
  zu dem geometrischen Problem, zu zwei gegebenen Punkten 
  $O$ und $P$ der euklidischen Ebene  
  einen Punkt $P'$ zu konstruieren, für den $\ov{OP} \cdot \ov{OP'}=1$ gilt. 
  Der Punkt $P'$ heißt Spiegelpunkt zu $P$, wenn er auf der Halbgeraden 
  $OP$ liegt.

  Abbildung~\ref{fig:01_inversion1} zeigt die Konstruktion des Spiegelpunktes. 
  Je nachdem, ob $P$ oder $P'$ gegeben ist, 
  erhält man den Punkt $R$ als Schnittpunkt 
  von $S^1$ mit dem Thaleskreis über $OP$ bzw. der auf 
  $OP'$ senkrecht stehenden Geraden durch $P'$. 
  Die Konstruktion der rechtwinkligen Dreiecke $OP'R$ bzw. $ORP$ 
  liefert den gesuchten Spiegelpunkt.

  Die Gleichung $\ov{OP}\cdot \ov{OP'}$ folgt aus der Ähnlichkeit der 
  beiden Dreiecke $OP'R$ und $ORP$. 
  Mit dieser gilt nämlich $1:\ov{OP'}=\ov{OP}:1$.
  \AntEnd

  \begin{center}
    \includegraphics{mp/01_inversion1}
    \captionof{figure}{Konstruktion des Spiegelpunkts $P'$ mit $\ov{OP}\cdot \ov{OP'}=1$.}
    \label{fig:01_inversion1}
  \end{center}

\end{antwort}






%: Question 90
\begin{frage}
  Wie erhält man durch eine einfache geometrische Konstruktion 
  für einen Punkt $z\in\CC$, $z\not=0$, den Punkt $z^{-1}$?
\end{frage}


\begin{antwort}[]
  \Ant Wegen $z=|z|/\ov{z}$ liegt $1/ \ov{z}$ auf der Geraden durch $z$ und 
  dem Nullpunkt der komplexen Ebene. Da ferner $|z|\cdot|1/\ov{z}|=1$ gilt, 
  ist $1/\ov{z}$ gerade der Spiegelpunkt von $z$ bezüglich 
  der Einheitskreislinie, man erhält ihn aus $z$ mit der 
  geometrischen Konstruktion aus Frage \ref{01_spiegelpunkt}. 
  Aus $1/\ov{z}$ erhält man nun $1/z$ durch Spiegelung an der reellen 
  Achse, denn $1/z$ ist die zu $1/\ov{z}$ konjugiert komplexe Zahl, 
  \sieheAbbildung\ref{fig:01_inverse}
  \AntEnd

  \begin{center}
    \includegraphics{mp/01_inverse}
    \captionof{figure}{Geometrische Konstruktion von $z^{-1}$ 
      im Komplexen durch Spiegelung am Einheitskreis.}
    \label{fig:01_inverse}
  \end{center}

\end{antwort}






%: Question 91
\begin{frage}\label{polar}
  \index{Polarkoordinaten}
  Was versteht man unter einer \bold{Polarkoordinatendarstellung} einer komplexen Zahl? 
\end{frage}

\begin{antwort}[]
  \Ant Jede komplexe Zahl $\not=0$ lässt sich durch die Angabe 
  ihres Betrages $r$ und des (im Bogenmaß gemessenen) Winkels  
  $\varphi$, den die Gerade durch $z$ und den Nullpunkt mit der 
  positiven reellen Achse einschließt, eindeutig identifizieren. 
  Mit den trigonometrischen Funktionen Sinus und Cosinus, wie man sie etwa 
  aus der Schule her kennt, gilt dann $\Re z=r\cos \varphi$ 
  und $\Im z=r\sin\varphi$. 

  Daraus kann man schließen, 
  dass jede komplexe Zahl $z$ eine Darstellung der Form 
  \begin{equation*} 
    \boxed{z = r ( \cos \varphi + \i \sin \varphi ).} \tag{$\ast$}
  \end{equation*}
  besitzt. Man nennt ($\ast$) eine 
  \textit{Polarkoordinatendarstellung} von $z$, 
  $\varphi$ heißt ein \textit{Argument von $z$}, 
  \sieheAbbildung\ref{fig:01_polar}. Für $z=0$ ist $r=0$ und $\varphi\in\RR$ beliebig. \AntEnd

  \medskip
  \noindent
  Zur Abkürzung setzen wir 
  \[
  E(\varphi) = \cos\varphi + \i\sin\varphi.
  \]
  Wie wir später sehen werden gilt die \slanted{Euler'sche Formel} 
  $E(\varphi) = \exp(\i\varphi)$ und damit $z=r\exp(\i\varphi)$.
  \begin{center}
    \includegraphics{mp/01_polar}
    \captionof{figure}{In einer Polarkoordinatendarstellung wird $z\in \CC$ durch $r$ 
      und $\varphi$ ausgedrückt.}
    \label{fig:01_polar}
  \end{center}
\end{antwort}



%: Question 92
\begin{frage}
  Inwiefern ist die Polarkoordinatendarstellung einer 
  komplexen Zahl eindeutig?
\end{frage}

\begin{antwort}
  Die Zahl $\varphi$ ist in der Polarkoordinatendarstellung 
  {\astref} nicht eindeutig bestimmt, denn für alle $k\in\ZZ$ 
  ist mit {\astref} auch 
  \[
  r \big( \cos (\varphi +2k\pi)+\i\sin( \varphi + 2k\pi ) \big)
  \]
  eine Darstellung der Zahl $z$. Das ist ein Ausdruck der 
  anschaulichen Tatsache, dass man zum Ausgangspunkt zurückkommt, 
  wenn man in der komplexen Ebene den Ursprung $k$-mal umrundet. 
  Das Argument von $z$ ist also nur bis auf die Addition 
  eines ganzzahligen Vielfachen von $2\pi$ bestimmt. Streng 
  genommen sollte man $\arg z$ deswegen auch nicht als reelle 
  Zahl auffassen, sondern als \textit{\"Aquivalenzklasse reeller Zahlen}, die 
  durch die Relation $\varphi \sim \psi \LLa \varphi-\psi \in 2\pi\ZZ$ 
  bestimmt ist. 
  In der Regel wählt man $\varphi\in \ropen{0,2\pi}$ 
  oder $\varphi \in \lopen{-\pi,\pi}$. Für $\varphi \in \lopen{-\pi,\pi}$ 
  spricht man vom \slanted{Hauptwert des Arguments}\index{Hauptwert des Arguments einer komplexen Zahl}. 
  Die Abbildung von $\CC$ auf den Hauptwert des Arguments wird 
  mit $\mathrm{Arg}$ bezeichnet. 
  So gilt zum Beispiel $\mathrm{Arg}\,\i = \frac\pi2$, 
  $\mathrm{Arg}(-\i)=-\frac\pi2$ und $\mathrm{Arg}\,(-1)=-\pi$.
  \AntEnd
\end{antwort}






%: Question 93
\begin{frage}\index{Multiplikation komplexer Zahlen}\label{polarmult}
  Wie multipliziert man zwei komplexe Zahlen $z$ und $w$, wenn diese 
  durch eine Polarkoordinatendarstellung gegeben sind?
\end{frage}

\begin{antwort}
  Mit der Polarkoordinatendarstellung zweier komplexer Zahlen $z_1$ und 
  $z_2$ erhält man
  \begin{align*}
    z_1\cdot z_2 &= r_1( \cos \varphi_1 + \i \sin  \varphi_1 )\cdot 
    r_2 ( \cos \varphi_2 + \i\sin \varphi_2 ) \\
    &=  r_1 r_2 \left( \cos\varphi_1\cos\varphi_2 - 
      \sin\varphi_1 \sin\varphi_2 + \i
      ( \cos\varphi_1\sin\varphi_2 - \sin\varphi_1 \cos\varphi_2 ) \right) \\
    &= r_1 r_2 \left( \cos( \varphi_1 + \varphi_2) + 
      \i\sin( \varphi_1 + \varphi_2 ) \right).\asttag
  \end{align*}
  Im letzten Schritt wurden dabei die \textit{Additionstheoreme} des 
  Sinus und Cosinus angewendet (vgl. Frage \ref{05_trei}\,\desc{5}). 
  \AntEnd
\end{antwort} 






%: Question 94
\begin{frage}\index{Drehstreckung}
  Ist $w=a+\i b$ eine komplexe Zahl $\not=0$, dann wird durch die  
  Abbildung $\mu_w(z):=z\mapsto wz$ 
  eine \bold{Drehstreckung} $\mu_w\fd\CC \to \CC$   
  definiert. Können Sie das zeigen?
\end{frage}

\begin{antwort}
  Als Abbildung $\RR^2 \to \RR^2$ betrachtet ist $\mu_w$ $\RR$-linear und wird 
  durch die Matrix $
  A =  \left( \begin{smallmatrix} a &  -b \\ b & a \end{smallmatrix} \right)$ 
  beschrieben. Diese besitzt die Zerlegung 
  \[
  \begin{pmatrix} 
    \sqrt{\det A}  &  0 \\
    0 & \sqrt{\det A} 
  \end{pmatrix}
  \cdot 
  \begin{pmatrix}
    \alpha & -\beta \\
    \beta &  \alpha
  \end{pmatrix}
  \]
  mit $\alpha:=a/\sqrt{\det A}$ und $\beta:= b/\sqrt{\det A}$. 
  Die vordere Matrix beschreibt offensichtlich eine Streckung. 
  Für die Einträge der hinteren gilt 
  \[
  \alpha^2 + \beta^2 = 
  \left(\frac{a}{\sqrt{\det A}}\right)^2 + 
  \left(\frac{b}{\sqrt{\det A}}\right)^2 = \frac{a^2+b^2}{\det A} = 
  \frac{\det A}{\det A} =1.
  \]
  Die Punkte $(\alpha,\beta)$ liegen also auf dem Einheitskreis, 
  damit gibt es eine Zahl $\vartheta\in\ropen{0,2\pi}$, sodass 
  \[
  \begin{pmatrix}
    \alpha & -\beta \\
    \beta & \alpha 
  \end{pmatrix} = 
  \begin{pmatrix} 
    \cos \vartheta & -\sin\vartheta \\
    \sin\vartheta & \cos \vartheta
  \end{pmatrix}.
  \]
  Eine Matrix dieser Form beschreibt bekanntlich eine Drehung im $\RR^2$. 
  $\mu_w$ setzt sich also zusammen aus einer Drehung, gefolgt von einer 
  Streckung. 
  \AntEnd
\end{antwort}






%: Question 95
\begin{frage}\index{Multiplikation!komplexer Zahlen}
  Wie lässt sich die Multiplikation komplexer Zahlen geometrisch deuten?
\end{frage}


\begin{antwort}[]%
  \Ant Mit {\astref} bekommt man folgende geometrische Deutung:  
  \slanted{Zwei komplexe Zahlen werden nach miteinander multipliziert, 
    indem man ihre Beträge multipliziert und ihre Argumente addiert.} 
  Das entspricht einer Drehstreckung im $\RR^2$, 
  \sieheAbbildung\ref{fig:01_mult_kompl}

  Mit der \slanted{Euler'schen Formel} $E(\varphi) E(\psi) = E(\varphi+\psi)$
  lässt sich der Sachverhalt auch in der Form 
  \[
  z\cdot w = |z| e^{\i\arg z} \cdot |w| e^{\i\arg w} = 
  |zw| e^{\i( \arg z+\arg w)}
  \]
  ausdrücken, wodurch der Zusammenhang mit der 
  \slanted{Funktionalgleichung} (s. Frage \ref{05_exp_funktionalgleichung}) 
  der komplexen Exponentialfunktion sichtbar wird.  

  Dabei gelte 
  \[
  \arg(z)+\arg(w) := \{ \varphi + \psi \sets \varphi\in\arg(z), \psi \in\arg(w) \}.
  \]
  \AntEnd 

  \begin{center}
    \includegraphics{mp/01_mult_kompl}
    \captionof{figure}{
      Für $zw$ gilt $\mathrm{Arg}\,(zw)=\mathrm{Arg}\,(z)+\mathrm{Arg}\,(w) \pmod {2\pi}$ 
      und $|zw| = |z|\cdot |w|$
    }
    \label{fig:01_mult_kompl}
  \end{center}
\end{antwort}







%: Question 96
\begin{frage}\index{Einheitswurzeln, $n$-te}
  Was versteht man unter der 
  \bold{$\mathbf{n}$-ten Einheitswurzel} in $\CC$ ($n\in\NN$)? 
  Wie viele $n$-te Einheitswurzeln gibt es?
\end{frage}

\begin{antwort}
  Eine $n$-te Einheitswurzel ist definitionsgemäß eine Lösung der 
  Gleichung $z^n-1=0$ in $\CC$, also eine komplexe Zahl $\zeta$, für 
  die $\zeta^n=1$ gilt. 

  Für eine $n$-te Einheitswurzel 
  $\zeta := r(\cos \varphi + \i\sin\varphi)$ 
  hat man aufgrund einer leicht zu beweisenden Verallgemeinerung der Antwort 
  zu vorigen Frage die Darstellung
  \[
  \zeta^n = r^n (\cos n\varphi + \i \sin n \varphi)=1.
  \]
  Wegen $\sin n \varphi=0$ folgt hieraus 
  $r=1$ und $n\varphi=2k\pi$ mit $k\in \ZZ$. 
  Andersherum sind mit $\varphi_k := 2k\pi/n$ auch alle Zahlen
  \[
  \zeta_k := ( \cos \varphi_k + \i \sin  \varphi_k), \qquad k\in\ZZ
  \]
  $n$-te Einheitswurzeln, aber nicht alle davon sind verschieden. 
  Für $\ell,j \in \ZZ$ gilt
  \[
  \zeta_\ell = \zeta_j \LLa \frac{2\ell\pi}{n} - 
  \frac{2j\pi}{n} \in 2\pi\ZZ 
  \LLa \ell - j \in n\ZZ.
  \]
  Es gibt also genau $n$ verschiedene $n$-te Einheitswurzeln, 
  nämlich 
  $\zeta_0, \zeta_1, \ldots, \zeta_{n-1}$. 
  Für diese hat man auch eine 
  explizite Darstellung
  \begin{equation*}
    \zeta_\ell = \cos \frac{2\ell\pi}{n} + \i \sin  
    \frac{2 \ell \pi }{n},\qquad 
    \ell=0,\ldots,n-1.
  \end{equation*}

  \begin{center}
    \includegraphics{mp/01_einheitswurzeln}
    \captionof{figure}{Dritte, vierte und fünfte Einheitswurzeln in $\CC$.}
    \label{fig:01_einheitswurzeln}
  \end{center}

  Die $n$-ten Einheitswurzeln liegen auf der Einheitskreislinie und 
  bilden die Eckpunkte eines regulären $n$-Ecks (vgl. \Abb\ref{fig:01_einheitswurzeln}). 
  Mit $\zeta_\nu$ und $\zeta_\mu$ ist 
  auch $\zeta_\nu \zeta_\mu$ eine $n$-te Einheitswurzel. 
  Aus den Eigenschaften der komplexen Multiplikation 
  ergibt sich leicht, dass die $n$-ten Einheitswurzeln bezüglich 
  der Multiplikation eine Gruppe bilden, die isomorph 
  zur Restklassengruppe $\ZZ/n\ZZ$ ist. \AntEnd
\end{antwort}






%: Question 97
\begin{frage}\index{nte Wurzeln@$n$-te Wurzeln komplexer Zahlen}
  Warum gibt es zu jeder komplexen Zahl $w\not=0$ genau $n$ Lösungen 
  der Gleichung 
  \[
  z^n=w, \quad( n\in\NN),
  \]
  und wie kann man die Lösungen explizit beschreiben?
\end{frage}

\begin{antwort}
  Mit 
  $z = |z| (\cos \varphi + i \sin \varphi)$ und 
  $w=  |w| (\cos \psi + i \sin \psi)$ folgt aus der Gleichung  
  $|z|^n = |w|$ und $n\varphi = \psi + 2k\pi$ für ein 
  $k\in \ZZ$.

  Andersherum ist für jedes $k\in\ZZ$ 
  \[
  z_k := \left|w\right|^{1/n}  ( \cos \varphi_k + i\sin \varphi_k ) 
  \quad\text{ mit }\quad
  \varphi_k := \frac{\psi + 2k\pi}{n}
  \]
  eine Lösung der Gleichung. Zwei dieser Lösungen $z_l$ und 
  $z_j$ sind identisch genau dann, wenn $\varphi_j - \varphi_k$ 
  ein ganzzahliges Vielfaches von $2\pi$ ist. 
  Das ist genau dann der Fall, wenn 
  $j$ und $l$ sich durch ein ganzzahliges Vielfaches 
  von $n$ unterscheiden. Es folgt, dass es genau $n$ Lösungen 
  der Gleichung gibt, nämlich $z_0,\ldots,z_{n-1}$. 
  Für diese gilt:
  \begin{equation}
    z_l = \left| w \right|^{1/n} \left( \cos\frac{\psi + 2l\pi}{n} + 
      i\sin \frac{\psi + 2l\pi}{n} \right), \qquad l = 1,\ldots, n-1. 
    \notag
  \end{equation}
  Im Übrigen hat man damit 
  einen Spezialfall des 
  \slanted{Fundamentalsatzes der Algebra} bewiesen. \AntEnd
\end{antwort} 




\section{Die Standardvektorräume $\RR^n$ und $\CC^n$}


Für das Studium geometrischer Probleme, bei denen auch Längen und Winkel 
eine Rolle spielen , ist die Vekrorraumstruktur nicht ausreichend. Man benötigt 
eine zusätzliche Struktur, die es z.\,B. ermöglicht, auch Längen von Vektoren 
und Winkel zwischen Vektoren zu definieren, insbesondere möchte man ausdrücken 
können, dass zwei Vektoren "`aufeinander senkrecht stehen"'. 

Eine solche Zusatzstruktur erhält man durch Einführung eines geigneten 
\slanted{Skalarprodukts}. Wir orientieren uns bei deren Einführung 
zunächst an den Standardvekorräumen, aber auch für allgemeinere Räume, etwa 
Funktionenräume, lassen sich Skalarprodukte definieren.




%: Question 98
\begin{frage}\index{Standardskalarprodukt!in $\RR^n$}
  \label{skalarRR}
  Wie ist im Standardvektorraum $\RR^n$ das 
  \bold{Standardskalarprodukt} definiert? 
  Welche Haupteigenschaften hat es?
  \nomenclature{$\langle \,\; \rangle$}{Skalarprodukt}
\end{frage}

\begin{antwort}
  Das Standardskalarprodukt im $\RR^n$ ist eine Abbildung $\RR^n\times \RR^n\to\RR$. 
  Für zwei Vektoren $x=(x_1,\ldots, x_n)$ und 
  $y=(y_1,\ldots,y_n)$ aus $\in \RR^n$ ist es definiert 
  durch
  \[
  \langle x,y \rangle = x_1y_1 + \cdots + x_n y_n.
  \]
  Die Haupteigenschaften des Standardskalarprodukts sind
  {\setlength{\labelsep}{5mm}
    \begin{itemize}[3mm]
    \item[\desc{i}] \textit{Bilinearität:}
      \[
      \langle \lambda(x+x'), y \rangle 
      = \lambda \langle x,y \rangle 
      + \lambda \langle x',y \rangle, \qquad
      \langle x, \lambda(y+y') \rangle 
      = \lambda \langle x,y \rangle 
      + \lambda \langle x,y' \rangle, 
      \]
    \item[\desc{ii}] \textit{Symmetrie:}\quad 
      $\langle x,y \rangle = \langle y,x \rangle$,\\[-3.5mm]
    \item[\desc{iii}] \textit{Positive Definitheit:}\quad 
      $\langle x,x \rangle \ge 0$, \quad 
      $\langle x,x \rangle =0 \Longleftrightarrow x=0$.
      \AntEnd
    \end{itemize}}
\end{antwort}






%: Question 99
\begin{frage}\index{Standardskalarprodukt!in $\CC^n$}
  \label{skalarCC}
  Wie ist im $\CC$-Vektorraum $\CC^n$ das Standardskalarprodukt definiert 
  und welche Haupteigenschaften hat es?
\end{frage}

\begin{antwort}
  Für zwei Vektoren $z=(z_1,\ldots,z_n)$ und $w=(w_1,\ldots,w_n)$ aus 
  $\CC^n$ ist das kanonische Skalarprodukt 
  $\langle\;,\;\rangle_c\, : \, \CC^n \times \CC^n \to \CC$ definiert durch 
  \[
  \langle z,w \rangle_c := z_1\ov{w}_1 + \cdots + z_n \ov{w}_n.
  \]
  Die Eigenschaften \desc{i} und \desc{iii} des reellen Skalarprodukts gelten 
  genauso im komplexen 
  Fall, wie man leicht nachrechnet. Statt der Eigenschaft \desc{ii} gilt 
  aber jetzt
  \[
  \text{\desc{ii}$_c$} \quad \quad\langle z, w 
  \rangle_c = \ov{\langle w,z \rangle_c}.
  \]
  Für $z,w \in \RR^n$ ist $\langle z, w \rangle_c = \langle z, w \rangle$. 
  Das Standardskalarprodukt im $\CC^n$ kann also als Fortsetzung des 
  reellen Standardskalarprodukts verstanden werden. 
  Man beachte allerdings, dass bezüglich der Abbildung
  \[
  \CC^n \to \RR^{2n}, \quad z=(x_1+iy_1,\ldots, x_n+iy_n) 
  \mapsto (x_1,y_1,\ldots,x_n,y_n) = v
  \]
  die beiden Werte $\langle z, z'\rangle_c$ und $\langle v, v' \rangle$ 
  im Allgemeinen verschieden sind.
  \AntEnd
\end{antwort}






%: Question 100
\begin{frage}\index{Skalarprodukt}
  Was versteht man in einem $\KK$-Vektorraum 
  ($\KK=\RR$ oder $\KK=\CC$) allgemein unter einem \bold{Skalarprodukt}?
\end{frage}

\begin{antwort}
  Für $\KK=\RR$ ist ein Skalarprodukt eine \textit{positiv definite, symmetrische 
    Bilinearform}, das heißt eine Abbildung $\RR\times \RR^n \to \RR^n$, die  
  die drei Eigenschaften \desc{i}, \desc{ii} und \desc{iii} 
  aus Frage \ref{skalarRR} erfüllt.

  Für $K=\CC$ ist ein Skalarprodukt eine Abbildung $\CC\times\CC\to\CC$, 
  die die Eigenschaften \desc{i} und \desc{iii} aus Aufgabe \ref{skalarRR} 
  sowie Eigenschaft 
  \desc{ii}$_c$ aus Aufgabe \ref{skalarCC} erfüllt. 
  Man spricht dann von einer 
  \textit{positiv definiten hermiteschen} Form. \index{hermitesche Form (positiv definite)}
  \AntEnd
\end{antwort}







%: Question 101
\begin{frage}\index{euklidische Norm}
  Wie ist die \bold{euklidische Norm} im $\RR^n$ definiert?
\end{frage}

\begin{antwort}
  Die euklidische Norm ist eine Abbildung $\|\;\;\|\,:\, \RR^n \to \RR^n$, 
  die für $x=(x_1,\ldots,x_n)\in \RR^n$ definiert ist durch
  \[
  \| x \| = \sqrt{ x_1^2 +\cdots x_n^2 } = 
  \sqrt{ \langle x,x \rangle } .
  \]
  Der Zusammenhang mit dem Standardskalarprodukt im $\RR^n$ liegt 
  in der zweiten Gleichung begründet.
  \AntEnd
\end{antwort}






%: Question 102
\begin{frage}\index{Cauchy-Schwarz'sche Ungleichung}
  \index{Cauchy@\textsc{Cauchy}, Augustin Louis (1789-1857)}
  \index{Schwarz@\textsc{Schwarz}, Hermann Amandus (1843-1921)}
  Wie lautet die 
  \bold{Cauchy-Schwarz'sche Ungleichung} in einem 
  $K$-Vektorraum mit Skalarprodukt ($K=\RR$ bzw. $K=\CC$)?
\end{frage}

\begin{antwort}
  Für Vektoren $x,y\in K^n$ lautet die Cauchy-Schwarz'sche Ungleichung 
  \begin{equation}
    | \langle x,y \rangle | \le \| x\|\cdot  \|y\|,  \tag{$\ast$}
  \end{equation}
  dabei bezeichnet $\langle \;,\; \rangle$ das Standardskalarprodukt 
  im $\RR^n$ bzw. $\CC^n$, je nachdem, ob $\KK=\RR$ oder $\KK=\CC$ gilt.
  \AntEnd
\end{antwort}


%: Question 103
\begin{frage}
  Wie lautet die Cauchy-Schwarz'sche Ungleichung speziell für 
  $\RR^n$ bzw. $\CC^n$ und das jeweilige Standardskalarprodukt?
\end{frage}

\begin{antwort}
  Für $x,y\in \RR^n$ bzw. $z,w\in\CC^n$ besitzt die Cauchy-Schwarz'sche 
  Ungleichung bezüglich des jeweiligen Standardskalarprodukts die Form
  \[
  |x_1y_1+\ldots + x_ny_n|  \le  
  \sqrt{ x_1^2 + \ldots + x_n^2 } \cdot 
  \sqrt{ y_1^2+ \ldots + y_n^2 } 
  \]
  bzw.
  \begin{equation}
    |z_1 \ov{w}_1 + \ldots z_n \ov{w_n}|  \le  
    \sqrt{ \left|z_1\right|^2 + \ldots + \left|z_n\right|^2 } \cdot 
    \sqrt{ \left|w_1\right|^2 + \ldots + \left|w_n\right|^2 }. \EndTag
  \end{equation}
\end{antwort}






%: Question 104
\begin{frage}\label{01_pnorm}\index{Norm}
  Welche \bold{Normen} im 
  $\KK^n$ ($\KK=\RR$ oder $\KK=\CC$) sind Ihnen geläufig und kennen 
  Sie Beziehungen (Ungleichungen) zwischen ihnen?
\end{frage}

\begin{antwort}
  Die für $\RR^n$ und $\CC^n$ geläufigsten Normen sind die 
  sogenannten $p$-Normen, die eine Verallgemeinerung der euklidischen 
  Norm darstellen. Für jedes $p\ge 1$ ($p$ kann natürlich, rational 
  oder reell sein) ist die \textit{$p$-Norm} $\|\;\;\|_p$ 
  eines Vektors $z\in \KK^n$ 
  definiert durch
  \[
  \| z \|_p := \left( \sum_{\nu=1}^n \left|z_{\nu}\right|^p \right)^{1/p}.
  \]
  Für $p=2$ ist das gerade die euklidische Norm, für $p=1$ erhält man die 
  \textit{$1$-Norm} $\| z \|_1 := |z_1|+\ldots + |z_n|$. 

  Aus $p>q$ folgt stets 
  \begin{equation}
    \| z \|_p \le \| z \|_q. \tag{$\ast$}
  \end{equation}  
  Das erkennt man, wenn man mit 
  $ |z_M| := \max \{ |z_1|,\ldots, |z_n| \}$ die 
  Normen in der Form
  \begin{equation}
    \| z \|_p = |z_M| \left( 1+ \sum_{\substack{i=1 \\ i\not=M}}^n 
      \left( \frac{|z_i|}{|z_M|} \right)^p \right)^{1/p}, \quad 
    \| z \|_q = |z_M|\left( 1+ \sum_{\substack{i=1 \\ i\not=M}}^n 
      \left( \frac{|z_i|}{|z_M|} \right)^q \right)^{1/q} \tag{$\ast\ast$}
  \end{equation}
  schreibt und sich überlegt, welche Zahlen hier größer und welche 
  kleiner als $1$ sind und wie die 
  Exponenten $p,\,q, \,\frac{1}{p}$ und $\frac{1}{q}$ auf diese Zahlen wirken.

  \noindent
  Der Darstellung ($\ast\ast$) kann man außerdem die Beziehung 
  $\lim_{p\to\infty} \| z \|_p = |z_M|$
  entnehmen. Aus diesem Grund definiert man als \textit{Maximumsnorm} 
  \[
  \| z \|_{\infty} := \max \{|z_1|, \ldots ,|z_n| \}.
  \]
  Abbildung~\ref{fig:01_normen2} zeigt die Einheitskreisscheibe 
  im $\RR^2$ bezüglich unterschiedlicher Normen.
  
  \begin{center}
    \includegraphics{mp/01_normen2}
    \captionof{figure}{Einheitskreisscheiben einiger $p$-Normen und der 
      Maximumsnorm im $\RR^2$}  
    \label{fig:01_normen2}
  \end{center}
  
  Für jedes $p\ge 1$ lässt sich der Wert $\|z\|_p$ durch die 
  Maximumsnorm auch nach oben abschätzen. Es gilt nämlich
  \[ 
  \|z\|_p  \le \left( n \cdot \left|z_M\right|^p \right)^{1/p} 
  \le \sqrt[p]{n} \|z\|_\infty. 
  \]

  \noindent
  Zusammen mit den ersten Ungleichungen folgt daraus, dass zu 
  je zwei Zahlen $p,q$ mit $p>q$ eine Konstante $C$ existiert mit 
  \[
  \| z \|_p \le \| z \|_q  \le C \|z\|_p.
  \]
  Konkret kann man $C:=\sqrt[p]{n}$ wählen. Die 
  Ungleichungskette besagt, dass alle $p$-Normen in $\RR^n$ 
  \textit{äquivalent} sind. Abbildung \ref{fig:01_normaq} veranschaulicht 
  das für die Normen $\|\;\,\|_2$ und $\|\;\,\|_\infty$.
  \AntEnd

  \begin{center}
    \includegraphics{mp/01_normaq}
    \captionof{figure}{Zur Äquivalenz der $p$-Normen.}
    \label{fig:01_normaq}
  \end{center}

\end{antwort}






%: Question 105
\begin{frage}\label{01_parallelogramm}
  Warum muss für eine Norm 
  $\|\;\; \|$ auf einem $ \KK$-Vektorraum $V$
  ($\KK=\RR$ oder $\KK=\CC$), 
  die aus einem Skalarprodukt auf $V$ abgeleitet ist, 
  die sogenannte \bold{Parallelo\-gramm\-identität}
  \[
  \| v+w \|^2 + \| v-w \|^2 = 2 \left( \|v\|^2 + \|w\|^2 \right)
  \]
  gelten?
\end{frage}

\begin{antwort}
  Die Norm ist in diesem Fall gegeben durch 
  $\| v \|=\sqrt{\langle v,v \rangle }$. Die Parallelogrammidentität 
  ergibt sich aufgrund der Bilinearität des Skalarprodukts
  \begin{align}
    \| v+w \|^2 &+ \| v-w \|^2 = 
    \langle v+w,v+w \rangle + \langle v-w, v-w \rangle \notag \\
    &=
    \langle v,v \rangle + \langle v,w \rangle + \langle w,v \rangle
    +\langle w,w \rangle + \langle v,v \rangle - \langle v,w \rangle
    - \langle w,v \rangle + \langle w,w  \rangle \notag \\
    &= 2 \langle v,v \rangle + 2\langle w,w \rangle
    = 2( \|v \|^2 + \|w\|^2 ). \EndTag
  \end{align}
  \noindent
  Bemerkung: Nicht jede Norm auf $\RR^n$ stammt von einem Skalarprodukt, 
  das gilt \zB für die Maximumsnorm auf $\RR^n$. Gilt jedoch in einem 
  normierten $\KK$-Vektorraum ($\KK=\RR$ oder $\KK=\CC$) die 
  Parallelogrammidentität, so kann man zeigen, dass die Norm von einem 
  Skalarprodukt induziert wird 
  (Satz von J.~von~Neumann\index{Satz!von J.~von~Neumann}). 
  Einen Beweis dieses Satzes findet man bei \citep{Huppert}.
  \AntEnd
\end{antwort}

\section{Einige wichtige Ungleichungen}

Die folgenden Ungleichungen sind für die Analysis unverzichtbar.


%: Question 106
\begin{frage}\label{01_ungleichungen}
  \index{Ungleichung!zwischen arithmetischem und geometrischem Mittel}
  \index{Ungleichung!zwischen harmonischem und geometrischem Mittel}
  \index{Ungleichung!zwischen arithmetischem und quadratischem Mittel}
  
  Warum gilt für positive reelle Zahlen $a$ und $b$ die Ungleichungskette 
  \[
  \min \{ a,b \} \le \frac{2ab}{a+b} 
  \le \sqrt{ab} \le \frac{a+b}{2} 
  \le \sqrt{\frac{a^2+b^2}{2}}
  \le \max\{ a,b \}.
  \]
  Man bezeichnet 
  $\frac{2ab}{a+b}$ als \bold{harmonisches}, 
  $\sqrt{ab}$ als \bold{geometrisches},
  $\frac{a+b}{2}$ als \bold{arithmetisches} und 
  $\sqrt{\frac{a^2+b^2}{2}}$ als \bold{quadratisches Mittel 
    von $a$ und $b$.}
\end{frage}

\begin{antwort}
  Wir zeigen die Ungleichungen der Reihe nach
  {\setlength{\labelsep}{4mm}
    \begin{itemize}
    \item[\desc{1}] Sei \oBdA\ $\min \{ a,b \}=a$. Dann folgt 
      $\frac{2ab}{a+b} \ge \frac{2ab}{b+b} = a.$ \\[-2mm]
    \item[\desc{2}] Wegen 
      \[
      \frac{2ab}{a+b} \le \sqrt{ab} \LLa 
      \frac{2ab}{\sqrt{ab}} \le a+b \LLa 
      \sqrt{ab} \le \frac{a+b}{2} 
      \]
      folgt diese Ungleichung aus der folgenden.\\[-2mm] 
    \item[\desc{3}] Es ist 
      $0 \le \left(\sqrt{a} -\sqrt{b} \right)^2=a-2\sqrt{ab} +b \ge 0$ 
      und damit, also $a+b \ge \sqrt{ab}$. \\[-2mm] 
    \item[\desc{4}] Aus $(a-b)^2 \ge 0$ folgt 
      $2a^2-2ab +2b^2 \ge a^2+b^2$ und damit 
      $2a^2+2b^2 \ge(a+b)^2$, also 
      $\sqrt{a^2+b^2} \ge \frac{a+b}{\sqrt{2}}$. Multiplikation mit 
      $\frac{1}{\sqrt{2}}$ auf beiden Seiten liefert die behauptete 
      Ungleichung. \\[-2mm]
    \item[\desc{5}] \OBdA\ sei $b = \max \{ a,b \}$, dann ist 
      $\sqrt{\frac{a^2+b^2}{2}}\le \sqrt{ \frac{2b^2}{2} } = b = \max\{a,b\}$. 
      \AntEnd
    \end{itemize}}
\end{antwort}






%: Question 107
\begin{frage}
  \label{q:bernoulli-ungleichung}
  \index{Bernouillische Ungleichung@Bernoulli'sche Ungleichung}
  Was besagt die \bold{Bernoulli'sche Ungleichung}?
\end{frage}

\begin{antwort}
  Für alle $n\in \NN_0$ und 
  alle reelle Zahlen $x\ge -1$ gilt
  \[
  (1+x)^n \ge 1+nx.
  \]
  Das ist die Bernoulli'sche Ungleichung. 
  Beweisen lässt sie sich mit vollständiger Induktion. 
  Für $n=0$ ist sie wegen 
  $(1+x)^0 = 1 \ge 1$ richtig. Ist sie für ein $n\in\NN$ bereits gezeigt, 
  dann folgt daraus
  \[
  (1+x)^{n+1}=(1+x)^n(1+x) \ge (1+nx)(1+x) = 1+nx+x+nx^2 \ge 1+(n+1)x,
  \]
  also ihre Gültigkeit für $n+1$.
  \AntEnd
\end{antwort}






%: Question 108
\begin{frage}\index{Youngsche Ungleichung@Young'sche Ungleichung}
  Was besagt die 
  \textbf{Young'sche Ungleichung} und wie kann man sie beweisen?
\end{frage}

\begin{antwort}
  Die Young'sche Ungleichung lautet (in ihrer nicht allgemeinsten, 
  aber gebräuchlichsten Formulierung, in der sie eine fundamentale Rolle 
  beim Beweis der Hölder'schen Ungleichung spielt): 
  Seien $a,b \ge 0$ und $p,q\in\RR$ mit $p,q>1$ 
  und $\frac{1}{p}+\frac{1}{q}=1$. Dann gilt
  \[
  a b \le \frac{a^p}{p}+ \frac{b^p}{q}.
  \]
  Man kann diese Ungleichung nur mit analytischen Methoden beweisen.  
  Eine Möglichkeit besteht darin, die Konvexität der Exponentialfunktion 
  auszunutzen (\sieheAbbildung\ref{fig:01_exp}), 
  also die Tatsache, dass für alle $x,y \in \RR$ und alle $t\in [0,1]$ gilt 
  \[
  \exp\big( x+ t(y-x) \big) \le \exp(x) + t\big(\exp(y)-\exp(x)\big).
  \]
  Daraus folgt dann nämlich mit $x=\log a$ und $y=\log b$ für 
  $a,b \not=0$  
  \begin{align}
    a\cdot b &= \exp( x+ y ) = \exp \left(\frac{1}{p} px+\frac{1}{q}qy\right)
    = 
    \exp \left( px + \frac{1}{q} ( qy-px ) \right) \notag \\
    &\le
    e^{px} + \frac{1}{q} ( e^{qy} - e^{px} ) = 
    \frac{1}{p} e^{px} + \frac{1}{q} e^{qy} = \frac{a^p}{p}+\frac{b^q}{q}.\EndTag
  \end{align}

  \begin{center}
    \includegraphics{mp/01_exp}
    \captionof{figure}{Beim Beweis der Young'sche Ungleichung 
      wird die Konvexität der Exponentialfunktion ausgenutzt: 
      Die Verbindungsstrecke zweier Punkte auf dem Graphen verläuft oberhalb des 
      Graphen.}
    \label{fig:01_exp}
  \end{center}

\end{antwort}

%: Question 109
\begin{frage}
  \label{01_youngspecial}
  Ein Spezialfall der Young'schen Ungleichung lautet
  \[
  |a| |b| = |ab| \le \frac{1}{2} (|a|^2+|b|^2), \quad (a,b \in\CC).
  \]
  Wie kann man diese Ungleichung elementar beweisen?
\end{frage}

\begin{antwort}
  Die Ungleichung ist nichts anderes als die in der Antwort zu Frage \ref{01_ungleichungen} 
  gezeigte Ungleichung 
  \[
  \sqrt{ab} \le \sqrt{\frac{a^2+b^2}{2}} 
  \]
  Sie folgt sofort aus 
  $(|a|-|b|)^2 = |a|^2-2|ab|+|b|^2 \ge0$.
  \AntEnd
  
\end{antwort}






%: Question 110
\begin{frage}\index{Cauchy-Schwarz'sche Ungleichung}
  \index{Cauchy@\textsc{Cauchy}, Augustin Louis (1789-1857)}
  \index{Schwarz@\textsc{Schwarz}, Hermann Amandus (1843-1921)}
  Wie kann man mit der Ungleichung aus Frage \ref{01_youngspecial} 
  die Cauchy-Schwarz'sche 
  Ungleichung in $\CC^n$ oder $\RR^n$ beweisen?
\end{frage}

\begin{antwort}
  Man kann gleich den allgemeineren Fall $z,w \in \CC^n$ angehen.      
  Wegen der Dreiecksungleichung 
  gilt $|\langle z,w \rangle | \le \sum_{k=1}^n |z_k w_k|$, und deshalb 
  genügt es, die Ungleichung 
  \[
  \sum_{k=1}^n |z_k| |w_k| \le  
  \underbrace{ \left(\sum_{k=1}^n |z_k|^2 \right)}_{ := A } \cdot 
  \underbrace{ \left(\sum_{k=1}^n |w_k|^2 \right)}_{ := B }
  \]
  zu zeigen. 
  Mit $\zeta_k := z_k /A$ und $\omega_k := w_k/B$ für $k=1,\ldots,n$ lautet 
  diese
  \[
  \sum_{k=1}^n  |\zeta_k \omega_k | \le 1,
  \]
  und in dieser Form erhält man sie nun leicht aus der Ungleichung 
  aus Frage \ref{01_youngspecial}
  \[
  \sum_{k=1}^n  |\zeta_k \omega_k | \le 
  \frac{1}{2}\sum_{k=1}^n (|\zeta_k|^2 + |\omega_k|^2) = 
  \frac{1}{2}\left( \sum_{k=1}^n |\zeta_k|^2 + \sum_{k=1}^n |\omega_k|^2 \right)
  = \frac{1}{2}(1+1)=1.
  \]
  Insgesamt beweist das die Cauchy-Schwarz'sche Ungleichung.
  \AntEnd
\end{antwort}






%: Question 111
\begin{frage}
  \label{minkowski}
  \index{Minkowskische Ungleichung@Minkowski'sche Ungleichung}
  \index{Minkowski@\textsc{Minkowski}, Hermann (1864-1909)}
  Wie lautet die \bold{Minkowski'sche Ungleichung} in $\RR^n$ bzw. $\CC^n$?
\end{frage}

\begin{antwort}
  Die Minkowski'sche Ungleichung ist die Verallgemeinerung der 
  Dreiecksungleichung für $p$-Normen. Sie lautet: Für $p\ge 1$ gilt 
  \[
  \|z + w \|_p \le \| z \|_p + \| w \|_p, 
  \qquad z,w \in \CC^n.
  \]
  Im speziellen Fall der euklidischen Norm ($p=2$) besitzt sie die 
  Darstellung
  \[
  \left( \sumkn |z_k +w_k |^2 \right)^{1/2} \le 
  \left( \sumkn |z_k|^2 \right)^{1/2} +
  \left( \sumkn |w_k|^2 \right)^{1/2},
  \]
  die man mithilfe der Cauchy-Schwarz'schen Ungleichung beweisen 
  kann. Dazu betrachte man
  \begin{eqnarray*}
    \sumkn |z_k+w_k|^2 &\le& 
    \sumkn |z_k+w_k| |z_k| + \sumkn |z_k+w_k||w_k| \\
    &\le& \left( \sumkn |z_k+w_k|^2 \right)^{1/2} \cdot  
    \left[ \left( \sumkn |z_k|^2 \right)^{1/2} + 
      \left( \sumkn |w_k|^2 \right)^{1/2} \right].
  \end{eqnarray*}
  Die letzte Ungleichung folgt aus der Cauchy-Schwarz'schen Ungleichung. 
  Die Minkowski'sche Ungleichung für $p=2$ erhält man jetzt, indem man 
  beide Seiten durch $(\sumkn |z_k+w_k|^2)^{1/2}$ dividiert.

  Den allgemeinen Fall der Minkowski'schen Ungleichung beweist man 
  mit der Hölder'schen Ungleichung. \AntEnd
\end{antwort}






%: Question 112
\begin{frage}
  \label{hoelder}\index{Hoeldersche@Hölder'sche Ungleichung}
  \index{Hoelder@\textsc{Hölder}, Otto (1859-1937)} 
  Wie lautet die \bold{Hölder'sche Ungleichung} in $\RR^n$ bzw. $\CC^n$?
\end{frage}

\begin{antwort}
  Die Hölder'sche Ungleichung lautet: 
  Seien $p,q>1$ (natürliche, rationale 
  oder reelle) Zahlen mit $\frac{1}{p}+\frac{1}{q}=1$. Dann gilt 
  für alle $z,w\in\CC^n$:
  \[
  \sumkn |z_k w_k | \le \| z \|_p \cdot \| w\|_q. 
  \]
  Die Hölder'sche Ungleichung erhält man aus der Young'schen Ungleichung. 
  Nach dieser gilt nämlich 
  \[
  \frac{|z_k|}{\|z\|_p} \frac{|w_k|}{\|w\|_q} \le 
  \frac{1}{p} \frac{ |z_k|^p }{ \| z \|_p^p } + 
  \frac{1}{q} \frac{ |w_k|^q }{ \| w \|_q^q },
  \]
  und die Hölder'sche Ungleichung folgt nun durch Summation 
  beider Seiten
  \begin{equation}
    \frac{1}{ \| z\|_p\, \| w \|_q } \sumkn |z_k| |w_k| \le 
    \frac{1}{p}+\frac{1}{q} =1. \EndTag
  \end{equation}
\end{antwort}






%: Question 113
\begin{frage}\label{01_hldi}
  Kennen Sie eine Verallgemeinerung der Hölder'schen Ungleichung 
  aus Frage \ref{hoelder} für Integrale?
\end{frage}

\begin{antwort}
  Sei $A\subset \RR$ ein Intervall und $f$ und $g$ Funktionen 
  $\RR \to \CC$, {\sd} -- relativ zu dem benutzten Integralbegriff --   
  $f$ und $g$ über $A$ integrierbar sind. (Je nachdem, welchen 
  Integralbegriff man benutzt, steht auch fest, ob $A$ kompakt ist oder 
  auch offen und unbegrenzt sein darf.) Weiter 
  seien $p,q \ge 1$ Zahlen, für die $\frac{1}{p}+\frac{1}{q}=1$ gilt. 

  Unter diesen Voraussetzungen lautet die Integralversion der 
  Hölder'schen Ungleichung. 
  \begin{equation}
    \int_A |f(x)g(x)| \difx \le 
    \left( \int_A \left|f(x)\right|^p \difx \right)^{1/p} \cdot 
    \left( \int_A \left|g(x)\right|^q \difx \right)^{1/q}.
    \EndTag
  \end{equation}
\end{antwort}






%: Question 114
\begin{frage}
  Wie erhält man aus der Hölder'schen Ungleichung die Cauchy-Schwarz'sche 
  Ungleichung?
\end{frage}

\begin{antwort}
  Die Cauchy-Schwarz'sche Ungleichung ist ein Spezialfall der Hölder'schen 
  Ungleichung für $p=q=2$.
  \AntEnd
\end{antwort}   


%: Question 115
\begin{frage}
  Wie kann man die Ungleichung 
  \[
  \| z_1+z_2+\ldots+z_n\| \le \|z_1\|+\|z_2\|+\ldots + \|z_n\| 
  \]
  für $(z_1, \ldots, z_n) \in K^n$ ($K=\RR$ oder $K=\CC$) und beliebige 
  Normen $\|\quad \|\,:\,K^n \to \CC$ beweisen?
\end{frage}


\begin{antwort}
  Für $n=2$ ist die Ungleichung gerade die Dreiecksungleichung, 
  die definitionsgemäß von jeder Norm erfüllt ist. 
  Der Rest folgt dann durch vollständige 
  Induktion. Aus der Gültigkeit der Ungleichung für $n$ Vektoren 
  folgt nämlich mit der Dreiecksungleichung
  \begin{equation}
    \| z_1+\ldots + z_n + z_{n+1} \| \le 
    \| z_1+\ldots +z_n \|+\| z_{n+1} \| \le 
    \| z_1 \|+\ldots +\|z_{n}\|+\|z_{n+1}\|.
    \EndTag
  \end{equation}
\end{antwort}


%%% Local Variables: 
%%% mode: latex
%%% TeX-master: "master"
%%% End: 
